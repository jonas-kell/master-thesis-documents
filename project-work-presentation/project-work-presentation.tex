\documentclass[aspectratio=169]{beamer} % includes \pause in render
% \documentclass[aspectratio=169, handout]{beamer} % do not include \pause in render

\usetheme[neutralbackground]{uniamntf}

\title[Computer aided calculations]{\vspace{-2em}Computer aided Analytical Calculations for Physical Many-Body Problems}
\subtitle{ - Project Work Presentation - }

\author{Jonas Kell}
\institute[TP III]{Chair for theoretical Physics III}

\date[15.05.2024]{$15^{\text{th}}$ of Mai 2024}

\acknowledgement{
    Jonas Kell\\
    University of Augsburg\\
    jonas.kell@student.uni-augsburg.de\\
    www.uni-augsburg.de
}

\newcommand{\blankfootnote}[1]{%
\let\thefootnote\relax\footnotetext{#1}%
}
\newcommand{\tab}{%
\,\,\,\,
}

% bibtex/biber
\usepackage[backend=biber, style=phys, biblabel=brackets]{biblatex} % citations with "modern" backend and an physics-accepted citation style
\addbibresource{literature.bib}

\newenvironment{wideitemize}{\itemize\addtolength{\itemsep}{0.3em}}{\enditemize}

%! notes control

%\setbeameroption{hide notes} % Only slides
%\setbeameroption{show only notes} % Only notes
\setbeameroption{show notes on second screen=right} % Both, use for pympress mode

% presentation tool https://github.com/Cimbali/pympress
% run: pympress project-work-presentation.pdf 

%! notes control

\begin{document}
    \begin{frame}[t,plain] 
        \maketitle
    \end{frame}

    \note[enumerate]{
        \item Welcome
        \item Presentation of "Project Work" (Projektarbeit)
        \item Runs parallel to the "Practical Training" (Fachpraktikum)
        \item Presentation in this Group-Slot
    }

    \begin{frame}
        \frametitle{Outline}
        \tableofcontents[pausesections] % pause toc before each section
        % \tableofcontents % all toc at once

        % toc notes do need to live inside the frame, to appear on all animated slides
        \note[item] {
            Introduction of theme worked on in Project-Work, Practical Training and Master Thesis
            \begin{itemize}
                \item The mathematical Problem of Many-Body Physics
            \end{itemize}
        }
        \note[item] {
            Focus of the report is on the mathematical details. I have a history in Computer-Science, therefore would like here to present some techniques I learned in my work as well as studies as a Computer scientist, that I think might be quite useful to apply in the context of such a working group.
            \begin{itemize}
                \item The computational/notational problem of "doing maths easily"
            \end{itemize}
        }
        \note[item] {
            What was there to make it easier (Off the shelf solutions)
        }
        \note[item] {
            What I did:
            \begin{itemize}
                \item Math Manipulator
                \item Tricks to use to improve your Python
                \item Latex for presentations
            \end{itemize}
        }
        \note[item] {
            Gist: after I (somewhat of a computer scientist by trade) learned the process of being a theoretical physicist, I want to bring back some tools/workflows to maybe improve someones life here at TP III  
        }
    \end{frame}


    \section{TODO}
    \begin{frame}[t]
        \frametitle{TODO}
        
        \vspace{-0.5em}
        \begin{itemize}
            \item TODO
        \end{itemize}

        % notes 
        \onslide % on all slides of frame
        \note[item] {
            TODO
        }
    \end{frame}

    \section{Math-Manipulator}
    \begin{frame}
        \frametitle{title}
    \end{frame}

    \subsection{What did I do?}
        \begin{frame}
            \frametitle{title}
        \end{frame}

    \subsection{How can you use it}
        \begin{frame}
            \frametitle{title}
        \end{frame}

    \section{Custom Python Scripts (SymPy)}
    \begin{frame}
        \frametitle{title}
    \end{frame}

    \note[enumerate]{
        \item asdadasdadsa
    }
    \section{Presentations? - Custom Beamer Template}
    \begin{frame}{Beamer: LaTeX way of writing Presentations}
        \begin{itemize}
            \item Write presentations like your papers/thesis in \LaTeX \pause
            \item Reuse formulas/images/code/sources \pause
            \item Consistent style \& references    \pause
            \item Version control    \pause
            \item Easier collaboration
        \end{itemize}

        \blankfootnote{Beamer \cite{beamerPackageCtan} \tab{} Template \cite{selfBeamerTemplateMNTF}}

        % notes 
        \onslide % on all slides of frame
        \note[item] {
            Re-use tooling, editor, setup (even this on overleaf alone much better that shared powerpoint presentation)
        }
        \note[item] {
            This presentation, report and thesis all live in one git-repository and share resources
        }
        \note[item] {
            Everything I ever work on is hosted in Git.
            \begin{itemize}
                \item Without Version Control it is no longer possible for me to work effectively
                \item Every workflow is massively ensured by it
                \item Also for single-person work, but built in collaboration, synchronization and backup
            \end{itemize}
        }
    \end{frame}

    \begin{frame}{Minimal example for beamer presentation}
        \begin{columns}
            \column{0.4\textwidth}
                \includegraphics[width=1.1\textwidth]{./beamer-minimal-example/code.png}
            \column{0.4\textwidth}
                \includegraphics[width=0.9\textwidth]{./beamer-minimal-example/result.png}
        \end{columns}

        % notes 
        \onslide % on all slides of frame
        \note[item] {
            29 lines gets you a basic presentation
        }
        \note[item] {
            First time very much slower as putting together by hand in PowerPoint
        }
        \note[item] {
            If you want to do crazy stuff, possible but really hard
        }
        \note[item] {
            Actually quite a timesaver when you have an example/template to work of and do not require crazy levels of features
        }
        \note[item] {
            When it works, perfectly portable (just a pdf), high performant and versatile (output rendering of presentation, notes, animations from one source)
        }
        \note[item] {
            If proper presentation tools, has all the bells and whistles (presentation timer, pointer, multi-view presentation and more)
        }
    \end{frame}



    \section*{Summary \& Conclusion}
    {
        \setbeamertemplate{frametitle}[uniamntfack] % use the acknowledgment-style for this slide
        \begin{frame}[plain]{Acknowledgment}
            \vspace{1cm}

            Thank you for your kind attention\\

            \vspace{1cm}
            \makebox[0.4\textwidth][c]{
                \includegraphics[width=0.3\textwidth]{extra-slides/qr-github.png}
            }
        \end{frame}
    }

    \note[enumerate]{
        \item Thank you for your kind attention
        \item All tools and other resources are referenced in the presentation
        \item You can also find everything on my Github
    }

    \begingroup % group to not count pages from here
        \begin{frame}[allowframebreaks,noframenumbering]
            \frametitle{References}
            \nocite{*}
            \printbibliography[title={Bibliography}]
        \end{frame}
        
        \section*{Extra slides}
            
\begin{frame}[noframenumbering]{Theme alternative: Faculty of App. Computer Science}
    \hspace{1cm}\includegraphics[height=.7\paperheight]{./../latex-beamer-template/rendered-preview-pictures/FAICompilation.png}
\end{frame}

\note[enumerate]{
    \item Template does also do "Faculty of Applied Computer Science" presentations out of the box
    \item Alternate styles (e.g. title slide) available
}

\begin{frame}[noframenumbering]{Backup-Solution Problem 1}
    \hspace{2cm}\includegraphics[height=.72\paperheight]{./math-manipulator-calculations/fkp_page_15_backup_solution.png}
\end{frame}

\begin{frame}[noframenumbering]{Backup-Solution Problem 2}
    \hspace{0.6cm}\includegraphics[height=.72\paperheight]{./math-manipulator-calculations/fkp_page_33_backup_solution.png}
\end{frame}

\begin{frame}[noframenumbering, t]{Backup: Simplification V-Parts}
    \vspace{-0.2cm}
    \begin{minipage}[t]{0.6\textwidth}
        \vspace{0pt}
        \hspace{-2em}
        \includegraphics[width=.5\paperwidth]{./math-manipulator-calculations/simplification_cparts_backup_solution.png}
    \end{minipage}%
    \begin{minipage}[t]{0.6\textwidth}
        \vspace{0pt}
        \hspace{-3em}
        \includegraphics[width=.41\paperwidth]{./math-manipulator-calculations/simplification_cparts.pdf}
    \end{minipage}
\end{frame}

\note[enumerate]{
    \item Better optimization was found, rewriting in number operators DOES save on terms (however only for B and C, A still same number of terms and computational requirement, even if it looks nicer)
    \item Doesn't help at all with computing more efficiently, so python script still exceptinally usefull
    \item Wouldn't have come so far, if simple "brute-force" solution wasn't done in the first place
}

    \endgroup

\end{document}
