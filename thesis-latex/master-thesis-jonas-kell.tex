\documentclass[
headings=optiontohead,              % allows double headers
12pt,                               % fontsize 
DIV=13,                             % koma script diveider amount. tells koma how much of the site can be written to
twoside=false,                      % if set to true, automatically formats as book style with different left and right pages
open=right,                         % starting page on twosided texts 
BCOR=00mm,                          % correction that accounts for the center of the pages being glued in
toc=bibliographynumbered            % bibliography gets a number and is listed in the table of contents
]{scrreport}

\usepackage[utf8]{inputenc}                     % correct encoding of output, technically not needed anymore
\usepackage[T1]{fontenc}                        % correct encoding of output, technically not needed anymore
\usepackage[english]{babel}                     % font that supports English
\usepackage{upgreek}                            % non-cursive Greek letters
\usepackage[stretch=10,shrink=10,protrusion=true,expansion=true,final]{microtype} % prettier block format
\usepackage{hyperref}                           % links for everything
\usepackage{color}                              % allows for setting in different colors
\usepackage[autooneside=false,automark]{scrlayer-scrpage} % page-style with "Kolumnentitel" (title of current chapter is displayed at the top)
\usepackage[sb]{libertinus}                     % use the font libertinus (needs to be installed from the web)
\usepackage[slantedGreek]{libertinust1math}     % math mode improvement for libertinus
\usepackage{siunitx}                            % physical units setting
\usepackage{icomma}                             % commas in lists get extra space if needed                        
\usepackage{amsfonts,amssymb,amstext,amsmath,amsthm,bbm} % better math mode (\mathrm and \text) and symbols
\usepackage{xspace}                             % works to improve own commands and provides "\xspace"-command, that puts a space if needed
\usepackage{ifthen}                             % more control over non-obligatory parameters
\usepackage{titling}                            % get title values as macros
\usepackage[onehalfspacing]{setspace}           % control the spacing between lines and in enumeration lists
\usepackage[backend=biber, style=phys, biblabel=brackets]{biblatex} % citations with "modern" backend and an physics-accepted citation style
\usepackage{graphicx}                           % work with graphics 
\usepackage{ragged2e}                           % ragged-commands (when no block format is wanted)
\usepackage{pdfpages}                           % allows including of pdfs into this pdf
\usepackage{booktabs}                           % better table formatting
\usepackage{multicol}                           % allows for the definition of multi-columns in tables
\usepackage{multirow}                           % allows for the definition of multi-row-tables instead of just multi-column
\usepackage[section]{placeins}                  % provides the command "\FloatBarrier" to control the end of floatable regions for figures/tables
\usepackage{float}                              % provides the "H" option for forcing placement of a figure
\usepackage{floatpag}                           % make it possible for float-pages to not have a page number
\usepackage{url}                                % sometimes needed by biblatex, technically no longer needed
\usepackage{minted}                             % nice code highlighting (needs Python Package to compile!!)
\usepackage{accents}                            % better control over accents
\usepackage{mathtools}                          % more math control possibilities
\usepackage[autostyle=true]{csquotes}           % context-sensitive-quotes -> quotation marks that are set correctly for the context
\usepackage{physics}                            % bra-ket and more
\usepackage{nicematrix}                         % label row/cols on matrix
\usepackage{caption}                            % caption of different environments
\usepackage{sidecap}                            % caption may be next to the figure
\sidecaptionvpos{figure}{c}
\usepackage{tikz}
\usetikzlibrary{calc}
  % required packages

\title{Classical networks for the Hubbard model with a tilted potential}
\author{Jonas Kell}
\date{13$^\text{th}$ February 2025}
\newcommand{\pdfkeywords}{neural networks, ai, physics, quantum mechanics, machine learning, analytical calculations, classical networks, hubbard model}
\newcommand{\pdfsubject}{Master Thesis}

\graphicspath{{./../shared-latex/images/}}              % custom paths for folders in that graphics can be found

\hypersetup
{
bookmarksnumbered=true,
bookmarksopen=true,
bookmarksopenlevel=2,
colorlinks=true,
linkcolor=dblue,                                    % dark blue linkcolor
urlcolor=dblue,                                     % dark blue linkcolor
citecolor=dblue,                                    % dark blue linkcolor
pdfauthor = {Jonas Kell},                           % write details into the expanded file properties
pdftitle = {\thetitle},
pdfkeywords = {\pdfkeywords},  
pdfsubject = {\pdfsubject}                      
}

\sisetup                                % setup for siunitx
{
detect-all,
locale=US,                              % language setup for siunitx
range-phrase={ \text{to} },             % word that is put into an si range
range-units = single,                 % better display of error ranges
per-mode=symbol-or-fraction,            % more dynamic frac usage in inline/displaymath mode
separate-uncertainty,                   % for better +- , \pm when including an error range 
}

\AtBeginDocument{
	\let\mathbb\relax
	\DeclareMathAlphabet\PazoBB{U}{fplmbb}{m}{n}
	\newcommand{\mathbb}{\PazoBB}
}       %more options to the \mathbb command

\setminted[]{
    xleftmargin=0cm,
    xrightmargin=0cm,
    frame=single,
    framesep=.25cm,
    linenos,
    tabsize=2,
    breaklines,
    breakafter=.],
    breakaftersymbolpre= ,
}           %configure the minted code-highlighting style

\addbibresource{literature.bib}              %initialize bibtex with correct file

\NiceMatrixOptions{
code-for-first-row = \color{dblue} ,
code-for-last-row = \color{dblue} ,
code-for-first-col = \color{dblue} ,
code-for-last-col = \color{dblue}
}
                % another file that holds the package/document configuration
\linespread{1.1}                                    % line-spacing can be controlled here

\clubpenalty10000                                   % Schusterjunge, orphan
\widowpenalty10000                                  % Hurenkind, Witwe
\displaywidowpenalty=10000                          % Make document obey stricter rules considering "Schusterjungen" and "Hurenkinder"
\setcounter{biburlnumpenalty}{100}
\setcounter{biburlucpenalty}{100}
\setcounter{biburllcpenalty}{100}
\renewcommand{\topfraction}{0.8}                    % allows for more chilled "text to image ratio" 
\renewcommand{\bottomfraction}{0.8}
\renewcommand{\textfraction}{0.1}
\renewcommand{\floatpagefraction}{0.8}

% \renewcaptionname{ngerman}{\figurename}{Abb.}       %"Figure" becomes "Fig." in English
\setcapindent{0cm}                                  %useful if image captions have multiple lines. Removers indentation below "Fig."
\setlength{\parindent}{0cm}                         %removes indentation at start of new paragraphs
\setlength{\abovecaptionskip}{0.0cm}

%bibliography slots are redefined/modified here
\DeclareFieldFormat{journaltitle}{\textsl{#1}\isdot}
\DeclareFieldFormat{titlecase}{{#1}}


%COLORS
\definecolor{dblue}{rgb}{0,0,0.5}
\definecolor{dred}{rgb}{0.5,0,0}
\definecolor{dgrey}{rgb}{0.5,0.5,0.5}
\definecolor{lgrey}{rgb}{0.8,0.8,0.8}
\definecolor{textred}{RGB}{170,0,0}
\definecolor{textyellow}{RGB}{170,170,0}
\definecolor{textgreen}{RGB}{0,170,0}
\definecolor{textblue}{RGB}{0,0,170}


%overwrite the coma-script definitions
\addtokomafont{pagehead}{\normalfont\color{dgrey}}                  %overwrite the coma-script definitions
\addtokomafont{sectioning}{\rmfamily\color{dblue}\boldmath}         %rmfamily puts headings in "normal" "serif-font" instead of "sans-serif"  boldmath ensures a bold math font in subscripts
\addtokomafont{captionlabel}{\bfseries\footnotesize}                %better Fig. format
\addtokomafont{caption}{\footnotesize}                          


% headline spacing
\RedeclareSectionCommand[beforeskip=0cm,afterskip=1cm]{chapter}                % another file that holds format information
%! Ref-Commands 
\newcommand*{\fullref}[1]{\hyperref[{#1}]{\textit{\autoref*{#1} \nameref*{#1}}}}
\newcommand*{\fullpage}[1]{\hyperref[{#1}]{Seite \pageref*{#1}}}
\newcommand*{\fullpages}[1]{\hyperref[{#1}]{Seiten \pageref*{#1}ff}}

%! Math operators and other small conveniences
\newcommand\thickbar[1]{\accentset{\rule{.6em}{.8pt}}{#1}}
\DeclareMathOperator{\ggt}{ggT}
\DeclareMathOperator{\kgv}{kgV}
\DeclarePairedDelimiter\ceil{\lceil}{\rceil}
\DeclarePairedDelimiter\floor{\lfloor}{\rfloor}
\renewcommand*{\arraystretch}{0.8}

%! qm commands
\newcommand{\hamiltonian}{\ensuremath{\mathcal{H}}\xspace}
\newcommand{\operator}{\ensuremath{\hat{\mathrm{O}}}\xspace}
\newcommand{\up}{\ensuremath{\uparrow}\xspace}
\newcommand{\down}{\ensuremath{\downarrow}\xspace}

\newcommand{\filepath}[2]{\colorbox{lgrey}{#1: \texttt{#2}}\xspace}

%! replace German quotation marks with English ones (I always write the left and right quotes explicitly with the German command variants as they are nice to debug and already in my macros. Then in the end, I can swap them with the desired replacement quotes of my choice)
\renewcommand*{\glqq}{``}
\renewcommand*{\grqq}{''}

\newcommand{\bigo}[1]{\ensuremath{\mathcal{O}(#1)}}

\newcommand{\schroedingerPicture}{\text{S}}
\newcommand{\interactionPicture}{\text{I}}

\newcommand{\generalop}[6]{\ensuremath{\hat{\text{#1}}^{#2\xspace #3}_{#4#5#6}}\xspace}
\newcommand{\withspinop}[4][]{\generalop{c}{#4}{#1}{#2}{,\,}{#3}}
\newcommand{\cop}[3][]{\generalop{c}{#3}{#1}{#2}{}{}}
\newcommand{\dop}[3][]{\generalop{d}{#3}{#1}{#2}{}{}}
\newcommand{\nop}[3][]{\generalop{n}{}{#1}{#2}{,\,}{#3}}

\newcommand{\pictureHamiltonian}[1][]{\ensuremath{\hamiltonian^{#1}}\xspace}
\newcommand{\Vhamiltonian}[1][]{\ensuremath{\text{V}}^{#1}\xspace}
\newcommand{\HzeroHamiltonian}[1][]{\ensuremath{\hamiltonian_0}^{#1}\xspace}

\newcommand{\lsum}[1][l]{\ensuremath{\sum\limits_{#1}}}
\newcommand{\neighborsum}[3][]{\ensuremath{\sum\limits_{\left\langle#2, #3\right\rangle#1}}}
\newcommand{\neighborsumWSpin}[3]{\neighborsum[,\,#3]{#1}{#2}}

\newcommand{\epsl}[1][l]{\ensuremath{\varepsilon_{#1}}}
              % another file that holds predefined commands

%! hyphenation commonly used words can be spelled here to provide latex with the correct places to make line breaks
\hyphenation{Li-pid-mono-lage}           % another file that holds predefined hyphenation helpers for unknown words


\begin{document}

% ! Bibliography, page numbering and Title setups
\thispagestyle{empty}                           % make sure title page is not numbered or anything else


\newcommand{\mail}{jonas.kell@student.uni-augsburg.de}



\begin{titlepage}
\makebox[\textwidth][c]{\includegraphics[width=0.5\textwidth]{logo_uni_augsburg.jpg}}
    
\color{dblue}

\begin{center}
    \vspace*{2cm}
    \Huge
    \textbf{\thetitle}

    \vspace*{1.5cm}
    \color{black}
    \textbf{Master Thesis}

    \vspace*{1cm}
    \normalsize
    submitted by\\
    \LARGE
    \theauthor\\\vspace*{0.3cm}
    \normalsize
    on \thedate

    \vspace{1.8cm}
    \color{black}
    \emph{University of Augsburg}\\
    \emph{Faculty of Mathematics, Natural Sciences, and Materials Engineering}\\
    \emph{Institute of Physics}\\
    \emph{Chair for theoretical Physics III}

    \vfill

    \begin{tabular}{rl}
        1$^\text{st}$ Corrector: &Prof. Dr. Markus Heyl\\
        2$^\text{nd}$ Corrector: &??\\ %TODO
    \end{tabular}
\end{center}

\end{titlepage}
                               % include title-page
\cleardoublepage                                % make sure, that if double-page is active, to reset the double page counter
\pagestyle{scrheadings}                         % puts current chapters title into the header in small gray font
\pagenumbering{roman}                           % number the pages of the table of contents in roman numerals
\renewcommand{\contentsname}{Table of Contents} % title of table of contents
\tableofcontents                                % table of contents
\noindent\\\\

\addsec*{List of Abbreviations}
\begin{tabular}[h]{p{3cm}|l}
	Abbreviation & Meaning\\
	\hline
	TP III & Chair for theoretical Physics III\\
	AI & Artificial Intelligence\\
	RBM & Restricted Boltzmann Machine\\ 
	GPT & Generative Pre-Trained Transformer\\
	MC & Monte Carlo\\
	VCN & Variational Classical Network\\ 
	TDVP & Time-dependent Variational Principle\\ 
	HPC & High Performance Computing\\ 
	LiCCA & Linux Compute Cluster Augsburg\\ 
\end{tabular}\\\\

\newpage                                   % list of abbreviations, figures, etc
\cleardoublepage                                % make sure, that if double-page is active, to reset the double page counter
\pagenumbering{arabic}                          % number the pages of the main document in Arabic numerals

% After this, the redefinition of the "Kolumnentitel" takes place
\clearpairofpagestyles
\ihead{\leftmark}
\ohead{\Ifstr{\leftmark}{\rightmark}{}{\rightmark}}
\cfoot*{\pagemark}
% End of the "Kolumnentitel" redefinition

%  Continuously number equations (basically all equations are in chapter 2, that makes it unnecessary and if we have over 100 equations anyway, it is cleaner to just number them through globally)
\renewcommand{\theequation}{\arabic{equation}} % Use global numbering

% ! Main Document Body
\chapter{Introduction}
\label{sec:introduction}
\section{TODO}
    \begin{frame}[t]
        \frametitle{TODO}
        
        \vspace{-0.5em}
        \begin{itemize}
            \item TODO
        \end{itemize}

        % notes 
        \onslide % on all slides of frame
        \note[item] {
            TODO
        }
    \end{frame}

\FloatBarrier

\chapter{Theory}
\label{sec:theory}

% Disclaimer, this is partly taken from report
Some parts of \fullref{sec:theory} were first written for a \emph{practical training report}\\ \filepath{\cite{selfDocument}}{/practical-training-latex} that was part of initial works on the thesis-subject.
The calculations and text was supplemented and adapted for this final thesis. 
How the parts have been adapted, as well as the original author being the thesis author, can be verified in the version history of the repository \cite{selfDocument}.

    \section{Physical Basics}
    \label{sec:theory-physics}
    \section{TODO}
    \begin{frame}[t]
        \frametitle{TODO}
        
        \vspace{-0.5em}
        \begin{itemize}
            \item TODO
        \end{itemize}

        % notes 
        \onslide % on all slides of frame
        \note[item] {
            TODO
        }
    \end{frame}

    \FloatBarrier
        \subsection{Different Particle-Types, their Operators and Mappings}
        \label{sec:particles}
        To describe different particle-classes in the setting of many-body-physics on discrete lattices, the \emph{ladder operator formalism} is a useful tool. 
The inherit properties of the respective particle are guaranteed to be upheld by the \emph{(anti-) commutation relations}.

In some cases, transformations may be used to map problems formulated in one particle type onto equivalent formulations in a different particle type.

For the examples a discrete lattice of arbitrary enumeration - which also takes care of potential additional spin-degrees - is assumed.

\subsubsection*{Bosons}

\begin{equation}
    \label{eq:boson-commutators}
    \begin{split}
        \left[\bop{l}{},\,\bop{m}{}\right] &= 0\\
        \left[\bop{l}{\dagger},\,\bop{m}{\dagger}\right] &= 0\\
        \left[\bop{l}{},\,\bop{m}{\dagger}\right] &= \delta_{l,\,m}\\
    \end{split}
\end{equation}

The bosonic operators follow the commutation-relations described in \autoref{eq:boson-commutators}, with $\delta$ the \emph{Kronecker-Delta} \cite{schwablBookII}, which allows for occupation numbers of all natural numbers or $0$ per site.

\subsubsection*{Fermions}

For fermionic systems it is important, that their wave-function anti-symmetrizes. This results in the anti-commutation relations of  \autoref{eq:fermion-commutators}.

\begin{equation}
    \label{eq:fermion-commutators}
    \begin{split}
        \left\{\cop{l}{},\,\cop{m}{}\right\} &= 0\\
        \left\{\cop{l}{\dagger},\,\cop{m}{\dagger}\right\} &= 0\\
        \left\{\cop{l}{},\,\cop{m}{\dagger}\right\} &= \delta_{l,\,m}\\
    \end{split}
\end{equation}

From \autoref{eq:fermion-commutators}, one can derive that the operator combination $\cop{l}{\dagger}\cop{l}{}$, the \emph{number operator}, is \emph{idempotent}: 

\begin{equation}
    \label{eq:fermion-counting-op-idempotent}
    \left(\cop{l}{\dagger}\cop{l}{}\right)^n = 
    \cop{l}{\dagger}\cop{l}{} \qquad \forall n \in \mathbb{N}
\end{equation}

This is a property that is necessary for performing a specific calculation later and makes sense, once one realizes, this restricts the  fermionic occupations to the values $0$ and $1$.

\subsubsection*{Hard-Core Bosons}

While we require the number operator for the investigated particle type to be idempotent, it would be convenient to work with an operator that commutes and doesn't anti-commute like the fermions. 
In this regard, the hard-core bosons behave like a typical boson, but with extra restrictions that limit the occupation to either $0$ or $1$, meaning $\hop{l}{\dagger}\hop{l}{\dagger} = \hop{l}{}\hop{l}{} = 0$ \cite{hardCoreBosonsBasics}.
The required properties are fulfilled, if the particles obey the commutation-relations from \autoref{eq:hard-core-commutators} \cite{hardCoreBosonsFullCommutationRelation}, which in turn makes it possible to derive the idempotence-relation in \autoref{eq:hc-counting-op-idempotent}.

\begin{equation}
    \label{eq:hard-core-commutators}
    \begin{split}
        \left[\hop{l}{},\,\hop{m}{}\right] &= 0\\
        \left[\hop{l}{\dagger},\,\hop{m}{\dagger}\right] &= 0\\
        \left[\hop{l}{},\,\hop{m}{\dagger}\right] &= \left(1 - 2 \cdot \hop{m}{\dagger}\hop{m}{} \right)\delta_{l,\,m}
    \end{split}
\end{equation}

\begin{equation}
    \label{eq:hc-counting-op-idempotent}
    \left(\hop{l}{\dagger}\hop{l}{}\right)^n = 
    \hop{l}{\dagger}\hop{l}{} \qquad \forall n \in \mathbb{N}
\end{equation}

This additionally grants the helpful property $\hop{l}{\dagger}\hop{l}{} + \hop{l}{}\hop{l}{\dagger}  = 1$.

\subsubsection*{Spins / Pauli-Matrices}

Basic Hamiltonians, like the \emph{Ising-Model} are often expressed in the \emph{spin basis}. 
The operators that act on this basis are the \emph{Pauli matrices}, which have the useful property of being a basis for complex matrices.
Most \emph{quantum computing} formalisms are expressed in the spin basis and heavily rely on the Pauli matrices \cite{quantumBookPaulisAndBasics}:

\begin{equation}
    \label{eq:pauli-matrices}
    \begin{split}
        \pauli{0} = \one = \left(\begin{matrix}
            1& 0 \\
            0& 1
        \end{matrix}\right) 
        \qquad & \qquad
        \pauli{1} = \pauli{x} = \left(\begin{matrix}
            0& 1 \\
            1& 0
        \end{matrix}\right) \\
        \pauli{2} = \pauli{y} = \left(\begin{matrix}
            0& -i \\
            i& 0
        \end{matrix}\right) 
        \qquad & \qquad
        \pauli{3} = \pauli{z} = \left(\begin{matrix}
            1& 0 \\
            0& -1
        \end{matrix}\right)
    \end{split}
\end{equation}

For spin-particles on a lattice, the operators need to obey the relations in \autoref{eq:pauli-commutators}.

\begin{equation}
    \label{eq:pauli-commutators}
    \begin{split}
        \left\{\pauli[l]{$\alpha$},\,\pauli[l]{$\beta$}\right\} &= 2 \cdot \delta_{\alpha,\,\beta}\\
        \left[\pauli[l]{$\alpha$},\,\pauli[m]{$\beta$}\right] &= 2 \cdot i \cdot \varepsilon_{\alpha,\,\beta,\,\gamma} \cdot \delta_{l,\,m}\cdot \pauli[l]{$\gamma$}
    \end{split}
\end{equation}

With the complex unit $i$ and the \emph{Levi-Civita-Symbol} $\varepsilon_{\alpha,\,\beta,\,\gamma}$.

\subsubsection*{Transformation: Express Spins as Fermions}
For mapping between fermions and spins, there exists a well known transformation, the \emph{Jordan-Wigner-Transformation} \cite{jordanWignerBaseCase} (\autoref{eq:jordan-wigner-classical}).

The following transformations use \autoref{eq:jordan-wigner-p-factor}, where the equality follows from the fact that these operators are all different representations of the number operator.

\begin{equation}
    \label{eq:jordan-wigner-p-factor}
\left(-1\right)^{\sum\limits_{k=0}^{l-1} \left( \pauli[k]{z} +1 \right)/2 } = 
\left(-1\right)^{\sum\limits_{k=0}^{l-1}\cop{k}{\dagger}\cop{k}{} } =
\left(-1\right)^{\sum\limits_{k=0}^{l-1}\hop{k}{\dagger}\hop{k}{} } = 
\text{P}(l)
\end{equation}

With the easily verifiable property $\text{P}(l)\cdot \text{P}(l) = 1$. Also $\text{P}(l)$ commutates with all $\cop{m}{}$, $\cop{m}{\dagger}$, $\hop{m}{}$ and $\hop{m}{\dagger}$ if $l\leq m$ and $\left[\text{P}(l),\, \text{P}(m)\right] = 0$.

\begin{equation}
    \label{eq:jordan-wigner-classical}
    \begin{split}
        \cop{l}{\dagger} &= \text{P}(l) \cdot \frac{\pauli[l]{x} + i\cdot \pauli[l]{y}}{2} = \text{P}(l) \cdot \pauli[l]{+}\\
        \cop{l}{} &= \text{P}(l) \cdot \frac{\pauli[l]{x} - i\cdot \pauli[l]{y}}{2} = \text{P}(l) \cdot \pauli[l]{-}\\
        \cop{l}{\dagger}\cop{l}{} &= \left( \pauli[l]{z} +1 \right)/2\\
        &\Longleftrightarrow \\
        \pauli[l]{x} &= \left(\cop{l}{} + \cop{l}{\dagger}\right) \cdot \text{P}(l) \\
        \pauli[l]{y} &= i \cdot \left(\cop{l}{} - \cop{l}{\dagger}\right) \cdot \text{P}(l)\\
        \pauli[l]{z} &= 2 \cdot \cop{l}{\dagger}\cop{l}{} -1
    \end{split}
\end{equation}

For performing commutation-relation calculations, a helper program was written in conjunction with this thesis and related submissions.
The program is called \emph{Math-Manipulator} \cite{selfMathManipulator} and the transformations in this chapter have been validated with it in a supplementary repository:\\ \filepath{\cite{selfMathManipulatorCalculations}}{/jordan-wigner-transformation}.


\subsubsection*{Transformation: Express Fermions as Hard-Core Bosons}
The tensor-network library \emph{ITensor} provides in their documentation an example for a mapping between fermions and hard-core bosons, used to simplify their calculations \cite{itensorFermionizationLibrary}.
The transformations may be called \emph{fermionization/bosonization} or respectively also Jordan-Wigner-Transformation.

\autoref{eq:jordan-wigner-hcb-fermions} presents the transformation:

\begin{equation}
    \label{eq:jordan-wigner-hcb-fermions}
    \begin{split}
        \cop{l}{\dagger} &= \text{P}(l) \cdot \hop{l}{\dagger}\\
        \cop{l}{} &= \text{P}(l) \cdot \hop{l}{}\\
        &\Longleftrightarrow \\
        \hop{l}{\dagger} &= \text{P}(l) \cdot \cop{l}{\dagger}\\
        \hop{l}{} &= \text{P}(l) \cdot \cop{l}{}\\
    \end{split}
\end{equation}

\subsubsection*{Transformation: Express Spins as Hard-Core Bosons}
Finally this confirms the transformation that will be required later: expressing a measurement in the spin-basis in terms of hard-core bosonic operators.

\autoref{eq:jordan-wigner-spin-hcb} can be directly obtained, by plugging \autoref{eq:jordan-wigner-hcb-fermions} into \autoref{eq:jordan-wigner-classical}.

\begin{equation}
    \label{eq:jordan-wigner-spin-hcb}
    \begin{split}
        \pauli[l]{x} &= \hop{l}{} + \hop{l}{\dagger} \\
        \pauli[l]{y} &= i \cdot \left(\hop{l}{} - \hop{l}{\dagger}\right) \\
        \pauli[l]{z} &= 2 \cdot \hop{l}{\dagger}\hop{l}{} -1
    \end{split}
\end{equation}

This connection shows, that spin-$\frac{1}{2}$ systems and hard-core bosonic systems can be used to investigate the same problems in different notations.
\autoref{eq:spin-hcb-state-mapping} shows, what states are equivalent for these two notations.

\begin{equation}
    \label{eq:spin-hcb-state-mapping}
    \begin{split}
        \ketN[\down] &= \left(\begin{matrix}
            0 \\
            1
        \end{matrix}\right) = \ketN[0] \\
        \pauli[]{+}\ketN[\down] = \ketN[\up] &= \left(\begin{matrix}
            1 \\
            0
        \end{matrix}\right) = \ketN[1] = \hop{}{\dagger}\ketN[0] \\
    \end{split}
\end{equation}

This exact mapping is important to conserve sign-consistency across all calculations. 
E.g. swapping the $\ketN[0] = \ketN[\down]$ correspondence to $\ketN[0] = \ketN[\up]$, would introduce a hidden minus sign in observables depending on $\pauli[]{y}$, because of the implicit order of basis states in the definition of $\pauli[]{y}$.
        \FloatBarrier

        \subsection{Hubbard Model and discussed Hamiltonian}
        \label{sec:theory-hubbard-hamiltonian}
        The system in question is described by a \emph{Hubbard Model Hamiltonian}, that includes an electrical field. It is shown in \autoref{eq:main-hamiltonian}.


\begin{equation}
    \label{eq:main-hamiltonian}
    \pictureHamiltonian = \HzeroHamiltonian + \Vhamiltonian
\end{equation}

\begin{equation}
    \label{eq:main-hamiltonian-h0}
    \HzeroHamiltonian = U \cdot \lsum \nop{l}{\up}\nop{l}{\down} + \lsum[l,\,\sigma] \underbrace{\left(\vec{E} \cdot \vec{r_l}\right)}_{\text{\normalsize \epsl}} \nop{l}{\sigma}
\end{equation}

\begin{equation}
    \label{eq:main-hamiltonian-pertubation}
    \Vhamiltonian =  - J \cdot \neighborsumWSpin{l}{m}{\sigma} \left(\withspinop{l}{\sigma}{\dagger} \withspinop{m}{\sigma}{} + \withspinop{m}{\sigma}{\dagger} \withspinop{l}{\sigma}{} \right)
\end{equation}

$\vec{E}$ describes the vector of said electrical field and $\vec{r_l}$ the position of the site with index $l$.
The number operators $\nop{l}{\sigma} = \withspinop{l}{\sigma}{\dagger}\withspinop{l}{\sigma}{}$ measure the occupation on site $l$ with the respective spin $\sigma$. 

$U$, $J$ and \epsl[] are constants that describe the interaction strength. 
They all have the unit of energy. 
The scalar Product that is defined as \epsl[] can be evaluated, based on the system geometry. 
In this case, the system described by a regular square pattern that can be seen in \autoref{fig:geometry-of-square-system}. 
For such a system, the energy difference that is acquired from hopping from site $l$ to $m$ ($\Delta E_{l \rightarrow m}$) (provided, $l$ and $m$ are nearest neighbors) is described by \autoref{eq:energy-difference-hopping}. By assuming a default value for \epsl[0] one gets the relation from \autoref{eq:epsl} for \epsl{} (using the telescoping sum over nearest-neighbors-hopping along the path from $\vec{r_0}$ to $\vec{r_l}$). 

\begin{figure}[htbp]
    \centering
            While the optimizations described in the previous section greatly increase computational efficiency, they need to be applied for each $\HNOft$-order and each $\Phi_\ast(N)$ separately.
The problem is that for two dimensions the number of interactions to consider grows quickly, while visualizing the symmetries of what terms are equivalent gets progressively harder.
When looking at a chain, each higher order expands the interaction range by a one-site-step. 
So taking one more order into account, two sites are added to the possibilities for each interaction - this linearity is still somewhat manageable.
In two dimensions, the number of affected sites in relation to the interaction range grows quadratically, as visualized in \autoref{fig:growing-interaction-range}.

\begin{figure}[htbp]
    \centering
    \begin{tikzpicture}[scale=0.38]
    \definecolor{relevantsitecol}{HTML}{00AA00} % green

    \definecolor{concolhor}{HTML}{AA0000} % red
    \definecolor{concolver}{HTML}{FF8E00} % orange

    \def\m{5} % Change this value to adjust the grid size (m x m)
    \def\side{3} % Change this value to adjust the square side length
    \def\labelsize{\side/6} % Adjust label size
    
    \def\sp{1} % Change this value to adjust horizontal skip between the grids
    \def\order{2}

    \foreach \orderindex in {0,...,\numexpr\order\relax} {

        % Draw connections
        \foreach \x in {0,...,\numexpr\m-1\relax} {
            \foreach \y in {0,...,\numexpr\m-1\relax} {
                \pgfmathsetmacro{\orderindexincr}{\orderindex + 1}

                % Get shift amount
                \pgfmathsetmacro{\shft}{(\m + \sp - 1)*\orderindex*\side}
                
                % Get current node position
                \pgfmathsetmacro{\cx}{\x*\side + \side/2 + \shft}
                \pgfmathsetmacro{\cy}{\y*\side + \side/2 + \side/4}

                % Compute CURRENT Manhattan distance from center
                \pgfmathtruncatemacro{\dista}{abs(\x - (\m-1)/2) + abs(\y - (\m-1)/2)}


                % Connect to right neighbor
                \ifnum\x<\numexpr\m-1\relax

                    \pgfmathsetmacro{\usex}{\x+1}
                    \pgfmathsetmacro{\usey}{\y}

                    \pgfmathtruncatemacro{\distb}{abs(\usex - (\m-1)/2) + abs(\usey - (\m-1)/2)}

                    \pgfmathsetmacro{\nx}{\usex*\side + \side/2 + \shft}
                    \pgfmathsetmacro{\ny}{\usey*\side + \side/2 + \side/4}

                    % Color if both are smaller Manhattan distance
                    \ifnum\dista<\orderindexincr
                        \ifnum\distb<\orderindexincr
                            \ifnum\orderindex=1
                                \edef\useconcol{concolhor}
                            \else
                                \edef\useconcol{gray}
                            \fi
                        \else
                            \edef\useconcol{black}
                        \fi
                    \else
                        \edef\useconcol{black}
                    \fi

                    \draw[line width=2\pgflinewidth, color=\useconcol] (\cx,\cy) -- (\nx,\ny);
                \fi
                

                % Connect to top neighbor
                \ifnum\y<\numexpr\m-1\relax

                    \pgfmathsetmacro{\usex}{\x}
                    \pgfmathsetmacro{\usey}{\y+1}

                    \pgfmathtruncatemacro{\distb}{abs(\usex - (\m-1)/2) + abs(\usey - (\m-1)/2)}

                    \pgfmathsetmacro{\nx}{\usex*\side + \side/2 + \shft}
                    \pgfmathsetmacro{\ny}{\usey*\side + \side/2 + \side/4}

                    % Color if both are smaller Manhattan distance
                    \ifnum\dista<\orderindexincr
                        \ifnum\distb<\orderindexincr
                            \ifnum\orderindex=1
                                \edef\useconcol{concolver}
                            \else
                                \edef\useconcol{gray}
                            \fi
                        \else
                            \edef\useconcol{black}
                        \fi
                    \else
                        \edef\useconcol{black}
                    \fi

                    \draw[line width=2\pgflinewidth, color=\useconcol] (\cx,\cy) -- (\nx,\ny);
                \fi
            }
        }

        % Draw dots and labels with custom text using a loop
        \foreach \x in {0,...,\numexpr\m-1\relax} {
            \foreach \y in {0,...,\numexpr\m-1\relax} {
                % Get shift amount
                \pgfmathsetmacro{\shft}{(\m + \sp - 1)*\orderindex*\side}
                \pgfmathtruncatemacro{\index}{(\m-\y-1) * \m + \x}

                % Get current node position
                \pgfmathsetmacro{\cx}{\x*\side + \side/2 + \shft}
                \pgfmathsetmacro{\cy}{\y*\side + \side/2 + \side/4}

                % Compute Manhattan distance from center
                \pgfmathtruncatemacro{\dist}{abs(\x - (\m-1)/2) + abs(\y - (\m-1)/2)}
                
                % Change color if distance equals \orderindex
                \pgfmathsetmacro{\orderindexincr}{\orderindex + 1}
                \ifnum\dist<\orderindexincr
                    \fill[color=relevantsitecol] (\cx,\cy) circle (0.4);
                \else
                    \fill[color=dblue] (\cx,\cy) circle (0.4);
                \fi
            }
        }
    }

\end{tikzpicture}
    \caption{
        A depiction of the relevant sites and interactions that might be affected when a modification-event (e.g. a single flip on the center site) occurs.
        From left to right, the graphic shows the case for \emph{base-energy only}, \emph{first order perturbation theory} and \emph{second order perturbation theory}.
        In case of a modification to the center-site the occupation of all green sites possibly influences the outcome of the calculation (in the corresponding order). For the first and second order the terms can be identified with their corresponding colored edges.
    }
    \label{fig:growing-interaction-range}
\end{figure}

As show in \fullref{sec:theory-optimizations-analytical}, the described methods require calculating the differences between the effective Hamiltonians on two states that only differ in localized modifications.
Each modification flips the occupation of at least one site. 
To guarantee the differences of the effective Hamiltonians are evaluated correctly, only the terms that contain a modified site-occupation need to be evaluated. 
The terms that only contain non-modified occupation-numbers are guaranteed to cancel.

For the zeroth-order (base-energy) this is swiftly calculated and can be written down in simple terms like in e.g. equations \ref{eq:new-n-flipping} or \ref{eq:simplified-base-energy-hopping}.
In the case of the first order, it is simple enough to identify the terms.
Looking back at \autoref{eq:hn-integrated-first-order}, it becomes clear that all terms are described by \emph{edges} of two adjacent sites.
These relevant edges are colored red and orange in \autoref{fig:growing-interaction-range}. 
The symmetry is clear enough to spot that all horizontal and vertical edges respectively have the same analytical value.
This is because the $\VhamiltonianAnalyticalPartIntegrated{$\ast$}{l}{m}{t}$ are translationally invariant and are complex conjugates of each other when inverting the indices - which is the reason for arriving at $2\cdot 3 \cdot 2 = 12$ terms in \autoref{eq:psi-and-c-choices}, of which 6 are identified in \autoref{table:first-order-identification} (times two for horizontal/vertical versions).

\begin{table}[htbp]
    \centering
    \begin{tabular}{cc|cccccccccccccccc} 
        \toprule
             $l$ & \up      & 0 & 0 & 0 & 0   & 0 & 0 & 0 & 0   & 1 & 1 & 1 & 1   & 1 & 1 & 1 & 1    \\
             $l$ & \down    & 0 & 0 & 0 & 0   & 1 & 1 & 1 & 1   & 0 & 0 & 0 & 0   & 1 & 1 & 1 & 1    \\
             $m$ & \up      & 0 & 0 & 1 & 1   & 0 & 0 & 1 & 1   & 0 & 0 & 1 & 1   & 0 & 0 & 1 & 1    \\
             $m$ & \down    & 0 & 1 & 0 & 1   & 0 & 1 & 0 & 1   & 0 & 1 & 0 & 1   & 0 & 1 & 0 & 1    \\
        \midrule   
    \multicolumn{2}{c|}{$\VhamiltonianAnalyticalPartIntegrated{$\ast$}{l}{m}{t}$}
                            &   &   &   &    
                                              & A &   & C & 
                                                                & A & C &   &   
                                                                                  & B & A & A &      \\
    \multicolumn{2}{c|}{$\VhamiltonianAnalyticalPartIntegrated{$\ast$}{m}{l}{t}$}
                            &   & A & A & B  
                                              &   &   & C & A
                                                                &   & C &   & A 
                                                                                  &   &   &   &      \\
        \bottomrule
    \end{tabular}
    \vspace{0.5cm}
    \caption{
        A list of all occupation-configurations for the first order terms.
        The indices of the two involved sites are $l$ and $m$. Each hold a site for spin up and down particles \up and \down.
        Not all 16 configurations have a representative term that results from the perturbation theory.
        The ones that do reference the letters A, B and C from the $\VhamiltonianAnalyticalPartIntegrated{$\ast$}{l}{m}{t}$ in \autoref{eq:hn-integrated-first-order}.
        A second line takes the other combination $l \leftrightarrow m$ into account.
    }
    \label{table:first-order-identification}
\end{table}

Looking at all 16 occupation-number-combinations, the complexity is already extreme. 
Some combinations contribute to no terms, some contribute multiple times and some terms have more states contributing to them, some less.
Identifying and analytically optimizing the first order therefore is \emph{just} possible, albeit complicated.
For the variational parametrization, it would be possible and maybe even more efficient to just have 16 parameters, mapping one to one onto the occupation-number-combination of the respective bond. 
In second order, however, this all breaks down.
Taking a second look at \autoref{fig:growing-interaction-range}, each term in \autoref{eq:hn-integrated-second-order-final} now corresponds to \emph{two pairs of connections}, of which are 34 different ones with the edges adjacent (assuming no edge-cases, because at the border all is different again) and even more disconnected ones that might or might not be cases from the first order.
% 6 L -> * 4 for turning
% 5 I -> * 2 for turning
For identifying these, one must take rotational and mirror symmetries into account and the check for the border is vastly more complicated.
Because each of these double-bounds is formed by 3 sites with each two spin-degrees, there are in total $2^{3\cdot 2} = 64$ base-cases that need to be mapped.
The numbers of base-cases 16 and 64 here are only comparably low, as they take symmetries into account. 
For brute-forcing all combinations without the manual identification of symmetries, one would get $2^{5\cdot 2} = 1.024$ cases for the first order and $2^{13\cdot 2} = 67.108.864$ - it would most definitely not be a viable strategy to convert each of these into their own parameter.
Even if it was possible, this would diminish the physical bias we hope to get from defining the $\Phi_\ast(N)$ according to the cumulant expansion.

Even more problematic is the case of double-flips, because the interaction-spheres of these two modifications might be independent, overlapping slightly or even so much that both modifications are inside one bond.
While not impossible for the second order, it is highly complicated to analytically compute the most efficient formula for the difference of effective Hamiltonians.
And the process is not easily generalizable to higher orders because of order-specific symmetries.


\subsubsection*{Pre-Computed Interaction Spheres}

In this case, this was solved by defining pre-computed sets of indices that must be taken into account.

Given a formula that needs to sum over all indices once (like \autoref{eq:hn-integrated-first-order}), or once over all indices for each index (like \autoref{eq:hn-integrated-second-order-final} or \autoref{eq:v-squared-hard-computation}), it is clear that they require a runtime-complexity of \bigo{\text{\#}(\text{sites})} or even \bigo{(\text{\#}(\text{sites}))^2} to evaluate.
Yet all of them have so far also appeared as differences of two sums - one for the un-modified state and one for a state that had undergone a localized modification - of which most elements cancel.

One can assume it is possible to compute the neighbors of a site in constant time - still it takes computational time.
This and the problem from before can be solved by taking time at the start of the program to generate a cache of all tuples of indices, which would appear in the sum that is being optimized, but only restricted to a pre-set range around the index of the modification - a \emph{sphere of influence}.
Because even if the generation takes \bigo{(\text{\#}(\text{sites}))^2} to generate this cache for a two-site-modification - after generation the lookup becomes practically \bigo{1} (and is faster than computing the neighbor indices, even if all of that has asymptotically constant complexity).

This means effectively always a constant number of indices needs to be taken into account for all calculations (which then can be evaluated in constant time).
While it may be possible to analytically derive a more optimal solution with more canceling terms, this method easily scales.
Only the radius of the pre-computed cache needs to be chosen to be large enough to be compatible with the used order of the expansion.
These geometry-dependent caches are generated in \filepath{\cite{selfCode}}{/computation-scripts/systemgeometry.py} and the caches are used at various places.

The difference that this makes depends on the system size. 
As the number of influenced indices is constant, the strategy gets relatively more effective the more sites the system has (so that a comparably large portion of them is left out).
For a chain, optimizations will be noticed comparably earlier than for the square lattice - this is because in two dimensions the sphere of influence grows as an area and not linearly at the two edges.
But in the limit this strategy allows to keep the targeted complexity class, while scaling to the order of the expansion.
    \vspace{0.8cm}
    \caption{Graphical representation of how the examined square system is laid out. The sites are labeled with the index $l$ they later can be identified by. Each of the sites has 4 nearest neighbors, except the ones on the borders, that have 3 or only 2 nearest neighbors (as the system is not periodic). A system of side-length $M$ with size $N$ is depicted, which means that it has $M^2 = N$ sites.
    In green, the Electrical field vector $\vec{E}$ is depicted, it is parametrized by the field strength $E$ and the angle $\varphi$.}
    \label{fig:geometry-of-square-system}
\end{figure}

\begin{equation}
    \label{eq:energy-difference-hopping}
    \begin{split}
        \Delta E_{l \rightarrow m} &=  \epsl[m] - \epsl[l] = \vec{E} \cdot \left(\vec{r_m}-\vec{r_l}\right)\\
        &= E \cdot \left[\cos(\varphi)\cdot \left(m \% M-l \% M\right) + \sin(\varphi)\cdot \left(\floor[\bigg]{\frac{m}{M}}-\floor[\bigg]{\frac{l}{M}}\right)\right]
    \end{split}
\end{equation}
\vspace{0.5cm} % equations are not well enough separated otherwise % TODO check in final render
\begin{equation}
    \label{eq:epsl}
    \begin{split}
        \epsl[0] = 0 \Rightarrow
        \epsl[l] &= \epsl[l] - \epsl[0] = \Delta E_{0 \rightarrow x} + \dots + \Delta E_{y \rightarrow l}\\
        &= E \cdot \left[\cos(\varphi)\cdot l \% M + \sin(\varphi)\cdot \floor[\bigg]{\frac{l}{M}}\right]
    \end{split}
\end{equation}

The Hubbard Model \cite{hubbardModelOriginalDerivation} is a spin-depending quantum-mechanical model. 
Because of this, the Hamiltonian includes terms for both the spin directions $\up$ and $\down$ (most of the time denoted by a summation over $\sigma$).

For ease of notation (most importantly in the code), the two spin degrees of freedom may be described by the alternate naming scheme provided in \autoref{eq:alt-naming-scheme}.

\begin{equation}
    \label{eq:alt-naming-scheme}
    \withspinop{l}{\up}{(\dagger)} \leftrightarrow \cop{l}{(\dagger)}
    \qquad\qquad
    \withspinop{l}{\down}{(\dagger)} \leftrightarrow \dop{l}{(\dagger)}
\end{equation}

For simplification purposes, the two spin directions are assumed to have the same coupeling constants ($U$, $J$ and \epsl).

% TODO tex the translated hamiltonian

        \FloatBarrier

        
        \subsection{Interaction Picture and Time Evolution}
        \label{sec:theory-interaction-picture}
        Solving the quantum-mechanical many-body time-evolution problem in general requires at least a time-complexity of \bigo{2^n}.
For some special cases it is possible to solve analytically.
E.g. for 1-dimensional systems, the Hubbard Model can be solved analytically even with the addition of an electrical field \cite{exactSolutionExampleHubbardEvenElectricalField}.

For arbitrary geometry in two or even more dimensions, this is no longer possible.
In that case, approximations are required to lower the computational complexity to get it to a reasonable level to obtain the required measurements.
Here, the calculation is performed in the \emph{Interaction Picture} (relevant equations from \cite{schwablBook}), along the lines of the calculation performed in \cite{variationalClassicalNetworksPaper}.

The equations that were provided so far, show operators from the \emph{Schrödinger Picture}. 
For this section, the operators and states will be labeled whether they are Schrödinger- or Interaction Picture. All un-labeled elements are assumed to be Schrödinger Picture.

A general state in the Schrödinger Picture can be expanded in a basis like shown in \autoref{eq:base-expansion-state}. It is time-evolved using the full Hamiltonian in the Schrödinger Picture (\autoref{eq:time-evolution-schroedinger}, all units so that $\hbar = 1$).

\begin{equation}
    \label{eq:base-expansion-state}
    \ketpsi[\schroedingerPicture]{} = \ketpsiof[\schroedingerPicture]{t=0} = 
    \lsum[N] \ketN{}\underbrace{\braketHelper{N}{\picturePsi[\schroedingerPicture]{}}}_{\text{\normalsize\psiN}} = \lsum[N] \psiN \ketN{}
\end{equation}

\begin{equation}
    \label{eq:time-evolution-schroedinger}
    \ketpsiof[\schroedingerPicture]{t} = e^{-i \pictureHamiltonian[\schroedingerPicture] t} \ketpsiof[\schroedingerPicture]{t=0} = e^{-i \pictureHamiltonian[\schroedingerPicture] t} \ketpsi[\schroedingerPicture]{}
\end{equation}

In the Interaction Picture, operators have a time evolution, that depends only on the \HzeroHamiltonian[\schroedingerPicture]{}-part, shown in \autoref{eq:time-evolution-operator-interaction} (assuming the operators are time-independent in the Schrödinger Picture). 

\begin{equation}
    \label{eq:time-evolution-operator-interaction}
    \begin{split}
        \AopOfT[\interactionPicture] = &e^{i \HzeroHamiltonian[\schroedingerPicture]{} t} \Aop[\schroedingerPicture] e^{-i \HzeroHamiltonian[\schroedingerPicture]{} t}
        \quad \Rightarrow \quad 
        \HzeroHamiltonian[\schroedingerPicture]{} = \HzeroHamiltonian[\interactionPicture]{}= \HzeroHamiltonian[]{}\\
        \text{especially:} \qquad \VhamiltonianOfT[\interactionPicture] = &e^{i \HzeroHamiltonian[\schroedingerPicture]{} t} \Vhamiltonian[\schroedingerPicture] e^{-i \HzeroHamiltonian[\schroedingerPicture]{} t}
    \end{split}
\end{equation}

The time-evolution of a general state in the Interaction Picture on the other hand is obtained like \autoref{eq:time-evolution-interaction}.

\begin{equation}
    \label{eq:time-evolution-interaction}
    \ketpsiof[\interactionPicture]{t} = e^{i \HzeroHamiltonian{} t} \ketpsiof[\schroedingerPicture]{t} \stackrel{\ref{eq:time-evolution-schroedinger}}{=} e^{i \HzeroHamiltonian{} t} e^{-i \pictureHamiltonian[\schroedingerPicture]{} t} \ketpsi[\schroedingerPicture]{} 
    \quad \Rightarrow \quad 
    \ketpsiof[\schroedingerPicture]{t} = e^{-i \HzeroHamiltonian{} t} \ketpsiof[\interactionPicture]{t}
\end{equation}

Differentiating \autoref{eq:time-evolution-interaction} and substituting the special case in \autoref{eq:time-evolution-operator-interaction} along the way, one gets \autoref{eq:equation-of-motion-interaction-picture}.

\begin{equation}
    \label{eq:equation-of-motion-interaction-picture}
    \begin{split}
        \difft{\ketpsiof[\interactionPicture]{t}} &\stackrel{\ref{eq:time-evolution-interaction}}{=}
        \difft{} \left(e^{i \HzeroHamiltonian{} t} e^{-i \pictureHamiltonian[\schroedingerPicture]{} t} \right)\ketpsi[\schroedingerPicture]{} \\
        &\stackrel{\phantom{\ref{eq:time-evolution-interaction}}}{=} -i \cdot e^{i \HzeroHamiltonian{} t}\left( \pictureHamiltonian[\schroedingerPicture]{} - \HzeroHamiltonian{} \right) e^{-i \pictureHamiltonian[\schroedingerPicture]{} t} \ketpsi[\schroedingerPicture]{}\\
        &\stackrel{\ref{eq:main-hamiltonian}}{=} -i \cdot e^{i \HzeroHamiltonian{} t} \Vhamiltonian[\schroedingerPicture] \cdot 1 \cdot e^{-i \pictureHamiltonian[\schroedingerPicture]{} t} \ketpsi[\schroedingerPicture]{}\\
        &\stackrel{\phantom{\ref{eq:main-hamiltonian}}}{=} -i \cdot e^{i \HzeroHamiltonian{} t} \Vhamiltonian[\schroedingerPicture] e^{-i \HzeroHamiltonian{} t}e^{i \HzeroHamiltonian{} t} e^{-i \pictureHamiltonian[\schroedingerPicture]{} t} \ketpsi[\schroedingerPicture]{}\\
        &\stackrel{\ref{eq:time-evolution-interaction}}{=} -i \cdot e^{i \HzeroHamiltonian{} t} \Vhamiltonian[\schroedingerPicture] e^{-i \HzeroHamiltonian{} t}\ketpsiof[\interactionPicture]{t}\\
        &\stackrel{\ref{eq:time-evolution-operator-interaction}}{=} -i \cdot \VhamiltonianOfT[\interactionPicture] \ketpsiof[\interactionPicture]{t}
    \end{split}
\end{equation}

The equation of motion in \autoref{eq:equation-of-motion-interaction-picture}, is solved by the ansatz in \autoref{eq:solve-equation-of-motion} with the \emph{time-evolution-operator} \timeEvolutionOperator[\interactionPicture] and the \emph{time-ordering-operator} \timeOrderingOperator.

\begin{equation}
    \label{eq:solve-equation-of-motion}
    \begin{split}
        \ketpsiof[\interactionPicture]{t} &= \timeOrderingOperator \left\lbrace e^{-i\int\limits_0^t \text{d} t' \VhamiltonianOfTPrime[\interactionPicture]}\right\rbrace \ketpsiof[\interactionPicture]{0}\\
        TODO
    \end{split}
\end{equation}

% TODO
        \FloatBarrier
        
        \subsection{Operators in the Interaction Picture}
        \label{sec:theory-approximation-evaluation}
        As \autoref{eq:cumulant-expansion} reveals, by this process the problem was converted into finding $\VhamiltonianOf[\interactionPicture]{t'}$ to be able to calculate the first two orders of the expansion.

Operators in the Interaction Picture can be derived from their Schrödinger Picture variants like previously shown in \autoref{eq:time-evolution-operator-interaction}. 
With this, one can derive a useful expression for their \emph{equation of motion} (\autoref{eq:schroedinger-op-to-interaction-op}).

\begin{equation}
    \label{eq:schroedinger-op-to-interaction-op}
    \begin{split}
        \difft{}\AopOfT[\interactionPicture] &\stackrel{\phantom{\ref{eq:time-evolution-operator-interaction}}}{=} i \left[\HzeroHamiltonian[\schroedingerPicture], \AopOfT[\interactionPicture]\right]
        %
        \stackrel{\ref{eq:time-evolution-operator-interaction}}{=}
        i \left[\HzeroHamiltonian[\schroedingerPicture],\,e^{i \HzeroHamiltonian[\schroedingerPicture]{} t} \Aop[\schroedingerPicture] e^{-i \HzeroHamiltonian[\schroedingerPicture]{} t}\right]\\
        %
        &\stackrel{\phantom{\ref{eq:time-evolution-operator-interaction}}}{=}
        i e^{i \HzeroHamiltonian[\schroedingerPicture]{} t}\left[\HzeroHamiltonian[\schroedingerPicture],\, \Aop[\schroedingerPicture]\right]e^{-i \HzeroHamiltonian[\schroedingerPicture]{} t}
        %
        \stackrel{\phantom{\ref{eq:time-evolution-operator-interaction}}}{=}
        i \left\{\left[\HzeroHamiltonian[\schroedingerPicture],\, \Aop[\schroedingerPicture]\right]\right\}(t)
    \end{split}
\end{equation}

\autoref{eq:schroedinger-op-to-interaction-op} also defines the operator $\left\{\cdot \right\}(t)$, which simply is $e^{i \HzeroHamiltonian[\schroedingerPicture]{} t} \left\{\cdot \right\} e^{-i \HzeroHamiltonian[\schroedingerPicture]{} t}$.
This section uses $(t)$ for indicating that operators in the Interaction Picture have a time dependence. 
This is superficial and only for better readability, as in fact $\hopOfT[\interactionPicture]{m}{\dagger} = \hop[\interactionPicture]{m}{\dagger} = \left\{\hop[\schroedingerPicture]{m}{\dagger} \right\}(t)$.

If now $\left[\HzeroHamiltonian[\schroedingerPicture],\, \Aop[\schroedingerPicture]\right]$ now is a function of $\Aop[\schroedingerPicture]$, one gets a \emph{differential equation} that can be solved to obtain $\AopOfT[\interactionPicture]$.

For the repetitive calculation of the objects of type $\left[\HzeroHamiltonian[\schroedingerPicture],\, \Aop[\schroedingerPicture]\right]$ for various \Aop[\schroedingerPicture], the tool \emph{Math-Manipulator} was specifically developed. 
Like the transformations in \ref{sec:particles}, calculations in the following section have been validated or calculated from scratch in this tool.
The relevant files can be found in \filepath{\cite{selfMathManipulatorCalculations}}{/calculate-V}, and \filepath{\cite{selfMathManipulatorCalculations}}{/time-evolution}.
The equations \ref{eq:stuff-from-math-manipulator-dd} and \ref{eq:stuff-from-math-manipulator-cm} list exemplary results that were computed with the tool.

\begin{equation}
    \label{eq:stuff-from-math-manipulator-dd}
    \difft{}\left(\dop[\interactionPicture]{m}{\dagger}\dop[\interactionPicture]{m}{}\right)(t)
    = i \left\{\left[\HzeroHamiltonian[],\,\dop[\schroedingerPicture]{m}{\dagger}\dop[\schroedingerPicture]{m}{}\right]\right\}(t) \stackrel{\text{MM}}{=} 0 \quad \Rightarrow \quad \left(\dop[\interactionPicture]{m}{\dagger}\dop[\interactionPicture]{m}{}\right)(t) = \dop[\schroedingerPicture]{m}{\dagger}\dop[\schroedingerPicture]{m}{}
\end{equation}

\begin{equation}
    \label{eq:stuff-from-math-manipulator-cm}
    \begin{split}
        \difft{}\hopOfT[\interactionPicture]{m}{\dagger}
        &\stackrel{\phantom{\ref{eq:time-evolution-operator-interaction}}}{=} i \left\{\left[\HzeroHamiltonian[]{},\,\hop[\schroedingerPicture]{m}{\dagger}\right]\right\}(t) \stackrel{\text{MM}}{=} 
        i \left\{ \left(\epsl[m] + U \dop[\schroedingerPicture]{m}{\dagger}\dop[\schroedingerPicture]{m}{}\right)\hop[\schroedingerPicture]{m}{}\right\}(t)\\
        &\stackrel{\phantom{\ref{eq:time-evolution-operator-interaction}}}{=}i e^{i \HzeroHamiltonian[\schroedingerPicture]{} t} \left(\epsl[m] + U \dop[\schroedingerPicture]{m}{\dagger}\dop[\schroedingerPicture]{m}{}\right)\hop[\schroedingerPicture]{m}{}e^{-i \HzeroHamiltonian[\schroedingerPicture]{} t}\\
        &\stackrel{\phantom{\ref{eq:time-evolution-operator-interaction}}}{=}i \left(\epsl[m] + U e^{i \HzeroHamiltonian[\schroedingerPicture]{} t}\dop[\schroedingerPicture]{m}{\dagger}\dop[\schroedingerPicture]{m}{}e^{-i \HzeroHamiltonian[\schroedingerPicture]{} t}\right)e^{i \HzeroHamiltonian[\schroedingerPicture]{} t}\hop[\schroedingerPicture]{m}{}e^{-i \HzeroHamiltonian[\schroedingerPicture]{} t}\\
        &\stackrel{\ref{eq:time-evolution-operator-interaction}}{=}
        i \left(\epsl[m] + U \left(\dop[\interactionPicture]{m}{\dagger}\dop[\interactionPicture]{m}{}\right)(t)\right) \hopOfT[\interactionPicture]{m}{\dagger}\\
    \end{split}
\end{equation}

With this and the idempotence-relation \autoref{eq:hc-counting-op-idempotent} one can first derive \autoref{eq:operator-out-from-exponent} and finally use an exponential function as a natural ansatz to solve \autoref{eq:stuff-from-math-manipulator-cm} with \autoref{eq:the-final-time-dependence-cop}.

\begin{equation}
    \label{eq:operator-out-from-exponent}
    \begin{split}
        e^{a \cdot \withspinhcop[\schroedingerPicture]{l}{\sigma}{\dagger}\withspinhcop[\schroedingerPicture]{l}{\sigma}{}} &= \sum\limits_{m=0}^\infty \frac{a^m \left(\withspinhcop[\schroedingerPicture]{l}{\sigma}{\dagger}\withspinhcop[\schroedingerPicture]{l}{\sigma}{}\right)^m}{m!}
        \stackrel{\ref{eq:hc-counting-op-idempotent}}{=} 1 + \left[\sum\limits_{m=1}^\infty \frac{a^m}{m!}\right]\cdot \left(\withspinhcop[\schroedingerPicture]{l}{\sigma}{\dagger}\withspinhcop[\schroedingerPicture]{l}{\sigma}{}\right)\\
        &= 1 + \left[\sum\limits_{m=0}^\infty \frac{a^m}{m!}-1\right]\cdot \left(\withspinhcop[\schroedingerPicture]{l}{\sigma}{\dagger}\withspinhcop[\schroedingerPicture]{l}{\sigma}{}\right)
         = 1 + \left(e^a-1\right)\cdot \withspinhcop[\schroedingerPicture]{l}{\sigma}{\dagger}\withspinhcop[\schroedingerPicture]{l}{\sigma}{}
    \end{split}
\end{equation}


\begin{equation}
    \label{eq:the-final-time-dependence-cop}
    \begin{split}
        \stackrel{\ref{eq:stuff-from-math-manipulator-cm},\, \ref{eq:stuff-from-math-manipulator-dd}}{\Longrightarrow}
        \hopOfT[\interactionPicture]{m}{\dagger}
        &\stackrel{\phantom{\ref{eq:stuff-from-math-manipulator-dd}}}{=}
        e^{i \cdot \epsl[m]\cdot  t + i \cdot U \left(\dop[\interactionPicture]{m}{\dagger}\dop[\interactionPicture]{m}{}\right)(t) \cdot  t}  \hop[\schroedingerPicture]{m}{\dagger} 
        \stackrel{\ref{eq:stuff-from-math-manipulator-dd}}{=}
        e^{i \cdot \epsl[m]\cdot  t + i \cdot U \cdot \dop[\schroedingerPicture]{m}{\dagger}\dop[\schroedingerPicture]{m}{} \cdot  t}  \hop[\schroedingerPicture]{m}{\dagger} \\
        &\stackrel{\ref{eq:operator-out-from-exponent}}{=}
        e^{i \cdot \epsl[m]\cdot  t}\left(1 + \left(e^{i  \cdot U  \cdot  t}-1\right) \dop[\schroedingerPicture]{m}{\dagger}\dop[\schroedingerPicture]{m}{}\right)  \hop[\schroedingerPicture]{m}{\dagger}
    \end{split}
\end{equation}

From analogous calculations follows:

\begin{equation*}
    \begin{split}
        \hopOfT[\interactionPicture]{m}{\dagger}  &= e^{i \cdot \epsl[m]\cdot  t}\left(1 + \left(e^{i  \cdot U  \cdot  t}-1\right) \dop[\schroedingerPicture]{m}{\dagger}\dop[\schroedingerPicture]{m}{}\right)  \hop[\schroedingerPicture]{m}{\dagger}\\
        \hopOfT[\interactionPicture]{m}{}  &= e^{-i \cdot \epsl[m]\cdot  t}\left(1 + \left(e^{-i  \cdot U  \cdot  t}-1\right) \dop[\schroedingerPicture]{m}{\dagger}\dop[\schroedingerPicture]{m}{}\right)  \hop[\schroedingerPicture]{m}{}\\
        \dopOfT[\interactionPicture]{m}{\dagger}  &= e^{i \cdot \epsl[m]\cdot  t}\left(1 + \left(e^{i  \cdot U  \cdot  t}-1\right) \hop[\schroedingerPicture]{m}{\dagger}\hop[\schroedingerPicture]{m}{}\right)  \dop[\schroedingerPicture]{m}{\dagger}\\
        \dopOfT[\interactionPicture]{m}{}  &= e^{-i \cdot \epsl[m]\cdot  t}\left(1 + \left(e^{-i  \cdot U  \cdot  t}-1\right) \hop[\schroedingerPicture]{m}{\dagger}\hop[\schroedingerPicture]{m}{}\right)  \dop[\schroedingerPicture]{m}{}\\
    \end{split}
\end{equation*}

Injecting these derivations into \autoref{eq:main-hamiltonian-perturbation-full-sum}, finally \VhamiltonianOfT[\interactionPicture] can be derived in \autoref{eq:interaction-picture-v-ham-full}.

\begin{equation}
    \label{eq:interaction-picture-v-ham-full}
    \begin{split}
        \VhamiltonianOfT[\interactionPicture] &
        \stackrel{\phantom{\text{MM}}}{=}
        \left\{\Vhamiltonian[\schroedingerPicture]\right\}(t) \stackrel{\ref{eq:main-hamiltonian-perturbation-full-sum}}{=}
        - J \cdot \fullneighborsum{l}{m}  \left\{
              \left(\hop[\schroedingerPicture]{l}{\dagger}\hop[\schroedingerPicture]{m}{} + \dop[\schroedingerPicture]{l}{\dagger}\dop[\schroedingerPicture]{m}{} \right)
        \right\}(t) \\
        &\stackrel{\phantom{\text{MM}}}{=}
        -J \cdot \fullneighborsum{l}{m}\left(\hopOfT[\interactionPicture]{l}{\dagger}\hopOfT[\interactionPicture]{m}{} + \dopOfT[\interactionPicture]{l}{\dagger}\dopOfT[\interactionPicture]{m}{} \right)\\
        &\stackrel{\text{MM}}{=}
        %
        -J \cdot \fullneighborsum{l}{m} \left[
            \VhamiltonianAnalyticalPart{A}{l}{m}{t} \cdot \VhamiltonianOperatorPart{A}{l}{m} + 
            \VhamiltonianAnalyticalPart{B}{l}{m}{t} \cdot \VhamiltonianOperatorPart{B}{l}{m} + 
            \VhamiltonianAnalyticalPart{C}{l}{m}{t} \cdot \VhamiltonianOperatorPart{C}{l}{m} 
        \right]
    \end{split}
\end{equation}

It is possible to express these operator-strings as single hoppings, decorated with number operators. One receives \autoref{eq:interaction-picture-v-ham-parts}, that is symmetrical in \up and \down (with $\overline{\up} = \down$ and  $\overline{\down} = \up$).

\begin{equation}
    \label{eq:interaction-picture-v-ham-parts}
    \begin{split}
        \VhamiltonianAnalyticalPart{A}{l}{m}{t} \stackrel{\text{MM}}{=} e^{i\cdot \left(\epsl-\epsl[m]\right)\cdot t} \qquad
        \VhamiltonianOperatorPart{A}{l}{m} &\stackrel{\text{MM}}{=} 
        \lsum[\sigma \in \left\{\up,\,\down\right\}]
        \withspinhcop[\schroedingerPicture]{l}{\sigma}{\dagger}\withspinhcop[\schroedingerPicture]{m}{\sigma}{}
        \left(
            1+
            2 \cdot \nop[\schroedingerPicture]{l}{\overline{\sigma}}\nop[\schroedingerPicture]{m}{\overline{\sigma}}
            - \nop[\schroedingerPicture]{l}{\overline{\sigma}}
            - \nop[\schroedingerPicture]{m}{\overline{\sigma}}
        \right)
        \\
        \VhamiltonianAnalyticalPart{B}{l}{m}{t} \stackrel{\text{MM}}{=} e^{i\cdot \left(\epsl-\epsl[m] + U\right)\cdot t} \qquad
        \VhamiltonianOperatorPart{B}{l}{m} &\stackrel{\text{MM}}{=} 
        \lsum[\sigma \in \left\{\up,\,\down\right\}]        
        \withspinhcop[\schroedingerPicture]{l}{\sigma}{\dagger}\withspinhcop[\schroedingerPicture]{m}{\sigma}{}
        \left(
            \nop[\schroedingerPicture]{l}{\overline{\sigma}}
            - \nop[\schroedingerPicture]{l}{\overline{\sigma}}\nop[\schroedingerPicture]{m}{\overline{\sigma}}
        \right)
        \\
        \VhamiltonianAnalyticalPart{C}{l}{m}{t} \stackrel{\text{MM}}{=} e^{i\cdot \left(\epsl-\epsl[m] - U\right)\cdot t} \qquad
        \VhamiltonianOperatorPart{C}{l}{m} &\stackrel{\text{MM}}{=} 
        \lsum[\sigma \in \left\{\up,\,\down\right\}]
        \withspinhcop[\schroedingerPicture]{l}{\sigma}{\dagger}\withspinhcop[\schroedingerPicture]{m}{\sigma}{}
        \left(
            \nop[\schroedingerPicture]{m}{\overline{\sigma}}
            - \nop[\schroedingerPicture]{m}{\overline{\sigma}}\nop[\schroedingerPicture]{l}{\overline{\sigma}}
        \right)
        \\
    \end{split}
\end{equation}

The expressions for the dressed operators in \autoref{eq:interaction-picture-v-ham-parts} can be rearranged, to reflect more their physical effects.
\autoref{eq:interaction-picture-v-ham-parts-simplified} can be obtained and verified by comparing the measurements in a truth table (the \nop[\schroedingerPicture]{}{} measure either a $0$ or $1$ on a state, without modifying it).

\begin{equation}
    \label{eq:interaction-picture-v-ham-parts-simplified}
    \begin{split}
        \VhamiltonianOperatorPart{A}{l}{m} &\stackrel{\text{MM}}{=} 
        \lsum[\sigma \in \left\{\up,\,\down\right\}]
        \withspinhcop[\schroedingerPicture]{l}{\sigma}{\dagger}\withspinhcop[\schroedingerPicture]{m}{\sigma}{}
        \left(\nop[\schroedingerPicture]{l}{\overline{\sigma}}
            \equivalentOperator 
            \nop[\schroedingerPicture]{m}{\overline{\sigma}}
        \right)
        \\
        \VhamiltonianOperatorPart{B}{l}{m} &\stackrel{\text{MM}}{=} 
        \lsum[\sigma \in \left\{\up,\,\down\right\}]        
        \withspinhcop[\schroedingerPicture]{l}{\sigma}{\dagger}\withspinhcop[\schroedingerPicture]{m}{\sigma}{}
        \nop[\schroedingerPicture]{l}{\overline{\sigma}}
        \left(
            1 - \nop[\schroedingerPicture]{m}{\overline{\sigma}}
        \right)
        \\
        \VhamiltonianOperatorPart{C}{l}{m} &\stackrel{\text{MM}}{=} 
        \lsum[\sigma \in \left\{\up,\,\down\right\}]
        \withspinhcop[\schroedingerPicture]{l}{\sigma}{\dagger}\withspinhcop[\schroedingerPicture]{m}{\sigma}{}
        \nop[\schroedingerPicture]{m}{\overline{\sigma}}
        \left(
           1 - \nop[\schroedingerPicture]{l}{\overline{\sigma}}
        \right)
        \\
    \end{split}
\end{equation}

The \equivalentOperator here means \glqq $1$ if the measurements of the operators are equivalent, $0$ otherwise\grqq.

\paragraph*{Base Energy}\makebox{}\\

In \autoref{eq:time-evolution-target}, the effective Hamiltonian $\HeffOft = -i E_0(N) t + \HNOft$ was defined.
This construct will be necessary in the following sections. 
$E_0(N)$ is luckily calculated quite swiftly, as one can see in \autoref{eq:base-energy} (with $n_{l,\,\sigma}$ - the occupation-number in $\ketN$, not the operator).

\begin{equation}
    \label{eq:base-energy}
    \begin{split}
        E_0(N) &\stackrel{\phantom{\ref{eq:main-hamiltonian-h0}}}{=} \frac{\bracketHelper{N}{\HzeroHamiltonian[]}{N}}{\braketHelper{N}{N}} \\
        %
        &\stackrel{\ref{eq:main-hamiltonian-h0}}{=} \bracketHelper{N}{U \cdot \lsum \nop{l}{\up}\nop{l}{\down}}{N} + \bracketHelper{N}{\lsum[l,\,\sigma] \epsl \nop{l}{\sigma}}{N}\\
        %
        &\stackrel{\phantom{\ref{eq:main-hamiltonian-h0}}}{=} U \cdot \lsum n_{l,\,\up}n_{l,\,\down} + \lsum \epsl n_{l,\,\sigma}
    \end{split}
\end{equation}

\paragraph*{Calculate the first order of \HNOft}\makebox{}\\

The last missing element for $\HeffOft = -i E_0(N) t + \HNOft$ now is \HNOft.
At least for the first order, the integration in \autoref{eq:cumulant-expansion} is easily doable, with now knowing the form of \VhamiltonianOf[\interactionPicture]{t}.

The result of the integration can be seen in \autoref{eq:hn-integrated-first-order}.

\begin{equation}
    \label{eq:hn-integrated-first-order}
    \begin{split}
        \HNOftOrder{1} &\stackrel{\ref{eq:cumulant-expansion}}{=} -i \int\limits_0^t \mathrm{d}t' \frac{\bracketHelper{N}{\VhamiltonianOf[\interactionPicture]{t'}}{\picturePsi[\schroedingerPicture]{}}}{\braketHelper{N}{\picturePsi[\schroedingerPicture]{}}}\\
        %
        &\stackrel{\ref{eq:base-expansion-state}}{=}
        -i \frac{1}{\psiN} \int\limits_0^t \mathrm{d}t' \lsum[K] \bracketHelper{N}{\VhamiltonianOf[\interactionPicture]{t'}}{K} \psiN[K]\\
        &\stackrel{\ref{eq:interaction-picture-v-ham-full}}{=}
        i \cdot J \lsum[K] \fullneighborsum{l}{m} \frac{\psiN[K]}{\psiN} \int\limits_0^t \mathrm{d}t' 
        \left[
            e^{i\cdot \left(\epsl-\epsl[m]\right)\cdot t'} \cdot 
            \bracketHelper{N}{
                \VhamiltonianOperatorPart{A}{l}{m} 
            }{K}
            + 
            \right.\\
            &\qquad
            \left.
            e^{i\cdot \left(\epsl-\epsl[m] + U\right)\cdot t'} \cdot 
            \bracketHelper{N}{
                \VhamiltonianOperatorPart{B}{l}{m}
            }{K}
            + 
            e^{i\cdot \left(\epsl-\epsl[m] - U\right)\cdot t'} \cdot 
            \bracketHelper{N}{
                \VhamiltonianOperatorPart{C}{l}{m} 
            }{K}
        \right]
        \\
        %
        &\stackrel{\phantom{\ref{eq:interaction-picture-v-ham-full}}}{=}
        J \lsum[K] \fullneighborsum{l}{m} \frac{\psiN[K]}{\psiN}
        \left[
            \frac{e^{i\cdot \left(\epsl-\epsl[m]\right)\cdot t}-1}{\epsl-\epsl[m]} \cdot 
            \bracketHelper{N}{
                \VhamiltonianOperatorPart{A}{l}{m} 
            }{K}
            + 
            \right.\\
            &\qquad
            \left.
            \frac{
                e^{i\cdot \left(\epsl-\epsl[m] + U\right)\cdot t}-1
            }{\epsl-\epsl[m] + U}
             \cdot 
            \bracketHelper{N}{
                \VhamiltonianOperatorPart{B}{l}{m}
            }{K}
            + 
            \frac{
                e^{i\cdot \left(\epsl-\epsl[m] - U\right)\cdot t}-1
            }{\epsl-\epsl[m] - U}
             \cdot 
            \bracketHelper{N}{
                \VhamiltonianOperatorPart{C}{l}{m} 
            }{K}
        \right]\\
        %
        &\stackrel{\phantom{\ref{eq:interaction-picture-v-ham-full}}}{=}
        J \lsum[K] \fullneighborsum{l}{m} \frac{\psiN[K]}{\psiN}
        \left[
            \VhamiltonianAnalyticalPartIntegrated{A}{l}{m}{t} \cdot 
            \bracketHelper{N}{
                \VhamiltonianOperatorPart{A}{l}{m} 
            }{K}
            + 
            \right.\\
            &\qquad
            \left.
            \VhamiltonianAnalyticalPartIntegrated{B}{l}{m}{t}
             \cdot 
            \bracketHelper{N}{
                \VhamiltonianOperatorPart{B}{l}{m}
            }{K}
            + 
            \VhamiltonianAnalyticalPartIntegrated{C}{l}{m}{t}
             \cdot 
            \bracketHelper{N}{
                \VhamiltonianOperatorPart{C}{l}{m} 
            }{K}
        \right]
    \end{split}
\end{equation}

In the second to last step, the $i$ is moved into the integral.

While it seems that $\lsum[K] \fullneighborsum{l}{m}$ is even a larger summation than the previously problematic ones, because the operators $\VhamiltonianOperatorPart{A}{l}{m}$, $\VhamiltonianOperatorPart{B}{l}{m}$ and $\VhamiltonianOperatorPart{C}{l}{m}$ have interaction range limited to nearest neighbors, actually the summation is really sparse and most entries are $0$.
This summation can be efficiently evaluated in \bigo{\text{\#}(\text{sites}) \cdot \text{\#}(\text{nearest neighbors})}.
How this could be done, can be looked up in the implementation \filepath{\cite{selfCode}}{/computation-scripts/hamiltonian.py} in the specialization class \emph{HardcoreBosonicHamiltonian}.

\paragraph*{Calculate the second order of \HNOft}\makebox{}\\

For the calculation of the second order, the time-ordering-operator is relevant again. 
It however is not necessary for the right side of the integral, as the two integrals can be fully localized to one factor each.
For the left side, it is necessary to integrate different cases separately.

\begin{equation}
    \label{eq:hn-integrated-second-order}
    \begin{split}
        \HNOftOrder{2} &\stackrel{\ref{eq:cumulant-expansion}}{=} - \frac{1}{2} \int\limits_0^t\mathrm{d}t_1 \int\limits_0^t\mathrm{d}t_2
        \left(
        \frac{\bracketHelper{N}{\timeOrderingOperator
        \VhamiltonianOf[\interactionPicture]{t_1}\VhamiltonianOf[\interactionPicture]{t_2}
        }{\picturePsi[\schroedingerPicture]{}}}{\braketHelper{N}{\picturePsi[\schroedingerPicture]{}}} -  \frac{\bracketHelper{N}{\VhamiltonianOf[\interactionPicture]{t_1}}{\picturePsi[\schroedingerPicture]{}} \cdot \bracketHelper{N}{\VhamiltonianOf[\interactionPicture]{t_2}}{\picturePsi[\schroedingerPicture]{}}}{\braketHelper{N}{\picturePsi[\schroedingerPicture]{}}^2}
        \right)\\
        %
        &\stackrel{\ref{eq:base-expansion-state}}{=}
        - \frac{1}{2} \lsum[K] \frac{\psiN[K]}{\psiN} \int\limits_0^t \int\limits_0^t        \bracketHelper{N}{\timeOrderingOperator
        \VhamiltonianOf[\interactionPicture]{t_1}\VhamiltonianOf[\interactionPicture]{t_2}
        }{K}            \mathrm{d}t_1\mathrm{d}t_2 \\
        &\stackrel{\phantom{\ref{eq:base-expansion-state}}}{+} 
        \frac{1}{2}
        \lsum[M]\lsum[L]
        \frac{\psiN[M]\psiN[L]}{\psiN{}^2}
        \int\limits_0^t \bracketHelper{N}{\VhamiltonianOf[\interactionPicture]{t_1}}{M}  \mathrm{d}t_1
        \cdot
        \int\limits_0^t \bracketHelper{N}{\VhamiltonianOf[\interactionPicture]{t_2}}{L}  \mathrm{d}t_2\\
        %
        &\stackrel{\ref{eq:interaction-picture-v-ham-full}}{=}
        - \frac{J^2}{2} \lsum[K]\lsum[X,Y \in \left\{\text{A}, \text{B}, \text{C}\right\}] \fullneighborsum{l}{m}\fullneighborsum{j}{k}
        \frac{\psiN[K]}{\psiN} \\&\quad
        \int\limits_0^t \int\limits_0^t        \bracketHelper{N}{\timeOrderingOperator
        \VhamiltonianAnalyticalPart{$X$}{l}{m}{t_1} \VhamiltonianOperatorPart[t_1]{$X$}{l}{m}
        \cdot 
        \VhamiltonianAnalyticalPart{$Y$}{j}{k}{t_2} \VhamiltonianOperatorPart[t_2]{$Y$}{j}{k}
        }{K}            \mathrm{d}t_1\mathrm{d}t_2 \\
        &\stackrel{\phantom{\ref{eq:interaction-picture-v-ham-full}}}{+} 
        \frac{J^2}{2}
        \lsum[M]\lsum[L]\lsum[V,W \in \left\{\text{A}, \text{B}, \text{C}\right\}]
        \fullneighborsum{o}{p}\fullneighborsum{q}{r}
        \frac{\psiN[M]\psiN[L]}{\psiN{}^2}\\&\quad
        \int\limits_0^t \bracketHelper{N}{\VhamiltonianAnalyticalPart{$V$}{o}{p}{t_1} \cdot \VhamiltonianOperatorPart{$V$}{o}{p}}{M}  \mathrm{d}t_1
        \cdot
        \int\limits_0^t \bracketHelper{N}{\VhamiltonianAnalyticalPart{$W$}{q}{r}{t_2} \cdot \VhamiltonianOperatorPart{$W$}{q}{r}}{M}  \mathrm{d}t_2\\
        %
        &=
        - \frac{J^2}{2} \lsum[K]\lsum[X,Y \in \left\{\text{A}, \text{B}, \text{C}\right\}] \fullneighborsum{l}{m}\fullneighborsum{j}{k}
        \frac{\psiN[K]}{\psiN} \\&\quad
        \int\limits_0^t \int\limits_0^t  
        \VhamiltonianAnalyticalPart{$X$}{l}{m}{t_1}
        \cdot 
        \VhamiltonianAnalyticalPart{$Y$}{j}{k}{t_2}
        \bracketHelper{N}{\timeOrderingOperator
            \VhamiltonianOperatorPart[t_1]{$X$}{l}{m}
        \cdot 
            \VhamiltonianOperatorPart[t_2]{$Y$}{j}{k}
        }{K}            \mathrm{d}t_1\mathrm{d}t_2 \\
        &-\frac{J^2}{2}
        \lsum[M]\lsum[L]\lsum[V,W \in \left\{\text{A}, \text{B}, \text{C}\right\}]
        \fullneighborsum{o}{p}\fullneighborsum{q}{r}
        \frac{\psiN[M]\psiN[L]}{\psiN{}^2}\\&\quad
        \VhamiltonianAnalyticalPartIntegrated{$V$}{o}{p}{t} 
        \cdot
        \VhamiltonianAnalyticalPartIntegrated{$W$}{q}{r}{t}
        \bracketHelper{N}{\VhamiltonianOperatorPart{$V$}{o}{p}}{M}
        \bracketHelper{N}{\VhamiltonianOperatorPart{$W$}{q}{r}}{M}
    \end{split}
\end{equation}

Notice in the second to last step, the \VhamiltonianOperatorPart[t_\ast]{$\ast$}{\ast}{\ast} is indicated with a $t_\ast$ to keep its time-ordering association, even after writing out the \VhamiltonianOf[\interactionPicture]{t_\ast}.
In the last step, the sign of the second term flips, because of a $-1 = i \cdot i$ that was introduced, to allow for integration of \VhamiltonianAnalyticalPart{$\ast$}{\ast}{\ast}{t_\ast} like in \autoref{eq:hn-integrated-first-order}.\\

\begin{equation}
    \label{eq:u-eps-substitution}
    \begin{split}
        \epsl-\epsl[m] &= \ueps{A}{l}{m}\\
        \epsl-\epsl[m]+U &= \ueps{B}{l}{m}\\
        \epsl-\epsl[m]-U &= \ueps{C}{l}{m}\\
    \end{split}
\end{equation}

Inverting the steps performed in \cite{dissectTimeOrderingIntegrals}, one can easily see how to split integrals with the time-ordering-operator for evaluation in \autoref{eq:split-integral-time-ordering}.

\begin{equation}
    \label{eq:split-integral-time-ordering}
    \begin{split}
        &\int\limits_0^t \mathrm{d}t_1 \int\limits_0^{t_1} \mathrm{d}t_2 A(t_1) B(t_2)  = \int\limits_0^t \mathrm{d}t_1 \int\limits_0^{t_1} \mathrm{d}t_2  \timeOrderingOperator A(t_1) B(t_2) \\
        &\int\limits_0^t \mathrm{d}t_1 \int\limits_{t_1}^t \mathrm{d}t_2  B(t_2) A(t_1)  = \int\limits_0^t \mathrm{d}t_1 \int\limits_{t_1}^t \mathrm{d}t_2  \timeOrderingOperator A(t_1) B(t_2) \\\\
        %
        &\int\limits_0^t \mathrm{d}t_1 \left( \int\limits_0^{t_1} \mathrm{d}t_2 \timeOrderingOperator A(t_1) B(t_2) +  \int\limits_{t_1}^t \mathrm{d}t_2 \timeOrderingOperator A(t_1) B(t_2)\right) =
        \int\limits_0^t \mathrm{d}t_1\int\limits_0^t \mathrm{d}t_2 \timeOrderingOperator A(t_1) B(t_2)
    \end{split}
\end{equation}

What is left, is to evaluate the integral that needs to respect the time-ordering.

\begin{equation}
    \label{eq:time-ordered-integral}
    \begin{split}
        &\int\limits_0^t \int\limits_0^t  
        \VhamiltonianAnalyticalPart{$X$}{l}{m}{t_1}
        \cdot 
        \VhamiltonianAnalyticalPart{$Y$}{j}{k}{t_2}
        \bracketHelper{N}{\timeOrderingOperator
            \VhamiltonianOperatorPart[t_1]{$X$}{l}{m}
        \cdot 
            \VhamiltonianOperatorPart[t_2]{$Y$}{j}{k}
        }{K}            \mathrm{d}t_1\mathrm{d}t_2
        \stackrel{\ref{eq:u-eps-substitution},\,\ref{eq:interaction-picture-v-ham-parts}}{=}\\
        %
        &\int\limits_0^t \int\limits_0^t  
        e^{
            i \cdot \ueps{$X$}{l}{m} \cdot t_1
        }
        \cdot 
        e^{
            i \cdot \ueps{$Y$}{j}{k} \cdot t_2
        }
        \bracketHelper{N}{\timeOrderingOperator
            \VhamiltonianOperatorPart[t_1]{$X$}{l}{m}
        \cdot 
            \VhamiltonianOperatorPart[t_2]{$Y$}{j}{k}
        }{K}            \mathrm{d}t_1\mathrm{d}t_2 \stackrel{\ref{eq:split-integral-time-ordering}}{=} \\
        %
        &\int\limits_0^t  \int\limits_0^{t_1}
        e^{
            i \cdot \ueps{$X$}{l}{m} \cdot t_1
        }
        e^{
            i \cdot \ueps{$Y$}{j}{k} \cdot t_2
        }
        \mathrm{d}t_2
        \mathrm{d}t_1
        \cdot 
        \bracketHelper{N}{
            \VhamiltonianOperatorPart[]{$X$}{l}{m}
            \VhamiltonianOperatorPart[]{$Y$}{j}{k}
        }{K}            + \\
        &\int\limits_0^t  \int\limits_{t_1}^t
        e^{
            i \cdot \ueps{$X$}{l}{m} \cdot t_1
        }
        e^{
            i \cdot \ueps{$Y$}{j}{k} \cdot t_2
        }
        \mathrm{d}t_2
        \mathrm{d}t_1
        \cdot 
        \bracketHelper{N}{
            \VhamiltonianOperatorPart[]{$Y$}{j}{k}
            \VhamiltonianOperatorPart[]{$X$}{l}{m}
        }{K} 
    \end{split}
\end{equation}

        \FloatBarrier

    \section{Observable Measurements}
    \label{sec:theory-observables}
    \section{TODO}
    \begin{frame}[t]
        \frametitle{TODO}
        
        \vspace{-0.5em}
        \begin{itemize}
            \item TODO
        \end{itemize}

        % notes 
        \onslide % on all slides of frame
        \note[item] {
            TODO
        }
    \end{frame}

    \FloatBarrier

        \subsection{Observable Sampling}
        \label{sec:theory-observables-sampling}
        The evaluation of observables is a primary requirement for this application. 
First it is necessary to formulate a more general formula, that one later can plug specific observables into.

From \autoref{eq:time-evolution-target} and applying \emph{dagger} $\dagger$ to said equation, \autoref{eq:time-evolved-state-and-dagger} follows.


\begin{equation}
    \label{eq:time-evolved-state-and-dagger}
    \begin{split}
        \braketHelper{N}{\psiOfT[\schroedingerPicture]} &\stackrel{\ref{eq:time-evolution-target}}{=} \braN{} \lsum[K] e^{\HeffOft[K]} \psiN[K] \ketN[K]{}\\
        & \stackrel{\phantom{\ref{eq:time-evolution-target}}}{=} 
        \lsum[K]   e^{\HeffOft[K]} \psiN[K] \underbrace{\braketHelper{N}{K}}_\text{$\delta_{N,K}$} = e^{\HeffOft} \psiN{} \\
        %
        \stackrel{\dagger}{\Rightarrow} \braketHelper{\psiOfT[\schroedingerPicture]}{N} &\stackrel{\phantom{\ref{eq:time-evolution-target}}}{=} e^{\HeffOftStar} \psiNStar
    \end{split}
\end{equation}

In \autoref{eq:expectation-value} by starting with the default way of calculating a normalized \emph{expectation-value} for an observable \ObservableOp \cite{monteCarloObservableSampling} and applying multiple modifications, one gets to formula that can be effectively sampled later. 

% TODO check that in final compile this breaks properly

\begin{equation}
    \label{eq:expectation-value}
    \begin{split}
        &\frac{\bracketHelper{\psiOfT[\schroedingerPicture]}{\ObservableOp}{\psiOfT[\schroedingerPicture]}}{\braketHelper{\psiOfT[\schroedingerPicture]}{\psiOfT[\schroedingerPicture]}}
        %
        \stackrel{\phantom{\ref{eq:time-evolution-target}}}{=} 
        \frac{
            \lsum[N] \braketHelper{\psiOfT[\schroedingerPicture]}{N} \bracketHelper{N}{\ObservableOp}{\psiOfT[\schroedingerPicture]}
        }{
            \lsum[K] \braketHelper{\psiOfT[\schroedingerPicture]}{K} \braketHelper{K}{\psiOfT[\schroedingerPicture]}
        }\\
        %
        &\stackrel{\ref{eq:time-evolved-state-and-dagger}}{=} 
        \frac{
            \lsum[N] e^{\HeffOftStar} \psiNStar \bracketHelper{N}{\ObservableOp}{\psiOfT[\schroedingerPicture]}
        }{
            \lsum[K] e^{\HeffOftStar[K]} \psiNStar[K] e^{\HeffOft[K]} \psiN[K] 
        }\\
        & \stackrel{\phantom{\ref{eq:time-evolved-state-and-dagger}}}{=} 
        \frac{
            \lsum[N] e^{\HeffOftStar} \psiNStar \bracketHelper{N}{\ObservableOp}{\psiOfT[\schroedingerPicture]}
        }{
            \lsum[K] \absSquare{e^{\HeffOft[K]}} \absSquare{\psiN[K]} 
        } \cdot 
        \frac{\braketHelper{N}{\psiOfT[\schroedingerPicture]}}{\braketHelper{N}{\psiOfT[\schroedingerPicture]}}\\
        & \stackrel{\phantom{\ref{eq:time-evolved-state-and-dagger}}}{=} 
        \lsum[N]
        \frac{
            e^{\HeffOftStar} \psiNStar e^{\HeffOft} \psiN{} \braketHelper{N}{\psiOfT[\schroedingerPicture]}
        }{
            \lsum[K] \absSquare{e^{\HeffOft[K]}} \absSquare{\psiN[K]} 
        }  \cdot 
        \frac{\bracketHelper{N}{\ObservableOp}{\psiOfT[\schroedingerPicture]}}{\braketHelper{N}{\psiOfT[\schroedingerPicture]}}\\
        & \stackrel{\ref{eq:time-evolved-state-and-dagger}}{=} 
        \lsum[N]
        \underbrace{\frac{
            \absSquare{e^{\HeffOft}} \absSquare{\psiN} 
        }{
            \lsum[K] \absSquare{e^{\HeffOft[K]}} \absSquare{\psiN[K]} 
        }}_{\probabilityOf{N}{t}} \cdot 
        \frac{\bracketHelper{N}{\ObservableOp}{\psiOfT[\schroedingerPicture]}}{\braketHelper{N}{\psiOfT[\schroedingerPicture]}}\\
        & \stackrel{\phantom{\ref{eq:time-evolved-state-and-dagger}}}{=} \lsum[N]
        \probabilityOf{N}{t}
        \frac{\bracketHelper{N}{\ObservableOp}{\psiOfT[\schroedingerPicture]}}{\braketHelper{N}{\psiOfT[\schroedingerPicture]}}
        %
        \stackrel{\phantom{\ref{eq:time-evolved-state-and-dagger}}}{=} \lsum[N]
        \probabilityOf{N}{t}
        \frac{
            \lsum[K] \bracketHelper{N}{\ObservableOp}{K} \braketHelper{K}{\psiOfT[\schroedingerPicture]} 
        }{
            \braketHelper{N}{\psiOfT[\schroedingerPicture]}
        }\\
        & \stackrel{\ref{eq:time-evolved-state-and-dagger}}{=} \lsum[N]
        \probabilityOf{N}{t}
        \frac{
            \lsum[K] \bracketHelper{N}{\ObservableOp}{K} e^{\HeffOft[K]} \psiN[K]
        }{
            e^{\HeffOft[N]} \psiN[N]
        }\\
        %
        &\stackrel{\phantom{\ref{eq:time-evolved-state-and-dagger}}}{=} \lsum[N]
        \probabilityOf{N}{t}
        \underbrace{\lsum[K] \bracketHelper{N}{\ObservableOp}{K} e^{\HeffOft[K]-\HeffOft[N]}
        \frac{
            \psiN[K]
        }{
            \psiN[N]
        }}_{\localObservable{N}{t}}
        %
        \stackrel{\phantom{\ref{eq:time-evolved-state-and-dagger}}}{=} \lsum[N]
        \probabilityOf{N}{t}
        \localObservable{N}{t}
    \end{split}
\end{equation}

By inserting \autoref{eq:time-evolved-state-and-dagger} into \autoref{eq:expectation-value}, a probability-dependent formula is obtained, that can be used to plug in specific observables.  
The probability of a base-state at a specified time \probabilityOf{N}{t} is important for this calculation. 
The \fullref{sec:theory-monte-carlo} will discuss the viability of obtaining these probabilities and strategies of doing so.
The matrix-element $\bracketHelper{N}{\ObservableOp}{K}$ has the important property of being $0$ for almost all combinations of $\braN$ and $\ketN[K]$, depending on the chosen observable \ObservableOp, which is what makes the calculation of the \emph{local-observable} \localObservable{N}{t} possible in the first place.

\paragraph*{Double-Occupation}\makebox{}\\

One possibly interesting observable would be the measurement of \emph{double-occupation} (one particle of spin-up and one of spin-down) per site.

The operator $\doubleOccupationOperator{l}$ that measure the double-occupation on site $l$ can be easily written as $\doubleOccupationOperator{l} = \nop{l}{\up}\nop{l}{\down}$.

As the used basis-states are eigenstates of the number-operators, the evaluation of this operator is quite straight forward. 
$\bracketHelper{N}{\doubleOccupationOperator{l}}{K}$ becomes $\delta_{N,\,K}\cdot n_{l,\,\up}\cdot n_{l,\,\down}$ with the occupation-number (not operator) of the specific site and spin in \braN, $n_{l,\,\sigma}$. 
With that \localObservable{N}{t} can be simply evaluated to the result in \autoref{eq:double-occupation-loc}.

\begin{equation}
    \label{eq:double-occupation-loc}
    \localObservable{N}{t} = \lsum[K] \bracketHelper{N}{\doubleOccupationOperator{l}}{K} e^{\HeffOft[K]-\HeffOft[N]}
    \frac{
        \psiN[K]
    }{
        \psiN[N]
    } = n_{l,\,\up}\cdot n_{l,\,\down}
\end{equation}

\paragraph*{Spin-Polarized Kinetics}\makebox{}\\

For measuring the spin-polarized flow of particles, a slightly more complex observable must be employed. 
In \autoref{eq:spin-polarized-kinetics-operator-definition} a possible observable is listed, that measures such kinetics direction in-dependent (\spinPolarizedKineticsOperator{l}{m}{\sigma}) or direction dependent (\spinPolarizedKineticsOperatorDir{l}{m}{\sigma}, caution: needs $i$ to obtain hermitian operator).

\begin{equation}
    \label{eq:spin-polarized-kinetics-operator-definition}
    \begin{split}
        \spinPolarizedKineticsOperator{l}{m}{\sigma} &= - J \left(\withspinop{l}{\sigma}{\dagger}\withspinop{m}{\sigma}{} + \withspinop{m}{\sigma}{\dagger}\withspinop{l}{\sigma}{}\right)\\
        \spinPolarizedKineticsOperatorDir{l}{m}{\sigma} &= i J \left(\withspinop{m}{\sigma}{\dagger}\withspinop{l}{\sigma}{} - \withspinop{l}{\sigma}{\dagger}\withspinop{m}{\sigma}{}\right)\\
    \end{split}
\end{equation}

In this case, the used basis-states are not eigenstates of the operators. 
$\bracketHelper{N}{\withspinop{m}{\sigma}{\dagger}\withspinop{l}{\sigma}{}}{K}$ becomes $\delta_{N,\,\tilde{N}}\cdot (1-n_{l,\,\sigma} )\cdot n_{m,\,\sigma}$, where \ketN[{\vphantom{N}\smash{\tilde{N}}}] is the state obtained when the particle number on site $m,\, \sigma$ is transferred to site $l,\, \sigma$ (this \emph{hopping} is only possible, when there is a particle on the original site and no particle yet on the target site, which is ensured by the occupation numbers).
Evaluating the whole operator with the signs and $i$s correctly and in a efficient manner can best be looked up in the implementation \filepath{\cite{selfCode}}{/computation-scripts/observables.py}.

Overall, the evaluation of this observable requires knowing the value of an object of the form presented in \autoref{eq:form-heff-difference}.

\begin{equation}
    \label{eq:form-heff-difference}
    \frac{\psiN[\tilde{N}]}{\psiN} e^{\HeffOft[\tilde{N}]-\HeffOft[N]}
\end{equation}

The \fullref{sec:theory-special-cases} will go into how to compute this efficiently for states \ketN[N] and \ketN[{\vphantom{N}\smash{\tilde{N}}}] that are connected via the hopping between nearest-neighbor lattice sites.

It is finally important to notice, that the objects from \autoref{eq:form-heff-difference} are not real-valued, but complex.
Only the complete observable from the full sum over all basis-states \ketN over \localObservable{N}{t} has a fully vanishing imaginary part. 
Especially when the observable is approximated with an incomplete set of basis-states, computationally an imaginary component remains. 
It is important to monitor the magnitude of this value and monitor it going to $0$.
If it doesn't fully vanish, this indicates an error in the sampling strategy or the implementation or the number of sampled states is not great enough.

        \FloatBarrier

        \subsection{Occupation Types}
        \label{sec:theory-observables-occupations}
        % \subsubsection*{Double-Occupation}

One possibly interesting observable would be the measurement of \emph{double-occupation} (one particle of spin-up and one of spin-down) per site. The operator $\doubleOccupationOperator{l}$, measuring the double-occupation on site $l$, can be written as $\doubleOccupationOperator{l} = \nop{l}{\up}\nop{l}{\down}$.

As the used basis-states are eigenstates of the number-operators, the evaluation of this operator is quite straight forward. 
$\bracketHelper{N}{\doubleOccupationOperator{l}}{K}$ becomes $\delta_{N,\,K}\cdot n_{l,\,\up}\cdot n_{l,\,\down}$ with the occupation-number (not operator) of the specific site and spin in \braN, $n_{l,\,\sigma}$. 
Hereby, \localObservable{N}{t} can be simply evaluated to the result shown in \autoref{eq:double-occupation-loc}.

\begin{equation}
    \label{eq:double-occupation-loc}
    \localObservable{N}{t} = \lsum[K] \bracketHelper{N}{\doubleOccupationOperator{l}}{K} e^{\HeffOft[K]-\HeffOft[N]}
    \frac{
        \psiN[K]
    }{
        \psiN[N]
    } = n_{l,\,\up}\cdot n_{l,\,\down}
\end{equation}

The operator measuring \emph{single-occupation} $\singleOccupationOperator{l}{\sigma}$ is derived in the exact same way.

\begin{equation}
    \label{eq:single-occupation-loc}
    \localObservable{N}{t} = \lsum[K] \bracketHelper{N}{\singleOccupationOperator{l}{\sigma}}{K} e^{\HeffOft[K]-\HeffOft[N]}
    \frac{
        \psiN[K]
    }{
        \psiN[N]
    } = n_{l,\,\sigma}
\end{equation}
        \FloatBarrier

        \subsection{Spin-Current}
        \label{sec:theory-observables-current}
        % \subsubsection*{Spin-Polarized Kinetics}

For measuring the spin-polarized flow of particles, a slightly more complex observable must be employed. 
In \autoref{eq:spin-polarized-kinetics-operator-definition} a possible observable is listed, that measures such kinetics direction in-dependent (\spinPolarizedKineticsOperator{l}{m}{\sigma}) or direction dependent (\spinPolarizedKineticsOperatorDir{l}{m}{\sigma}, caution: needs $i$ to obtain hermitian operator).

\begin{equation}
    \label{eq:spin-polarized-kinetics-operator-definition}
    \begin{split}
        \spinPolarizedKineticsOperator{l}{m}{\sigma} &= - J \left(\withspinhcop{l}{\sigma}{\dagger}\withspinhcop{m}{\sigma}{} + \withspinhcop{m}{\sigma}{\dagger}\withspinhcop{l}{\sigma}{}\right)\\
        \spinPolarizedKineticsOperatorDir{l}{m}{\sigma} &= i J \left(\withspinhcop{m}{\sigma}{\dagger}\withspinhcop{l}{\sigma}{} - \withspinhcop{l}{\sigma}{\dagger}\withspinhcop{m}{\sigma}{}\right)\\
    \end{split}
\end{equation}

In this case, the used basis-states are not eigenstates of the operators. 
$\bracketHelper{N}{\withspinhcop{m}{\sigma}{\dagger}\withspinhcop{l}{\sigma}{}}{K}$ becomes $\delta_{N,\,\tilde{N}}\cdot (1-n_{l,\,\sigma} )\cdot n_{m,\,\sigma}$, where \ketN[{\vphantom{N}\smash{\tilde{N}}}] is the state obtained when the particle number on site $m,\, \sigma$ is transferred to site $l,\, \sigma$ (this \emph{hopping} is only possible, when there is a particle on the original site and no particle yet on the target site, which is ensured by the occupation numbers).
Evaluating the whole operator with the signs and $i$s correctly and in a efficient manner can best be looked up in the implementation \filepath{\cite{selfCode}}{/computation-scripts/observables.py}.

Overall, the evaluation of this observable requires knowing the value of an object of the form presented in \autoref{eq:form-heff-difference}.

\begin{equation}
    \label{eq:form-heff-difference}
    \frac{\psiN[\tilde{N}]}{\psiN} e^{\HeffOft[\tilde{N}]-\HeffOft[N]}
\end{equation}

The \fullref{sec:theory-optimizations-analytical} will go into how to compute this efficiently for states \ketN[N] and \ketN[{\vphantom{N}\smash{\tilde{N}}}] that are connected via the hopping between nearest-neighbor lattice sites.

It is finally important to notice, that the objects from \autoref{eq:form-heff-difference} are not real-valued, but complex.
Only the complete observable from the full sum over all basis-states \ketN over \localObservable{N}{t} has a fully vanishing imaginary part. 
Especially when the observable is approximated with an incomplete set of basis-states, computationally an imaginary component remains. 
It is important to monitor the magnitude of this value and monitor it going to $0$.
If it doesn't fully vanish, this indicates an error in the sampling strategy or the implementation or the number of sampled states is not great enough.

        \FloatBarrier

        \subsection{Reduced Density Matrix}
        \label{sec:theory-observables-density-matrix}
        Typically, the \emph{density matrix} $\rho = \ketpsiof[]{t}\brapsiof[]{t}$ encodes the full quantum information of the state. 
Because the number of entries in the matrix is the number of base-states squared - which is exponential in the number of sites, exactly what the approximation is supposed to avoid - it is not feasible to access this quantity.

While it is not possible to access the complete matrix, there exists one closely related entity, that has multiple uses that are going to be utilized in the subsequent section.
The \emph{reduced density matrix} $\rho_A$ is obtained upon \emph{tracing out} (\autoref{eq:tracing-out}, \cite{partialTraceEntanglementOfSubsystemsBlochVector}) the part $B$ from a system that is completely partitioned into two parts $A$ and $B$.
The basis of $A$ is $\left\{\ketN[\Phi_k]\right\}_{k=0}^{2^{2}-1 = 3 = N} = \left\{\ketN[11],\,\ketN[10],\,\ketN[01],\,\ketN[00]\right\}$ and the basis of $B$ $\left\{\ketN[\Chi_k]\right\}_{k=0}^{2^{2 \cdot \#\left(\text{sites}\right) - 2}-1 = M}$.
Notice, the order of basis states starting with $\ketN[1\dots1]$ then $\ketN[1\dots10]$ and ending with $\ketN[0\dots0]$. As stated in \autoref{eq:spin-hcb-state-mapping}, this is the convention that is necessary to keep consistent sings, as $\ketN[0\dots0] \rightarrow \ketN[1\dots1]$ would introduce exactly one $-1$ into all measurements with \pauli{y}.

\begin{equation}
    \label{eq:tracing-out}
    \rho_A = \partialTrace{B}{\rho_{AB}} = \sum\limits_{k=0}^{M} \left(\one[A]\otimes\braN[\Chi_k] \right)\rho_{AB}\left(\one[A]\otimes\ketN[\Chi_k]\right)
\end{equation}

If - for this application - one reduces the part $A$ to contain only two sides $l, \sigma$ and $m, \mu$, then $\rho_A$ will be a complex $4\times 4$ matrix.
As it is again not feasible to obtain via tracing out, the matrix can be calculated by rewriting it in the basis of Pauli matrices (\ref{sec:particles}):

\begin{equation}
    \label{eq:reduced-density-matrix-via-paulis}
    \rho_A (t) = \rho_{l,\,\sigma,\,m,\,\mu} (t) = \frac{1}{4} \sum\limits_{\alpha,\beta\in\left\{0,\text{x},\text{y},\text{z}\right\}}
    {\left\langle \pauli[l,\,\sigma]{$\alpha$}\pauli[m,\,\mu]{$\beta$} \right\rangle}_{\psiOfT} \left(\pauli[l,\,\sigma]{$\alpha$}\otimes\pauli[m,\,\mu]{$\beta$}\right)
\end{equation}

\autoref{eq:reduced-density-matrix-via-paulis} uses the property, that the 16 matrices $\pauli{$\alpha$}\otimes\pauli{$\beta$}$ are a complete basis of the complex $4\times 4$ matrices.
The observable $\pauli[l,\,\sigma]{$\alpha$}\pauli[m,\,\mu]{$\beta$}$ can be translated into hard-core bosonic operators, thanks to the previously derived \autoref{eq:jordan-wigner-spin-hcb} and with that can easily be measured like all other observables.
The factor $\frac{1}{4}$ follows from a normalization argument in \autoref{eq:reduced-density-matrix-normalization}, that uses properties of density matrices \cite{partialTraceEntanglementOfSubsystemsBlochVector} and algebraic properties of the Pauli matrices.

\begin{equation}
    \label{eq:reduced-density-matrix-normalization}
    \begin{split}
        {\left\langle \pauli[l,\,\sigma]{$\alpha$}\pauli[m,\,\mu]{$\beta$} \right\rangle}_{\psiOfT} &= \partialTrace{}{\rho_{l,\,\sigma,\,m,\,\mu} (t) \left(\pauli[l,\,\sigma]{$\alpha$}\otimes\pauli[m,\,\mu]{$\beta$}\right)}  \\
        &\stackrel{\ref{eq:reduced-density-matrix-via-paulis}}{=}\sum\limits_{\alpha',\beta'\in\left\{0,\text{x},\text{y},\text{z}\right\}} \frac{{\left\langle \pauli[l,\,\sigma]{$\alpha'$}\pauli[m,\,\mu]{$\beta'$} \right\rangle}_{\psiOfT}}{4}
        \partialTrace{}{\left(\pauli{$\alpha'$}\otimes\pauli{$\beta'$}\right)\left(\pauli{$\alpha$}\otimes\pauli{$\beta$}\right)}\\
        &\stackrel{\phantom{\ref{eq:reduced-density-matrix-via-paulis}}}{=}\sum\limits_{\alpha',\beta'\in\left\{0,\text{x},\text{y},\text{z}\right\}} \frac{{\left\langle \pauli[l,\,\sigma]{$\alpha'$}\pauli[m,\,\mu]{$\beta'$} \right\rangle}_{\psiOfT}}{4}
        \delta_{\alpha',\,\alpha}\delta_{\beta',\,\beta}\partialTrace{}{\one[4\times 4]}\\
        &\stackrel{\phantom{\ref{eq:reduced-density-matrix-via-paulis}}}{=}\frac{4}{4} \cdot {\left\langle \pauli[l,\,\sigma]{$\alpha$}\pauli[m,\,\mu]{$\beta$} \right\rangle}_{\psiOfT}
    \end{split}
\end{equation}
        \FloatBarrier
        
        \subsection{Entanglement Measurements}
        \label{sec:theory-observables-entanglement}
        The $\rho_{l,\,\sigma,\,m,\,\mu} (t)$ from \autoref{eq:reduced-density-matrix-via-paulis} has been specifically chosen to give the full quantum information of a subsystem of two arbitrary site-indices and spin-directions.
Most importantly, as the research is performed on quantum systems, it is a goal to extract quantum properties and the prime candidate for this is extracting the \emph{entanglement}.

Already research in \cite{isingDynamicsWithClassicalNetworks} used the same way of obtaining the reduced density matrix as presented in \autoref{eq:reduced-density-matrix-via-paulis} with the goal of measuring the entanglement.
There the number of sampled states, that is required to obtain the measurements of the observables, was called out as a limiting factor.
Monte-Carlo sampling was suggested as a fix for this limitation, which will actually be introduced in \fullref{sec:theory-optimizations-monte-carlo}.

In this work, the \emph{concurrence} was chosen as the \emph{entanglement monotone} to provide insight into the entanglement between different sites.
The concurrence has been introduced in \cite{concurrenceMainPaper} as a measure to obtain the entanglement of two spins from the reduced density matrix that contains these two spins.
A concurrence measurement does not give the value for the \emph{entanglement of formation} directly, but it needs to be rescaled as shown in \autoref{fig:entanglement}.
As one can see there, the rescaling does not introduce major additional features and for that reason the concurrence will be studied without rescaling to the entanglement.
The point of minimal entanglement (maximum separability) and maximal entanglement map to the same values and the intermediate values communicate the same qualitative information.

\begin{SCfigure}[2.0][htbp]
    \centering
    \includegraphics[width=0.35\textwidth]{plotgeneration/binary-entropy-function/binary-entropy-function.pdf}
    \caption{
        Plot of the function that connects the entanglement $\text{E}_{\psiOfT}$ of a subsystem with the concurrence of a subsystem $C(\psiOfT)$ in the relevant region $0\leq x \leq 1$.
        According to \cite{concurrenceMainPaper}: $\text{E}_{\psiOfT} = \text{E}(C(\psiOfT))$, with the function $\text{E}(x) = H(\frac{1}{2} + \frac{1}{2} \sqrt{1-x^2})$ and the \emph{binary entropy function} $H(x) = -\left[x \cdot \log_2(x) + (1-x) \cdot \log_2(1-x)\right]$.
        Comparing with the identity $y=x$, it shows that the values at the borders of the region $x=0$ and $x=1$ are the same for both functions. 
        The examined function $\text{E}(x)$ increases monotonically for the relevant input range and lies reasonably close to the identity. 
    }
    \label{fig:entanglement}
\end{SCfigure}

\filepath{\cite{selfCode}}{/calculation-helpers/concurrence/} holds multiple scripts used to validate the calculations in the following section give the same result, no matter which of the alternative implementation-/calculation-strategies are used.

\cite{concurrenceMainPaper} introduces the \emph{magic basis} that is shown in \autoref{eq:magic-basis} as opposed to the counting basis that was introduced in the beginning of \fullref{sec:theory-observables-density-matrix}. The magic basis consists of \emph{Bell states} with particular phases.

\begin{equation}
    \label{eq:magic-basis}
    \begin{split}
        \ketN[m_1] &= \frac{1}{\sqrt{2}} \left(\ketN[11] + \ketN[00]\right) \\
        \ketN[m_2] &= \frac{1}{\sqrt{2}} i \left(\ketN[11] - \ketN[00]\right) \\
        \ketN[m_3] &= \frac{1}{\sqrt{2}} i \left(\ketN[10] + \ketN[01]\right) \\
        \ketN[m_4] &= \frac{1}{\sqrt{2}} \left(\ketN[10] - \ketN[01]\right) \\
    \end{split}
\end{equation}

The Bell state is a construct tightly linked to entanglement, consequently it is by design straight forward, to write down the concurrence of a pure state expressed in the magic basis.
For $\ketN[\psiOfT] = \sum\limits_{i=1}^4 \alpha_i \ketN[m_i]$ the concurrence $C(\psiOfT)$ is $C(\psiOfT) = \left|\sum\limits_{i=1}^4 \alpha_i^2 \right|$.
This is equivalent to saying for a pure state expressed in the counting basis $\ketN[\psiOfT] = \alpha \ketN[00] +  \beta \ketN[01] +  \gamma \ketN[10] +  \delta\ketN[11]$, the concurrence is calculated $C(\psiOfT) = 2 \left|\alpha \delta - \beta \gamma\right|$.
This gives the same value as for all calculations that extract the concurrence from the density matrix that is calculated from these pure state coefficients.
But, while all states obtained via time evolution $\ketN[{\psiOfT[\schroedingerPicture]}]$ - as described in previous chapters - are pure states, through the reduction to a \glqq two-spin-system\grqq{} of particles $l,\,\sigma$ and $m,\,\mu$ to $\rho_{l,\,\sigma,\,m,\,\mu} (t)$ the remaining system is likely to be a mixed state \cite{mixedStateFromPureState}.

The calculation necessarily needed to be generalized to take the density matrix as an input.
\cite{concurrenceRewording} gives two alternate ways of calculating the concurrence from the density matrix.
Both require the \emph{spin-flip} operation $\tilde{\rho}$.
The original paper \cite{concurrenceMainPaper} describes this operation for density matrix $\rho$ expressed in the counting basis to be: transforming it into to magic basis \autoref{eq:magic-basis}, taking the complex conjugate, then transforming it back.
As per \cite{concurrenceRewording} this is equivalent to the operation described in \autoref{eq:spin-flip}, where $\rho$ and \pauli[]{y} (\autoref{eq:pauli-matrices}) are expressed in the standard counting basis.

\begin{equation}
    \label{eq:spin-flip}
    \begin{split}
        \tilde{\rho} = \left(\pauli[]{y} \otimes \pauli[]{y}\right) \rho^\ast  \left(\pauli[]{y} \otimes \pauli[]{y}\right)
    \end{split}
\end{equation}

\filepath{\cite{selfCode}}{/calculation-helpers/concurrence/basis-transformation.py} verifies this equivalence and shows the ${\tilde{\ast}}$-operation modifies a hermitian matrix like in \autoref{eq:demonstration-spin-flip}.

\begin{equation}
    \label{eq:demonstration-spin-flip}
    \begin{split}
        \rho = \left(\begin{matrix}
            a& b & c &d \\
            \overline{b}& e & f &g \\
            \overline{c}& \overline{f} & h &k \\
            \overline{d}& \overline{g} & \overline{k} &l \\
        \end{matrix}\right)  \quad\Rightarrow\quad\tilde{\rho} = 
        \left(\begin{matrix}
            l& -k & -g &d \\
            -\overline{k}& h & f &-c \\
            -\overline{g}& \overline{f} & e &-b \\
            \overline{d}& -\overline{c} & -\overline{b} &a \\
        \end{matrix}\right) 
    \end{split}
\end{equation}

With that knowledge \autoref{eq:concurrence-calculation} states how to calculate the concurrence $C(\rho)$ from the reduced density matrix $\rho$.

\begin{equation}
    \label{eq:concurrence-calculation}
    \begin{gathered}
        R = \sqrt{\sqrt{\rho}\tilde{\rho}\sqrt{\rho}}\\
        \lambda_1 \geq \lambda_2 \geq \lambda_3 \geq \lambda_4, \quad \left\{\lambda_i\right\} = \text{Eigenvalues}(R)\\
        C(\rho) = \max (0, \lambda_1 - \lambda_2 - \lambda_3 - \lambda_4)
    \end{gathered}
\end{equation}

In this formula, the square root $\sqrt{M}$ of a square matrix $M$ is a square matrix of the same dimensionality, that fulfills the property $\sqrt{M}\cdot \sqrt{M} = M$.
A candidate can be obtained numerically, by calculating the eigenvalues of the matrix (which for a hermitian matrix are all positive real numbers), taking their square root and re-assembling the matrix with the eigenvectors like in \autoref{eq:matrix-square-root} ($\lambda_i$ different values than in \autoref{eq:concurrence-calculation}).

\begin{equation}
    \label{eq:matrix-square-root}
    \begin{gathered}
        M  = Q \Lambda Q^\text{T}\\
        \Lambda = \text{diag}(\lambda_1,\, \lambda_2,\, \lambda_3,\, \lambda_4)\\
        \sqrt{\Lambda} = \text{diag}(\sqrt{\lambda_1},\, \sqrt{\lambda_2},\,\sqrt{\lambda_3},\,\sqrt{\lambda_4})\\
        \sqrt{M}  = Q \sqrt{\Lambda} Q^\text{T}
    \end{gathered}
\end{equation}

While this operation is quite complicated, it is required only once per calculation and only for a matrix of fixed size $4\times 4$.
To avoid it, \cite{concurrenceRewording} alternatively provides that the $\lambda_i$ from \autoref{eq:concurrence-calculation} are the square roots of the non-hermitian matrix $\rho \tilde{\rho}$.
For exact calculations both ways of calculating yield the same results, which is implemented and checked. 
If there are any inconsistencies in the order of basis vectors however, the results diverge. Which in fact is a good check that the bases are consistent across the whole implementation.

        \FloatBarrier
        
        \subsection{Energy}
        \label{sec:theory-observables-energy}
        % TODO


%TODO swapping == double flipping if controlled....
        \FloatBarrier

    \section{Variational Classical Networks}
    \label{sec:theory-variational-classical-networks}
    \section{TODO}
    \begin{frame}[t]
        \frametitle{TODO}
        
        \vspace{-0.5em}
        \begin{itemize}
            \item TODO
        \end{itemize}

        % notes 
        \onslide % on all slides of frame
        \note[item] {
            TODO
        }
    \end{frame}

    \FloatBarrier
        
        \subsection{General Theory of VCN}
        \label{sec:theory-variational-classical-networks-basics}
        %TODO

        

\cite{originalDerivationTimeDependendVariationalPrinciple} % tdvp historic source
\cite{TDVPcomplexPrefactors} % how to treat complex pre-factors, complex differentiation avoided through real differentiation
\cite{complexDifferentiation} % in this case the complex derivative is fine
\cite{probabilitySamplingRequirementVCN} % VCN general derivation, eta dot derivation and that it needs to sample from the probability distribution of the current state(t)
\cite{variationalClassicalNetworksPaper} % Also source that does the same derivation on different system 
        \FloatBarrier


        \subsection{Application to this Task}
        \label{sec:theory-variational-classical-networks-application}
        Depending on the chosen parametrization, different \HvcnOft{\vec{\eta}} might be obtained. 
This changes the value of the variational derivatives \autoref{eq:variational-derivatives-definition}.
The reference \cite{VCNsolutionForRBM} states how to apply the method to \emph{restricted Boltzmann machines} (RBM) \cite{neuralNetworkQuantumStates}.
In this case, a simple linear dependency on the parameters is chosen:

\begin{equation}
    \label{eq:hamiltonian-vcn}
    \HNOft = \lsum[i] C_i(t) \cdot \Phi_i(N) \leftrightarrow \HvcnOft{\vec{\eta}} = \lsum[i] \eta_i(t) \cdot \Phi_i(N)
\end{equation}

$C_\ast(t)$ are here analytically calculated functions, while the $\eta_\ast(t)$ are complex parameters to replace them.
The choice of the $\Phi_i(N)$ dictates the range of interaction that could technically be covered.
For the structure in \autoref{eq:hamiltonian-vcn}, the comparison with the cumulant-expansion yields expressions for $\Phi_l(N)$ and $C_l(t)$ in a controlled manner.
By identifying the terms in \autoref{eq:hn-integrated-first-order}, a possible choice for the first terms could be chosen like in \autoref{eq:psi-and-c-choices} (absolute lattice-site enumeration like in \autoref{fig:geometry-of-square-system}). 

% TODO large formula, make sure it stays at the correct location

\begin{equation}
    \label{eq:psi-and-c-choices}
    \begin{split}
        C_1(t) = \VhamiltonianAnalyticalPartIntegrated{A}{0}{1}{t}
        \quad &\quad
        \Phi_1(N) = J \lsum[l] \lsum[\biggerNeighborX{l}{m}] 
        \lsum[K] \frac{\psiN[K]}{\psiN}
            \bracketHelper{N}{
                \VhamiltonianOperatorPart{A}{l}{m} 
            }{K}\\
        C_2(t) = \VhamiltonianAnalyticalPartIntegrated{B}{0}{1}{t}
        \quad &\quad
        \Phi_2(N) = J \lsum[l] \lsum[\biggerNeighborX{l}{m}] 
        \lsum[K] \frac{\psiN[K]}{\psiN}
            \bracketHelper{N}{
                \VhamiltonianOperatorPart{B}{l}{m} 
            }{K}\\
        C_3(t) = \VhamiltonianAnalyticalPartIntegrated{C}{0}{1}{t}
        \quad &\quad
        \Phi_3(N) = J \lsum[l] \lsum[\biggerNeighborX{l}{m}] 
        \lsum[K] \frac{\psiN[K]}{\psiN}
            \bracketHelper{N}{
                \VhamiltonianOperatorPart{C}{l}{m} 
            }{K}\\
        C_4(t) = \VhamiltonianAnalyticalPartIntegrated{A}{1}{0}{t}
        \quad &\quad
        \Phi_4(N) = J \lsum[l] \lsum[\smallerNeighborX{l}{m}] 
        \lsum[K] \frac{\psiN[K]}{\psiN}
            \bracketHelper{N}{
                \VhamiltonianOperatorPart{A}{l}{m} 
            }{K}\\
        C_5(t) = \VhamiltonianAnalyticalPartIntegrated{B}{1}{0}{t}
        \quad &\quad
        \Phi_5(N) = J \lsum[l] \lsum[\smallerNeighborX{l}{m}] 
        \lsum[K] \frac{\psiN[K]}{\psiN}
            \bracketHelper{N}{
                \VhamiltonianOperatorPart{B}{l}{m} 
            }{K}\\
        C_6(t) = \VhamiltonianAnalyticalPartIntegrated{C}{1}{0}{t}
        \quad &\quad
        \Phi_6(N) = J \lsum[l] \lsum[\smallerNeighborX{l}{m}] 
        \lsum[K] \frac{\psiN[K]}{\psiN}
            \bracketHelper{N}{
                \VhamiltonianOperatorPart{C}{l}{m} 
            }{K}\\
        C_7(t) = \VhamiltonianAnalyticalPartIntegrated{A}{0}{M}{t}
        \quad &\quad
        \Phi_7(N) = J \lsum[l] \lsum[\biggerNeighborY{l}{m}] 
        \lsum[K] \frac{\psiN[K]}{\psiN}
            \bracketHelper{N}{
                \VhamiltonianOperatorPart{A}{l}{m} 
            }{K}\\
        \cdots\quad \quad \quad\quad \quad &\quad\quad \quad\quad\quad \quad \cdots
    \end{split}
\end{equation}

The dissection separates the structure into 12 different factors. The terms $C_1(t)$ to $C_6(t)$ describe the neighbor interaction in x-direction, the terms $C_7(t)$ to $C_{12}(t)$ are identical with the interactions in y-direction (not printed, as they are identical with the x swapped for y and the 1 swapped for an $M$).
For the chain-geometry only the terms 1 through 6 are relevant, as there are no neighbors in y-direction.

        \FloatBarrier

        
        \subsection{Explicit Time-Dependency}
        \label{sec:theory-variational-classical-networks-time-dependency}
        The method described so far seems to be the minimal viable parametrization at first.
Of course it should be possible to introduce more terms that are inspired from higher order perturbation theory.
Currently for the 2-dimensional grid the last section suggested 12 parameters. 
Considering that half of them can be expressed as complex conjugates of some other, this leaves only 6 distinct parameters.
While this doesn't provide many degrees of freedom to encode information, it should be sufficient for some basic calculations and a verification of the method.
At least for validating the hypothesis it should be better than the first-order model without variational parameters.

In previous sections, the TDVP-equation was derived from maximizing the overlap of a variational state that has its parameters numerically integrated to a future time and the same state being time-evolved with the Hamiltonian for a small time-step.
However as \cite{TDVPcomplexPrefactors} shows, the TDVP-formalism can also be derived from an \emph{action principle}.
And as \emph{Noether's theorem} states: if a method keeps the \emph{Lagrangian invariant} (like the mentioned principle that derives parameter-variations which are symmetries of the action), the system's energy must be \emph{exactly} conserved over time \cite{energyConservationFromActionPrinciple} (independent of how \glqq bad\grqq{} the parametrization is).

In order to adhere to the structure of the thesis, at this point some numerical results are revealed in advance. 
As experiments in \fullref{sec:experiments-vcn-fails} show, this simple parametrization does not result in the energy being conserved at all.
Upon closer inspection, the resulting changes to the variational parameters seem to be more or less independent from the initial values.
After implementation reasons for this behavior were ruled out to the best possible degree, it seemed unclear what was the cause for this.

Yet the algorithm \emph{must} conserve the system's energy - for small enough time-steps - most importantly already at the very first variational time-step.
This leads to the solution, because the only thing that is left to influence the effective Hamiltonian (at $t=0$, where all introduced variational parameters are approximately zero) is the base-energy with its explicit time-dependency.

Presenting this realization in this case out-of-order (presenting and doing experiments on a clearly still unviable model before presenting the correct one) is supposed to reflect the work-process that was necessary to arrive at this final version.
In first theoretical discussions about the application of TDVP, the algorithm's explicit dependency on time in the base-energy case had been touched and even been presented in the correct form to solve this shortcoming.
Yet, only after going through two alternate base-energy parametrizations unsuccessfully and arriving at the final working solution independently of this hint, the relevance of the statement became clear.
In hindsight it is straight forward to see, why the explicit time-dependency contribution of the base-energy to the effective Hamiltonian is detrimental, when looking at equations \ref{eq:eta-dot-calculation} and \ref{eq:variational-derivatives-definition}.

Starting at a point $t$ in time, it comes natural to numerically integrate a quantity $\vec{\eta}$ if one has access to the time-derivative $\dot{\vec{\eta}}$.
In the method up to this point through the partial derivative with respect to the variational parameters: $\frac{\partial \HeffOft[N,\,\vec{\eta}]}{\partial \vec{\eta}_k} = \frac{\partial \left(-i E_0(N) t + \Hvcn{\vec{\eta}}\right)}{\partial \vec{\eta}_k} = \frac{\partial \Hvcn{\vec{\eta}} }{\partial \vec{\eta}_k}$, the explicit time dependency is dropped.
However classically the time-derivative is obtained as per the chain rule:
$\frac{\mathrm{d} f(t,\,x(t))}{\mathrm{d} t} = \frac{\partial f}{\partial t} + \frac{\partial f}{\partial x} \cdot \frac{\mathrm{d} x}{\mathrm{d} t}$, which rightfully does not drop the explicit dependency.
So to avoid having to derive the algorithm again, but with explicit time-dependency in mind, the most direct strategy would be to eliminate all explicit time-dependencies.
This would result in a completely variational effective Hamiltonian $\Heffvcn{N}{\vec{\eta}}$, where all that is time-dependent is encoded in the current state of $\vec{\eta}$.

Following the linear strategy of \autoref{eq:hamiltonian-vcn}, the resulting structure is presented in \autoref{eq:h-eff-vcn}.

\begin{equation}
    \label{eq:h-eff-vcn}
    \begin{split}
        \Heffvcn{N}{\vec{\eta}} &= \sum\limits_{n=1}^{\text{\#}(\text{var.})} \eta^\text{var.}_{n} \cdot \Phi^\text{var.}_{n}(N)
        + \sum\limits_{l=0}^{L} \eta^\text{b.e.}_{l} \cdot \Phi^\text{b.e.}_{l}(N)
         = \lsum[l] \eta_l \cdot \Phi_l(N)\\
         \vec{\eta} &= \left(
            \eta^\text{var.}_{1},\,\eta^\text{var.}_{2},\, \dots,\, \eta^\text{var.}_{\text{\#}(\text{var.})},\,
            \eta^\text{b.e.}_{0},\,\eta^\text{b.e.}_{1},\, \dots,\, \eta^\text{b.e.}_{L}
           \right)^\text{T}
    \end{split}
\end{equation}

The $\eta^\text{var.}_{n}$ and $\Phi^\text{var.}_{n}(N)$ are the same as in \autoref{sec:theory-variational-classical-networks-application} and still depend on the chosen parametrization of $\HNOft$.
When comparing with \autoref{eq:base-energy}, \autoref{eq:base-energy-variational-parameters} presents itself as an ideal choice for $\eta^\text{b.e.}_{l}$ and $\Phi^\text{b.e.}_{l}(N)$, as it upholds $\sum\limits_{l=0}^{L} \eta^\text{b.e.}_{l} \cdot \Phi^\text{b.e.}_{l}(N) = -i \cdot E_0(N) \cdot t$ for a lattice with $L$ sites.

\begin{equation}
    \label{eq:base-energy-variational-parameters}
    \begin{split}
        E_0(N) &\stackrel{\ref{eq:base-energy}}{=} U \cdot \lsum[m] n_{m,\,\up}n_{m,\,\down} + \lsum[l,\,\sigma] \epsl n_{l,\,\sigma}\\
        \eta^\text{b.e.}_{l} &\stackrel{\phantom{\ref{eq:base-energy}}}{=} \begin{cases}
            l=0: \quad -i\cdot U \cdot t             \\
            l>0: \quad -i\cdot \epsl[l] \cdot t
        \end{cases}\\
        \Phi^\text{b.e.}_{l}(N) &\stackrel{\phantom{\ref{eq:base-energy}}}{=} \begin{cases}
            l=0: \quad \lsum[m] n_{m,\,\up}n_{m,\,\down}             \\
            l>0: \quad \lsum[\sigma] n_{l,\,\sigma}
        \end{cases}
    \end{split}
\end{equation}

Other partitions of the parameters are also possible, e.g. having multiple parameters for the encoding of the double-occupation term.
With this modification, the equations in \autoref{sec:theory-variational-classical-networks-basics} can be used as derived and the explicit time-dependency is eliminated.
        \FloatBarrier


    \section{Time-Complexity Optimizations}
    \label{sec:theory-optimizations}
    \section{TODO}
    \begin{frame}[t]
        \frametitle{TODO}
        
        \vspace{-0.5em}
        \begin{itemize}
            \item TODO
        \end{itemize}

        % notes 
        \onslide % on all slides of frame
        \note[item] {
            TODO
        }
    \end{frame}

    \FloatBarrier

        \subsection{Monte Carlo Sampling}
        \label{sec:theory-optimizations-monte-carlo}
        Mathematically, the problem of finding the solution to such a quantum-mechanical many-body model is solved by the calculations provided up to this point. 
However when looking at \autoref{eq:expectation-value}, one can easily see why in practice this is not viable with larger systems. 
The calculation requires summing over all base-states. 
Their number is given in \autoref{eq:system-size} (factor 2 in the exponent for the spin-degree) to be of exponential size in regards to the number of lattice-sites.
Considering a physical material has at least in the order of $10^{23}$ (from \emph{Avogadros-Constant}) degrees of freedom, an \bigo{2^{\text{\#}(\text{sites})}} computational-complexity is not good enough to simulate interestingly-sized systems. 

\begin{equation}
    \label{eq:system-size}
    \text{\#}(\text{states}) = 2^{\text{\#}(\text{sites}) \cdot 2}
\end{equation}

Multiple strategies have already been suggested to circumvent this limitation.
Here, the focus will be put onto randomized algorithms that use \emph{Monte-Carlo sampling} in order generate a limited number of states, representative of the real probability-distribution, in order to not having to sum over the complete \emph{Hilbert-Space}.
The strategy of solving problems of this kind with the strategy of \emph{variational Monte-Carlo} is e.g. suggested in \cite{metropolisAlgorithmAndVariationalMonteCarlo}.
This means starting with a random state and applying modifications to it (= variation).
When the accepting/rejecting of these modifications is coupled to the probability the state has, based on its energy in the systems energy-distribution, the resulting states will have an energy/probability distribution, that resembles the one as if they were sampled from the full Hilbert-Space.
The derivation for the whole process is shown in \cite{monteCarloObservableSampling}.

However computing the normalized probability for the states is still required, in order to compare against a (e.g. thermal) reference-distribution.
As one sees in the probability in \autoref{eq:probability-isolated} that was defined in \autoref{eq:expectation-value}, still a full sum over all states is required to get the normalization factor.
\begin{equation}
    \label{eq:probability-isolated}
    \probabilityOf{N}{t} = \frac{
        \absSquare{e^{\HeffOft}} \absSquare{\psiN} 
    }{
        \lsum[K] \absSquare{e^{\HeffOft[K]}} \absSquare{\psiN[K]} 
    }
\end{equation}
However by employing the \emph{Metropolis-Hastings Algorithm} instead of only the \emph{Metropolis-Algorithm} to decide on the acceptance/rejection of proposed states, one can drop the requirement of normalizing the probability.
In \autoref{eq:transition-probability-final} the \emph{acceptance-ratio} $\alpha$ is derived that is necessary for the Metropolis-Hastings Algorithm to work (as per \cite{metropolisHastingsAlgorithmGeneral}) and decide on the transition from state \ketN[N] to \ketN[{\vphantom{N}\smash{\tilde{N}}}].
The algorithm proposes to generate a new state \ketN[{\vphantom{N}\smash{\tilde{N}}}] and calculate the corresponding $\alpha$. Then it requires generating a random number $u$ between $0$ and $1$, pulled from a uniform distribution.
If $u \leq \alpha$, \ketN[{\vphantom{N}\smash{\tilde{N}}}] is accepted and will be used as the new current state in the next step of the sampling. 
If $u>\alpha$, the proposed state is rejected and \ketN[N] will be used again as the next state.
The function $f(N,t)$, provided in \autoref{eq:proportional-function}, fulfills the necessary property of being proportional to the probability  \probabilityOf{N}{t}.

\begin{equation}
    \label{eq:proportional-function}
    f(N,t) = \absSquare{e^{\HeffOft}} \absSquare{\psiN} \propto \probabilityOf{N}{t}
\end{equation}

\begin{equation}
    \label{eq:transition-probability-final}
    \begin{split}
        \alpha &\stackrel{\phantom{\ref{eq:proportional-function}}}{=} \frac{\probabilityOf{\vphantom{N}\smash{\tilde{N}}}{t}}{\probabilityOf{N}{t}} =  \frac{f(\vphantom{N}\smash{\tilde{N}},t)}{f(N,t)}
        \stackrel{\ref{eq:proportional-function}}{=}
        \frac{
            \absSquare{e^{\HeffOft[\vphantom{N}\smash{\tilde{N}}]}} \absSquare{\psiN[\vphantom{N}\smash{\tilde{N}}]}
        }{
            \absSquare{e^{\HeffOft}} \absSquare{\psiN}
        }\\
        &\stackrel{\phantom{\ref{eq:proportional-function}}}{=}
        \frac{\absSquare{\psiN[\vphantom{N}\smash{\tilde{N}}]}}{\absSquare{\psiN}}
        \frac{
            e^{\real\left(\HeffOft[\vphantom{N}\smash{\tilde{N}}]\right) + i\imaginary\left(\HeffOft[\vphantom{N}\smash{\tilde{N}}]\right) +\real\left(\HeffOft[\vphantom{N}\smash{\tilde{N}}]\right)-i \imaginary\left(\HeffOft[\vphantom{N}\smash{\tilde{N}}]\right)}
        }{
            e^{\real\left(\HeffOft\right) + i\imaginary\left(\HeffOft\right) +\real\left(\HeffOft\right)-i \imaginary\left(\HeffOft\right)}
        }\\
        &\stackrel{\phantom{\ref{eq:proportional-function}}}{=}
        \frac{\absSquare{\psiN[\vphantom{N}\smash{\tilde{N}}]}}{\absSquare{\psiN}}
        e^{2\cdot \real\left(\HeffOft[\vphantom{N}\smash{\tilde{N}}]\right) - 2\cdot \real\left(\HeffOft\right)}\\
        &\stackrel{\phantom{\ref{eq:proportional-function}}}{=}
        \frac{\absSquare{\psiN[\vphantom{N}\smash{\tilde{N}}]}}{\absSquare{\psiN}}
        e^{2\cdot \real\left(\HeffOft[\vphantom{N}\smash{\tilde{N}}] - \HeffOft\right)}
    \end{split}
\end{equation}

Multiple factors must be considered, which kind of variation is applied to the states in this schema. 
However probably the most important criteria is how quickly one can evaluate \autoref{eq:transition-probability-final}, as this will be done tens to hundreds of times between each sampled state.

Finally, the method of sampling a selection of distributed states results in \autoref{eq:final-sampling-equation} as a rewrite to \autoref{eq:expectation-value} \cite{monteCarloObservableSampling}.

\begin{equation}
    \label{eq:final-sampling-equation}
    \frac{\bracketHelper{\psiOfT[\schroedingerPicture]}{\ObservableOp}{\psiOfT[\schroedingerPicture]}}{\braketHelper{\psiOfT[\schroedingerPicture]}{\psiOfT[\schroedingerPicture]}} \stackrel{\ref{eq:expectation-value}}{=} 
    \lsum[N]
    \probabilityOf{N}{t}
    \localObservable{N}{t} \cdot
    \stackrel{\text{\cite{monteCarloObservableSampling}}}{\approx} \frac{1}{\left|\left\{N\right\}_\text{MC}\right|} \lsum[\left\{N\right\}_\text{MC}]\localObservable{N}{t}
\end{equation}
        \FloatBarrier

        \subsection{Analytical Simplifications for Special Cases}
        \label{sec:theory-optimizations-analytical}
        Generally, handling the computation of the difference of the effective Hamiltonian for two states $\HeffOft[\vphantom{N}\smash{\tilde{N}}]-\HeffOft[N]$ is the most expensive computation that is required up to this point.
For one it is e.g. required in \autoref{eq:form-heff-difference} for the calculation of the spin-polarized kinetics.
Furthermore - as stated in \fullref{sec:theory-optimizations-monte-carlo} - it is also needed to compute the transition probability between two sampled states in \autoref{eq:transition-probability-final}.
The calculation of $\HeffOft[N]$ on its own requires an effort of \bigo{\text{\#}(\text{sites}) \cdot \text{\#}(\text{nearest neighbors})}.
However, for states \ketN[N] and \ketN[{\vphantom{N}\smash{\tilde{N}}}] that are connected with only small changes, most of the elements in the sum cancel and the complexity may here even be reduced to \bigo{$\text{\#}(\text{nearest neighbors})$}.
This is quite attractive, because it de-couples the per-step computational costs from the number of lattice sites - a vital step for simulating large systems efficiency - however this requires extra analytical calculations.

In this section analytical simplifications are presented to get the objects $E_0(N)-E_0(\vphantom{N}\smash{\tilde{N}})$ and $\HNOft - \HNOft[\vphantom{N}\smash{\tilde{N}}]$ (Caution, do not miss the extra \emph{minus} required, to calculate the required $\HeffOft[\vphantom{N}\smash{\tilde{N}}]-\HeffOft[N]$ from this!).

Evidently, this also needs to hold for the variational versions of the effective Hamiltonian, that were introduced in \autoref{sec:theory-variational-classical-networks-time-dependency}.
As the $\Phi_\ast(N)$ are inspired by the form of the base-energy-terms in $E_0(N)$ or parts of $\HNOft$, all the same arguments apply: 
$\Heffvcn{\vphantom{N}\smash{\tilde{N}}}{\vec{\eta}} - \Heffvcn{N}{\vec{\eta}} = \lsum[l] \eta_l \cdot \left[\Phi_l(\vphantom{N}\smash{\tilde{N}}) - \Phi_l(N)\right]$.
Yet all of the re-writes must be done precisely to produce the correct results.

\subsubsection*{Initial State}

The initial state of the system before the start of the time-evolution is encoded in $\psiN$, defined by \autoref{eq:base-expansion-state}. Different configurations can be considered here.
The best choice for this would logically be the one that save the most computational effort and provides the biggest symmetry for the terms to promote the possibility of terms canceling.

A choice that fulfills these requirements trivially, would be the perfectly uniform distribution of probability for all the base-states.
For most of the sources that describe similar methods (like e.g. \cite{isingDynamicsWithClassicalNetworks} or \cite{variationalClassicalNetworksPaper}, where the latter describes how the approach also works for other initial states) this initial state has been chosen for its advantages.

The calculation of the uniform probability is straight-forward and listed in \autoref{eq:psi-n-homogenous}, while the state would be written down like \autoref{eq:psi-zero}.

\begin{equation}
    \label{eq:psi-n-homogenous}
    \psiN{} = \frac{1}{\sqrt{\text{\#}(\text{states})}} \stackrel{\ref{eq:system-size}}{=} \frac{1}{\sqrt{2^{\text{\#}(\text{sites}) \cdot 2}}} = \frac{1}{2^{\text{\#}(\text{sites})}}
\end{equation}

\begin{equation}
    \label{eq:psi-zero}
    \ketpsiof[\schroedingerPicture]{t=0} = \bigotimes\limits_{l=1}^{\text{\#}(\text{states})} \frac{1}{2} \left( 1 + \withspinhcop[\schroedingerPicture]{l}{\up}{\dagger} + \withspinhcop[\schroedingerPicture]{l}{\down}{\dagger} + \withspinhcop[\schroedingerPicture]{l}{\up}{\dagger}  \withspinhcop[\schroedingerPicture]{l}{\down}{\dagger} \right) \ketN[0]
\end{equation}

Central advantage of the uniform distribution of base-state-probabilities is, that all $\psiN$ are equal, which makes it possible to cancel most terms from the sums, as $\psiN{}/\psiN[\vphantom{N}\smash{\tilde{N}}]{} = 1$.
The following two optimizations require this assumption.

As the states under consideration are \ketN[N] and \ketN[{\vphantom{N}\smash{\tilde{N}}}], the occupation-numbers $n_{l,\,\sigma}$ will describe the occupation of \ketN[N] and $\tilde{n}_{l,\,\sigma}$ the occupation of \ketN[{\vphantom{N}\smash{\tilde{N}}}].
Both of the following modifications restrict the values $\tilde{n}_{l,\,\sigma}$ to make them dependent on $n_{l,\,\sigma}$.

\subsubsection*{Single Flip Modification}

In this simplification, the maximum difference that can happen, is that a \emph{flipping} event occurs on site $i$ and spin $\sigma_i$. 
In the language of hard-core bosons this means one occupation turning form a 0 into a 1 or the inverse of that.
This operation doesn't conserve the number of particles, so depending on the application this might not be a desired modification for a Monte-Carlo-step.
Still, the calculation of the reduced density-matrix as described in \autoref{sec:theory-observables-density-matrix} requires calculation of the effective Hamiltonian difference between two states connected by such a modification.

The flipping results in the values for $\tilde{n}_{l,\,\sigma}$ are given in \autoref{eq:new-n-flipping}.

\begin{equation}
    \label{eq:new-n-flipping}
    \tilde{n}_{l,\,\sigma} = \begin{cases}
        l = i \land \sigma = \sigma_i\text{ : }  &  (1 - n_{i,\,\sigma_i})  \\
        \text{else: }   &    n_{l,\,\sigma} 
    \end{cases}
\end{equation}

This simplifies the energy difference $E_0(N)-E_0(\vphantom{N}\smash{\tilde{N}})$ like equation \autoref{eq:simplified-base-energy-flipping} presents.

\begin{equation}
    \label{eq:simplified-base-energy-flipping}
    \begin{split}
        E_0(N)-E_0(\vphantom{N}\smash{\tilde{N}}) 
        &\stackrel{\phantom{\ref{eq:new-n-flipping}}}{=} U\lsum n_{l,\,\down}n_{l,\,\up}-U\lsum \tilde{n}_{l,\,\down}\tilde{n}_{l,\,\up} 
        + \lsum[l,\,\sigma] \epsl n_{l,\,\sigma} - \lsum[l,\,\sigma] \epsl \tilde{n}_{l,\,\sigma}\\
        &\stackrel{\ref{eq:new-n-flipping}}{=} \epsl[i] \left(2 n_{i,\,\sigma_i} - 1\right) +
        U\cdot \begin{cases}
            \sigma_i = \,\up\text{: }   & n_{i,\,\down} (2 n_{i,\,\up} - 1)\\
            \sigma_i = \,\down\text{: } & n_{i,\,\up} (2 n_{i,\,\down} - 1)  
        \end{cases}
    \end{split}
\end{equation}

The optimized calculation of $\HNOft - \HNOft[\vphantom{N}\smash{\tilde{N}}]$ or the derived $\Phi_l(N) - \Phi_l(\vphantom{N}\smash{\tilde{N}})$ is more involved to get correct.
Still the process is mathematically straight forward. 
For hints and further resources on how this was implemented, see \fullref{sec:implementation-details-script-generation}.
How the correctness of the optimizations is assured, is briefly touched in \fullref{sec:implementation-details-simplification-verifications}.
Finally, a generalized (if maybe not optimally efficient) method is outlined in \fullref{sec:theory-optimizations-geometry}.

\subsubsection*{Hopping, Swapping or Double-Flip Modification}

A possible Monte-Carlo-modification that conserves particle number is the change that occurs when a particle from a specific site and spin (occupation 1, then 0) is transferred to a different site and spin combination that currently is un-occupied (occupation 0, then 1).
This so-called \emph{hopping} event occurs in this example from site $i$ and spin $\sigma_i$ to the site $j$ and spin $\sigma_j$. 

The hopping is a subset of the \emph{swapping} modification, that occurs in the case of two occupation-number being exchanged.
The difference being, that while swapping two identical occupations is possible, for the hopping having exactly one particle on the relevant two sites is a pre-requirement.

Depending on the use-case this is a relevant difference, yet in case of calculation of the differences of effective Hamiltonians it is irrelevant: 
On comparison with e.g. \autoref{eq:form-heff-difference}, one can see that the term containing $\Delta = \HeffOft[\vphantom{N}\smash{\tilde{N}}]-\HeffOft[N]$ is multiplied with 0 in the cases where hopping is not possible.

Furthermore, the only way one could possibly get a contribution by the difference of effective Hamiltonians is in the case where $n_{i,\,\sigma_i} \neq n_{j,\,\sigma_j}$.
Because if they are the same, swapping doesn't change the state and a check can be performed beforehand, to save on computational resources (if $\ketN[\vphantom{N}\smash{\tilde{N}}] = \ketN[N]$, trivially  $\Delta = \HeffOft[\vphantom{N}\smash{\tilde{N}}]-\HeffOft[N] = 0$).

Thirdly, one could also think about an event, where simultaneously two occupation-numbers on two different site- / spin-indices are flipped, or \emph{double-flipping}.
In the case where hopping is possible, this is yet again the exact same operation (the 0 changes to a 1 and the 1 on the second site to a 0).
Differences are shown in \autoref{table:hopping-is-swapping}.

\begin{table}[htbp]
    \centering
    \begin{tabular}{l|cc|cc|c} 
        \toprule
        Type     &  $n_{i,\,\sigma_i}$ & $n_{j,\,\sigma_j}$ &
                    $\tilde{n}_{i,\,\sigma_i}$ & $\tilde{n}_{j,\,\sigma_j}$ & $\Delta$  \\  
        \midrule 
        hopping  & 0 & 0    & \texttimes & \texttimes & \texttimes \\
                 & 0 & 1    &     1      &      0     & $\Delta_1$ \\
                 & 1 & 0    &     0      &      1     & $\Delta_2$ \\
                 & 1 & 1    & \texttimes & \texttimes & \texttimes \\
        \midrule   
        swapping & 0 & 0    &     0      &      0     &     0      \\
                 & 0 & 1    &     1      &      0     & $\Delta_1$ \\
                 & 1 & 0    &     0      &      1     & $\Delta_2$ \\
                 & 1 & 1    &     1      &      1     &     0      \\
        \midrule   
        double-  & 0 & 0    &     1      &      1     & $\Delta_3$ \\
        flipping & 0 & 1    &     1      &      0     & $\Delta_1$ \\
                 & 1 & 0    &     0      &      1     & $\Delta_2$ \\
                 & 1 & 1    &     0      &      0     & $\Delta_4$ \\
        \bottomrule
    \end{tabular}
    \vspace{0.5cm}
    \caption{
        Basic tabular depiction of what the three different modification-operations do to the occupation-numbers of a state. 
        For a difference of effective Hamiltonians $ \Delta $ a placeholder value is listed.
        Where the exact value is known to be zero this is noted, otherwise a variable is listed.
        The same variable indicates the same value, or simply put in these cases the methods are exchangeable.
    }
    \label{table:hopping-is-swapping}
\end{table}

For the case of swapping, \autoref{eq:simplified-base-energy-hopping} shows the base-energy difference $E_0(N)-E_0(\vphantom{N}\smash{\tilde{N}})$ - the double flipping implementation would be constructed analogously to \autoref{eq:simplified-base-energy-flipping}.

\begin{equation}
    \label{eq:simplified-base-energy-hopping}
    \begin{split}
        E_0(N)-E_0(\vphantom{N}\smash{\tilde{N}}) 
        & = U\lsum n_{l,\,\down}n_{l,\,\up}-U\lsum \tilde{n}_{l,\,\down}\tilde{n}_{l,\,\up} 
        + \lsum[l,\,\sigma] \epsl n_{l,\,\sigma} - \lsum[l,\,\sigma] \epsl \tilde{n}_{l,\,\sigma}\\
        & =  \left(\epsl[i]-\epsl[j]\right)\left(n_{i,\,\sigma_i} - n_{j,\,\sigma_j}\right) +
        U\cdot \begin{cases}
            \sigma_i = \sigma_j\text{ : } & \left(n_{i,\,\up}-n_{j,\,\up}\right) \left(n_{i,\,\down}-n_{j,\,\down}\right)\\
            \sigma_i \neq \sigma_j\text{ : } & \left(n_{i,\,\up}-n_{j,\,\down}\right) \left(n_{i,\,\down}-n_{j,\,\up}\right)
        \end{cases}\\
        & = \left(\epsl[i]-\epsl[j]\right)\left(n_{i,\,\sigma_i} - n_{j,\,\sigma_j}\right) +
        U \left(n_{i,\,\sigma_i}-n_{j,\,\sigma_j}\right) \left(n_{i,\,\overline{\sigma_i}}-n_{j,\,\overline{\sigma_j}}\right)
    \end{split}
\end{equation}

To get the simplest possible implementation for more complicated terms, is is advised to implement double-flipping first.
It is the simplest of the three and (with the correct pre-factors that restrict the starting occupation-numbers) can be used to implement the other two easily.
This verified \filepath{\cite{selfCode}}{/computation-scripts/compareobservables.py}.
        \FloatBarrier

        \subsection{Geometry-Dependent Interaction-Range}
        \label{sec:theory-optimizations-geometry}
        While the optimizations described in the previous section greatly increase computational efficiency, they need to be applied for each $\HNOft$-order and each $\Phi_\ast(N)$ separately.
The problem is that for two dimensions the number of interactions to consider grows quickly, while visualizing the symmetries of what terms are equivalent gets progressively harder.
When looking at a chain, each higher order expands the interaction range by a one-site-step. 
So taking one more order into account, two sites are added to the possibilities for each interaction - this linearity is still somewhat manageable.
In two dimensions, the number of affected sites in relation to the interaction range grows quadratically, as visualized in \autoref{fig:growing-interaction-range}.

\begin{figure}[htbp]
    \centering
    \begin{tikzpicture}[scale=0.38]
    \definecolor{relevantsitecol}{HTML}{00AA00} % green

    \definecolor{concolhor}{HTML}{AA0000} % red
    \definecolor{concolver}{HTML}{FF8E00} % orange

    \def\m{5} % Change this value to adjust the grid size (m x m)
    \def\side{3} % Change this value to adjust the square side length
    \def\labelsize{\side/6} % Adjust label size
    
    \def\sp{1} % Change this value to adjust horizontal skip between the grids
    \def\order{2}

    \foreach \orderindex in {0,...,\numexpr\order\relax} {

        % Draw connections
        \foreach \x in {0,...,\numexpr\m-1\relax} {
            \foreach \y in {0,...,\numexpr\m-1\relax} {
                \pgfmathsetmacro{\orderindexincr}{\orderindex + 1}

                % Get shift amount
                \pgfmathsetmacro{\shft}{(\m + \sp - 1)*\orderindex*\side}
                
                % Get current node position
                \pgfmathsetmacro{\cx}{\x*\side + \side/2 + \shft}
                \pgfmathsetmacro{\cy}{\y*\side + \side/2 + \side/4}

                % Compute CURRENT Manhattan distance from center
                \pgfmathtruncatemacro{\dista}{abs(\x - (\m-1)/2) + abs(\y - (\m-1)/2)}


                % Connect to right neighbor
                \ifnum\x<\numexpr\m-1\relax

                    \pgfmathsetmacro{\usex}{\x+1}
                    \pgfmathsetmacro{\usey}{\y}

                    \pgfmathtruncatemacro{\distb}{abs(\usex - (\m-1)/2) + abs(\usey - (\m-1)/2)}

                    \pgfmathsetmacro{\nx}{\usex*\side + \side/2 + \shft}
                    \pgfmathsetmacro{\ny}{\usey*\side + \side/2 + \side/4}

                    % Color if both are smaller Manhattan distance
                    \ifnum\dista<\orderindexincr
                        \ifnum\distb<\orderindexincr
                            \ifnum\orderindex=1
                                \edef\useconcol{concolhor}
                            \else
                                \edef\useconcol{gray}
                            \fi
                        \else
                            \edef\useconcol{black}
                        \fi
                    \else
                        \edef\useconcol{black}
                    \fi

                    \draw[line width=2\pgflinewidth, color=\useconcol] (\cx,\cy) -- (\nx,\ny);
                \fi
                

                % Connect to top neighbor
                \ifnum\y<\numexpr\m-1\relax

                    \pgfmathsetmacro{\usex}{\x}
                    \pgfmathsetmacro{\usey}{\y+1}

                    \pgfmathtruncatemacro{\distb}{abs(\usex - (\m-1)/2) + abs(\usey - (\m-1)/2)}

                    \pgfmathsetmacro{\nx}{\usex*\side + \side/2 + \shft}
                    \pgfmathsetmacro{\ny}{\usey*\side + \side/2 + \side/4}

                    % Color if both are smaller Manhattan distance
                    \ifnum\dista<\orderindexincr
                        \ifnum\distb<\orderindexincr
                            \ifnum\orderindex=1
                                \edef\useconcol{concolver}
                            \else
                                \edef\useconcol{gray}
                            \fi
                        \else
                            \edef\useconcol{black}
                        \fi
                    \else
                        \edef\useconcol{black}
                    \fi

                    \draw[line width=2\pgflinewidth, color=\useconcol] (\cx,\cy) -- (\nx,\ny);
                \fi
            }
        }

        % Draw dots and labels with custom text using a loop
        \foreach \x in {0,...,\numexpr\m-1\relax} {
            \foreach \y in {0,...,\numexpr\m-1\relax} {
                % Get shift amount
                \pgfmathsetmacro{\shft}{(\m + \sp - 1)*\orderindex*\side}
                \pgfmathtruncatemacro{\index}{(\m-\y-1) * \m + \x}

                % Get current node position
                \pgfmathsetmacro{\cx}{\x*\side + \side/2 + \shft}
                \pgfmathsetmacro{\cy}{\y*\side + \side/2 + \side/4}

                % Compute Manhattan distance from center
                \pgfmathtruncatemacro{\dist}{abs(\x - (\m-1)/2) + abs(\y - (\m-1)/2)}
                
                % Change color if distance equals \orderindex
                \pgfmathsetmacro{\orderindexincr}{\orderindex + 1}
                \ifnum\dist<\orderindexincr
                    \fill[color=relevantsitecol] (\cx,\cy) circle (0.4);
                \else
                    \fill[color=dblue] (\cx,\cy) circle (0.4);
                \fi
            }
        }
    }

\end{tikzpicture}
    \caption{
        A depiction of the relevant sites and interactions that might be affected when a modification-event (e.g. a single flip on the center site) occurs.
        From left to right, the graphic shows the case for \emph{base-energy only}, \emph{first order perturbation theory} and \emph{second order perturbation theory}.
        In case of a modification to the center-site the occupation of all green sites possibly influences the outcome of the calculation (in the corresponding order). For the first and second order the terms can be identified with their corresponding colored edges.
    }
    \label{fig:growing-interaction-range}
\end{figure}

As show in \fullref{sec:theory-optimizations-analytical}, the described methods require calculating the differences between the effective Hamiltonians on two states that only differ in localized modifications.
Each modification flips the occupation of at least one site. 
To guarantee the differences of the effective Hamiltonians are evaluated correctly, only the terms that contain a modified site-occupation need to be evaluated. 
The terms that only contain non-modified occupation-numbers are guaranteed to cancel.

For the zeroth-order (base-energy) this is swiftly calculated and can be written down in simple terms like in e.g. equations \ref{eq:new-n-flipping} or \ref{eq:simplified-base-energy-hopping}.
In the case of the first order, it is simple enough to identify the terms.
Looking back at \autoref{eq:hn-integrated-first-order}, it becomes clear that all terms are described by \emph{edges} of two adjacent sites.
These relevant edges are colored red and orange in \autoref{fig:growing-interaction-range}. 
The symmetry is clear enough to spot that all horizontal and vertical edges respectively have the same analytical value.
This is because the $\VhamiltonianAnalyticalPartIntegrated{$\ast$}{l}{m}{t}$ are translationally invariant and are complex conjugates of each other when inverting the indices - which is the reason for arriving at $2\cdot 3 \cdot 2 = 12$ terms in \autoref{eq:psi-and-c-choices}, of which 6 are identified in \autoref{table:first-order-identification} (times two for horizontal/vertical versions).

\begin{table}[htbp]
    \centering
    \begin{tabular}{cc|cccccccccccccccc} 
        \toprule
             $l$ & \up      & 0 & 0 & 0 & 0   & 0 & 0 & 0 & 0   & 1 & 1 & 1 & 1   & 1 & 1 & 1 & 1    \\
             $l$ & \down    & 0 & 0 & 0 & 0   & 1 & 1 & 1 & 1   & 0 & 0 & 0 & 0   & 1 & 1 & 1 & 1    \\
             $m$ & \up      & 0 & 0 & 1 & 1   & 0 & 0 & 1 & 1   & 0 & 0 & 1 & 1   & 0 & 0 & 1 & 1    \\
             $m$ & \down    & 0 & 1 & 0 & 1   & 0 & 1 & 0 & 1   & 0 & 1 & 0 & 1   & 0 & 1 & 0 & 1    \\
        \midrule   
    \multicolumn{2}{c|}{$\VhamiltonianAnalyticalPartIntegrated{$\ast$}{l}{m}{t}$}
                            &   &   &   &    
                                              & A &   & C & 
                                                                & A & C &   &   
                                                                                  & B & A & A &      \\
    \multicolumn{2}{c|}{$\VhamiltonianAnalyticalPartIntegrated{$\ast$}{m}{l}{t}$}
                            &   & A & A & B  
                                              &   &   & C & A
                                                                &   & C &   & A 
                                                                                  &   &   &   &      \\
        \bottomrule
    \end{tabular}
    \vspace{0.5cm}
    \caption{
        A list of all occupation-configurations for the first order terms.
        The indices of the two involved sites are $l$ and $m$. Each hold a site for spin up and down particles \up and \down.
        Not all 16 configurations have a representative term that results from the perturbation theory.
        The ones that do reference the letters A, B and C from the $\VhamiltonianAnalyticalPartIntegrated{$\ast$}{l}{m}{t}$ in \autoref{eq:hn-integrated-first-order}.
        A second line takes the other combination $l \leftrightarrow m$ into account.
    }
    \label{table:first-order-identification}
\end{table}

Looking at all 16 occupation-number-combinations, the complexity is already extreme. 
Some combinations contribute to no terms, some contribute multiple times and some terms have more states contributing to them, some less.
Identifying and analytically optimizing the first order therefore is \emph{just} possible, albeit complicated.
For the variational parametrization, it would be possible and maybe even more efficient to just have 16 parameters, mapping one to one onto the occupation-number-combination of the respective bond. 
In second order, however, this all breaks down.
Taking a second look at \autoref{fig:growing-interaction-range}, each term in \autoref{eq:hn-integrated-second-order-final} now corresponds to \emph{two pairs of connections}, of which are 34 different ones with the edges adjacent (assuming no edge-cases, because at the border all is different again) and even more disconnected ones that might or might not be cases from the first order.
% 6 L -> * 4 for turning
% 5 I -> * 2 for turning
For identifying these, one must take rotational and mirror symmetries into account and the check for the border is vastly more complicated.
Because each of these double-bounds is formed by 3 sites with each two spin-degrees, there are in total $2^{3\cdot 2} = 64$ base-cases that need to be mapped.
The numbers of base-cases 16 and 64 here are only comparably low, as they take symmetries into account. 
For brute-forcing all combinations without the manual identification of symmetries, one would get $2^{5\cdot 2} = 1.024$ cases for the first order and $2^{13\cdot 2} = 67.108.864$ - it would most definitely not be a viable strategy to convert each of these into their own parameter.
Even if it was possible, this would diminish the physical bias we hope to get from defining the $\Phi_\ast(N)$ according to the cumulant expansion.

Even more problematic is the case of double-flips, because the interaction-spheres of these two modifications might be independent, overlapping slightly or even so much that both modifications are inside one bond.
While not impossible for the second order, it is highly complicated to analytically compute the most efficient formula for the difference of effective Hamiltonians.
And the process is not easily generalizable to higher orders because of order-specific symmetries.


\subsubsection*{Pre-Computed Interaction Spheres}

In this case, this was solved by defining pre-computed sets of indices that must be taken into account.

Given a formula that needs to sum over all indices once (like \autoref{eq:hn-integrated-first-order}), or once over all indices for each index (like \autoref{eq:hn-integrated-second-order-final} or \autoref{eq:v-squared-hard-computation}), it is clear that they require a runtime-complexity of \bigo{\text{\#}(\text{sites})} or even \bigo{(\text{\#}(\text{sites}))^2} to evaluate.
Yet all of them have so far also appeared as differences of two sums - one for the un-modified state and one for a state that had undergone a localized modification - of which most elements cancel.

One can assume it is possible to compute the neighbors of a site in constant time - still it takes computational time.
This and the problem from before can be solved by taking time at the start of the program to generate a cache of all tuples of indices, which would appear in the sum that is being optimized, but only restricted to a pre-set range around the index of the modification - a \emph{sphere of influence}.
Because even if the generation takes \bigo{(\text{\#}(\text{sites}))^2} to generate this cache for a two-site-modification - after generation the lookup becomes practically \bigo{1} (and is faster than computing the neighbor indices, even if all of that has asymptotically constant complexity).

This means effectively always a constant number of indices needs to be taken into account for all calculations (which then can be evaluated in constant time).
While it may be possible to analytically derive a more optimal solution with more canceling terms, this method easily scales.
Only the radius of the pre-computed cache needs to be chosen to be large enough to be compatible with the used order of the expansion.
These geometry-dependent caches are generated in \filepath{\cite{selfCode}}{/computation-scripts/systemgeometry.py} and the caches are used at various places.

The difference that this makes depends on the system size. 
As the number of influenced indices is constant, the strategy gets relatively more effective the more sites the system has (so that a comparably large portion of them is left out).
For a chain, optimizations will be noticed comparably earlier than for the square lattice - this is because in two dimensions the sphere of influence grows as an area and not linearly at the two edges.
But in the limit this strategy allows to keep the targeted complexity class, while scaling to the order of the expansion.
        \FloatBarrier
        

\chapter{Implementation and Calculation Details}
\label{sec:implementation-details}
\section{TODO}
    \begin{frame}[t]
        \frametitle{TODO}
        
        \vspace{-0.5em}
        \begin{itemize}
            \item TODO
        \end{itemize}

        % notes 
        \onslide % on all slides of frame
        \note[item] {
            TODO
        }
    \end{frame}

\FloatBarrier

    \section{Math-Manipulator}
    \cite{compilersDragonBook}

    \section{Programming Style Optimizations}

    \section{Python-Script Generation}

    \section{Numerical Alternatives}

    \section{Simplification Verifications}
    %TODO theoretically this is stated in the introduction of time-complexity optimizations

\chapter{Numerical Experiments}
\label{sec:numerical-experiments}
\section{TODO}
    \begin{frame}[t]
        \frametitle{TODO}
        
        \vspace{-0.5em}
        \begin{itemize}
            \item TODO
        \end{itemize}

        % notes 
        \onslide % on all slides of frame
        \note[item] {
            TODO
        }
    \end{frame}

\FloatBarrier

    \section{Time-Complexity Verification}

    \section{Perturbation-Approximation Convergence}

    \cite{variationalClassicalNetworksPaper}
    % TODO -> time order to where this should converge in VCN for dynamics paper
    % Let us remark that the cumulant expansion (11) goes
    % beyond conventional time-dependent perturbation the-
    % ory, since the corrections considered here effectively ac-
    % count for a resummation of several terms that appear in a
    % standard perturbative expansion31 . However, pCNs face
    % their own limitations, too. In particular, they are inher-
    % ently restricted to weak quantum fluctuations (small γ)
    % to ensure that we can safely truncate the expansion (11).
    % Besides, the description of the evolution of observables
    % will eventually break down, since resonant processes may
    % be present, giving rise to secular terms that limit a correct
    % description to timescales of order O(1/γ)31 . Nonetheless,
    % one can still benefit from the framework introduced here,
    % while mitigating the drawbacks mentioned before. This
    % is achieved by constructing adequate variational wave
    % functions with a network architecture that is inherited
    % from a corresponding pCN, as argued in the following.
    
    \cite{starkManyBodyLocalization} % what we want to observe for electrical field on a discrete lattice: fluctuations instead of transport

    \section{Monte-Carlo-Sampling Convergence}

    \section{Observable-Dependent Noise}

    \section{Improvements through VCN}

    \section{System-Size Dependency of Measurements}

\chapter{Conclusion}
\label{sec:conclusion}
While working on this research subject over the course of my practical training, my project work and finally the master thesis, the material turned out to be more challenging than expected.
Initially, work advanced with steady progress and clear goals.
Because of the seemingly very complicated commutation relations, I spent a lot of time on the development of the Math-Manipulator to streamline similar work.
I had the feeling that several people at the chair would benefit from a tool to validate and improve the calculations that are traditionally done on paper, without resorting to the full adoption of a computer-algebra-system.
Looking back at the calculations eventually needed for the thesis, this tool seems to have been under-utilized.

However, during the development I was able to significantly improve my understanding for the different types of operators, their properties and differences.
This not only directly lead to the calculations now being as straight forward as they are, but also - as my understanding increased - numerous recalculations were required.
In that context, having this consistent tool saved many hours.
So I can confidently say that the time for the side-project was not wasted but well invested.

Consequently, when it came down to practical realization of the mathematical derivations, I spent a lot of time writing extra alternative calculations and programmed checks to verify the correctness of the analytical calculations and numerical implementations.
This lead to many bugs and quirks being found - many of which with only small effects on the result. 
Still, this strategy (as opposed to taking the first \glqq working\grqq{} code) made me highly confident to state that the final implementation is bug-free and correctly reflects the described methods accurately to a high degree.
This would be especially important, as a logical next step would be to produce an even more efficient implementation in a lower-level, compiled language like C(++) or Rust to improve base level performance.

While this thesis and the corresponding code aspires to provide adequate comparisons between the methods and validate their viability, because of the amount of inefficient small operations in Python it lacks the efficiency required for taking on experiments on very large systems.
This shows that the described computational complexity class can be reached and the method can be used to handle the underlying problem - yet there still are a lot of micro-optimizations left to tackle.
Also, several generalizations needed in previous revisions/comparisons could be taken out, if but the final iteration of the method is needed.

Getting all processes to work as intended, turned out to be more challenging than first anticipated.
For one, a problem in two dimensions and non-periodic boundary conditions has a lot more edge-cases than the same method on a 1-dimensional, linear chain with periodic boundary conditions.
When tracking which interactions cancel, the overlapping \glqq spheres of influence\grqq{} produce a significantly higher number of interaction terms in this problem compared to descriptions of previous applications to simpler geometric lattices.

This, combined with two unexpectedly complex road-blocks caused my available time to erode.
Nearing the end of the allocated time-frame, the work was still containing less \glqq Physics\grqq{} (Concrete new insights, obtained by experimentally observing the effects of parameter variations) than we were hoping.
In the words of my supervisor Markus Heyl, in the cases where we can not imagine what is stopping the experiment from behaving as expected: \glqq That's science!\grqq{}.
Now at the end I can proudly state, that this motivated me to finally solve all remaining road-blocks before the deadline.

Which also is the reason I am happy to be able to provide a fully tested and detailed implementation and verification of the hypothesized solution.
To provide the biggest contribution to science, I believe in providing clean final code and transparent plotting process for making numerical methods reproducible and expandable.
And I hope the code can be build upon in future research, that may focus on honing the experiment's parameters and not being held up with debugging while replicating theoretical methods.

\FloatBarrier

% ! Addendum
\newpage
\nocite{*}

{
    \setlength{\emergencystretch}{5em} % should stop urls from going over the right side of the textwidth
    \sloppy                             % looser line-breaking rules
    \setcounter{biburllcpenalty}{9000}  % Try not to break urls at lower letters / in words (preferably more at /:-+= or other signs)

    \printbibliography[title={Bibliography}]
}
\chapter{Appendix}
\label{sec:appendix}

% minted to properly import and style code. ! Needs python libraries

\section{some code} \label{appendix:some-code}
    \filepath{\cite{selfCode}}{/computation-scripts/script.py}
    \inputminted[firstline=5, lastline=30]{python}{./../physics-code/computation-scripts/script.py}
    

\end{document}





