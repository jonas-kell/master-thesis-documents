As stated in \fullref{sec:theory}, some of the analytical calculations involving ladder-operators have been performed with the tool \emph{Math-Manipulator} \cite{selfMathManipulator}.
This tool was purpose-built to play with operators that obey specific commutation-relations in a way which lies in between writing on paper and using a \emph{computer-algebra-system}.
Originally, this started as a passion-project with the purpose of gaining knowledge about \emph{lexing}, \emph{parsing} and the handling of \emph{operator trees} to gain deeper experience with the working of compilers \cite{compilersDragonBook}.
After adding a \LaTeX-renderer and full browser support to minimize the friction of adopting such a new tool into established workflows, it became clear that it could be useful to perform some of the necessary analytical calculations.
Having encountered many exercise-problems with the main focus on ladder-operator commutation-relations over the course of the master curriculum, oftentimes the tediousness of such calculations started to obscure the real lessons of the exercises.
The project tries to establish itself as a low-friction option that fits into manual \glqq on-paper\grqq{} workflows and is yet powerful, extensible and provides reproducible calculations.

Some sections in the practical training report \filepath{\cite{selfDocument}}{/practical-training-latex} are dedicated to outlining the implementation, giving a basic overview of the theoretical backbone and redirecting to further resources.
\filepath{\cite{selfDocument}}{/project-work-presentation} is discussing advantages and disadvantages of classical \glqq on-paper\grqq{} calculation vs. fully strict computer-algebra-systems.
As both of these reports devote their focus to painting a more detailed picture, there will be no in-depth presentation of the side-project to this thesis.
In the case of further interest it is best to follow the linked resources.

As a summarizing statement to gauge the necessity and usefulness of this detour, only the next paragraph should be sufficient.

While there already exist better tools for similar use-cases, the introduction of this new one does not aim to replace, imitate or upstage any of them.
The focus of Math-Manipulator is to position itself as a playground to experiment with the behavior of non-commutative operators and related mathematical objects, while reducing the expense on repetitive manual calculations and providing results that are easier to reproduce.
During the development process and the first-hand usage on an actual scientific problem, it helped making the operator behavior tangible.
Even though the time spent on developing the tool might outweigh the time that would have been necessary for writing down the final version of the calculations, during re-calculations and iterations it already could save many hours.
Furthermore, only having such a consistent iteration cycle made it possible to arrive at the cleanliness of the final presented calculations.
And finally, the stored save-files of the calculations make it possible to share the equations that are not put into the final thesis in a standardized way -- without resorting to inclusion of large and complex parts in the appendix and supplementary material.
