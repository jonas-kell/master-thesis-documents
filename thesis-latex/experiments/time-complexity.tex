At multiple times in the theoretical derivation idealized time-complexities have been stated.
In practice, different hardware-related factors might come into play.
The code to verify the correctness of the simplifications is timed to allow for basic verification of the announced runtime complexities.

\begin{figure}[htbp]
    \centering
    \vspace{-0.7cm}
    \makebox[\textwidth][c]{\includegraphics[width=\textwidth]{plotgeneration/runtime-complexity/runtime.pdf}}
    \caption{
            Plot of a comparison of runtimes for calculating the difference of effective Hamiltonians for an arbitrary amount of randomly generated states for the three different described local modifications. 
            Solid lines represent the approximate time-dependency of the optimized version of the calculation function on the number of sites in the system.
            The dashed lines show the relation for the un-optimized version.
            The graph includes measurements for first (O1) and second order (O2) perturbation theory and for a first order variational classical network (VNC).
        }
    \label{fig:runtime-complexity}
\end{figure}

As shown in \autoref{fig:runtime-complexity}, the solid lines (that represent the runtime of the optimized versions) are approximately constant.
This verifies the functions having a runtime of \bigo{1}. 
For the un-optimized versions (dashed lines) as expected a polynomial dependency can be observed.
In the beginning the curves run steeper, while the influence of the border-cases is important. 
For larger systems the behavior tends to a linear course, which on a logarithmic plot verifies the polynomial time-dependency.

The plot further shows, that for this implementation the double flipping is the most expensive modification, closely followed by swapping.
This is expected, as they are equivalent except for some cases being trivially zero in the case of hopping.
Single flips logically are the cheapest modification.

As the second order implementation is the most complicated and least optimized it is distinctly the slowest.
And because of the variational classical network implementation being more general than the first order perturbation theory, it too has slightly higher runtime.
Lastly, the square geometry always requires more calculation time than the chain one, as for the same number of sites a two-dimensional system has more neighbor interactions than a one-dimensional one.