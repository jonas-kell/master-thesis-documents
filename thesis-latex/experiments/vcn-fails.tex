As previously hinted in the theoretical derivation, this section does not present successful measurements.
For a comparably long time, an unknown error was preventing the suggested variational classical networks algorithm from working.
The problem lay, as previously revealed, in the explicit time-dependency of the base-energy that hade not been properly dealt with.
Before coming back to successful verifications of the described methods, this segment is meant to briefly outline the discovery process that led to finding the reason of the misbehavior.

In \autoref{fig:vcn-fail-energy-variance} an energy and variance measurement is presented. 
The plot and experiment do thereby not differ from the ones from previous sections, except for the inclusion of VCN measurements in comparison to the established perturbative effective Hamiltonians.
In the appendix at \ref{appendix:vcn-fail-parameters}, the VCN-parameters and their analytical comparisons (as described in \fullref{sec:theory-variational-classical-networks-application}) are plotted.

\begin{figure}[htbp]
    \centering
    \vspace{-0.2cm}
    \makebox[\textwidth][c]{\includegraphics[width=\textwidth]{plotgeneration/vcn-eff-stepsize/energy-variance.pdf}}
    \caption{
            Plot of energy and variance of a system with chain geometry and $n=6$ lattice sites.
            The energy plot additionally contains the imaginary part of the observed quantities, which is depicted by the circle-symbol.
            For the imaginary parts, the right y-axis is used.
            Four different VCN-series are plotted. These differ by the effective step size, describing the magnitude of the time-step for numerical integration.
            Exact sampling was used, to restrict possible problems to the effective Hamiltonian without having to deal with Monte-Carlo noise.
        }
    \label{fig:vcn-fail-energy-variance}
\end{figure}

At the start of the debugging-process, a relatively large remaining imaginary part kept appearing for the measurements of all observables, as long as an effective Hamiltonian parametrized by VCN was used.
However, this problem could be isolated to a sub-class in the implementation, that was incorrectly missing an overwrite for one of the methods of its parent class.
The imaginary part that was remaining after this had been fixed, is sufficiently small to be attributed to numerical noise, as studying \autoref{fig:vcn-fail-energy-variance} reveals.

Yet, the resulting architecture is still not able to keep the energy nor variance constant - which is an inherent property of the theoretical model.
One could hypothesize, that the problem lies in a numerical integration-step that is too large.
For this reason, the experiment varies the effective step size that is used for integration.
While this results in changes to the resulting values of the observables, it does not result in differences of the ability of the parametrization to conserve the energy.
A convergence can be spotted, yet definitely no convergence to a constant energy.

As a second hypothesis, the problem was suspected to be related to the initialization. 
Therefore the experiment in \autoref{fig:vcn-fail-starting-energy-variance} also tries to compute the energy and variance, but with purposefully large and different initial values for the variational parameters.
In this case, the corresponding variational parameters are more interesting and can be found in the appendix at \ref{appendix:vcn-fail-starting-parameters}.

The misbehavior of the model was already appearing on the very first time-step.
Therefore it could be assumed, that maybe an unlucky choice of initial values was causing the divergence of the VCN-parameters (compare appendix, \ref{appendix:vcn-fail-starting-parameters}).
However, it can be quickly spotted that neither the energy, nor the variance, nor the variational parameters seem to have a different course.
All the trends seem to have just been offset by the initial value.

\begin{figure}[htbp]
    \centering
    \vspace{-0.2cm}
    \makebox[\textwidth][c]{\includegraphics[width=\textwidth]{plotgeneration/vcn-starting-values/energy-variance.pdf}}
    \caption{
            Plot of energy and variance of a system with chain geometry and $n=8$ lattice sites.
            Each run was performed with a different seed and the spread of the initial parameter values was purposefully set very high.
            Exact sampling has been used.
        }
    \label{fig:vcn-fail-starting-energy-variance}
\end{figure}

Consequently, this could only be explained in one way: the development of the parameters in this parametrization is independent of the parameters themself.
This however eliminates the \HNOft[N]-part from being responsible and only the explicitly time-dependent base-energy part remains as a possible candidate.
A repeated inspection of \autoref{fig:vcn-fail-energy-variance} supports this hypothesis, as the basic shape of the energy seems to strongly correlate with the shape of the series of zeroth order energy measurements.

As already revealed in the theoretical derivation section, eliminating the explicit time-dependency from the base-energy had been the necessary step to allow the model to perform its duty properly.