Giving absolute statements about the quality of a numerical method is naturally very difficult.
All approximations by definition omit some aspects in comparison to the exact result.
This however is the exact purpose of a model - simplifying by leaving out parts, but still capturing the relevant characteristics.
In conclusion, there are two ways of gauging the quality of a model: comparing to other models and checking if the approximation holds in the limit to a more general case.
Both of these checks will be employed to some degree in this or following sections.

To begin with, it should be repeated that a perturbative expansion is designed to capture the influence of a \glqq small\grqq{} perturbation to an underlying system.
The cumulant expansion, that was presented in \fullref{sec:theory-physics}, is based on the interaction energy $J$ being small compared to the energy scale of the base energy ($U$ and $\epsl[\ast]$).
In the limit of $J$ going to zero, the errors resulting from the expansion are expected to also become smaller.
It should be the goal, to show that the expansion is \emph{controlled} - meaning for a fixed $J$, a large enough expansion-order can be chosen, or equivalently, each order of the expansion is good enough to make statements for $J$ up to a specific magnitude in relation to $U$. 

For the first verification, therefore the \autoref{fig:j-sweep-single-occ-center} shows a family of measurements that were performed for different magnitudes of the parameter $J$.

\begin{figure}[htbp]
    \centering
    \vspace{-0.7cm}
    \makebox[\textwidth][c]{\includegraphics[width=\textwidth]{plotgeneration/j-sweep/single-occ-center.pdf}}
    \caption{
            Family of measurements of the observable that describes the occupation of a single site and spin direction near the center of a system with \emph{chain} geometry and $n=6$ sites.
            The magnitude of the parameter $J$ is varied in relation to the parameter $U$, while the remaining simulation-settings are kept constant.
            On the y-axis, the relative error (in relation to the measurement taken from exact diagonalization) is shown logarithmically scaled.
            The measurement was taken with exact sampling.
        }
    \label{fig:j-sweep-single-occ-center}
\end{figure}

To balance the amount of space that is taken up by the numerous plots in the subsequent sections, supplementary plots without a direct reference ore less importance are added in the end of the thesis.
In this case, three more plots (showing different observables but otherwise from the same experimental dataset) are included in the appendix at \ref{appendix:more-jsweep-plots}.

The measurements show the expected tendencies: The relative error is the smallest for times close to $t=0$. For larger times the relative error continuously grows.
In the start the perturbative measurement and the exact measurement are identical.
Then, for later times, the relative error constantly grows.
This is expected, as the perfect alignment from $t=0$ is constantly worsening due to errors, resulting from the truncated expansion, that accumulate in the perturbative calculation until the relative error is maximal at the point of total de-phasing.
The reference \cite{variationalClassicalNetworksPaper} suggests that the approximation on the basis of the cumulant expansion provides a correct description for timescales on the order of \bigo{J^{-1}}.
No quantitative analysis of that statement has been carried out, yet the suggested dependency seems to be in-line with the trend seen in the data.

Furthermore, the correct behavior for a limit of a small perturbative parameter $J$ can be also attested.
The data shows a clear downwards-shift to smaller relative errors for the whole measurement series, the closer $J/\,U$ comes to zero.
At the limit $J=0$, the error should also vanish, as there is no perturbation left in the system.

Viewed from the opposite side, it must be stressed that the expansion only is viable to predict the behavior of systems with weak quantum fluctuations. 
Were the interactions too large, the expansion could not be safely truncated after few terms \cite{variationalClassicalNetworksPaper}.

\begin{figure}[htbp]
    \centering
    \makebox[\textwidth][c]{\includegraphics[width=\textwidth]{plotgeneration/energy-variance/energy.pdf}}
    \caption{
            Plot of the course of the energy over time, for the case of a system with chain geometry and $n=4$ sites.
            The exact energy has been calculated with exact diagonalization for all times (real measurements).
            Both plots show the same data, but the zeroth order approximation has been omitted from the lower graph in order to better visualize the differences between first and second order.
            The sampling strategy for the perturbative approximations has been exact sampling.
        }
    \label{fig:energy-variance-energy}
\end{figure}
    
As stated by \autoref{eq:energy-conserved}, the energy and variance stay constant for exact measurements of the system.
It is therefore a measurement of the quality of a perturbative method, how good these two observables are being conserved over time.
\autoref{fig:energy-variance-energy} shows the course of the energy measurement, while \autoref{fig:energy-variance-variance} shows the corresponding values for the variance.

\begin{figure}[htbp]
    \centering
    \makebox[\textwidth][c]{\includegraphics[width=\textwidth]{plotgeneration/energy-variance/variance.pdf}}
    \caption{
            Plot of the variance measurements corresponding to the energy from \autoref{fig:energy-variance-energy}.
            Again, both plots show the same data, only the second graph omits the zeroth order and presents the differences between first and second order in a zoomed-in fashion.
            The measurements in the lower graph consist still only of individual data points, but the plot here is done with lines to allow to differentiate the curves in the beginning section.
        }
    \label{fig:energy-variance-variance}
\end{figure}

The conservation of energy and variance is perfectly validated by the measurements from exact diagonalization (here in blue).
From this point on, measurements will be labeled as having \glqq exact sampling\grqq, if for the calculations all possible base-states were examined and their normalized probabilities were calculated.
The chosen sampling strategy is independent of the used effective Hamiltonian. 

In this case, for the blue curves the effective Hamiltonian has been calculated by exact diagonalization and the sampling has been exact - making the full measurement having no approximations.
Because of the two spin-degrees per site and the exponential nature of the number of base states, a system of size $n=4$ is already close to the limit that can be reasonably diagonalized exactly - at least with the presented setup.
By applying more optimizations to the exact diagonalization, a few more sites can be considered, yet in this case the focus will be put on optimizing the perturbative approach.

The three degrees of perturbative measurements, while having exact sampling, rely on an effective Hamiltonian that is obtained from the truncated cumulant expansion - making the methods approximations and not exact.
Effects of additionally replacing the sampling strategy with non-exact ones will be investigated in following sections.

First of all, it is easily seen that the zeroth order perturbative effective Hamiltonian does not conserve the energy or variance at all.
While for the variance the center of the oscillation stays around the exact value and for the energy it at least comes back to it in the end of the measurement, the observables clearly instantly deviate from the energy at $t=0$ starting at the first time-step.
This was to be expected, as the energy-measurement takes the interactions (mediated by the interaction strength $J$) into account, but the effective Hamiltonian is truncated to not have any knowledge of $J$.
Yet, the zeroth order curves can be taken as a reference of scale: while all approximations can have fluctuations, no approximation should have larger fluctuations than this trivial case.

Inspecting the effects of the proposed approximation, represented by the green and red curves, the consistency is significantly better than for zeroth order.
Both methods keep the observables far more constant than the zeroth order experiment and all oscillations are now around the values they are supposed to have in the exact case.
For the variance, the second order calculations seem to be clearly closer to the expected solution.
In the case of the energy, for the times smaller 1 (in the presented units) a bigger gap can be seen between the second order calculation and the exact result than between the first order and the exact value.

It is always necessary to look at the overall energy scale and check if this might be expected or the result of a mistake.
A general rule of thumb for this approximation would be to state that an approximation from cumulant expansion to order $x$ should have an error in the order of \bigo{\frac{J^2}{U^2}}. 
For the presented measurement this is generally fulfilled. 
So both approximations seem to be correctly derived and implemented and should be capable of calculating helpful results.
If it was worth deriving the second order effective Hamiltonian will be examined in a third experiment for this section.

Lastly, a more involved observable will be calculated.
As \fullref{sec:theory-observables-density-matrix} describes, the reduced density matrix of a small subsystem can be extracted only from the measurement of local observables.
This process was performed for a system with chain geometry and $n=4$ (to allow still having access to the exact diagonalization for the purpose of comparison).
In \autoref{fig:concurrence-first-experiment} and \autoref{fig:purity-first-experiment}, the concurrence and purity of such a density matrix of two adjacent spin-up particles are displayed.
The values of the individual observables for the Pauli operator basis for the same subsystem can be found in the appendix at \ref{appendix:pauli-measurements}.


\begin{figure}[htbp]
    \centering
    \vspace{-0.3cm}
    \makebox[\textwidth][c]{\includegraphics[width=\textwidth]{plotgeneration/concurrence-from-spin/concurrence-comparison.pdf}}
    \caption{
            Concurrence measurements extracted from the reduced density matrix of a subsystem of two adjacent spin-up particles in a system with chain geometry and $n=4$ sites.
            The results from exact diagonalization have been calculated by two independent means. 
            Once by replacing the effective Hamiltonian with one gained from exact diagonalization, once by calculating the full time-evolution of the complete density matrix and tracing out the required subsystem (here labeled as \glqq External\grqq).
            This is used for verification. 
            As the perturbative algorithm does not use the full density matrix anywhere, this needed to be implemented independently.
            The measurement uses exact sampling.
        }
    \label{fig:concurrence-first-experiment}
\end{figure}


\begin{figure}[htbp]
    \centering
    \vspace{-0.3cm}
    \makebox[\textwidth][c]{\includegraphics[width=\textwidth]{plotgeneration/concurrence-from-spin/purity-comparison.pdf}}
    \caption{
            Purity measurement, belonging to the experiment in \autoref{fig:concurrence-first-experiment}.
        }
    \label{fig:purity-first-experiment}
\end{figure}

To start off the analysis, it should be noted how the external measurements and the ones that reconstruct the density matrix from observables (albeit the values being obtained by exact diagonalization and exact sampling) match exactly.
This verifies the described method of obtaining a reduced density matrix without access to the full density matrix.

The concurrence measurements now clearly show the superiority of the second order cumulant expansion.
Position and shape of the second peak are clearly replicated much more precise by the second order calculations, than by the first order ones.
It can also be noted, that while the purity measurement for the zeroth order is not totally off, the zeroth order calculation completely fails to replicate the features which are necessary to calculate the concurrence.
Not even the first peak (which also the first order replicates perfectly) can be reproduced while the interaction $J$ is ignored.

Looking at the observables from which this density matrix was obtained (see appendix, \ref{appendix:pauli-measurements}), it can be clearly seen that the observables corresponding to Pauli-z seem to be the most difficult to replicate.
A Pauli-z measurement is equivalent to a re-scaled measurement of a single occupation.
