Giving absolute statements about the quality of a numerical method is naturally very difficult.
All approximations by definition omit some aspects in comparison to the exact result.
This however is the exact purpose of a model - simplifying by leaving out parts, but still capturing the relevant characteristics.
In conclusion, there are two ways of gauging the quality of a model: comparing to other models and checking if the approximation holds in the limit to a more general case.
Both of these checks will be employed to some degree in this or following sections.

To begin with, it should be repeated that a perturbative expansion is designed to capture the influence of a \glqq small\grqq{} perturbation to an underlying system.
The cumulant expansion, that was presented in \fullref{sec:theory-physics}, is based on the interaction energy $J$ being small compared to the energy scale of the base energy ($U$ and $\epsl[\ast]$).
In the limit of $J$ going to zero, the errors resulting from the expansion are expected to also become smaller.
It should be the goal, to show that the expansion is \emph{controlled} - meaning for a fixed $J$, a large enough expansion-order can be chosen, or equivalently, each order of the expansion is good enough to make statements for $J$ up to a specific magnitude in relation to $U$. 

For the first verification, therefore the \autoref{fig:j-sweep-single-occ-center} shows a family of measurements that were performed for different magnitudes of the parameter $J$.

\begin{figure}[htbp]
    \centering
    \vspace{-0.7cm}
    \makebox[\textwidth][c]{\includegraphics[width=\textwidth]{plotgeneration/j-sweep/single-occ-center.pdf}}
    \caption{
            Family of measurements of the observable that describes the occupation of a single site and spin direction near the center of a system with \emph{chain} geometry and $n=6$ sites.
            The magnitude of the parameter $J$ is varied in relation to the parameter $U$, while the remaining simulation-settings are kept constant.
            On the y-axis, the relative error (in relation to the measurement taken from exact diagonalization) is shown logarithmically scaled.
        }
    \label{fig:j-sweep-single-occ-center}
\end{figure}

To balance the amount of space that is taken up by the numerous plots in the subsequent sections, supplementary plots without a direct reference are added in the end of the thesis.
In this case, three more plots (showing different observables but otherwise from the same experimental dataset) are included in \ref{appendix:more-jsweep-plots}.

The measurements show the expected tendencies: The relative error is the smallest for times close to $t=0$. For larger times, the relative error continuously grows.
In the start, the perturbative measurement and the exact measurement are identical.
Then, for later times, the relative error constantly grows.
This is expected, as the perfect alignment from $t=0$ is constantly worsening due to errors, resulting from the truncated expansion, that accumulate in the perturbative calculation.
The reference \cite{variationalClassicalNetworksPaper} suggests that the approximation on the basis of the cumulant expansion provides a correct description for timescales on the order of \bigo{J^{-1}}.
No quantitative analysis of that statement has been carried out, yet the suggested dependency seems to be in-line with the trend seen in the data.

Furthermore, the correct behavior for a limit of a small perturbative parameter $J$ can be also attested.
The data shows a clear downwards-shift to smaller relative errors for the whole measurement series, the closer $J/\,U$ comes to zero.
At the limit $J=0$, the error should also vanish, as the is no perturbation left in the system.

Viewed from the opposite side, it must be stressed that the expansion only is viable to predict the behavior of systems with weak quantum fluctuations. 
Were the interactions too large, the expansion could not be safely truncated after few terms \cite{variationalClassicalNetworksPaper}.



% energy variance
% concurrence-from-spin