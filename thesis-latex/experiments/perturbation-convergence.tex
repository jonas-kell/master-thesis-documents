
\cite{variationalClassicalNetworksPaper}
% TODO -> time order to where this should converge in VCN for dynamics paper
% Let us remark that the cumulant expansion (11) goes
% beyond conventional time-dependent perturbation the-
% ory, since the corrections considered here effectively ac-
% count for a resummation of several terms that appear in a
% standard perturbative expansion31 . However, pCNs face
% their own limitations, too. In particular, they are inher-
% ently restricted to weak quantum fluctuations (small γ)
% to ensure that we can safely truncate the expansion (11).
% Besides, the description of the evolution of observables
% will eventually break down, since resonant processes may
% be present, giving rise to secular terms that limit a correct
% description to timescales of order O(1/γ)31 . Nonetheless,
% one can still benefit from the framework introduced here,
% while mitigating the drawbacks mentioned before. This
% is achieved by constructing adequate variational wave
% functions with a network architecture that is inherited
% from a corresponding pCN, as argued in the following.

\cite{starkManyBodyLocalization} % what we want to observe for electrical field on a discrete lattice: fluctuations instead of transport

% energy variance
% j-sweep 
% concurrence-from-spin