So far, only exact sampling has been used. 
However, as illustrated in \fullref{sec:theory-optimizations-monte-carlo}, for systems with too many base states it is necessary to switch to a sampling method that only looks at a subset of these states.
An attempt to verify the convergence of the Monte-Carlo sampling strategy was made in \autoref{fig:mc-convergence-occupation}.
One additional measurement for a different observable is included in the appendix at \ref{appendix:mc-convergence-spin-current}.

\begin{figure}[htbp]
    \centering
    \vspace{-0.2cm}
    \makebox[\textwidth][c]{\includegraphics[width=\textwidth]{plotgeneration/monte-carlo-variance-test/occupation.pdf}}
    \caption{
            Standard deviation of occupation measurements, obtained from calculations on the same system and parameterization as in the previous section, but with Monte-Carlo sampling instead of exact sampling.
            Ten runs of the same experiment, with different seeds, have been calculated. 
            This resulted in ten slightly different datasets between which the standard deviation could be calculated.
            The five series differ in the number of samples which have been taken into account for each single observable calculation.
        }
    \label{fig:mc-convergence-occupation}
\end{figure}

Upon inspection of the trend of the standard deviation in relation to the number of states sampled by Monte-Carlo, a clear correspondence can be spotted.
For both measured observables, a higher number of Monte-Carlo samples directly corresponds to a lower standard deviation.
This behavior was expected, as theoretically for an infinite number of samples it should be possible to perfectly replicate the original probability distribution.
Taking this experiment as a reference, future Monte-Carlo sampling will be performed with \num{30000} samples as a default setting.
While this may be much more than would have been needed to sample this small test-system exactly, it is an incredible reduction for systems with $n \approx 100$ sites.

Finally, upon closely inspecting \autoref{fig:mc-convergence-occupation} again, one can spot that the overall spread of the single occupation is considerably higher than for the double occupation.
This was already noted at the end of the previous section, where the statement was presented how the measurement of single occupation (corresponding to Pauli-z measurements) seem to be harder to reproduce than the other Pauli observables.