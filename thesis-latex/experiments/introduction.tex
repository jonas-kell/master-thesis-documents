The last section of this thesis will present the numerical results that have been generated per the previously described methods.
All experiments and their exact configurations are stored in 
\filepath{\cite{selfCode}}{/calculation-helpers/aggregator.py}.
Because of that, it is possible to fully reproduce the measurements, identically to how everything is presented here.
This includes the elements that depend on randomness, as all random generators are consistently seeded and the seeds are included in the logs.

The code is implemented to be executed only on CPU, so running it requires no setup like configuring graphics cards - just having an up to date version of Python and some basic packages.
However, while small-scale tests will run fine on nearly every machine, more involved calculations require a lot of processing power.
Most of the computational work can be sped up by utilizing multiple CPU cores in parallel and the scripts support multithreading for all expensive operations.
In the case of this work, all calculations have been run using high performance computing (HPC) resources on the Linux Compute Cluster Augsburg (LiCCA).
Access to LiCCA for the scope of this thesis work has been graciously granted by the University of Augsburg.

To run experiments of this codebase on HPC, documentation and control-scripts are included in \filepath{\cite{selfCode}}{/hpc-augsburg/}.
In order to share the results produced by HPC-measurements, data files in a uniform zip format are produced.

In the spirit of making research as transparent and reproducible as possible, the repository \filepath{\cite{selfDocument}}{/thesis-latex/plotgeneration/} includes all scripts to generate every plot directly from the source data.
On compilation of the \LaTeX{} source the graphs get rendered, too.
