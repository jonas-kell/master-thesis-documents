The energy \energyOft (\autoref{eq:energy-main}) is a key measure of many quantum-mechanical experiments, with the observable being the hamiltonian that governs the time-evolution.
The variance \varianceOft of this quantity (\autoref{eq:variance-main}) describes the energy-fluctuations in the system.

\begin{equation}
    \label{eq:energy-main}
    \energyOft = \frac{\bracketHelper{\psiOfT[\schroedingerPicture]}{\pictureHamiltonian[\schroedingerPicture]}{\psiOfT[\schroedingerPicture]}}{\braketHelper{\psiOfT[\schroedingerPicture]}{\psiOfT[\schroedingerPicture]}}
       = \timeExpectationVal{\pictureHamiltonian[\schroedingerPicture]}
\end{equation}
\begin{equation}
    \label{eq:variance-main}
    \varianceOft = \frac{\bracketHelper{\psiOfT[\schroedingerPicture]}{{\pictureHamiltonian[\schroedingerPicture]}^2}{\psiOfT[\schroedingerPicture]}}{\braketHelper{\psiOfT[\schroedingerPicture]}{\psiOfT[\schroedingerPicture]}}-\left(\frac{\bracketHelper{\psiOfT[\schroedingerPicture]}{\pictureHamiltonian[\schroedingerPicture]}{\psiOfT[\schroedingerPicture]}}{\braketHelper{\psiOfT[\schroedingerPicture]}{\psiOfT[\schroedingerPicture]}}\right)^2
       = \timeExpectationVal{{\pictureHamiltonian[\schroedingerPicture]}^2} - {\timeExpectationVal{\pictureHamiltonian[\schroedingerPicture]}}^2
\end{equation}

Without any additional constrictions %(like e.g. conservation of particle number) 
the energy - as well as all other cumulants of the hamiltonian operator - must be conserved under normal time evolution, as \autoref{eq:energy-conserved} shows. 
This comes from the fact, that an exact hamiltonian commutes with itself in all instances and allows to show that the energy $\energyOft = \energyOft[t=0]$.
With equivalent reasoning, also the higher order cumulants must necessarily be constant for all times.
The energy as first cumulant % https://en.wikipedia.org/wiki/Cumulant (yes, energy is the first, not the zeroth) 
being 0 or constant comes down to the choice of energy reference.

\begin{equation}
    \label{eq:energy-conserved}
    \begin{split}
        \energyOft &\stackrel{\ref{eq:energy-main}}{=}
        \frac{\bracketHelper{\psiOfT[\schroedingerPicture]}{\pictureHamiltonian[\schroedingerPicture]}{\psiOfT[\schroedingerPicture]}}{\braketHelper{\psiOfT[\schroedingerPicture]}{\psiOfT[\schroedingerPicture]}}
        \stackrel{\ref{eq:time-evolution-schroedinger}}{=}
        \frac{\bracketHelper{\picturePsi[\schroedingerPicture]{}}{e^{i \pictureHamiltonian[\schroedingerPicture] t}{\pictureHamiltonian[\schroedingerPicture]}e^{-i \pictureHamiltonian[\schroedingerPicture] t}}{\picturePsi[\schroedingerPicture]{}}}{\bracketHelper{\picturePsi[\schroedingerPicture]{}}{e^{i \pictureHamiltonian[\schroedingerPicture] t}e^{-i \pictureHamiltonian[\schroedingerPicture] t}}{\picturePsi[\schroedingerPicture]{}}}\\
        &\stackrel{\phantom{\ref{eq:time-evolution-schroedinger}}}{=}
        \frac{\bracketHelper{\picturePsi[\schroedingerPicture]{}}{e^{i \pictureHamiltonian[\schroedingerPicture] t}e^{-i \pictureHamiltonian[\schroedingerPicture] t}{\pictureHamiltonian[\schroedingerPicture]}}{\picturePsi[\schroedingerPicture]{}}}{\bracketHelper{\picturePsi[\schroedingerPicture]{}}{e^{i \pictureHamiltonian[\schroedingerPicture] t}e^{-i \pictureHamiltonian[\schroedingerPicture] t}}{\picturePsi[\schroedingerPicture]{}}} = 
        \frac{\bracketHelper{\picturePsi[\schroedingerPicture]{}}{\pictureHamiltonian[\schroedingerPicture]}{\picturePsi[\schroedingerPicture]{}}}{\braketHelper{\picturePsi[\schroedingerPicture]{}}{\picturePsi[\schroedingerPicture]{}}}
         = \energyOft[t=0]
    \end{split}
\end{equation}

The higher order cumulants vanish, depending on the initial state. 
\autoref{eq:variance-constant} shows, the variance calculation expanded in the eigenbasis of the hamiltonian.
For initial states that are comprised of only one element of that basis $\ketN[{\picturePsi[\schroedingerPicture]{}}] \propto \ketN[E_i],\,\,\pictureHamiltonian[\schroedingerPicture]\ketN[E_i] = E_i\cdot\ketN[E_i]$, the higher order cumulants vanish. 
Looking at the last statement, in \autoref{eq:variance-constant}, in the case that all $c(E_i) = 0$ except for one specific $i$, the difference must vanish.
Otherwise the variance is non-zero, but still constant in time, because of the argument in \autoref{eq:energy-conserved}.

\begin{equation}
    \label{eq:variance-constant}
    \begin{split}
        \varianceOft[t=0] &\stackrel{\ref{eq:variance-main}}{=} \frac{\bracketHelper{\picturePsi[\schroedingerPicture]{}}{{\pictureHamiltonian[\schroedingerPicture]}^2}{\picturePsi[\schroedingerPicture]{}}}{\braketHelper{\picturePsi[\schroedingerPicture]{}}{\picturePsi[\schroedingerPicture]{}}}-\left(\frac{\bracketHelper{\picturePsi[\schroedingerPicture]{}}{\pictureHamiltonian[\schroedingerPicture]}{\picturePsi[\schroedingerPicture]{}}}{\braketHelper{\picturePsi[\schroedingerPicture]{}}{\picturePsi[\schroedingerPicture]{}}}\right)^2
        \\ &\stackrel{\phantom{\ref{eq:variance-main}}}{=} 
        \lsum[i]
        \lsum[j]
        \frac{\bracketHelper{E_i}{c^\ast(E_i)c(E_j){\pictureHamiltonian[\schroedingerPicture]}^2}{E_j}}{\braketHelper{\picturePsi[\schroedingerPicture]{}}{\picturePsi[\schroedingerPicture]{}}}
        -\left(
            \lsum[i]
            \lsum[j]
            \frac{\bracketHelper{E_i}{c^\ast(E_i)c(E_j)\pictureHamiltonian[\schroedingerPicture]}{E_j}}{\braketHelper{\picturePsi[\schroedingerPicture]{}}{\picturePsi[\schroedingerPicture]{}}}
        \right)^2
        \\ &\stackrel{\phantom{\ref{eq:variance-main}}}{=} 
        \frac{
            \lsum[i]
            \absSquare{c(E_i)}{E_i}^2}{\braketHelper{\picturePsi[\schroedingerPicture]{}}{\picturePsi[\schroedingerPicture]{}}}
        -\left(
            \lsum[i]
            \frac{\absSquare{c(E_i)}E_i}{\braketHelper{\picturePsi[\schroedingerPicture]{}}{\picturePsi[\schroedingerPicture]{}}}
        \right)^2
        =
        \frac{
            \lsum[i]
            \absSquare{c(E_i)}{E_i}^2 
            -\left(
                \lsum[i]
                \absSquare{c(E_i)}E_i
            \right)^2
        }{\braketHelper{\picturePsi[\schroedingerPicture]{}}{\picturePsi[\schroedingerPicture]{}}}
    \end{split}
\end{equation}

While it might not seem to give much insight to calculate the time evolved measurements of these values if they are constant, in reality this serves the purpose of \emph{checking} the accuracy of potential approximations.
If the time evolution is performed with an approximated, effective hamiltonian, then the exact one will no longer commute with it and the energy and variance will vary over time.

Measuring \localObservable{N}{t} for $\ObservableOp = \pictureHamiltonian[\schroedingerPicture]$ is straight forward, as \autoref{eq:energy-local-observable} shows.

\begin{equation}
    \label{eq:energy-local-observable}
    \begin{split}
        \localObservable{N}{t} &\stackrel{\ref{eq:expectation-value}}{=}
        \lsum[K] \bracketHelper{N}{\pictureHamiltonian[\schroedingerPicture]}{K} e^{\HeffOft[K]-\HeffOft[N]}
        \frac{
            \psiN[K]
        }{
            \psiN[N]
        } \\
        &\stackrel{\ref{eq:main-hamiltonian-h0},\,\ref{eq:main-hamiltonian-perturbation}}{=}
        %
        \lsum[K] \bracketHelper{N}{U \cdot \lsum \nop{l}{\up}\nop{l}{\down} + \lsum[l,\,\sigma] \epsl \nop{l}{\sigma}}{K} e^{\HeffOft[K]-\HeffOft[N]}
        \frac{
            \psiN[K]
        }{
            \psiN[N]
        }
        %
        \\&\stackrel{\phantom{\ref{eq:base-energy}}}{-}
        %
        J \cdot 
        \lsum[K] \bracketHelper{N}{\neighborsumWSpin{l}{m}{\sigma} \left(\withspinhcop{l}{\sigma}{\dagger} \withspinhcop{m}{\sigma}{} + \withspinhcop{m}{\sigma}{\dagger} \withspinhcop{l}{\sigma}{} \right)}{K} e^{\HeffOft[K]-\HeffOft[N]}
        \frac{
            \psiN[K]
        }{
            \psiN[N]
        }
        \\&\stackrel{\phantom{\ref{eq:base-energy}}}{=}
        %
        \lsum \lsum[K] \bracketHelper{N}{U \cdot  \nop{l}{\up}\nop{l}{\down} + \epsl \left(\nop{l}{\up} + \nop{l}{\down}\right)}{K} e^{\HeffOft[K]-\HeffOft[N]}
        \frac{
            \psiN[K]
        }{
            \psiN[N]
        }
        %
        \\&\stackrel{\phantom{\ref{eq:base-energy}}}{-}
        %
        J \cdot \neighborsumWSpin{l}{m}{\sigma}
        \lsum[K] \bracketHelper{N}{ \withspinhcop{l}{\sigma}{\dagger} \withspinhcop{m}{\sigma}{} + \withspinhcop{m}{\sigma}{\dagger} \withspinhcop{l}{\sigma}{} }{K} e^{\HeffOft[K]-\HeffOft[N]}
        \frac{
            \psiN[K]
        }{
            \psiN[N]
        }
        \\&\stackrel{\phantom{\ref{eq:base-energy}}}{=}
        %
        \lsum \left(U \cdot  n_{l,\,\up}n_{l,\,\down} + \epsl \left(n_{l,\,\up} + n_{l,\,\down}\right)\right)
        %
        \\&\stackrel{\phantom{\ref{eq:base-energy}}}{-}
        %
        J \cdot \neighborsumWSpin{l}{m}{\sigma}
        \lsum[K] \bracketHelper{N}{ \withspinhcop{l}{\sigma}{\dagger} \withspinhcop{m}{\sigma}{} + \withspinhcop{m}{\sigma}{\dagger} \withspinhcop{l}{\sigma}{} }{K} e^{\HeffOft[K]-\HeffOft[N]}
        \frac{
            \psiN[K]
        }{
            \psiN[N]
        }\\
        %
        &\stackrel{\ref{eq:base-energy}}{=} E_0(N,\,t)-
        %
        J \cdot \neighborsumWSpin{l}{m}{\sigma}
        \left(n_{l,\,\sigma} (1- n_{m,\,\sigma}) + (1-n_{l,\,\sigma}) n_{m,\,\sigma}\right)
         e^{\HeffOft[\vphantom{N}\smash{\tilde{N}}]-\HeffOft[N]}
        \frac{
            \psiN[\vphantom{N}\smash{\tilde{N}}]
        }{
            \psiN[N]
        }\\
        %
        &\stackrel{\phantom{\ref{eq:base-energy}}}{=} E_0(N,\,t)-
        %
        J \cdot \neighborsumWSpin{l}{m}{\sigma}
        \left(n_{l,\,\sigma} \neq n_{m,\,\sigma}\right)
         e^{\HeffOft[\vphantom{N}\smash{\tilde{N}}]-\HeffOft[N]}
        \frac{
            \psiN[\vphantom{N}\smash{\tilde{N}}]
        }{
            \psiN[N]
        }\\
        %
        &\stackrel{\phantom{\ref{eq:base-energy}}}{=} E_0(N,\,t) + E_\text{V}(N,\,t)
    \end{split}
\end{equation}

\ketN[\vphantom{N}\smash{\tilde{N}}] is here the state that is obtained, when starting on state \ketN and performing a swapping of the occupations of sites $l,\,\sigma$ and $m,\,\sigma$.
On comparison with \autoref{eq:base-energy} one can see, that the definition for $E_0(N,\,t)$ is consistent for the expansion and the observable calculation.
Newly defined is the energy $E_\text{V}(N,\,t)$, originating from \Vhamiltonian[].
As hinted for the other observables earlier, the difference of hamiltonians with local modifications $\HeffOft[\vphantom{N}\smash{\tilde{N}}]-\HeffOft[N]$ can be evaluated in constant time.
However the energy is the first observable that requires \bigo{\text{\#}(\text{sites})} on its own to evaluate. This can not be reduced, however as will become apparent in the calculation of the variance, this should be an upper bound for observable-calculations.

The variance can be evaluated by splitting the expectation values for the parts \HzeroHamiltonian[\schroedingerPicture] and \Vhamiltonian[\schroedingerPicture], using the \emph{covariance} as presented in \autoref{eq:variance-split}.

\begin{equation}
    \label{eq:variance-split}
    \begin{split}
        \varianceOft &\stackrel{\ref{eq:variance-main}}{=} \timeExpectationVal{{\pictureHamiltonian[\schroedingerPicture]}^2} - {\timeExpectationVal{\pictureHamiltonian[\schroedingerPicture]}}^2
        \stackrel{\ref{eq:main-hamiltonian}}{=} \timeExpectationVal{\left(\HzeroHamiltonian[\schroedingerPicture] + \Vhamiltonian[\schroedingerPicture]\right)^2} - {\timeExpectationVal{\HzeroHamiltonian[\schroedingerPicture] + \Vhamiltonian[\schroedingerPicture]}}^2
        \\&\stackrel{\phantom{\ref{eq:main-hamiltonian}}}{=}
        \timeExpectationVal{{\HzeroHamiltonian[\schroedingerPicture]}^2} - {\timeExpectationVal{\HzeroHamiltonian[\schroedingerPicture]}}^2
        +
        \timeExpectationVal{{\Vhamiltonian[\schroedingerPicture]}^2} - {\timeExpectationVal{\Vhamiltonian[\schroedingerPicture]}}^2
        \\&\stackrel{\phantom{\ref{eq:main-hamiltonian}}}{+}
        \timeExpectationVal{\HzeroHamiltonian[\schroedingerPicture]\Vhamiltonian[\schroedingerPicture] + \Vhamiltonian[\schroedingerPicture]\HzeroHamiltonian[\schroedingerPicture]}
        -        2\cdot 
        \timeExpectationVal{\HzeroHamiltonian[\schroedingerPicture]}\cdot \timeExpectationVal{\Vhamiltonian[\schroedingerPicture]}
    \end{split}
\end{equation}

It is important to notice, that only $\HzeroHamiltonian[\schroedingerPicture]\Vhamiltonian[\schroedingerPicture] + \Vhamiltonian[\schroedingerPicture]\HzeroHamiltonian[\schroedingerPicture]$ is a hermitian operator, $\HzeroHamiltonian[\schroedingerPicture]\Vhamiltonian[\schroedingerPicture]$ alone isn't.
The derivation for half of the terms can be taken from \autoref{eq:energy-local-observable}, only $\timeExpectationVal{\HzeroHamiltonian[\schroedingerPicture]\Vhamiltonian[\schroedingerPicture] + \Vhamiltonian[\schroedingerPicture]\HzeroHamiltonian[\schroedingerPicture]}$, $\timeExpectationVal{{\HzeroHamiltonian[\schroedingerPicture]}^2}$ and $\timeExpectationVal{{\Vhamiltonian[\schroedingerPicture]}^2}$ require closer inspection.

For the case of $\timeExpectationVal{\HzeroHamiltonian[\schroedingerPicture]\Vhamiltonian[\schroedingerPicture] + \Vhamiltonian[\schroedingerPicture]\HzeroHamiltonian[\schroedingerPicture]}$ and $\timeExpectationVal{{\HzeroHamiltonian[\schroedingerPicture]}^2}$, \autoref{eq:simple-variance-terms} shows the expressions for \localObservable{N}{t}. Comparing with \autoref{eq:energy-local-observable}, the sums can be fully separated, as the \ketN[N] are eigenstates of \HzeroHamiltonian[\schroedingerPicture] and the \HzeroHamiltonian[\schroedingerPicture]-sum can always be moved outside without interfering with the states that are next to \Vhamiltonian[\schroedingerPicture].


\begin{equation}
    \label{eq:simple-variance-terms}
    \begin{split}
        \ObservableOp ={\HzeroHamiltonian[\schroedingerPicture]}^2   &: \quad \localObservable{N}{t} =
        \left(E_0(N,\,t)\right)^2
        \\
        \ObservableOp =\HzeroHamiltonian[\schroedingerPicture]\Vhamiltonian[\schroedingerPicture]   &: \quad \localObservable{N}{t} = E_0(N,\,t) \cdot E_\text{V}(N,\,t)
        \\
        \ObservableOp =\Vhamiltonian[\schroedingerPicture]\HzeroHamiltonian[\schroedingerPicture]   &: \quad \localObservable{N}{t} = \left[E_0(N,\,t) \cdot E_\text{V}(N,\,t) \right]^\ast
    \end{split}
\end{equation}

Sadly, calculating \localObservable{N}{t} for $\ObservableOp = {\Vhamiltonian[\schroedingerPicture]}^2$ in the exact same way has a way higher runtime complexity that all previous observables.
Plugging it into \autoref{eq:expectation-value} like previous operators, one can not use the strategy of pulling out one sum early, as \ketN[N] is not an eigenvector of \Vhamiltonian[\schroedingerPicture].
And the full expression requires \bigo{\left[\text{\#}(\text{sites})\right]^2} operations to evaluate. 

There are multiple ways to get around this though. First there is the possibility of introducing advantageous terms as presented in \autoref{eq:v-squared-hard-computation}, yielding the desired expression.

\begin{equation}
    \label{eq:v-squared-hard-computation}
    \begin{split}
        \localObservable{N}{t} &\stackrel{\ref{eq:expectation-value}}{=}
        \lsum[K] \bracketHelper{N}{{\Vhamiltonian[\schroedingerPicture]}^2}{K} e^{\HeffOft[K]-\HeffOft[N]}
        \frac{
            \psiN[K]
        }{
            \psiN[N]
        }  - \left(E_V(N,\,t)\right)^2 + \left(E_V(N,\,t)\right)^2\\
        &\stackrel{\ref{eq:main-hamiltonian-perturbation},\,\ref{eq:energy-local-observable}}{=}
        J^2 \fullneighborsum[,\,\sigma]{l}{m}\fullneighborsum[,\,\mu]{a}{b}
        \left(
            \lsum[K]
            e^{\HeffOft[K]-\HeffOft[N]}
            \frac{
                \psiN[K]
            }{
                \psiN[N]
            }
            \bracketHelper{N}{\withspinhcop[\schroedingerPicture]{l}{\sigma}{\dagger}\withspinhcop[\schroedingerPicture]{m}{\sigma}{}\withspinhcop[\schroedingerPicture]{a}{\mu}{\dagger}\withspinhcop[\schroedingerPicture]{b}{\mu}{}}{K}
        \right.
        \\&\quad-
        \left.
            \lsum[M]
            \lsum[L]
            \frac{e^{\HeffOft[M]-\HeffOft[N]}}{e^{\HeffOft[N]-\HeffOft[L]}}
            \frac{
                \psiN[M]\psiN[L]
            }{
                \psiN[N]
            }
            \bracketHelper{N}{\withspinhcop[\schroedingerPicture]{l}{\sigma}{\dagger}\withspinhcop[\schroedingerPicture]{m}{\sigma}{}}{L}
            \bracketHelper{N}{\withspinhcop[\schroedingerPicture]{a}{\mu}{\dagger}\withspinhcop[\schroedingerPicture]{b}{\mu}{}}{M}
        \right) 
        \\&\quad + \left(E_V(N,\,t)\right)^2
        \\&\stackrel{\phantom{\ref{eq:expectation-value}}}{=}
        J^2 \fullneighborsum[,\,\sigma]{l}{m}\fullneighborsum[,\,\mu]{a}{b}
        \left(
            e^{\HeffOft[\vphantom{N}\smash{\tildetilde{N}}(l,m,\sigma,a,b,\mu)]-\HeffOft[N]}
            \frac{
                \psiN[\vphantom{N}\smash{\tildetilde{N}}(l,m,\sigma,a,b,\mu)]
            }{
                \psiN[N]
            }
            I_\text{occ}\left(N,l,m,\sigma,a,b,\mu\right)
        \right.
        \\&\stackrel{\phantom{\ref{eq:expectation-value}}}{-}
        \left.
            \frac{e^{\HeffOft[\vphantom{N}\smash{\tilde{N}}(a,b,\mu)]-\HeffOft[N]}}{e^{\HeffOft[N]-\HeffOft[\vphantom{N}\smash{\tilde{N}}(l,m,\sigma)]}}
            \frac{
                \psiN[\vphantom{N}\smash{\tilde{N}}(a,b,\mu)]\psiN[\vphantom{N}\smash{\tilde{N}}(l,m,\sigma)]
            }{
                \psiN[N]
            }
            \left(n_{l,\,\sigma} \neq n_{m,\,\sigma}\right)
            \left(n_{a,\,\mu} \neq n_{b,\,\mu}\right)
        \right) 
        \\&\stackrel{\phantom{\ref{eq:expectation-value}}}{+} \left(E_V(N,\,t)\right)^2
    \end{split}
\end{equation}

$I_\text{occ}\left(N,l,m,\sigma,a,b,\mu\right)$ is in this relation a four-way interaction term depending on the occupations of the state \ketN and the indices $l$, $m$, $a$ and $b$, as well as the spin directions $\sigma$ and $\mu$. It must take the cases into account, where $\sigma = \mu$ and $l=a$ or $l=b$ or $m=a$ or $m=b$ and treat them accordingly.
Through the introduction of the extra terms, \autoref{eq:v-squared-hard-computation} first seems overly complicated and yet still \bigo{\left[\text{\#}(\text{sites})\right]^2}, because of the two sums over $l$ and $a$. 
However because of the limited interaction range of \HeffOft[\ast] - based on the order of expansion that is taken into account - under specific conditions (which will be touched in \fullref{sec:theory-optimizations-geometry}), the bracket will in fact cancel for terms with sites $l$ and $a$ far enough apart.
For larger systems basically all sites are \glqq far enough\grqq{} apart, meaning the expression can be reduced to a single sum \bigo{\text{\#}(\text{sites})} that for each site needs to only cover the \emph{sphere of possible interactions}.
All terms between the states $\ketN[\vphantom{N}\smash{\tilde{N}}(\ast)] \leftrightarrow \ketN$  have been previously stated to be possible to be evaluated efficiently if connected by a two-occupation-swap.
And also the four-occupation-interactions between $\ketN[\vphantom{N}\smash{\tildetilde{N}}(\ast)] \leftrightarrow \ketN$ can each be reduced to two two-occupation-swaps.
All this together shows that it is possible to reduce second order operators to a computational effort of \bigo{\text{\#}(\text{sites})} times the size of the interaction sphere. 

This reduction is implemented in the codebase and verified in 

\filepath{\cite{selfCode}}{/computation-scripts/compareenergyvariance.py}.

For a second way to acquire the desired values it is necessary derive a different expression for \localObservable{N}{t}, than the one previously presented in \autoref{eq:expectation-value}.
Starting with a operator that is a second power of any other operator and can be written $\ObservableOp = {\ObservableOp{}'}^2$, one can derive the alternate formulation in \autoref{eq:expectation-value-square-operator}.

\begin{equation}
    \label{eq:expectation-value-square-operator}
    \begin{split}
        &\frac{\bracketHelper{\psiOfT[\schroedingerPicture]}{{\ObservableOp{}'}{\ObservableOp{}'}}{\psiOfT[\schroedingerPicture]}}{\braketHelper{\psiOfT[\schroedingerPicture]}{\psiOfT[\schroedingerPicture]}}
        %
        \stackrel{\phantom{\ref{eq:time-evolution-target}}}{=} 
        \frac{
            \lsum[N] \bracketHelper{\psiOfT[\schroedingerPicture]}{{\ObservableOp{}'}}{N} \bracketHelper{N}{{\ObservableOp{}'}}{\psiOfT[\schroedingerPicture]}
        }{
            \lsum[K] \braketHelper{\psiOfT[\schroedingerPicture]}{K} \braketHelper{K}{\psiOfT[\schroedingerPicture]}
        }\\
        %
        & \stackrel{\ref{eq:time-evolved-state-and-dagger}}{=} 
        \lsum[N]
        \frac{
            \bracketHelper{\psiOfT[\schroedingerPicture]}{{\ObservableOp{}'}}{N} \bracketHelper{N}{{\ObservableOp{}'}}{\psiOfT[\schroedingerPicture]}
        }{
            \lsum[K] e^{\HeffOftStar[K]} \psiNStar[K] e^{\HeffOft[K]} \psiN[K] 
        }
        \cdot
        \frac{\braketHelper{\psiOfT[\schroedingerPicture]}{N}}{\braketHelper{\psiOfT[\schroedingerPicture]}{N}}
        \cdot 
        \frac{\braketHelper{N}{\psiOfT[\schroedingerPicture]}}{\braketHelper{N}{\psiOfT[\schroedingerPicture]}}
        \\
        & \stackrel{\phantom{\ref{eq:time-evolved-state-and-dagger}}}{=} 
        \lsum[N]
        \frac{
            \braketHelper{\psiOfT[\schroedingerPicture]}{N} \braketHelper{N}{\psiOfT[\schroedingerPicture]}
        }{
            \lsum[K] \absSquare{e^{\HeffOft[K]}} \absSquare{\psiN[K]} 
        } \cdot 
        \frac{\bracketHelper{\psiOfT[\schroedingerPicture]}{{\ObservableOp{}'}}{N}}{\braketHelper{\psiOfT[\schroedingerPicture]}{N}}
        \cdot 
        \frac{\bracketHelper{N}{{\ObservableOp{}'}}{\psiOfT[\schroedingerPicture]}}{\braketHelper{N}{\psiOfT[\schroedingerPicture]}}
        \\
        & \stackrel{\ref{eq:time-evolved-state-and-dagger}}{=} 
        \lsum[N]
        \frac{
            e^{\HeffOftStar} \psiNStar e^{\HeffOft} \psiN{}
        }{
            \lsum[K] \absSquare{e^{\HeffOft[K]}} \absSquare{\psiN[K]} 
        }  \cdot 
        \absSquare{\frac{\bracketHelper{N}{{\ObservableOp{}'}}{\psiOfT[\schroedingerPicture]}}{\braketHelper{N}{\psiOfT[\schroedingerPicture]}}}
        \\& \stackrel{\phantom{\ref{eq:time-evolved-state-and-dagger}}}{=} 
        \lsum[N]
        \underbrace{\frac{
            \absSquare{e^{\HeffOft}} \absSquare{\psiN} 
        }{
            \lsum[K] \absSquare{e^{\HeffOft[K]}} \absSquare{\psiN[K]} 
        }}_{\probabilityOf{N}{t}} \cdot 
        \absSquare{\frac{\bracketHelper{N}{{\ObservableOp{}'}}{\psiOfT[\schroedingerPicture]}}{\braketHelper{N}{\psiOfT[\schroedingerPicture]}}}\\
        &\stackrel{\ref{eq:expectation-value}}{=} \lsum[N]
        \probabilityOf{N}{t}
        \underbrace{\absSquare{\lsum[K] \bracketHelper{N}{{\ObservableOp{}'}}{K} e^{\HeffOft[K]-\HeffOft[N]}
        \frac{
            \psiN[K]
        }{
            \psiN[N]
        }}}_{\localObservable{N}{t}}
    \end{split}
\end{equation}

Comparing with the previous expression, the result is now much cheaper to compute - as \autoref{eq:energy-local-observable} pins the value of the sum in the absolute square to $E_V(N,\,t)$.
Combining the two methods of employing locality and strategic expansion of the time evolved wave-function makes it possible to efficiently evaluate most observables.
Also the result from \autoref{eq:simple-variance-terms} for $\ObservableOp = {\HzeroHamiltonian[\schroedingerPicture]}^2$ holds, as $E_0(N,\,t)$ is real-valued.

\begin{equation}
    \label{eq:v-squared-variance-last-term}
        \ObservableOp ={\Vhamiltonian[\schroedingerPicture]}^2: \quad \localObservable{N}{t} \stackrel{\ref{eq:expectation-value-square-operator},\,\ref{eq:energy-local-observable}}{=}
        \absSquare{E_\text{V}(N,\,t)}
\end{equation}

So to now get the full expectation value for the energy fluctuations, it is in fact only necessary to measure $E_0(N,\,t)$ and $E_V(N,\,t)$ for each sampled state - making it exactly as expensive to compute as the energy itself.
Then however it requires aggregating $E_0(N,\,t)$, $E_V(N,\,t)$, $\left(E_0(N,\,t)\right)^2$, $\absSquare{E_V(N,\,t)}$ and $E_0(N,\,t) \cdot E_\text{V}(N,\,t) + \left[E_0(N,\,t) \cdot E_\text{V}(N,\,t) \right]^\ast$ separately (scaled with the probability) for all sampled states, to be able to combine all terms according to \autoref{eq:variance-split} in the end.