% \subsubsection*{Spin-Polarized Kinetics}

For measuring the spin-polarized flow of particles, a slightly more complex observable must be employed. 
In \autoref{eq:spin-polarized-kinetics-operator-definition} a possible observable is listed, that measures such kinetics direction in-dependent (\spinPolarizedKineticsOperator{l}{m}{\sigma}) or direction dependent (\spinPolarizedKineticsOperatorDir{l}{m}{\sigma}, caution: needs $i$ to obtain hermitian operator).

\begin{equation}
    \label{eq:spin-polarized-kinetics-operator-definition}
    \begin{split}
        \spinPolarizedKineticsOperator{l}{m}{\sigma} &= - J \left(\withspinhcop{l}{\sigma}{\dagger}\withspinhcop{m}{\sigma}{} + \withspinhcop{m}{\sigma}{\dagger}\withspinhcop{l}{\sigma}{}\right)\\
        \spinPolarizedKineticsOperatorDir{l}{m}{\sigma} &= i J \left(\withspinhcop{m}{\sigma}{\dagger}\withspinhcop{l}{\sigma}{} - \withspinhcop{l}{\sigma}{\dagger}\withspinhcop{m}{\sigma}{}\right)\\
    \end{split}
\end{equation}

In this case, the used basis-states are not eigenstates of the operators. 
$\bracketHelper{N}{\withspinhcop{m}{\sigma}{\dagger}\withspinhcop{l}{\sigma}{}}{K}$ becomes $\delta_{N,\,\tilde{N}}\cdot (1-n_{l,\,\sigma} )\cdot n_{m,\,\sigma}$, where \ketN[{\vphantom{N}\smash{\tilde{N}}}] is the state obtained when the particle number on site $m,\, \sigma$ is transferred to site $l,\, \sigma$ (this \emph{hopping} is only possible, when there is a particle on the original site and no particle yet on the target site, which is ensured by the occupation numbers).
Evaluating the whole operator with the signs and $i$s correctly and in a efficient manner can best be looked up in the implementation \filepath{\cite{selfCode}}{/computation-scripts/observables.py}.

Overall, the evaluation of this observable requires knowing the value of an object of the form presented in \autoref{eq:form-heff-difference}.

\begin{equation}
    \label{eq:form-heff-difference}
    \frac{\psiN[\tilde{N}]}{\psiN} e^{\HeffOft[\tilde{N}]-\HeffOft[N]}
\end{equation}

The \fullref{sec:theory-optimizations-analytical} will go into how to compute this efficiently for states \ketN[N] and \ketN[{\vphantom{N}\smash{\tilde{N}}}] that are connected via the hopping between nearest-neighbor lattice sites.

It is finally important to notice, that the objects from \autoref{eq:form-heff-difference} are not real-valued, but complex.
Only the complete observable from the full sum over all basis-states \ketN over \localObservable{N}{t} has a fully vanishing imaginary part. 
Especially when the observable is approximated with an incomplete set of basis-states, computationally an imaginary component remains. 
It is important to monitor the magnitude of this value and monitor it going to $0$.
If it doesn't fully vanish, this indicates an error in the sampling strategy or the implementation or the number of sampled states is not great enough.
