Typically, the \emph{density matrix} $\rho = \ketpsiof[]{t}\brapsiof[]{t}$ encodes the full quantum information of the state. 
Because the number of entries in the matrix is the number of base-states squared - which is exponential in the number of sites, exactly what the approximation is supposed to avoid - it is not feasible to access this quantity.

While it is not possible to access the complete matrix, there exists one closely related entity, that has multiple uses that are going to be utilized in the subsequent section.
The \emph{reduced density matrix} $\rho_A$ is obtained upon \emph{tracing out} (\autoref{eq:tracing-out}, \cite{partialTraceEntanglementOfSubsystemsBlochVector}) the part $B$ from a system that is completely partitioned into two parts $A$ and $B$.
The basis of $A$ is $\left\{\ketN[\Phi_k]\right\}_{k=0}^{2^{2}-1 = 3 = N} = \left\{\ketN[11],\,\ketN[10],\,\ketN[01],\,\ketN[00]\right\}$ and the basis of $B$ $\left\{\ketN[\Chi_k]\right\}_{k=0}^{2^{2 \cdot \#\left(\text{sites}\right) - 2}-1 = M}$.
Notice, the order of basis states starting with $\ketN[1\dots1]$ then $\ketN[1\dots10]$ and ending with $\ketN[0\dots0]$. As stated in \autoref{eq:spin-hcb-state-mapping}, this is the convention that is necessary to keep consistent sings, as $\ketN[0\dots0] \rightarrow \ketN[1\dots1]$ would introduce exactly one $-1$ into all measurements with \pauli{y}.

\begin{equation}
    \label{eq:tracing-out}
    \rho_A = \partialTrace{B}{\rho_{AB}} = \sum\limits_{k=0}^{M} \left(\one[A]\otimes\braN[\Chi_k] \right)\rho_{AB}\left(\one[A]\otimes\ketN[\Chi_k]\right)
\end{equation}

If - for this application - one reduces the part $A$ to contain only two sides $l, \sigma$ and $m, \mu$, then $\rho_A$ will be a complex $4\times 4$ matrix.
As it is again not feasible to obtain via tracing out, the matrix can be calculated by rewriting it in the basis of Pauli matrices (\ref{sec:particles}):

\begin{equation}
    \label{eq:reduced-density-matrix-via-paulis}
    \rho_A (t) = \rho_{l,\,\sigma,\,m,\,\mu} (t) = \frac{1}{4} \sum\limits_{\alpha,\beta\in\left\{0,\text{x},\text{y},\text{z}\right\}}
    {\left\langle \pauli[l,\,\sigma]{$\alpha$}\pauli[m,\,\mu]{$\beta$} \right\rangle}_{\psiOfT} \left(\pauli[l,\,\sigma]{$\alpha$}\otimes\pauli[m,\,\mu]{$\beta$}\right)
\end{equation}

\autoref{eq:reduced-density-matrix-via-paulis} uses the property, that the 16 matrices $\pauli{$\alpha$}\otimes\pauli{$\beta$}$ are a complete basis of the complex $4\times 4$ matrices.
The observable $\pauli[l,\,\sigma]{$\alpha$}\pauli[m,\,\mu]{$\beta$}$ can be translated into hard-core bosonic operators, thanks to the previously derived \autoref{eq:jordan-wigner-spin-hcb} and with that can easily be measured like all other observables.
The factor $\frac{1}{4}$ follows from a normalization argument in \autoref{eq:reduced-density-matrix-normalization}, that uses properties of density matrices \cite{partialTraceEntanglementOfSubsystemsBlochVector} and algebraic properties of the Pauli matrices.

\begin{equation}
    \label{eq:reduced-density-matrix-normalization}
    \begin{split}
        {\left\langle \pauli[l,\,\sigma]{$\alpha$}\pauli[m,\,\mu]{$\beta$} \right\rangle}_{\psiOfT} &= \partialTrace{}{\rho_{l,\,\sigma,\,m,\,\mu} (t) \left(\pauli[l,\,\sigma]{$\alpha$}\otimes\pauli[m,\,\mu]{$\beta$}\right)}  \\
        &\stackrel{\ref{eq:reduced-density-matrix-via-paulis}}{=}\sum\limits_{\alpha',\beta'\in\left\{0,\text{x},\text{y},\text{z}\right\}} \frac{{\left\langle \pauli[l,\,\sigma]{$\alpha'$}\pauli[m,\,\mu]{$\beta'$} \right\rangle}_{\psiOfT}}{4}
        \partialTrace{}{\left(\pauli{$\alpha'$}\otimes\pauli{$\beta'$}\right)\left(\pauli{$\alpha$}\otimes\pauli{$\beta$}\right)}\\
        &\stackrel{\phantom{\ref{eq:reduced-density-matrix-via-paulis}}}{=}\sum\limits_{\alpha',\beta'\in\left\{0,\text{x},\text{y},\text{z}\right\}} \frac{{\left\langle \pauli[l,\,\sigma]{$\alpha'$}\pauli[m,\,\mu]{$\beta'$} \right\rangle}_{\psiOfT}}{4}
        \delta_{\alpha',\,\alpha}\delta_{\beta',\,\beta}\partialTrace{}{\one[4\times 4]}\\
        &\stackrel{\phantom{\ref{eq:reduced-density-matrix-via-paulis}}}{=}\frac{4}{4} \cdot {\left\langle \pauli[l,\,\sigma]{$\alpha$}\pauli[m,\,\mu]{$\beta$} \right\rangle}_{\psiOfT}
    \end{split}
\end{equation}