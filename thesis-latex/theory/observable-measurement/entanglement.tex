The $\rho_{l,\,\sigma,\,m,\,\mu} (t)$ from \autoref{eq:reduced-density-matrix-via-paulis} has been specifically chosen to give the full quantum information of a subsystem of two arbitrary site-indices and spin-directions.
Most importantly, as the research is performed on quantum systems, it is a goal to extract quantum properties and the prime candidate for this is extracting the \emph{entanglement}.

Research in \cite{isingDynamicsWithClassicalNetworks} has already used the same method of obtaining the reduced density matrix as presented in \autoref{eq:reduced-density-matrix-via-paulis}, with the goal of measuring the entanglement.
There, the number of sampled states, required to obtain the measurements of the observables, was called out as a limiting factor.
Monte-Carlo sampling was suggested as a fix for this limitation, which will actually be introduced in \fullref{sec:theory-optimizations-monte-carlo}.

In this work, the \emph{concurrence} was chosen as the \emph{entanglement monotone} to provide insight into the entanglement between different sites.
The concurrence has been introduced in \cite{concurrenceMainPaper} as a measure to obtain the entanglement of two spins from the reduced density matrix that contains these two spins.
A concurrence measurement does not give the value for the \emph{entanglement of formation} directly, but it needs to be rescaled as shown in \autoref{fig:entanglement}.
As can be see in the diagram, the rescaling does not introduce major additional features and for that reason the concurrence will be studied without rescaling to the entanglement.
The point of minimal entanglement (maximum separability) and maximal entanglement map to the same values and the intermediate values communicate the same qualitative information.

\begin{SCfigure}[2.0][htbp]
    \centering
    \includegraphics[width=0.35\textwidth]{plotgeneration/binary-entropy-function/binary-entropy-function.pdf}
    \caption{
        Plot of the function that connects the entanglement $\text{E}_{\psiOfT}$ of a subsystem with the concurrence of a subsystem $C(\psiOfT)$ in the relevant region $0\leq x \leq 1$.
        According to \cite{concurrenceMainPaper}: $\text{E}_{\psiOfT} = \text{E}(C(\psiOfT))$, with the function $\text{E}(x) = H(\frac{1}{2} + \frac{1}{2} \sqrt{1-x^2})$ and the \emph{binary entropy function} $H(x) = -\left[x \cdot \log_2(x) + (1-x) \cdot \log_2(1-x)\right]$.
        Compared with the identity $y=x$, it shows that the values at the borders of the region $x=0$ and $x=1$ are the same for both functions. 
        The examined function $\text{E}(x)$ increases monotonically for the relevant input range and lies reasonably close to the identity. 
    }
    \label{fig:entanglement}
\end{SCfigure}

\filepath{\cite{selfCode}}{/calculation-helpers/concurrence/} holds multiple scripts used to validate that the calculations in the following section give the same result, no matter which of the alternative implementation-/calculation-strategies are used.

\cite{concurrenceMainPaper} introduces the \emph{magic basis} shown in \autoref{eq:magic-basis} as opposed to the counting basis in the beginning of \fullref{sec:theory-observables-density-matrix}. 
The magic basis consists of \emph{Bell states} with particular phases.

\begin{equation}
    \label{eq:magic-basis}
    \begin{split}
        \ketN[m_1] &= \frac{1}{\sqrt{2}} \left(\ketN[11] + \ketN[00]\right) \\
        \ketN[m_2] &= \frac{1}{\sqrt{2}} i \left(\ketN[11] - \ketN[00]\right) \\
        \ketN[m_3] &= \frac{1}{\sqrt{2}} i \left(\ketN[10] + \ketN[01]\right) \\
        \ketN[m_4] &= \frac{1}{\sqrt{2}} \left(\ketN[10] - \ketN[01]\right) \\
    \end{split}
\end{equation}

The Bell state is a construct tightly linked to entanglement, consequently it is by design straight forward to write down the concurrence of a pure state expressed in the magic basis.
For $\ketN[\psiOfT] = \sum\limits_{i=1}^4 \alpha_i \ketN[m_i]$ the concurrence $C(\psiOfT)$ is $C(\psiOfT) = \left|\sum\limits_{i=1}^4 \alpha_i^2 \right|$.
This is equivalent to saying for a pure state expressed in the counting basis $\ketN[\psiOfT] = \alpha \ketN[00] +  \beta \ketN[01] +  \gamma \ketN[10] +  \delta\ketN[11]$, the concurrence is calculated $C(\psiOfT) = 2 \left|\alpha \delta - \beta \gamma\right|$.
This gives the same value as for all calculations that extract the concurrence from the density matrix that is calculated from these pure state coefficients.
But while all states obtained via time evolution $\ketN[{\psiOfT[\schroedingerPicture]}]$ - as described in previous chapters - are pure states, through the reduction to a \glqq two-spin-system\grqq{} of particles $l,\,\sigma$ and $m,\,\mu$ to $\rho_{l,\,\sigma,\,m,\,\mu} (t)$ the remaining system is likely to be a mixed state \cite{mixedStateFromPureState}.

The calculation necessarily needed to be generalized to take the density matrix as an input.
\cite{concurrenceRewording} gives two alternate ways of calculating the concurrence from the density matrix.
Both require the \emph{spin-flip} operation $\tilde{\rho}$.
The original paper \cite{concurrenceMainPaper} describes this operation for density matrix $\rho$ expressed in the counting basis to be: transforming it into the magic basis of \autoref{eq:magic-basis}, taking the complex conjugate, then transforming it back.
As per \cite{concurrenceRewording} this is equivalent to the operation described in \autoref{eq:spin-flip}, where $\rho$ and \pauli[]{y} (\autoref{eq:pauli-matrices}) are expressed in the standard counting basis.

\begin{equation}
    \label{eq:spin-flip}
    \begin{split}
        \tilde{\rho} = \left(\pauli[]{y} \otimes \pauli[]{y}\right) \rho^\ast  \left(\pauli[]{y} \otimes \pauli[]{y}\right)
    \end{split}
\end{equation}

\filepath{\cite{selfCode}}{/calculation-helpers/concurrence/basis-transformation.py} verifies this equivalence and shows the ${\tilde{\ast}}$-operation modifies a hermitian matrix like shown in \autoref{eq:demonstration-spin-flip}.

\begin{equation}
    \label{eq:demonstration-spin-flip}
    \begin{split}
        \rho = \left(\begin{matrix}
            a& b & c &d \\
            b^\ast& e & f &g \\
            c^\ast& f^\ast & h &k \\
            d^\ast& g^\ast & k^\ast &l \\
        \end{matrix}\right)  \quad\Rightarrow\quad\tilde{\rho} = 
        \left(\begin{matrix}
            l& -k & -g &d \\
            -k^\ast& h & f &-c \\
            -g^\ast& f^\ast & e &-b \\
            d^\ast& -c^\ast & -b^\ast &a \\
        \end{matrix}\right) 
    \end{split}
\end{equation}

With that knowledge \autoref{eq:concurrence-calculation} states how to calculate the concurrence $C(\rho)$ from the reduced density matrix $\rho$.

\begin{equation}
    \label{eq:concurrence-calculation}
    \begin{gathered}
        R = \sqrt{\sqrt{\rho} \cdot \tilde{\rho} \cdot \sqrt{\rho}}\\
        \lambda_1 \geq \lambda_2 \geq \lambda_3 \geq \lambda_4, \quad \left\{\lambda_i\right\} = \text{Eigenvalues}(R)\\
        C(\rho) = \max (0, \lambda_1 - \lambda_2 - \lambda_3 - \lambda_4)
    \end{gathered}
\end{equation}

In this formula, the square root $\sqrt{M}$ of a square matrix $M$ is a square matrix of the same dimensionality, fulfilling the property $\sqrt{M}\cdot \sqrt{M} = M$.
A candidate can be obtained numerically by calculating the eigenvalues of the matrix (which for a hermitian matrix are all positive real numbers), taking their square root and re-assembling the matrix with the eigenvectors like in \autoref{eq:matrix-square-root} ($\lambda_i$ different values than in \autoref{eq:concurrence-calculation}).

\begin{equation}
    \label{eq:matrix-square-root}
    \begin{gathered}
        M  = Q \Lambda Q^\text{T}\\
        \Lambda = \text{diag}(\lambda_1,\, \lambda_2,\, \lambda_3,\, \lambda_4)\\
        \sqrt{\Lambda} = \text{diag}(\sqrt{\lambda_1},\, \sqrt{\lambda_2},\,\sqrt{\lambda_3},\,\sqrt{\lambda_4})\\
        \sqrt{M}  = Q \sqrt{\Lambda} Q^\text{T}
    \end{gathered}
\end{equation}

While this operation is quite complicated, it is required only once per calculation and only for a matrix of fixed size $4\times 4$.
To avoid it, \cite{concurrenceRewording} alternatively provides that the $\lambda_i$ from \autoref{eq:concurrence-calculation} are the square roots of the non-hermitian matrix $\rho \tilde{\rho}$.
For exact calculations both ways of calculating yield the same results, which is implemented and checked. 
If there are any inconsistencies in the order of basis vectors however, the results diverge. Which in fact is a good check to verify the bases are consistent across the whole implementation.
