While the optimizations described in the previous section greatly increase computational efficiency, they need to be applied for each order and each $\Phi_\ast(N)$ separately.
The problem is, that for two dimensions the number of interactions to consider grows quickly, while visualizing the symmetries of what terms are equivalent gets progressively harder.
When looking at a chain, each higher order expands the interaction range by a one-site-step. 
So taking one more order into account, two sites are added to the possibilities for each interaction - this linearity is still somewhat manageable.
In two dimensions, the number of affected sites in relation to the interaction range grows quadratically, as visualized in \autoref{fig:growing-interaction-range}.

\begin{figure}[htbp]
    \centering
    \begin{tikzpicture}[scale=0.38]
    \definecolor{relevantcol}{HTML}{00AA00} % green

    \def\m{5} % Change this value to adjust the grid size (m x m)
    \def\side{3} % Change this value to adjust the square side length
    \def\labelsize{\side/6} % Adjust label size
    
    \def\sp{1} % Change this value to adjust horizontal skip between the grids
    \def\order{2}

    \foreach \orderindex in {0,...,\numexpr\order\relax} {

        % Draw connections
        \foreach \x in {0,...,\numexpr\m-1\relax} {
            \foreach \y in {0,...,\numexpr\m-1\relax} {
                \pgfmathsetmacro{\orderindexincr}{\orderindex + 1}

                % Get shift amount
                \pgfmathsetmacro{\shft}{(\m + \sp - 1)*\orderindex*\side}
                
                % Get current node position
                \pgfmathsetmacro{\cx}{\x*\side + \side/2 + \shft}
                \pgfmathsetmacro{\cy}{\y*\side + \side/2 + \side/4}

                % Compute CURRENT Manhattan distance from center
                \pgfmathtruncatemacro{\dista}{abs(\x - (\m-1)/2) + abs(\y - (\m-1)/2)}


                % Connect to right neighbor
                \ifnum\x<\numexpr\m-1\relax

                    \pgfmathsetmacro{\usex}{\x+1}
                    \pgfmathsetmacro{\usey}{\y}

                    \pgfmathtruncatemacro{\distb}{abs(\usex - (\m-1)/2) + abs(\usey - (\m-1)/2)}

                    \pgfmathsetmacro{\nx}{\usex*\side + \side/2 + \shft}
                    \pgfmathsetmacro{\ny}{\usey*\side + \side/2 + \side/4}

                    % Color if both are smaller Manhattan distance
                    \ifnum\dista<\orderindexincr
                        \ifnum\distb<\orderindexincr
                            \edef\useconcol{gray}
                        \else
                            \edef\useconcol{black}
                        \fi
                    \else
                        \edef\useconcol{black}
                    \fi

                    \draw[line width=2\pgflinewidth, color=\useconcol] (\cx,\cy) -- (\nx,\ny);
                \fi
                

                % Connect to top neighbor
                \ifnum\y<\numexpr\m-1\relax

                    \pgfmathsetmacro{\usex}{\x}
                    \pgfmathsetmacro{\usey}{\y+1}

                    \pgfmathtruncatemacro{\distb}{abs(\usex - (\m-1)/2) + abs(\usey - (\m-1)/2)}

                    \pgfmathsetmacro{\nx}{\usex*\side + \side/2 + \shft}
                    \pgfmathsetmacro{\ny}{\usey*\side + \side/2 + \side/4}

                    % Color if both are smaller Manhattan distance
                    \ifnum\dista<\orderindexincr
                        \ifnum\distb<\orderindexincr
                            \edef\useconcol{gray}
                        \else
                            \edef\useconcol{black}
                        \fi
                    \else
                        \edef\useconcol{black}
                    \fi

                    \draw[line width=2\pgflinewidth, color=\useconcol] (\cx,\cy) -- (\nx,\ny);
                \fi
            }
        }

        % Draw dots and labels with custom text using a loop
        \foreach \x in {0,...,\numexpr\m-1\relax} {
            \foreach \y in {0,...,\numexpr\m-1\relax} {
                % Get shift amount
                \pgfmathsetmacro{\shft}{(\m + \sp - 1)*\orderindex*\side}
                \pgfmathtruncatemacro{\index}{(\m-\y-1) * \m + \x}

                % Get current node position
                \pgfmathsetmacro{\cx}{\x*\side + \side/2 + \shft}
                \pgfmathsetmacro{\cy}{\y*\side + \side/2 + \side/4}

                % Compute Manhattan distance from center
                \pgfmathtruncatemacro{\dist}{abs(\x - (\m-1)/2) + abs(\y - (\m-1)/2)}
                
                % Change color if distance equals \orderindex
                \pgfmathsetmacro{\orderindexincr}{\orderindex + 1}
                \ifnum\dist<\orderindexincr
                    \fill[color=relevantcol] (\cx,\cy) circle (0.4);
                \else
                    \fill[color=dblue] (\cx,\cy) circle (0.4);
                \fi
            }
        }
    }

\end{tikzpicture}
    \caption{caption}
    \label{fig:growing-interaction-range}
\end{figure}
