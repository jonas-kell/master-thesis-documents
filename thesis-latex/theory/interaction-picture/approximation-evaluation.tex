As \autoref{eq:cumulant-expansion} reveals, by this process the problem was converted into finding $\VhamiltonianOf[\interactionPicture]{t'}$ to be able to calculate the first two orders of the expansion.

Operators in the Interaction Picture can be derived from their Schrödinger Picture variants like previously shown in \autoref{eq:time-evolution-operator-interaction}. 
With this, one can derive a useful expression for their \emph{equation of motion} (\autoref{eq:schroedinger-op-to-interaction-op}).

\begin{equation}
    \label{eq:schroedinger-op-to-interaction-op}
    \begin{split}
        \difft{}\AopOfT[\interactionPicture] &\stackrel{\phantom{\ref{eq:time-evolution-operator-interaction}}}{=} i \left[\HzeroHamiltonian[\schroedingerPicture], \AopOfT[\interactionPicture]\right]
        %
        \stackrel{\ref{eq:time-evolution-operator-interaction}}{=}
        i \left[\HzeroHamiltonian[\schroedingerPicture],\,e^{i \HzeroHamiltonian[\schroedingerPicture]{} t} \Aop[\schroedingerPicture] e^{-i \HzeroHamiltonian[\schroedingerPicture]{} t}\right]\\
        %
        &\stackrel{\phantom{\ref{eq:time-evolution-operator-interaction}}}{=}
        i e^{i \HzeroHamiltonian[\schroedingerPicture]{} t}\left[\HzeroHamiltonian[\schroedingerPicture],\, \Aop[\schroedingerPicture]\right]e^{-i \HzeroHamiltonian[\schroedingerPicture]{} t}
        %
        \stackrel{\phantom{\ref{eq:time-evolution-operator-interaction}}}{=}
        i \interactionTimeEvolution{\left[\HzeroHamiltonian[\schroedingerPicture],\, \Aop[\schroedingerPicture]\right]}
    \end{split}
\end{equation}

\autoref{eq:schroedinger-op-to-interaction-op} also defines the operator \interactionTimeEvolution{\ast}, which simply is $e^{i \HzeroHamiltonian[\schroedingerPicture]{} t} \left\{\ast \right\} e^{-i \HzeroHamiltonian[\schroedingerPicture]{} t}$.
This section uses $(t)$ for indicating that operators in the Interaction Picture have a time-dependence. 
This is superficial and only for better readability, as in fact $\hopOfT[\interactionPicture]{m}{\dagger} = \hop[\interactionPicture]{m}{\dagger} = \interactionTimeEvolution{\hop[\schroedingerPicture]{m}{\dagger}}$.

If $\left[\HzeroHamiltonian[\schroedingerPicture],\, \Aop[\schroedingerPicture]\right]$ now is a function of $\Aop[\schroedingerPicture]$, one gets a \emph{differential equation} that can be solved to obtain $\AopOfT[\interactionPicture]$.

For the repetitive calculation of the objects of type $\left[\HzeroHamiltonian[\schroedingerPicture],\, \Aop[\schroedingerPicture]\right]$ for various \Aop[\schroedingerPicture], the tool \emph{Math-Manipulator} was specifically developed. 
Like the transformations in \ref{sec:particles}, calculations in the following section have been validated or calculated from scratch in this tool.
The relevant files can be found in \filepath{\cite{selfMathManipulatorCalculations}}{/calculate-V}, and \filepath{\cite{selfMathManipulatorCalculations}}{/time-evolution}.
The equations \ref{eq:stuff-from-math-manipulator-dd} and \ref{eq:stuff-from-math-manipulator-cm} list exemplary results that were computed with the tool.

\begin{equation}
    \label{eq:stuff-from-math-manipulator-dd}
    \difft{}\left(\dop[\interactionPicture]{m}{\dagger}\dop[\interactionPicture]{m}{}\right)(t)
    = i \interactionTimeEvolution{\left[\HzeroHamiltonian[],\,\dop[\schroedingerPicture]{m}{\dagger}\dop[\schroedingerPicture]{m}{}\right]} \stackrel{\text{MM}}{=} 0 \quad \Rightarrow \quad \left(\dop[\interactionPicture]{m}{\dagger}\dop[\interactionPicture]{m}{}\right)(t) = \dop[\schroedingerPicture]{m}{\dagger}\dop[\schroedingerPicture]{m}{}
\end{equation}

\begin{equation}
    \label{eq:stuff-from-math-manipulator-cm}
    \begin{split}
        \difft{}\hopOfT[\interactionPicture]{m}{\dagger}
        &\stackrel{\phantom{\ref{eq:time-evolution-operator-interaction}}}{=} i \interactionTimeEvolution{\left[\HzeroHamiltonian[]{},\,\hop[\schroedingerPicture]{m}{\dagger}\right]} \stackrel{\text{MM}}{=} 
        i \interactionTimeEvolution{ \left(\epsl[m] + U \dop[\schroedingerPicture]{m}{\dagger}\dop[\schroedingerPicture]{m}{}\right)\hop[\schroedingerPicture]{m}{}}\\
        &\stackrel{\phantom{\ref{eq:time-evolution-operator-interaction}}}{=}i e^{i \HzeroHamiltonian[\schroedingerPicture]{} t} \left(\epsl[m] + U \dop[\schroedingerPicture]{m}{\dagger}\dop[\schroedingerPicture]{m}{}\right)\hop[\schroedingerPicture]{m}{}e^{-i \HzeroHamiltonian[\schroedingerPicture]{} t}\\
        &\stackrel{\phantom{\ref{eq:time-evolution-operator-interaction}}}{=}i \left(\epsl[m] + U e^{i \HzeroHamiltonian[\schroedingerPicture]{} t}\dop[\schroedingerPicture]{m}{\dagger}\dop[\schroedingerPicture]{m}{}e^{-i \HzeroHamiltonian[\schroedingerPicture]{} t}\right)e^{i \HzeroHamiltonian[\schroedingerPicture]{} t}\hop[\schroedingerPicture]{m}{}e^{-i \HzeroHamiltonian[\schroedingerPicture]{} t}\\
        &\stackrel{\ref{eq:time-evolution-operator-interaction}}{=}
        i \left(\epsl[m] + U \left(\dop[\interactionPicture]{m}{\dagger}\dop[\interactionPicture]{m}{}\right)(t)\right) \hopOfT[\interactionPicture]{m}{\dagger}\\
    \end{split}
\end{equation}

With this and the idempotence-relation \autoref{eq:hc-counting-op-idempotent} one can first derive \autoref{eq:operator-out-from-exponent} and finally use an exponential function as a natural ansatz to solve \autoref{eq:stuff-from-math-manipulator-cm} with \autoref{eq:the-final-time-dependence-cop}.

\begin{equation}
    \label{eq:operator-out-from-exponent}
    \begin{split}
        e^{a \cdot \withspinhcop[\schroedingerPicture]{l}{\sigma}{\dagger}\withspinhcop[\schroedingerPicture]{l}{\sigma}{}} &= \sum\limits_{m=0}^\infty \frac{a^m \left(\withspinhcop[\schroedingerPicture]{l}{\sigma}{\dagger}\withspinhcop[\schroedingerPicture]{l}{\sigma}{}\right)^m}{m!}
        \stackrel{\ref{eq:hc-counting-op-idempotent}}{=} 1 + \left[\sum\limits_{m=1}^\infty \frac{a^m}{m!}\right]\cdot \left(\withspinhcop[\schroedingerPicture]{l}{\sigma}{\dagger}\withspinhcop[\schroedingerPicture]{l}{\sigma}{}\right)\\
        &= 1 + \left[\sum\limits_{m=0}^\infty \frac{a^m}{m!}-1\right]\cdot \left(\withspinhcop[\schroedingerPicture]{l}{\sigma}{\dagger}\withspinhcop[\schroedingerPicture]{l}{\sigma}{}\right)
         = 1 + \left(e^a-1\right)\cdot \withspinhcop[\schroedingerPicture]{l}{\sigma}{\dagger}\withspinhcop[\schroedingerPicture]{l}{\sigma}{}
    \end{split}
\end{equation}


\begin{equation}
    \label{eq:the-final-time-dependence-cop}
    \begin{split}
        \stackrel{\ref{eq:stuff-from-math-manipulator-cm},\, \ref{eq:stuff-from-math-manipulator-dd}}{\Longrightarrow}
        \hopOfT[\interactionPicture]{m}{\dagger}
        &\stackrel{\phantom{\ref{eq:stuff-from-math-manipulator-dd}}}{=}
        e^{i \cdot \epsl[m]\cdot  t + i \cdot U \left(\dop[\interactionPicture]{m}{\dagger}\dop[\interactionPicture]{m}{}\right)(t) \cdot  t}  \hop[\schroedingerPicture]{m}{\dagger} 
        \stackrel{\ref{eq:stuff-from-math-manipulator-dd}}{=}
        e^{i \cdot \epsl[m]\cdot  t + i \cdot U \cdot \dop[\schroedingerPicture]{m}{\dagger}\dop[\schroedingerPicture]{m}{} \cdot  t}  \hop[\schroedingerPicture]{m}{\dagger} \\
        &\stackrel{\ref{eq:operator-out-from-exponent}}{=}
        e^{i \cdot \epsl[m]\cdot  t}\left(1 + \left(e^{i  \cdot U  \cdot  t}-1\right) \dop[\schroedingerPicture]{m}{\dagger}\dop[\schroedingerPicture]{m}{}\right)  \hop[\schroedingerPicture]{m}{\dagger}
    \end{split}
\end{equation}

From analogous calculations follows:

\begin{equation*}
    \begin{split}
        \hopOfT[\interactionPicture]{m}{\dagger}  &= e^{i \cdot \epsl[m]\cdot  t}\left(1 + \left(e^{i  \cdot U  \cdot  t}-1\right) \dop[\schroedingerPicture]{m}{\dagger}\dop[\schroedingerPicture]{m}{}\right)  \hop[\schroedingerPicture]{m}{\dagger}\\
        \hopOfT[\interactionPicture]{m}{}  &= e^{-i \cdot \epsl[m]\cdot  t}\left(1 + \left(e^{-i  \cdot U  \cdot  t}-1\right) \dop[\schroedingerPicture]{m}{\dagger}\dop[\schroedingerPicture]{m}{}\right)  \hop[\schroedingerPicture]{m}{}\\
        \dopOfT[\interactionPicture]{m}{\dagger}  &= e^{i \cdot \epsl[m]\cdot  t}\left(1 + \left(e^{i  \cdot U  \cdot  t}-1\right) \hop[\schroedingerPicture]{m}{\dagger}\hop[\schroedingerPicture]{m}{}\right)  \dop[\schroedingerPicture]{m}{\dagger}\\
        \dopOfT[\interactionPicture]{m}{}  &= e^{-i \cdot \epsl[m]\cdot  t}\left(1 + \left(e^{-i  \cdot U  \cdot  t}-1\right) \hop[\schroedingerPicture]{m}{\dagger}\hop[\schroedingerPicture]{m}{}\right)  \dop[\schroedingerPicture]{m}{}\\
    \end{split}
\end{equation*}

Injecting these derivations into \autoref{eq:main-hamiltonian-perturbation-full-sum}, finally \VhamiltonianOfT[\interactionPicture] can be derived in \autoref{eq:interaction-picture-v-ham-full}.

\begin{equation}
    \label{eq:interaction-picture-v-ham-full}
    \begin{split}
        \VhamiltonianOfT[\interactionPicture] &
        \stackrel{\phantom{\text{MM}}}{=}
        \interactionTimeEvolution{\Vhamiltonian[\schroedingerPicture]} \stackrel{\ref{eq:main-hamiltonian-perturbation-full-sum}}{=}
        - J \cdot \fullneighborsum{l}{m}  \interactionTimeEvolution{
              \left(\hop[\schroedingerPicture]{l}{\dagger}\hop[\schroedingerPicture]{m}{} + \dop[\schroedingerPicture]{l}{\dagger}\dop[\schroedingerPicture]{m}{} \right)
        } \\
        &\stackrel{\phantom{\text{MM}}}{=}
        -J \cdot \fullneighborsum{l}{m}\left(\hopOfT[\interactionPicture]{l}{\dagger}\hopOfT[\interactionPicture]{m}{} + \dopOfT[\interactionPicture]{l}{\dagger}\dopOfT[\interactionPicture]{m}{} \right)\\
        &\stackrel{\text{MM}}{=}
        %
        -J \cdot \fullneighborsum{l}{m} \left[
            \VhamiltonianAnalyticalPart{A}{l}{m}{t} \cdot \VhamiltonianOperatorPart{A}{l}{m} + 
            \VhamiltonianAnalyticalPart{B}{l}{m}{t} \cdot \VhamiltonianOperatorPart{B}{l}{m} + 
            \VhamiltonianAnalyticalPart{C}{l}{m}{t} \cdot \VhamiltonianOperatorPart{C}{l}{m} 
        \right]
    \end{split}
\end{equation}

It is possible to express these operator-strings as single hoppings, decorated with number operators. One receives \autoref{eq:interaction-picture-v-ham-parts}, which is symmetrical in \up and \down (with $\overline{\up} = \,\down$ and  $\overline{\down} = \,\up$).

\begin{equation}
    \label{eq:interaction-picture-v-ham-parts}
    \begin{split}
        \VhamiltonianAnalyticalPart{A}{l}{m}{t} \stackrel{\text{MM}}{=} e^{i\cdot \left(\epsl-\epsl[m]\right)\cdot t} \qquad
        \VhamiltonianOperatorPart{A}{l}{m} &\stackrel{\text{MM}}{=} 
        \lsum[\sigma \in \left\{\up,\,\down\right\}]
        \withspinhcop[\schroedingerPicture]{l}{\sigma}{\dagger}\withspinhcop[\schroedingerPicture]{m}{\sigma}{}
        \left(
            1+
            2 \cdot \nop[\schroedingerPicture]{l}{\overline{\sigma}}\nop[\schroedingerPicture]{m}{\overline{\sigma}}
            - \nop[\schroedingerPicture]{l}{\overline{\sigma}}
            - \nop[\schroedingerPicture]{m}{\overline{\sigma}}
        \right)
        \\
        \VhamiltonianAnalyticalPart{B}{l}{m}{t} \stackrel{\text{MM}}{=} e^{i\cdot \left(\epsl-\epsl[m] + U\right)\cdot t} \qquad
        \VhamiltonianOperatorPart{B}{l}{m} &\stackrel{\text{MM}}{=} 
        \lsum[\sigma \in \left\{\up,\,\down\right\}]        
        \withspinhcop[\schroedingerPicture]{l}{\sigma}{\dagger}\withspinhcop[\schroedingerPicture]{m}{\sigma}{}
        \left(
            \nop[\schroedingerPicture]{l}{\overline{\sigma}}
            - \nop[\schroedingerPicture]{l}{\overline{\sigma}}\nop[\schroedingerPicture]{m}{\overline{\sigma}}
        \right)
        \\
        \VhamiltonianAnalyticalPart{C}{l}{m}{t} \stackrel{\text{MM}}{=} e^{i\cdot \left(\epsl-\epsl[m] - U\right)\cdot t} \qquad
        \VhamiltonianOperatorPart{C}{l}{m} &\stackrel{\text{MM}}{=} 
        \lsum[\sigma \in \left\{\up,\,\down\right\}]
        \withspinhcop[\schroedingerPicture]{l}{\sigma}{\dagger}\withspinhcop[\schroedingerPicture]{m}{\sigma}{}
        \left(
            \nop[\schroedingerPicture]{m}{\overline{\sigma}}
            - \nop[\schroedingerPicture]{m}{\overline{\sigma}}\nop[\schroedingerPicture]{l}{\overline{\sigma}}
        \right)
        \\
    \end{split}
\end{equation}

The expressions for the dressed operators in \autoref{eq:interaction-picture-v-ham-parts} can be rearranged to better reflect their physical effects.
\autoref{eq:interaction-picture-v-ham-parts-simplified} can be obtained and verified by comparing the measurements in a truth table (the \nop[\schroedingerPicture]{\ast}{\ast} measure either a $0$ or $1$ on a state, without modifying it).

\begin{equation}
    \label{eq:interaction-picture-v-ham-parts-simplified}
    \begin{split}
        \VhamiltonianOperatorPart{A}{l}{m} &\stackrel{\text{MM}}{=} 
        \lsum[\sigma \in \left\{\up,\,\down\right\}]
        \withspinhcop[\schroedingerPicture]{l}{\sigma}{\dagger}\withspinhcop[\schroedingerPicture]{m}{\sigma}{}
        \left(\nop[\schroedingerPicture]{l}{\overline{\sigma}}
            \equivalentOperator 
            \nop[\schroedingerPicture]{m}{\overline{\sigma}}
        \right)
        \\
        \VhamiltonianOperatorPart{B}{l}{m} &\stackrel{\text{MM}}{=} 
        \lsum[\sigma \in \left\{\up,\,\down\right\}]        
        \withspinhcop[\schroedingerPicture]{l}{\sigma}{\dagger}\withspinhcop[\schroedingerPicture]{m}{\sigma}{}
        \nop[\schroedingerPicture]{l}{\overline{\sigma}}
        \left(
            1 - \nop[\schroedingerPicture]{m}{\overline{\sigma}}
        \right)
        \\
        \VhamiltonianOperatorPart{C}{l}{m} &\stackrel{\text{MM}}{=} 
        \lsum[\sigma \in \left\{\up,\,\down\right\}]
        \withspinhcop[\schroedingerPicture]{l}{\sigma}{\dagger}\withspinhcop[\schroedingerPicture]{m}{\sigma}{}
        \nop[\schroedingerPicture]{m}{\overline{\sigma}}
        \left(
           1 - \nop[\schroedingerPicture]{l}{\overline{\sigma}}
        \right)
        \\
    \end{split}
\end{equation}

The \equivalentOperator here means \glqq $1$ if the measurements of the operators are equivalent, $0$ otherwise\grqq.

\subsubsection*{Base Energy}

In \autoref{eq:time-evolution-target} the effective Hamiltonian $\HeffOft = -i E_0(N) t + \HNOft$ was defined.
This construct will be necessary in the following sections. 
$E_0(N)$ is calculated quite swiftly, as can be see in \autoref{eq:base-energy} (with $n_{l,\,\sigma}$ - the occupation-number in $\ketN$, not the operator).

\begin{equation}
    \label{eq:base-energy}
    \begin{split}
        E_0(N) &\stackrel{\phantom{\ref{eq:main-hamiltonian-h0}}}{=} \frac{\bracketHelper{N}{\HzeroHamiltonian[]}{N}}{\braketHelper{N}{N}} \\
        %
        &\stackrel{\ref{eq:main-hamiltonian-h0}}{=} \bracketHelper{N}{U \cdot \lsum \nop{l}{\up}\nop{l}{\down}}{N} + \bracketHelper{N}{\lsum[l,\,\sigma] \epsl \nop{l}{\sigma}}{N}\\
        %
        &\stackrel{\phantom{\ref{eq:main-hamiltonian-h0}}}{=} U \cdot \lsum n_{l,\,\up}n_{l,\,\down} + \lsum[l,\,\sigma] \epsl n_{l,\,\sigma}
    \end{split}
\end{equation}

\subsubsection*{Calculate the first order of \HNOft}

The last missing element for $\HeffOft = -i E_0(N) t + \HNOft$ now is \HNOft.
At least for the first order the integration in \autoref{eq:cumulant-expansion} is doable, with now knowing the form of \VhamiltonianOf[\interactionPicture]{t}.

The result of the integration can be seen in \autoref{eq:hn-integrated-first-order}.

\begin{equation}
    \label{eq:hn-integrated-first-order}
    \begin{split}
        \HNOftOrder{1} &\stackrel{\ref{eq:cumulant-expansion}}{=} -i \int\limits_0^t \mathrm{d}t' \frac{\bracketHelper{N}{\VhamiltonianOf[\interactionPicture]{t'}}{\picturePsi[\schroedingerPicture]{}}}{\braketHelper{N}{\picturePsi[\schroedingerPicture]{}}}\\
        %
        &\stackrel{\ref{eq:base-expansion-state}}{=}
        -i \frac{1}{\psiN} \int\limits_0^t \mathrm{d}t' \lsum[K] \bracketHelper{N}{\VhamiltonianOf[\interactionPicture]{t'}}{K} \psiN[K]\\
        &\stackrel{\ref{eq:interaction-picture-v-ham-full}}{=}
        i \cdot J \lsum[K] \fullneighborsum{l}{m} \frac{\psiN[K]}{\psiN} \int\limits_0^t \mathrm{d}t' 
        \left[
            e^{i\cdot \left(\epsl-\epsl[m]\right)\cdot t'} \cdot 
            \bracketHelper{N}{
                \VhamiltonianOperatorPart{A}{l}{m} 
            }{K}
            + 
            \right.\\
            &\qquad
            \left.
            e^{i\cdot \left(\epsl-\epsl[m] + U\right)\cdot t'} \cdot 
            \bracketHelper{N}{
                \VhamiltonianOperatorPart{B}{l}{m}
            }{K}
            + 
            e^{i\cdot \left(\epsl-\epsl[m] - U\right)\cdot t'} \cdot 
            \bracketHelper{N}{
                \VhamiltonianOperatorPart{C}{l}{m} 
            }{K}
        \right]
        \\
        %
        &\stackrel{\phantom{\ref{eq:interaction-picture-v-ham-full}}}{=}
        J \lsum[K] \fullneighborsum{l}{m} \frac{\psiN[K]}{\psiN}
        \left[
            \frac{e^{i\cdot \left(\epsl-\epsl[m]\right)\cdot t}-1}{\epsl-\epsl[m]} \cdot 
            \bracketHelper{N}{
                \VhamiltonianOperatorPart{A}{l}{m} 
            }{K}
            + 
            \right.\\
            &\qquad
            \left.
            \frac{
                e^{i\cdot \left(\epsl-\epsl[m] + U\right)\cdot t}-1
            }{\epsl-\epsl[m] + U}
             \cdot 
            \bracketHelper{N}{
                \VhamiltonianOperatorPart{B}{l}{m}
            }{K}
            + 
            \frac{
                e^{i\cdot \left(\epsl-\epsl[m] - U\right)\cdot t}-1
            }{\epsl-\epsl[m] - U}
             \cdot 
            \bracketHelper{N}{
                \VhamiltonianOperatorPart{C}{l}{m} 
            }{K}
        \right]\\
        %
        &\stackrel{\phantom{\ref{eq:interaction-picture-v-ham-full}}}{=}
        J \lsum[K] \fullneighborsum{l}{m} \frac{\psiN[K]}{\psiN}
        \left[
            \VhamiltonianAnalyticalPartIntegrated{A}{l}{m}{t} \cdot 
            \bracketHelper{N}{
                \VhamiltonianOperatorPart{A}{l}{m} 
            }{K}
            + 
            \right.\\
            &\qquad
            \left.
            \VhamiltonianAnalyticalPartIntegrated{B}{l}{m}{t}
             \cdot 
            \bracketHelper{N}{
                \VhamiltonianOperatorPart{B}{l}{m}
            }{K}
            + 
            \VhamiltonianAnalyticalPartIntegrated{C}{l}{m}{t}
             \cdot 
            \bracketHelper{N}{
                \VhamiltonianOperatorPart{C}{l}{m} 
            }{K}
        \right]
    \end{split}
\end{equation}

In the second to last step, the $i$ is moved into the integral.

While it seems like $\lsum[K] \fullneighborsum{l}{m}$ is even a larger summation than the previously problematic ones, because the operators $\VhamiltonianOperatorPart{A}{l}{m}$, $\VhamiltonianOperatorPart{B}{l}{m}$ and $\VhamiltonianOperatorPart{C}{l}{m}$ have interaction range limited to nearest neighbors, actually the summation is really sparse and most entries are $0$.
This summation can be efficiently evaluated in \bigo{\text{\#}(\text{sites}) \cdot \text{\#}(\text{nearest neighbors})}.
How this could be done, can be looked up in the implementation \filepath{\cite{selfCode}}{/computation-scripts/hamiltonian.py} in the specialization class \texttt{FirstOrderCanonicalHamiltonian}.

\subsubsection*{Calculate the second order of \HNOft}

For the calculation of the second order, the time-ordering-operator becomes relevant again. 
However, it is not necessary for the right side of the integral as the two integrals can be fully localized to one factor each.
For the left side it is necessary to integrate different cases separately.

\vspace{-0.2cm} % TODO check this page-breaks properly in the end

\begin{equation}
    \label{eq:hn-integrated-second-order}
    \begin{split}
        \HNOftOrder{2} &\stackrel{\ref{eq:cumulant-expansion}}{=} - \frac{1}{2} \int\limits_0^t\mathrm{d}t_1 \int\limits_0^t\mathrm{d}t_2
        \left(
        \frac{\bracketHelper{N}{\timeOrderingOperator
        \VhamiltonianOf[\interactionPicture]{t_1}\VhamiltonianOf[\interactionPicture]{t_2}
        }{\picturePsi[\schroedingerPicture]{}}}{\braketHelper{N}{\picturePsi[\schroedingerPicture]{}}} -  \frac{\bracketHelper{N}{\VhamiltonianOf[\interactionPicture]{t_1}}{\picturePsi[\schroedingerPicture]{}} \cdot \bracketHelper{N}{\VhamiltonianOf[\interactionPicture]{t_2}}{\picturePsi[\schroedingerPicture]{}}}{\braketHelper{N}{\picturePsi[\schroedingerPicture]{}}^2}
        \right)
        \hspace{-1.3cm} % put the equation number where there is space
        \\
        %
        &\stackrel{\ref{eq:base-expansion-state}}{=}
        - \frac{1}{2} 
        \left[
        \lsum[K] \frac{\psiN[K]}{\psiN} \int\limits_0^t \int\limits_0^t        \bracketHelper{N}{\timeOrderingOperator
        \VhamiltonianOf[\interactionPicture]{t_1}\VhamiltonianOf[\interactionPicture]{t_2}
        }{K}            \mathrm{d}t_1\mathrm{d}t_2 \right. \\
        &\stackrel{\phantom{\ref{eq:base-expansion-state}}}{-} 
        \left.
        \lsum[M]\lsum[L]
        \frac{\psiN[M]\psiN[L]}{\psiN{}^2}
        \int\limits_0^t \bracketHelper{N}{\VhamiltonianOf[\interactionPicture]{t_1}}{M}  \mathrm{d}t_1
        \cdot
        \int\limits_0^t \bracketHelper{N}{\VhamiltonianOf[\interactionPicture]{t_2}}{L}  \mathrm{d}t_2 \right]\\
        %
        &\stackrel{\ref{eq:interaction-picture-v-ham-full}}{=}
        - \frac{J^2}{2} 
        \left[
        \lsum[K]\lsum[X,Y \in \left\{\text{A}, \text{B}, \text{C}\right\}] \fullneighborsum{l}{m}\fullneighborsum{j}{k}
        \frac{\psiN[K]}{\psiN} \right. \\&\quad
        \int\limits_0^t \int\limits_0^t        \bracketHelper{N}{\timeOrderingOperator
        \VhamiltonianAnalyticalPart{$X$}{l}{m}{t_1} \VhamiltonianOperatorPart[t_1]{$X$}{l}{m}
        \cdot 
        \VhamiltonianAnalyticalPart{$Y$}{j}{k}{t_2} \VhamiltonianOperatorPart[t_2]{$Y$}{j}{k}
        }{K}            \mathrm{d}t_1\mathrm{d}t_2 \\
        &\stackrel{\phantom{\ref{eq:interaction-picture-v-ham-full}}}{-} 
        \lsum[M]\lsum[L]\lsum[V,W \in \left\{\text{A}, \text{B}, \text{C}\right\}]
        \fullneighborsum{o}{p}\fullneighborsum{q}{r}
        \frac{\psiN[M]\psiN[L]}{\psiN{}^2}\\&\quad
        \left.
        \int\limits_0^t \bracketHelper{N}{\VhamiltonianAnalyticalPart{$V$}{o}{p}{t_1} \cdot \VhamiltonianOperatorPart{$V$}{o}{p}}{M}  \mathrm{d}t_1
        \cdot
        \int\limits_0^t \bracketHelper{N}{\VhamiltonianAnalyticalPart{$W$}{q}{r}{t_2} \cdot \VhamiltonianOperatorPart{$W$}{q}{r}}{L}  \mathrm{d}t_2 \right]\\
        %
        &=
        - \frac{J^2}{2} \left[ 
        \lsum[K]\lsum[X,Y \in \left\{\text{A}, \text{B}, \text{C}\right\}] \fullneighborsum{l}{m}\fullneighborsum{j}{k}
        \frac{\psiN[K]}{\psiN} 
        \right.
        \\&\quad
        \int\limits_0^t \int\limits_0^t  
        \VhamiltonianAnalyticalPart{$X$}{l}{m}{t_1}
        \cdot 
        \VhamiltonianAnalyticalPart{$Y$}{j}{k}{t_2}
        \bracketHelper{N}{\timeOrderingOperator
            \VhamiltonianOperatorPart[t_1]{$X$}{l}{m}
        \cdot 
            \VhamiltonianOperatorPart[t_2]{$Y$}{j}{k}
        }{K}            \mathrm{d}t_1\mathrm{d}t_2 \\
        & - 
        \lsum[M]\lsum[L]\lsum[V,W \in \left\{\text{A}, \text{B}, \text{C}\right\}]
        \fullneighborsum{o}{p}\fullneighborsum{q}{r}
        \frac{\psiN[M]\psiN[L]}{\psiN{}^2}\\&\quad
        \left.
            (-1) \cdot \VhamiltonianAnalyticalPartIntegrated{$V$}{o}{p}{t} 
            \cdot
            \VhamiltonianAnalyticalPartIntegrated{$W$}{q}{r}{t}
            \bracketHelper{N}{\VhamiltonianOperatorPart{$V$}{o}{p}}{M}
            \bracketHelper{N}{\VhamiltonianOperatorPart{$W$}{q}{r}}{L}
            \vphantom{
                \lsum[M]
            }
        \right]
    \end{split}
\end{equation}

One might notice in the second to last step the \VhamiltonianOperatorPart[t_\ast]{$\ast$}{\ast}{\ast} is indicated with a $t_\ast$ to keep its time-ordering association, even after writing out the \VhamiltonianOf[\interactionPicture]{t_\ast}.
In the last step an additional -1 appears, because of a $-1 = i \cdot i$ that was introduced to allow for integration of \VhamiltonianAnalyticalPart{$\ast$}{\ast}{\ast}{t_\ast} like in \autoref{eq:hn-integrated-first-order}.

\begin{equation}
    \label{eq:u-eps-substitution}
    \begin{split}
        \epsl-\epsl[m] &= \ueps{A}{l}{m}\\
        \epsl-\epsl[m]+U &= \ueps{B}{l}{m}\\
        \epsl-\epsl[m]-U &= \ueps{C}{l}{m}\\
    \end{split}
\end{equation}

Inverting the steps performed in \cite{dissectTimeOrderingIntegrals}, the integrals can be split with the time-ordering-operator for evaluation in \autoref{eq:split-integral-time-ordering}.

\begin{equation}
    \label{eq:split-integral-time-ordering}
    \begin{split}
        &\int\limits_0^t \mathrm{d}t_1 \int\limits_0^{t_1} \mathrm{d}t_2 A(t_1) B(t_2)  = \int\limits_0^t \mathrm{d}t_1 \int\limits_0^{t_1} \mathrm{d}t_2  \timeOrderingOperator A(t_1) B(t_2) \\
        &\int\limits_0^t \mathrm{d}t_1 \int\limits_{t_1}^t \mathrm{d}t_2  B(t_2) A(t_1)  = \int\limits_0^t \mathrm{d}t_1 \int\limits_{t_1}^t \mathrm{d}t_2  \timeOrderingOperator A(t_1) B(t_2) \\\\
        %
        &\int\limits_0^t \mathrm{d}t_1 \left( \int\limits_0^{t_1} \mathrm{d}t_2 \timeOrderingOperator A(t_1) B(t_2) +  \int\limits_{t_1}^t \mathrm{d}t_2 \timeOrderingOperator A(t_1) B(t_2)\right) =
        \int\limits_0^t \mathrm{d}t_1\int\limits_0^t \mathrm{d}t_2 \timeOrderingOperator A(t_1) B(t_2)
    \end{split}
\end{equation}

What remains is the evaluation of the integral that needs to respect the time-ordering.

\begin{equation}
    \label{eq:time-ordered-integral}
    \begin{split}
        &\int\limits_0^t \int\limits_0^t  
        \VhamiltonianAnalyticalPart{$X$}{l}{m}{t_1}
        \cdot 
        \VhamiltonianAnalyticalPart{$Y$}{j}{k}{t_2}
        \bracketHelper{N}{\timeOrderingOperator
            \VhamiltonianOperatorPart[t_1]{$X$}{l}{m}
        \cdot 
            \VhamiltonianOperatorPart[t_2]{$Y$}{j}{k}
        }{K}            \mathrm{d}t_1\mathrm{d}t_2\\
        \stackrel{\ref{eq:u-eps-substitution},\,\ref{eq:interaction-picture-v-ham-parts}}{=}
        %
        &\int\limits_0^t \int\limits_0^t  
        e^{
            i \cdot \ueps{$X$}{l}{m} \cdot t_1
        }
        \cdot 
        e^{
            i \cdot \ueps{$Y$}{j}{k} \cdot t_2
        }
        \bracketHelper{N}{\timeOrderingOperator
            \VhamiltonianOperatorPart[t_1]{$X$}{l}{m}
        \cdot 
            \VhamiltonianOperatorPart[t_2]{$Y$}{j}{k}
        }{K}            \mathrm{d}t_1\mathrm{d}t_2\\
        \stackrel{\ref{eq:split-integral-time-ordering}}{=}
        %
        &\int\limits_0^t  \int\limits_0^{t_1}
        e^{
            i \cdot \ueps{$X$}{l}{m} \cdot t_1
        }
        e^{
            i \cdot \ueps{$Y$}{j}{k} \cdot t_2
        }
        \mathrm{d}t_2
        \mathrm{d}t_1
        \cdot 
        \bracketHelper{N}{
            \VhamiltonianOperatorPart[]{$X$}{l}{m}
            \VhamiltonianOperatorPart[]{$Y$}{j}{k}
        }{K}           \\ +
        &\int\limits_0^t  \int\limits_{t_1}^t
        e^{
            i \cdot \ueps{$X$}{l}{m} \cdot t_1
        }
        e^{
            i \cdot \ueps{$Y$}{j}{k} \cdot t_2
        }
        \mathrm{d}t_2
        \mathrm{d}t_1
        \cdot 
        \bracketHelper{N}{
            \VhamiltonianOperatorPart[]{$Y$}{j}{k}
            \VhamiltonianOperatorPart[]{$X$}{l}{m}
        }{K} 
    \end{split}
\end{equation}

\begin{equation}
    \label{eq:time-ordered-integral-part1}
    \begin{split}
        \Gamma_1 &\left(X,\,l,\,m,\,Y,\,j,\,k\right) = \\=&\int\limits_0^t  \int\limits_0^{t_1}
        e^{
            i \cdot \ueps{$X$}{l}{m} \cdot t_1
        }
        e^{
            i \cdot \ueps{$Y$}{j}{k} \cdot t_2
        }
        \mathrm{d}t_2
        \mathrm{d}t_1 = 
        %
        \int\limits_0^t 
        e^{
            i \cdot \ueps{$X$}{l}{m} \cdot t_1
        }
        \int\limits_0^{t_1}
        e^{
            i \cdot \ueps{$Y$}{j}{k} \cdot t_2
        }
        \mathrm{d}t_2
        \mathrm{d}t_1 \\=
        %
        &\int\limits_0^t 
        e^{
            i \cdot \ueps{$X$}{l}{m} \cdot t_1
        }
        \frac{
            e^{
                i \cdot \ueps{$Y$}{j}{k} \cdot t
            }
            -1
        }{
            i \cdot\ueps{$Y$}{j}{k}
        }
        \mathrm{d}t_1 =
        %
        \frac{
            e^{
                i \cdot \ueps{$X$}{l}{m} \cdot t
            }
            -1
        }{
            \ueps{$X$}{l}{m} \cdot\ueps{$Y$}{j}{k}
        } + 
        \int\limits_0^t 
        \frac{
            e^{
                i \cdot \left(\ueps{$X$}{l}{m} + \ueps{$Y$}{j}{k} \right) \cdot t_1
            }
        }{
            i \cdot\ueps{$Y$}{j}{k}
        }
        \mathrm{d}t_1\\=
        %
        &\begin{cases}
            \ueps{$X$}{l}{m} + \ueps{$Y$}{j}{k} = 0: \quad
            \frac{
                e^{
                    i \cdot \ueps{$X$}{l}{m} \cdot t
                }
                -1
            }{
                \ueps{$X$}{l}{m} \cdot\ueps{$Y$}{j}{k}
            } - \frac{i \cdot t}{\ueps{$Y$}{j}{k}}            \\
            %%%%%
            \ueps{$X$}{l}{m} + \ueps{$Y$}{j}{k} \neq 0: \quad
            \frac{
                e^{
                    i \cdot \ueps{$X$}{l}{m} \cdot t
                }
                -1
            }{
                \ueps{$X$}{l}{m} \cdot\ueps{$Y$}{j}{k}
            } - \frac{
                e^{
                    i \cdot \left(\ueps{$X$}{l}{m} + \ueps{$Y$}{j}{k} \right) \cdot t
                }
                -1
            }{\ueps{$Y$}{j}{k} \cdot \left(\ueps{$X$}{l}{m} + \ueps{$Y$}{j}{k} \right)}        
        \end{cases}
    \end{split}
\end{equation}


\begin{equation}
    \label{eq:time-ordered-integral-part2}
    \begin{split}
        \Gamma_2&\left(X,\,l,\,m,\,Y,\,j,\,k\right) =\\ =&\int\limits_0^t  \int\limits_{t_1}^t
        e^{
            i \cdot \ueps{$X$}{l}{m} \cdot t_1
        }
        e^{
            i \cdot \ueps{$Y$}{j}{k} \cdot t_2
        }
        \mathrm{d}t_2
        \mathrm{d}t_1 = 
        %
        \int\limits_0^t 
        e^{
            i \cdot \ueps{$X$}{l}{m} \cdot t_1
        }
        \int\limits_{t_1}^t
        e^{
            i \cdot \ueps{$Y$}{j}{k} \cdot t_2
        }
        \mathrm{d}t_2
        \mathrm{d}t_1 \\=
        %
        &\int\limits_0^t 
        e^{
            i \cdot \ueps{$X$}{l}{m} \cdot t_1
        }
        \frac{
            e^{
                i \cdot \ueps{$Y$}{j}{k} \cdot t
            }
            -
            e^{
                i \cdot \ueps{$Y$}{j}{k} \cdot t_1
            }
        }{
            i \cdot\ueps{$Y$}{j}{k}
        }
        \mathrm{d}t_1 \\=
        %
        &\frac{
            e^{
                i \cdot \ueps{$Y$}{j}{k} \cdot t
            }
            -
            e^{
                i \cdot \left(\ueps{$X$}{l}{m} + \ueps{$Y$}{j}{k} \right) \cdot t
            }
        }{
            \ueps{$X$}{l}{m} \cdot\ueps{$Y$}{j}{k}
        } -
        \int\limits_0^t 
        \frac{
            e^{
                i \cdot \left(\ueps{$X$}{l}{m} + \ueps{$Y$}{j}{k} \right) \cdot t_1
            }
        }{
            i \cdot\ueps{$Y$}{j}{k}
        }
        \mathrm{d}t_1\\=
        %
        &\begin{cases}
            \ueps{$X$}{l}{m} + \ueps{$Y$}{j}{k} = 0: \quad
            \frac{
                e^{
                    i \cdot \ueps{$Y$}{j}{k} \cdot t
                }
                -
                e^{
                    i \cdot \left(\ueps{$X$}{l}{m} + \ueps{$Y$}{j}{k} \right) \cdot t
                }
            }{
                \ueps{$X$}{l}{m} \cdot\ueps{$Y$}{j}{k}
            } + \frac{i \cdot t}{\ueps{$Y$}{j}{k}}            \\
            %%%%%
            \ueps{$X$}{l}{m} + \ueps{$Y$}{j}{k} \neq 0: \quad
            \frac{
                e^{
                    i \cdot \ueps{$Y$}{j}{k} \cdot t
                }
                -
                e^{
                    i \cdot \left(\ueps{$X$}{l}{m} + \ueps{$Y$}{j}{k} \right) \cdot t
                }
            }{
                \ueps{$X$}{l}{m} \cdot\ueps{$Y$}{j}{k}
            } + \frac{
                e^{
                    i \cdot \left(\ueps{$X$}{l}{m} + \ueps{$Y$}{j}{k} \right) \cdot t
                }
                -1
            }{\ueps{$Y$}{j}{k} \cdot \left(\ueps{$X$}{l}{m} + \ueps{$Y$}{j}{k} \right)}  
        \end{cases}
    \end{split}
\end{equation}

Adding \autoref{eq:time-ordered-integral-part1} and \autoref{eq:time-ordered-integral-part2}, \autoref{eq:time-ordered-integral-parts-sum} can be verified, which is necessary when comparing with the integration result of the second half of \autoref{eq:hn-integrated-second-order}.

\begin{equation}
    \label{eq:time-ordered-integral-parts-sum}
    \begin{split}
        &\Gamma_1 \left(X,\,l,\,m,\,Y,\,j,\,k\right) + \Gamma_2 \left(X,\,l,\,m,\,Y,\,j,\,k\right) = \\
        &=
        \frac{
            e^{
                i \cdot \ueps{$X$}{l}{m} \cdot t
            }
            -1
        }{
            \ueps{$X$}{l}{m} \cdot\ueps{$Y$}{j}{k}
        }
         + 
        \frac{
            e^{
                i \cdot \ueps{$Y$}{j}{k} \cdot t
            }
            -
            e^{
                i \cdot \left(\ueps{$X$}{l}{m} + \ueps{$Y$}{j}{k} \right) \cdot t
            }
        }{
            \ueps{$X$}{l}{m} \cdot\ueps{$Y$}{j}{k}
        } \\
        & = 
        -\frac{
            e^{
                i \cdot \ueps{$X$}{l}{m} \cdot t
            }
            -
            1
        }{
            \ueps{$X$}{l}{m}
        }
        \cdot 
        \frac{
            e^{
                i \cdot \ueps{$Y$}{j}{k} \cdot t
            }
            -
            1
        }{
            \ueps{$Y$}{j}{k}
        }
         = 
         - \VhamiltonianAnalyticalPartIntegrated{$X$}{l}{m}{t} 
         \cdot
         \VhamiltonianAnalyticalPartIntegrated{$Y$}{j}{k}{t}
    \end{split}
\end{equation}

With the integrals rewritten, \autoref{eq:hn-integrated-second-order} can be re-arranged to obtain the final result: 

\begin{equation}
    \label{eq:hn-integrated-second-order-final}
    \begin{split}
        &\HNOftOrder{2} \stackrel{\ref{eq:time-ordered-integral},\,\ref{eq:time-ordered-integral-part1},\,\ref{eq:time-ordered-integral-part2}}{=} 
        - \frac{J^2}{2} 
        \lsum[X,Y \in \left\{\text{A}, \text{B}, \text{C}\right\}] \fullneighborsum{l}{m}\fullneighborsum{j}{k}\\
        %
        \left[         \vphantom{\lsum[K]}        \right.
        &\phantom{+}
        \lsum[K]
        \frac{\psiN[K]}{\psiN} 
        \Gamma_1 \left(X,\,l,\,m,\,Y,\,j,\,k\right)
        \cdot 
        \bracketHelper{N}{
            \VhamiltonianOperatorPart[]{$X$}{l}{m}
            \cdot 
            \VhamiltonianOperatorPart[]{$Y$}{j}{k}
            }{K}
        \\
        &+ 
        \lsum[K]
        \frac{\psiN[K]}{\psiN} 
        \Gamma_2 \left(X,\,l,\,m,\,Y,\,j,\,k\right)
        \cdot 
        \bracketHelper{N}{
            \VhamiltonianOperatorPart[]{$Y$}{j}{k}
        \cdot 
            \VhamiltonianOperatorPart[]{$X$}{l}{m}
        }{K} \\
        &-
        \left.
        \lsum[M]\lsum[L]
        \frac{\psiN[M]\psiN[L]}{\psiN{}^2}
            \left[\Gamma_1 \left(X,\,l,\,m,\,Y,\,j,\,k\right) + \Gamma_2 \left(X,\,l,\,m,\,Y,\,j,\,k\right)\right] \cdot
            \bracketHelper{N}{\VhamiltonianOperatorPart{$X$}{l}{m}}{M}
            \cdot
            \bracketHelper{N}{\VhamiltonianOperatorPart{$Y$}{j}{k}}{L}
        \right]
        \hspace{-1cm} % put the equation number where there is space
    \end{split}
\end{equation}

As for the first order perturbation, this seems to have made this computationally more expensive than the full diagonalization.
But again, because of the limited interaction range (for second order the range is a sphere of at most two nearest-neighbor-hops), this can be efficiently evaluated in 
\bigo{\text{\#}(\text{sites}) \cdot \text{\#}(\text{nearest neighbors})^2} by using the sparse nature of the operators.

The Operator $\VhamiltonianOperatorPart[]{$X$}{l}{m}\VhamiltonianOperatorPart[]{$Y$}{j}{k}$ is actually already far too complicated to write down any of its mixed terms in an ordered, closed form.
To get around this, in the implementation the first half is applied to the left, the state modified accordingly and then the second half is applied - as this is much more efficient than calculating the closed form.

See \filepath{\cite{selfCode}}{/calculation-helpers/generatesecondorder.py} on how this could be done.

Close inspection discloses \autoref{eq:hn-integrated-second-order-final} being $0$ for all cases, where $l \neq j \land l \neq k \land m \neq j \land n \neq k$, if specific properties hold for \psiN[\ast].
In a subsequent section this optimization will be used.