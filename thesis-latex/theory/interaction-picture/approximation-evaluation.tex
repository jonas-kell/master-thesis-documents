As \autoref{eq:cumulant-expansion} reveals, by this process the problem was converted into finding $\VhamiltonianOf[\interactionPicture]{t'}$ to be able to calculate the first two orders of the expansion.

Operators in the Interaction Picture can be derived from their Schrödinger Picture variants like previously shown in \autoref{eq:time-evolution-operator-interaction}. 
With this, one can derive a useful expression for their \emph{equation of motion} (\autoref{eq:schroedinger-op-to-interaction-op}).

\begin{equation}
    \label{eq:schroedinger-op-to-interaction-op}
    \begin{split}
        \difft{}\AopOfT[\interactionPicture] &\stackrel{\phantom{\ref{eq:time-evolution-operator-interaction}}}{=} i \left[\HzeroHamiltonian[\schroedingerPicture], \AopOfT[\interactionPicture]\right]
        %
        \stackrel{\ref{eq:time-evolution-operator-interaction}}{=}
        i \left[\HzeroHamiltonian[\schroedingerPicture],\,e^{i \HzeroHamiltonian[\schroedingerPicture]{} t} \Aop[\schroedingerPicture] e^{-i \HzeroHamiltonian[\schroedingerPicture]{} t}\right]\\
        %
        &\stackrel{\phantom{\ref{eq:time-evolution-operator-interaction}}}{=}
        i e^{i \HzeroHamiltonian[\schroedingerPicture]{} t}\left[\HzeroHamiltonian[\schroedingerPicture],\, \Aop[\schroedingerPicture]\right]e^{-i \HzeroHamiltonian[\schroedingerPicture]{} t}
        %
        \stackrel{\phantom{\ref{eq:time-evolution-operator-interaction}}}{=}
        i \left\{\left[\HzeroHamiltonian[\schroedingerPicture],\, \Aop[\schroedingerPicture]\right]\right\}(t)
    \end{split}
\end{equation}

\autoref{eq:schroedinger-op-to-interaction-op} also defines the operator $\left\{\cdot \right\}(t)$, which simply is $e^{i \HzeroHamiltonian[\schroedingerPicture]{} t} \left\{\cdot \right\} e^{-i \HzeroHamiltonian[\schroedingerPicture]{} t}$.
This section uses $(t)$ for indicating that operators in the Interaction Picture have a time dependence. 
This is superficial and only for better readability, as in fact $\hopOfT[\interactionPicture]{m}{\dagger} = \hop[\interactionPicture]{m}{\dagger} = \left\{\hop[\schroedingerPicture]{m}{\dagger} \right\}(t)$.

If now $\left[\HzeroHamiltonian[\schroedingerPicture],\, \Aop[\schroedingerPicture]\right]$ now is a function of $\Aop[\schroedingerPicture]$, one gets a \emph{differential equation} that can be solved to obtain $\AopOfT[\interactionPicture]$.

For the repetitive calculation of the objects of type $\left[\HzeroHamiltonian[\schroedingerPicture],\, \Aop[\schroedingerPicture]\right]$ for various \Aop[\schroedingerPicture], the tool \emph{Math-Manipulator} was specifically developed. 
Like the transformations in \ref{sec:particles}, calculations in the following section have been validated or calculated from scratch in this tool.
The relevant files can be found in \filepath{\cite{selfMathManipulatorCalculations}}{/calculate-V}, and \filepath{\cite{selfMathManipulatorCalculations}}{/time-evolution}.
The equations \ref{eq:stuff-from-math-manipulator-dd} and \ref{eq:stuff-from-math-manipulator-cm} list exemplary results that were computed with the tool.

\begin{equation}
    \label{eq:stuff-from-math-manipulator-dd}
    \difft{}\left(\dop[\interactionPicture]{m}{\dagger}\dop[\interactionPicture]{m}{}\right)(t)
    = i \left\{\left[\HzeroHamiltonian[],\,\dop[\schroedingerPicture]{m}{\dagger}\dop[\schroedingerPicture]{m}{}\right]\right\}(t) \stackrel{\text{MM}}{=} 0 \quad \Rightarrow \quad \left(\dop[\interactionPicture]{m}{\dagger}\dop[\interactionPicture]{m}{}\right)(t) = \dop[\schroedingerPicture]{m}{\dagger}\dop[\schroedingerPicture]{m}{}
\end{equation}

\begin{equation}
    \label{eq:stuff-from-math-manipulator-cm}
    \begin{split}
        \difft{}\hopOfT[\interactionPicture]{m}{\dagger}
        &\stackrel{\phantom{\ref{eq:time-evolution-operator-interaction}}}{=} i \left\{\left[\HzeroHamiltonian[]{},\,\hop[\schroedingerPicture]{m}{\dagger}\right]\right\}(t) \stackrel{\text{MM}}{=} 
        i \left\{ \left(\epsl[m] + U \dop[\schroedingerPicture]{m}{\dagger}\dop[\schroedingerPicture]{m}{}\right)\hop[\schroedingerPicture]{m}{}\right\}(t)\\
        &\stackrel{\phantom{\ref{eq:time-evolution-operator-interaction}}}{=}i e^{i \HzeroHamiltonian[\schroedingerPicture]{} t} \left(\epsl[m] + U \dop[\schroedingerPicture]{m}{\dagger}\dop[\schroedingerPicture]{m}{}\right)\hop[\schroedingerPicture]{m}{}e^{-i \HzeroHamiltonian[\schroedingerPicture]{} t}\\
        &\stackrel{\phantom{\ref{eq:time-evolution-operator-interaction}}}{=}i \left(\epsl[m] + U e^{i \HzeroHamiltonian[\schroedingerPicture]{} t}\dop[\schroedingerPicture]{m}{\dagger}\dop[\schroedingerPicture]{m}{}e^{-i \HzeroHamiltonian[\schroedingerPicture]{} t}\right)e^{i \HzeroHamiltonian[\schroedingerPicture]{} t}\hop[\schroedingerPicture]{m}{}e^{-i \HzeroHamiltonian[\schroedingerPicture]{} t}\\
        &\stackrel{\ref{eq:time-evolution-operator-interaction}}{=}
        i \left(\epsl[m] + U \left(\dop[\interactionPicture]{m}{\dagger}\dop[\interactionPicture]{m}{}\right)(t)\right) \hopOfT[\interactionPicture]{m}{\dagger}\\
    \end{split}
\end{equation}

With this and the idempotence-relation \autoref{eq:hc-counting-op-idempotent} one can first derive \autoref{eq:operator-out-from-exponent} and finally use an exponential function as a natural ansatz to solve \autoref{eq:stuff-from-math-manipulator-cm} with \autoref{eq:the-final-time-dependence-cop}.

\begin{equation}
    \label{eq:operator-out-from-exponent}
    \begin{split}
        e^{a \cdot \withspinhcop[\schroedingerPicture]{l}{\sigma}{\dagger}\withspinhcop[\schroedingerPicture]{l}{\sigma}{}} &= \sum\limits_{m=0}^\infty \frac{a^m \left(\withspinhcop[\schroedingerPicture]{l}{\sigma}{\dagger}\withspinhcop[\schroedingerPicture]{l}{\sigma}{}\right)^m}{m!}
        \stackrel{\ref{eq:hc-counting-op-idempotent}}{=} 1 + \left[\sum\limits_{m=1}^\infty \frac{a^m}{m!}\right]\cdot \left(\withspinhcop[\schroedingerPicture]{l}{\sigma}{\dagger}\withspinhcop[\schroedingerPicture]{l}{\sigma}{}\right)\\
        &= 1 + \left[\sum\limits_{m=0}^\infty \frac{a^m}{m!}-1\right]\cdot \left(\withspinhcop[\schroedingerPicture]{l}{\sigma}{\dagger}\withspinhcop[\schroedingerPicture]{l}{\sigma}{}\right)
         = 1 + \left(e^a-1\right)\cdot \withspinhcop[\schroedingerPicture]{l}{\sigma}{\dagger}\withspinhcop[\schroedingerPicture]{l}{\sigma}{}
    \end{split}
\end{equation}


\begin{equation}
    \label{eq:the-final-time-dependence-cop}
    \begin{split}
        \stackrel{\ref{eq:stuff-from-math-manipulator-cm},\, \ref{eq:stuff-from-math-manipulator-dd}}{\Longrightarrow}
        \hopOfT[\interactionPicture]{m}{\dagger}
        &\stackrel{\phantom{\ref{eq:stuff-from-math-manipulator-dd}}}{=}
        e^{i \cdot \epsl[m]\cdot  t + i \cdot U \left(\dop[\interactionPicture]{m}{\dagger}\dop[\interactionPicture]{m}{}\right)(t) \cdot  t}  \hop[\schroedingerPicture]{m}{\dagger} 
        \stackrel{\ref{eq:stuff-from-math-manipulator-dd}}{=}
        e^{i \cdot \epsl[m]\cdot  t + i \cdot U \cdot \dop[\schroedingerPicture]{m}{\dagger}\dop[\schroedingerPicture]{m}{} \cdot  t}  \hop[\schroedingerPicture]{m}{\dagger} \\
        &\stackrel{\ref{eq:operator-out-from-exponent}}{=}
        e^{i \cdot \epsl[m]\cdot  t}\left(1 + \left(e^{i  \cdot U  \cdot  t}-1\right) \dop[\schroedingerPicture]{m}{\dagger}\dop[\schroedingerPicture]{m}{}\right)  \hop[\schroedingerPicture]{m}{\dagger}
    \end{split}
\end{equation}

From analogous calculations follows:

\begin{equation*}
    \begin{split}
        \hopOfT[\interactionPicture]{m}{\dagger}  &= e^{i \cdot \epsl[m]\cdot  t}\left(1 + \left(e^{i  \cdot U  \cdot  t}-1\right) \dop[\schroedingerPicture]{m}{\dagger}\dop[\schroedingerPicture]{m}{}\right)  \hop[\schroedingerPicture]{m}{\dagger}\\
        \hopOfT[\interactionPicture]{m}{}  &= e^{-i \cdot \epsl[m]\cdot  t}\left(1 + \left(e^{-i  \cdot U  \cdot  t}-1\right) \dop[\schroedingerPicture]{m}{\dagger}\dop[\schroedingerPicture]{m}{}\right)  \hop[\schroedingerPicture]{m}{}\\
        \dopOfT[\interactionPicture]{m}{\dagger}  &= e^{i \cdot \epsl[m]\cdot  t}\left(1 + \left(e^{i  \cdot U  \cdot  t}-1\right) \hop[\schroedingerPicture]{m}{\dagger}\hop[\schroedingerPicture]{m}{}\right)  \dop[\schroedingerPicture]{m}{\dagger}\\
        \dopOfT[\interactionPicture]{m}{}  &= e^{-i \cdot \epsl[m]\cdot  t}\left(1 + \left(e^{-i  \cdot U  \cdot  t}-1\right) \hop[\schroedingerPicture]{m}{\dagger}\hop[\schroedingerPicture]{m}{}\right)  \dop[\schroedingerPicture]{m}{}\\
    \end{split}
\end{equation*}

Injecting these derivations into \autoref{eq:main-hamiltonian-perturbation-full-sum}, finally \VhamiltonianOfT[\interactionPicture] can be derived in \autoref{eq:interaction-picture-v-ham-full}.

\begin{equation}
    \label{eq:interaction-picture-v-ham-full}
    \begin{split}
        \VhamiltonianOfT[\interactionPicture] &
        \stackrel{\phantom{\text{MM}}}{=}
        \left\{\Vhamiltonian[\schroedingerPicture]\right\}(t) \stackrel{\ref{eq:main-hamiltonian-perturbation-full-sum}}{=}
        - J \cdot \fullneighborsum{l}{m}  \left\{
              \left(\hop[\schroedingerPicture]{l}{\dagger}\hop[\schroedingerPicture]{m}{} + \dop[\schroedingerPicture]{l}{\dagger}\dop[\schroedingerPicture]{m}{} \right)
        \right\}(t) \\
        &\stackrel{\phantom{\text{MM}}}{=}
        -J \cdot \fullneighborsum{l}{m}\left(\hopOfT[\interactionPicture]{l}{\dagger}\hopOfT[\interactionPicture]{m}{} + \dopOfT[\interactionPicture]{l}{\dagger}\dopOfT[\interactionPicture]{m}{} \right)\\
        &\stackrel{\text{MM}}{=}
        %
        -J \cdot \fullneighborsum{l}{m} \left[
            \VhamiltonianAnalyticalPart{A}{l}{m}{t} \cdot \VhamiltonianOperatorPart{A}{l}{m} + 
            \VhamiltonianAnalyticalPart{B}{l}{m}{t} \cdot \VhamiltonianOperatorPart{B}{l}{m} + 
            \VhamiltonianAnalyticalPart{C}{l}{m}{t} \cdot \VhamiltonianOperatorPart{C}{l}{m} 
        \right]
    \end{split}
\end{equation}

It is possible to express these operator-strings as single hoppings, decorated with number operators. One receives \autoref{eq:interaction-picture-v-ham-parts}, that is symmetrical in \up and \down (with $\overline{\up} = \down$ and  $\overline{\down} = \up$).

\begin{equation}
    \label{eq:interaction-picture-v-ham-parts}
    \begin{split}
        \VhamiltonianAnalyticalPart{A}{l}{m}{t} \stackrel{\text{MM}}{=} e^{i\cdot \left(\epsl-\epsl[m]\right)\cdot t} \qquad
        \VhamiltonianOperatorPart{A}{l}{m} &\stackrel{\text{MM}}{=} 
        \lsum[\sigma \in \left\{\up,\,\down\right\}]
        \withspinhcop[\schroedingerPicture]{l}{\sigma}{\dagger}\withspinhcop[\schroedingerPicture]{m}{\sigma}{}
        \left(
            1+
            2 \cdot \nop[\schroedingerPicture]{l}{\overline{\sigma}}\nop[\schroedingerPicture]{m}{\overline{\sigma}}
            - \nop[\schroedingerPicture]{l}{\overline{\sigma}}
            - \nop[\schroedingerPicture]{m}{\overline{\sigma}}
        \right)
        \\
        \VhamiltonianAnalyticalPart{B}{l}{m}{t} \stackrel{\text{MM}}{=} e^{i\cdot \left(\epsl-\epsl[m] + U\right)\cdot t} \qquad
        \VhamiltonianOperatorPart{B}{l}{m} &\stackrel{\text{MM}}{=} 
        \lsum[\sigma \in \left\{\up,\,\down\right\}]        
        \withspinhcop[\schroedingerPicture]{l}{\sigma}{\dagger}\withspinhcop[\schroedingerPicture]{m}{\sigma}{}
        \left(
            \nop[\schroedingerPicture]{l}{\overline{\sigma}}
            - \nop[\schroedingerPicture]{l}{\overline{\sigma}}\nop[\schroedingerPicture]{m}{\overline{\sigma}}
        \right)
        \\
        \VhamiltonianAnalyticalPart{C}{l}{m}{t} \stackrel{\text{MM}}{=} e^{i\cdot \left(\epsl-\epsl[m] - U\right)\cdot t} \qquad
        \VhamiltonianOperatorPart{C}{l}{m} &\stackrel{\text{MM}}{=} 
        \lsum[\sigma \in \left\{\up,\,\down\right\}]
        \withspinhcop[\schroedingerPicture]{l}{\sigma}{\dagger}\withspinhcop[\schroedingerPicture]{m}{\sigma}{}
        \left(
            \nop[\schroedingerPicture]{m}{\overline{\sigma}}
            - \nop[\schroedingerPicture]{m}{\overline{\sigma}}\nop[\schroedingerPicture]{l}{\overline{\sigma}}
        \right)
        \\
    \end{split}
\end{equation}

The expressions for the dressed operators in \autoref{eq:interaction-picture-v-ham-parts} can be rearranged, to reflect more their physical effects.
\autoref{eq:interaction-picture-v-ham-parts-simplified} can be obtained and verified by comparing the measurements in a truth table (the \nop[\schroedingerPicture]{}{} measure either a $0$ or $1$ on a state, without modifying it).

\begin{equation}
    \label{eq:interaction-picture-v-ham-parts-simplified}
    \begin{split}
        \VhamiltonianOperatorPart{A}{l}{m} &\stackrel{\text{MM}}{=} 
        \lsum[\sigma \in \left\{\up,\,\down\right\}]
        \withspinhcop[\schroedingerPicture]{l}{\sigma}{\dagger}\withspinhcop[\schroedingerPicture]{m}{\sigma}{}
        \left(\nop[\schroedingerPicture]{l}{\overline{\sigma}}
            \equivalentOperator 
            \nop[\schroedingerPicture]{m}{\overline{\sigma}}
        \right)
        \\
        \VhamiltonianOperatorPart{B}{l}{m} &\stackrel{\text{MM}}{=} 
        \lsum[\sigma \in \left\{\up,\,\down\right\}]        
        \withspinhcop[\schroedingerPicture]{l}{\sigma}{\dagger}\withspinhcop[\schroedingerPicture]{m}{\sigma}{}
        \nop[\schroedingerPicture]{l}{\overline{\sigma}}
        \left(
            1 - \nop[\schroedingerPicture]{m}{\overline{\sigma}}
        \right)
        \\
        \VhamiltonianOperatorPart{C}{l}{m} &\stackrel{\text{MM}}{=} 
        \lsum[\sigma \in \left\{\up,\,\down\right\}]
        \withspinhcop[\schroedingerPicture]{l}{\sigma}{\dagger}\withspinhcop[\schroedingerPicture]{m}{\sigma}{}
        \nop[\schroedingerPicture]{m}{\overline{\sigma}}
        \left(
           1 - \nop[\schroedingerPicture]{l}{\overline{\sigma}}
        \right)
        \\
    \end{split}
\end{equation}

The \equivalentOperator here means \glqq $1$ if the measurements of the operators are equivalent, $0$ otherwise\grqq.

\paragraph*{Base Energy}\makebox{}\\

In \autoref{eq:time-evolution-target}, the effective Hamiltonian $\HeffOft = -i E_0(N) t + \HNOft$ was defined.
This construct will be necessary in the following sections. 
$E_0(N)$ is luckily calculated quite swiftly, as one can see in \autoref{eq:base-energy} (with $n_{l,\,\sigma}$ - the occupation-number in $\ketN$, not the operator).

\begin{equation}
    \label{eq:base-energy}
    \begin{split}
        E_0(N) &\stackrel{\phantom{\ref{eq:main-hamiltonian-h0}}}{=} \frac{\bracketHelper{N}{\HzeroHamiltonian[]}{N}}{\braketHelper{N}{N}} \\
        %
        &\stackrel{\ref{eq:main-hamiltonian-h0}}{=} \bracketHelper{N}{U \cdot \lsum \nop{l}{\up}\nop{l}{\down}}{N} + \bracketHelper{N}{\lsum[l,\,\sigma] \epsl \nop{l}{\sigma}}{N}\\
        %
        &\stackrel{\phantom{\ref{eq:main-hamiltonian-h0}}}{=} U \cdot \lsum n_{l,\,\up}n_{l,\,\down} + \lsum \epsl n_{l,\,\sigma}
    \end{split}
\end{equation}

\paragraph*{Calculate the first order of \HNOft}\makebox{}\\

The last missing element for $\HeffOft = -i E_0(N) t + \HNOft$ now is \HNOft.
At least for the first order, the integration in \autoref{eq:cumulant-expansion} is easily doable, with now knowing the form of \VhamiltonianOf[\interactionPicture]{t}.

The result of the integration can be seen in \autoref{eq:hn-integrated-first-order}.

\begin{equation}
    \label{eq:hn-integrated-first-order}
    \begin{split}
        \HNOftOrder{1} &\stackrel{\ref{eq:cumulant-expansion}}{=} -i \int\limits_0^t \mathrm{d}t' \frac{\bracketHelper{N}{\VhamiltonianOf[\interactionPicture]{t'}}{\picturePsi[\schroedingerPicture]{}}}{\braketHelper{N}{\picturePsi[\schroedingerPicture]{}}}\\
        %
        &\stackrel{\ref{eq:base-expansion-state}}{=}
        -i \frac{1}{\psiN} \int\limits_0^t \mathrm{d}t' \lsum[K] \bracketHelper{N}{\VhamiltonianOf[\interactionPicture]{t'}}{K} \psiN[K]\\
        &\stackrel{\ref{eq:interaction-picture-v-ham-full}}{=}
        i \cdot J \lsum[K] \fullneighborsum{l}{m} \frac{\psiN[K]}{\psiN} \int\limits_0^t \mathrm{d}t' 
        \left[
            e^{i\cdot \left(\epsl-\epsl[m]\right)\cdot t'} \cdot 
            \bracketHelper{N}{
                \VhamiltonianOperatorPart{A}{l}{m} 
            }{K}
            + 
            \right.\\
            &\qquad
            \left.
            e^{i\cdot \left(\epsl-\epsl[m] + U\right)\cdot t'} \cdot 
            \bracketHelper{N}{
                \VhamiltonianOperatorPart{B}{l}{m}
            }{K}
            + 
            e^{i\cdot \left(\epsl-\epsl[m] - U\right)\cdot t'} \cdot 
            \bracketHelper{N}{
                \VhamiltonianOperatorPart{C}{l}{m} 
            }{K}
        \right]
        \\
        %
        &\stackrel{\phantom{\ref{eq:interaction-picture-v-ham-full}}}{=}
        J \lsum[K] \fullneighborsum{l}{m} \frac{\psiN[K]}{\psiN}
        \left[
            \frac{e^{i\cdot \left(\epsl-\epsl[m]\right)\cdot t}-1}{\epsl-\epsl[m]} \cdot 
            \bracketHelper{N}{
                \VhamiltonianOperatorPart{A}{l}{m} 
            }{K}
            + 
            \right.\\
            &\qquad
            \left.
            \frac{
                e^{i\cdot \left(\epsl-\epsl[m] + U\right)\cdot t}-1
            }{\epsl-\epsl[m] + U}
             \cdot 
            \bracketHelper{N}{
                \VhamiltonianOperatorPart{B}{l}{m}
            }{K}
            + 
            \frac{
                e^{i\cdot \left(\epsl-\epsl[m] - U\right)\cdot t}-1
            }{\epsl-\epsl[m] - U}
             \cdot 
            \bracketHelper{N}{
                \VhamiltonianOperatorPart{C}{l}{m} 
            }{K}
        \right]
    \end{split}
\end{equation}

While it seems that $\lsum[K] \fullneighborsum{l}{m}$ is even a larger summation than the previously problematic ones, because the operators $\VhamiltonianOperatorPart{A}{l}{m}$, $\VhamiltonianOperatorPart{B}{l}{m}$ and $\VhamiltonianOperatorPart{C}{l}{m}$ have interaction range limited to nearest neighbors, actually the summation is really sparse and most entries are $0$.
This summation can be efficiently evaluated in \bigo{\text{\#}(\text{sites}) \cdot \text{\#}(\text{nearest neighbors})}.
How this could be done, can be looked up in the implementation \filepath{\cite{selfCode}}{/computation-scripts/hamiltonian.py} in the specialization class \emph{HardcoreBosonicHamiltonian}.