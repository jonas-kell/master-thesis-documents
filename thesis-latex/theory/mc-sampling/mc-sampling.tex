Mathematically, the problem of finding the solution to such a quantum-mechanical many-body model is solved by the calculations provided up to this point. 
However when looking at \autoref{eq:expectation-value}, one can easily see why in practice this is not viable with larger systems. 
The calculation requires summing over all base-states. 
Their number is given in \autoref{eq:system-size} (factor 2 in the exponent for the spin-degree) to be of exponential size in regards to the number of lattice-sites.
Considering a physical material has at least in the order of $10^{23}$ (from \emph{Avogadros-Constant}) degrees of freedom, an \bigo{2^{\text{\#}(\text{sites})}} computational-complexity is not good enough to simulate interestingly-sized systems. 

\begin{equation}
    \label{eq:system-size}
    \text{\#}(\text{states}) = 2^{\text{\#}(\text{sites}) \cdot 2}
\end{equation}

Multiple strategies have already been suggested to circumvent this limitation.
Here, the focus will be put onto randomized algorithms that use \emph{Monte-Carlo sampling} in order generate a limited number of states, representative of the real probability-distribution, in order to not having to sum over the complete \emph{Hilbert-Space}.
The strategy of solving problems of this kind with the strategy of \emph{variational Monte-Carlo} is e.g. suggested in \cite{metropolisAlgorithmAndVariationalMonteCarlo}.
This means starting with a random state and applying modifications to it (= variation).
When the accepting/rejecting of these modifications is coupled to the probability the state has, based on its energy in the systems energy-distribution, the resulting states will have an energy/probability distribution, that resembles the one as if they were sampled from the full Hilbert-Space.
The derivation for the whole process is shown in \cite{monteCarloObservableSampling}.

However computing the normalized probability for the states is still required, in order to compare against a (e.g. thermal) reference-distribution.
As one sees in the probability in \autoref{eq:probability-isolated} that was defined in \autoref{eq:expectation-value}, still a full sum over all states is required to get the normalization factor.
\begin{equation}
    \label{eq:probability-isolated}
    \probabilityOf{N}{t} = \frac{
        \absSquare{e^{\HeffOft}} \absSquare{\psiN} 
    }{
        \lsum[K] \absSquare{e^{\HeffOft[K]}} \absSquare{\psiN[K]} 
    }
\end{equation}
However by employing the \emph{Metropolis-Hastings Algorithm} instead of only the \emph{Metropolis-Algorithm} to decide on the acceptance/rejection of proposed states, one can drop the requirement of normalizing the probability.
In \autoref{eq:transition-probability-final} the \emph{acceptance-ratio} $\alpha$ is derived that is necessary for the Metropolis-Hastings Algorithm to work (as per \cite{metropolisHastingsAlgorithmGeneral}) and decide on the transition from state \ketN[N] to \ketN[{\vphantom{N}\smash{\tilde{N}}}].
The algorithm proposes to generate a new state \ketN[{\vphantom{N}\smash{\tilde{N}}}] and calculate the corresponding $\alpha$. Then it requires generating a random number $u$ between $0$ and $1$, pulled from a uniform distribution.
If $u \leq \alpha$, \ketN[{\vphantom{N}\smash{\tilde{N}}}] is accepted and will be used as the new current state in the next step of the sampling. 
If $u>\alpha$, the proposed state is rejected and \ketN[N] will be used again as the next state.
The function $f(N,t)$, provided in \autoref{eq:proportional-function}, fulfills the necessary property of being proportional to the probability  \probabilityOf{N}{t}.

\begin{equation}
    \label{eq:proportional-function}
    f(N,t) = \absSquare{e^{\HeffOft}} \absSquare{\psiN} \propto \probabilityOf{N}{t}
\end{equation}

\begin{equation}
    \label{eq:transition-probability-final}
    \begin{split}
        \alpha &\stackrel{\phantom{\ref{eq:proportional-function}}}{=} \frac{\probabilityOf{\tilde{N}}{t}}{\probabilityOf{N}{t}} =  \frac{f(\tilde{N},t)}{f(N,t)}
        \stackrel{\ref{eq:proportional-function}}{=}
        \frac{
            \absSquare{e^{\HeffOft[\tilde{N}]}} \absSquare{\psiN[\tilde{N}]}
        }{
            \absSquare{e^{\HeffOft}} \absSquare{\psiN}
        }\\
        &\stackrel{\phantom{\ref{eq:proportional-function}}}{=}
        \frac{\absSquare{\psiN[\tilde{N}]}}{\absSquare{\psiN}}
        \frac{
            e^{\real\left(\HeffOft[\tilde{N}]\right) + i\imaginary\left(\HeffOft[\tilde{N}]\right) +\real\left(\HeffOft[\tilde{N}]\right)-i \imaginary\left(\HeffOft[\tilde{N}]\right)}
        }{
            e^{\real\left(\HeffOft\right) + i\imaginary\left(\HeffOft\right) +\real\left(\HeffOft\right)-i \imaginary\left(\HeffOft\right)}
        }\\
        &\stackrel{\phantom{\ref{eq:proportional-function}}}{=}
        \frac{\absSquare{\psiN[\tilde{N}]}}{\absSquare{\psiN}}
        e^{2\cdot \real\left(\HeffOft[\tilde{N}]\right) - 2\cdot \real\left(\HeffOft\right)}\\
        &\stackrel{\phantom{\ref{eq:proportional-function}}}{=}
        \frac{\absSquare{\psiN[\tilde{N}]}}{\absSquare{\psiN}}
        e^{2\cdot \real\left(\HeffOft[\tilde{N}] - \HeffOft\right)}
    \end{split}
\end{equation}

Multiple factors must be considered, which kind of variation is applied to the states in this schema. 
However probably the most important criteria is how quickly one can evaluate \autoref{eq:transition-probability-final}, as this will be done tens to hundreds of times between each sampled state.

Finally, the method of sampling a selection of distributed states results in \autoref{eq:final-sampling-equation} as a rewrite to \autoref{eq:expectation-value} \cite{monteCarloObservableSampling}.

\begin{equation}
    \label{eq:final-sampling-equation}
    \frac{\bracketHelper{\psiOfT[\schroedingerPicture]}{\ObservableOp}{\psiOfT[\schroedingerPicture]}}{\braketHelper{\psiOfT[\schroedingerPicture]}{\psiOfT[\schroedingerPicture]}} \stackrel{\ref{eq:expectation-value}}{=} 
    \lsum[N]
    \probabilityOf{N}{t}
    \localObservable{N}{t} 
    \stackrel{\text{\cite{monteCarloObservableSampling}}}{\approx} \frac{1}{\left|\left\{N\right\}_\text{MC}\right|} \lsum[\left\{N\right\}_\text{MC}]\localObservable{N}{t}
\end{equation}