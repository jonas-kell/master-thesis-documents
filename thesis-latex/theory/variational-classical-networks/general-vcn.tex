For this application the way of approximating the wave-function is by use of a \emph{classical network}. 
The network parameters now should be obtained via the \emph{time-dependent variational principle} \cite{originalDerivationTimeDependendVariationalPrinciple}.
For this reason the method as a whole structure is dubbed a \emph{variational classical network} (VCN).

In \cite{variationalClassicalNetworksPaper} this strategy is used on comparable applications on similar quantum problems.
The research in \cite{probabilitySamplingRequirementVCN} even applies the method to two-dimensional geometries.
Both reach the same expressions for the advancement of the parameters to later times.
This section reiterates the equations, unifying the parts where different conventions might be used.

\HNOft from \autoref{eq:cumulant-expansion} should be approximated with a parametrized expression \HvcnOft.
\autoref{eq:hamiltonian-vcn} provides the used parametrization, which - compared with \autoref{eq:time-evolution-target} - means $\HeffOft = -i E_0(N) t + \HvcnOft$.

\begin{equation}
    \label{eq:hamiltonian-vcn}
    \HNOft = \lsum[i] C_i(t) \cdot \Phi_i(N) \leftrightarrow \HvcnOft = \lsum[i] \eta_i(t) \cdot \Phi_i(N)
\end{equation}

$C_\ast(t)$ are here analytically calculated functions, while the $\eta_\ast(t)$ are complex parameters to replace them.
For this structure, the comparison with the cumulant-expansion yields expressions for $\Phi_l(N)$ and $C_l(t)$ in a controlled manner.
By identifying the terms in \autoref{eq:hn-integrated-first-order}, a possible choice for the first terms could be chosen like in \autoref{eq:psi-and-c-choices} (absolute lattice-site enumeration like in \autoref{fig:geometry-of-square-system}). 

% TODO large formula, make sure it stays at the correct location

\begin{equation}
    \label{eq:psi-and-c-choices}
    \begin{split}
        C_1(t) = \VhamiltonianAnalyticalPartIntegrated{A}{0}{1}{t}
        \quad &\quad
        \Phi_1(N) = J \lsum[l] \lsum[\biggerNeighborX{l}{m}] 
        \lsum[K] \frac{\psiN[K]}{\psiN}
            \bracketHelper{N}{
                \VhamiltonianOperatorPart{A}{l}{m} 
            }{K}\\
        C_2(t) = \VhamiltonianAnalyticalPartIntegrated{B}{0}{1}{t}
        \quad &\quad
        \Phi_2(N) = J \lsum[l] \lsum[\biggerNeighborX{l}{m}] 
        \lsum[K] \frac{\psiN[K]}{\psiN}
            \bracketHelper{N}{
                \VhamiltonianOperatorPart{B}{l}{m} 
            }{K}\\
        C_3(t) = \VhamiltonianAnalyticalPartIntegrated{C}{0}{1}{t}
        \quad &\quad
        \Phi_3(N) = J \lsum[l] \lsum[\biggerNeighborX{l}{m}] 
        \lsum[K] \frac{\psiN[K]}{\psiN}
            \bracketHelper{N}{
                \VhamiltonianOperatorPart{C}{l}{m} 
            }{K}\\
        C_4(t) = \VhamiltonianAnalyticalPartIntegrated{A}{1}{0}{t}
        \quad &\quad
        \Phi_4(N) = J \lsum[l] \lsum[\smallerNeighborX{l}{m}] 
        \lsum[K] \frac{\psiN[K]}{\psiN}
            \bracketHelper{N}{
                \VhamiltonianOperatorPart{A}{l}{m} 
            }{K}\\
        C_5(t) = \VhamiltonianAnalyticalPartIntegrated{B}{1}{0}{t}
        \quad &\quad
        \Phi_5(N) = J \lsum[l] \lsum[\smallerNeighborX{l}{m}] 
        \lsum[K] \frac{\psiN[K]}{\psiN}
            \bracketHelper{N}{
                \VhamiltonianOperatorPart{B}{l}{m} 
            }{K}\\
        C_6(t) = \VhamiltonianAnalyticalPartIntegrated{C}{1}{0}{t}
        \quad &\quad
        \Phi_6(N) = J \lsum[l] \lsum[\smallerNeighborX{l}{m}] 
        \lsum[K] \frac{\psiN[K]}{\psiN}
            \bracketHelper{N}{
                \VhamiltonianOperatorPart{C}{l}{m} 
            }{K}\\
        C_7(t) = \VhamiltonianAnalyticalPartIntegrated{A}{0}{M}{t}
        \quad &\quad
        \Phi_7(N) = J \lsum[l] \lsum[\biggerNeighborY{l}{m}] 
        \lsum[K] \frac{\psiN[K]}{\psiN}
            \bracketHelper{N}{
                \VhamiltonianOperatorPart{A}{l}{m} 
            }{K}\\
        \cdots\quad \quad \quad\quad \quad &\quad\quad \quad\quad\quad \quad \cdots
    \end{split}
\end{equation}

The dissection separates the structure into 12 different factors. The terms $C_1(t)$ to $C_6(t)$ describe the neighbor interaction in x-direction, the terms $C_7(t)$ to $C_{12}(t)$ are identical with the interactions in y-direction (not printed, as they are identical with the x swapped for y and the 1 swapped for an $M$).
For the chain-geometry only the terms 1 through 6 are relevant, as there are no neighbors in y-direction.

\cite{TDVPcomplexPrefactors} % how to treat complex pre-factors, complex differentiation avoided through real differentiation
\cite{complexDifferentiation} % in this case the complex derivative is fine
\cite{probabilitySamplingRequirementVCN} % VCN general derivation, eta dot derivation and that it needs to sample from the probability distribution of the current state(t), 2d examples
\cite{variationalClassicalNetworksPaper} % Also source that does the same derivation on different system, 1d examples

% TODO short reference and maybe even explanation/derivation why the energy MUST be conserved under TDVP