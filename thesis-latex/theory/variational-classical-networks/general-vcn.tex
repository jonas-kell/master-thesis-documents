For this application the way of approximating the wave-function is by use of a \emph{classical network}. 
The network parameters now should be obtained via the \emph{time-dependent variational principle} \cite{originalDerivationTimeDependendVariationalPrinciple}.
For this reason the method as a whole structure is dubbed a \emph{variational classical network} (VCN).

In \cite{variationalClassicalNetworksPaper} this strategy is used on comparable applications on similar quantum problems.
The research in \cite{probabilitySamplingRequirementVCN} even applies the method to two-dimensional geometries.
Both reach the same expressions for the advancement of the parameters to later times.
This section reiterates the equations, unifying the parts where different conventions might be used.

\HNOft from \autoref{eq:cumulant-expansion} should be approximated with a parametrized expression \HvcnOft.
\autoref{eq:hamiltonian-vcn} provides the used parametrization, which - compared with \autoref{eq:time-evolution-target} - means $\HeffOft = -i E_0(N) t + \HvcnOft$.

\begin{equation}
    \label{eq:hamiltonian-vcn}
    \HNOft \leftrightarrow \HvcnOft = \lsum C_l(t) \Phi_l(N)
\end{equation}


\cite{TDVPcomplexPrefactors} % how to treat complex pre-factors, complex differentiation avoided through real differentiation
\cite{complexDifferentiation} % in this case the complex derivative is fine
\cite{probabilitySamplingRequirementVCN} % VCN general derivation, eta dot derivation and that it needs to sample from the probability distribution of the current state(t), 2d examples
\cite{variationalClassicalNetworksPaper} % Also source that does the same derivation on different system, 1d examples

% TODO short reference and maybe even explanation/derivation why the energy MUST be conserved under TDVP