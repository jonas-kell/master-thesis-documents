For this application the way of approximating the wave-function is by use of a \emph{classical network}. 
The network parameters now should be obtained via the \emph{time-dependent variational principle} \cite{originalDerivationTimeDependendVariationalPrinciple}.
For this reason the method as a whole is dubbed a \emph{variational classical network} (VCN).

In \cite{variationalClassicalNetworksPaper} this strategy is used on comparable applications on similar quantum problems.
The research in \cite{probabilitySamplingRequirementVCN} even applies the method to two-dimensional geometries.
Both reach the same expressions for the advancement of the parameters to later times.
This section reiterates the equations, unifying the parts where different conventions might be used.

\HNOft from \autoref{eq:cumulant-expansion} should be approximated with a parametrized expression \HvcnOft{\vec{\eta}}.
This means - compared with \autoref{eq:time-evolution-target} - the effective hamiltonian now is expressed as $\HeffOft = -i E_0(N) t + \HvcnOft{\vec{\eta}(t)}$.
The hamiltonian depends on the choice of the complex parameter vector $\vec{\eta}(t)$.

Historically the TDVP was derived for real-valued parameters \cite{originalDerivationTimeDependendVariationalPrinciple}.
To extend the principle to complex parameters, the networks structure could be adapted through the introduction of two real-valued parameters per complex parameter \cite{TDVPcomplexPrefactors}.
For \emph{holomorphic} functions the following also follows from an action-principle \cite{probabilitySamplingRequirementVCN} - or in short: for fully complex differentiable \cite{complexDifferentiation} parametrizations no special treatment is required.

Generally then the following method is applied: To begin with, $\vec{\eta}(t)$ is initialized to a known value for $t=0$ to try to achieve $\HvcnOft{\vec{\eta}(t=0)} \approx 0$.
The initialized parameters should additionally have some noise added to them, to prevent unwanted dead-points like an unstable equilibrium.
One then calculates the variational nudge $\dot{\vec{\eta}}$ to the parameters $\vec{\eta}$, so that the overlap of the states in \autoref{eq:best-overlap} is maximized.

\begin{equation}
    \label{eq:best-overlap}
    \begin{gathered}
        \ketpsiEta{\vec{\eta} + \delta \cdot \dot{\vec{\eta}}}{t} \leftrightarrow e^{-i \hamiltonian \delta} \ketpsiEta{\vec{\eta}}{t}
        \\
        \ketpsiEta{\vec{\eta}}{t} = \lsum[N] e^{-i E_0(N) t + \HvcnOft{\vec{\eta}(t)}} \psiN \ketN{}
    \end{gathered}
\end{equation}

This method tries to find the update to the parameters, that most accurately reflects what would happen if the system was time-evolved for a small time step $\delta$ with the full hamiltonian.
With a set of starting parameters for $t=0$ and an update-strategy for small time-steps, generating the parameters for later times is possible with repeated stepping.
The mentioned overlap is calculated with a measurement that uses the \emph{Fubini-Study metric} (\cite{variationalClassicalNetworksPaper} and \cite{probabilitySamplingRequirementVCN}, where \cite{probabilitySamplingRequirementVCN} contains a detailed derivation in the supplementary material).

This yields the \emph{TDVP-equation} \autoref{eq:eta-dot-calculation}, using the \emph{variational derivative} (\autoref{eq:variational-derivatives-definition}) and the \emph{local energy} (\autoref{eq:local-energy-definition}).
The \etaExpectationVal{\ast} denotes the expectation values with respect to the normalized probability distribution over all the states \ketpsiEta{\vec{\eta}}{t} \cite{probabilitySamplingRequirementVCN}.
Meaning the observable-sampling-expression from \autoref{eq:expectation-value} may be used as-is, if $\HeffOft$ is set to the parametrized version $-i E_0(N) t + \HvcnOft{\vec{\eta}(t)}$.

\begin{equation}
    \label{eq:eta-dot-calculation}
\end{equation}

\begin{equation}
    \label{eq:variational-derivatives-definition}
    \etaExpectationVal{}
\end{equation}

\begin{equation}
    \label{eq:local-energy-definition}
\end{equation}

% TODO short reference and maybe even explanation/derivation why the energy MUST be conserved under TDVP