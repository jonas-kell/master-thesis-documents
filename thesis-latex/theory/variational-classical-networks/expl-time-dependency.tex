The method described so far seems to be the minimal viable parametrization at first.
Of course it should be possible to introduce more terms that are inspired from higher order perturbation theory.
Currently, for the two-dimensional grid the last section suggested 12 parameters. 
Considering that half of them can be expressed as complex conjugates of some other, this leaves only 6 distinct parameters.
While this does not provide many degrees of freedom to encode information, it should be sufficient for some basic calculations and a verification of the method.
At least for validating the hypothesis it should be better than the first-order model without variational parameters.

In previous sections, the TDVP-equation was derived from maximizing the overlap of a variational state that has its parameters numerically integrated to a future time and the same state being time-evolved with the Hamiltonian for a small time-step.
However, as \cite{TDVPcomplexPrefactors} shows, the TDVP-formalism can also be derived from an \emph{action principle}.
And as \emph{Noether's theorem} states: if a method keeps the \emph{Lagrangian invariant} (like the mentioned principle that derives parameter-variations, which are symmetries of the action), the system's energy must be \emph{exactly} conserved over time \cite{energyConservationFromActionPrinciple} (independent of how \glqq bad\grqq{} the parametrization is).

In order to adhere to the structure of the thesis, at this point some numerical results are revealed in advance. 
As experiments in \fullref{sec:experiments-vcn-fails} show, this simple parametrization does not result in the energy being conserved at all.
Upon closer inspection, the resulting changes to the variational parameters seem to be more or less independent from the initial values.
After implementation reasons for this behavior were ruled out to the best possible degree, it seemed unclear what was the cause for this.

Yet the algorithm \emph{must} conserve the system's energy -- for small enough time-steps -- most importantly already at the very first variational time-step.
This leads to the solution, because the only remaining thing to influence the effective Hamiltonian (at $t=0$, where all introduced variational parameters are approximately zero) is the base-energy with its explicit time-dependency.

Presenting this realization in this case out-of-order (presenting and doing experiments on a clearly still unviable model before presenting the correct one) is supposed to reflect the work-process that was necessary to arrive at this final version.
In first theoretical discussions about the application of TDVP, the algorithm's explicit dependency on time in the base-energy case had been touched and even been presented in the correct form to solve this shortcoming.
Yet, only after going through two alternate base-energy parametrizations unsuccessfully and arriving at the final working solution independently of this hint, the relevance of the statement became clear.
In hindsight, it is straight forward to see why the explicit time-dependency contribution of the base-energy to the effective Hamiltonian is detrimental, when looking at equations \ref{eq:eta-dot-calculation} and \ref{eq:variational-derivatives-definition}.

Starting at a point $t$ in time, it comes natural to numerically integrate a quantity $\vec{\eta}$ if one has access to the time-derivative $\dot{\vec{\eta}}$.
In the method up to this point through the partial derivative with respect to the variational parameters: $\frac{\partial \HeffOft[N,\,\vec{\eta}]}{\partial \vec{\eta}_k} = \frac{\partial \left(-i E_0(N) t + \Hvcn{\vec{\eta}}\right)}{\partial \vec{\eta}_k} = \frac{\partial \Hvcn{\vec{\eta}} }{\partial \vec{\eta}_k}$, the explicit time dependency is dropped.
However classically the time-derivative is obtained as per the chain rule:
$\frac{\mathrm{d} f(t,\,x(t))}{\mathrm{d} t} = \frac{\partial f}{\partial t} + \frac{\partial f}{\partial x} \cdot \frac{\mathrm{d} x}{\mathrm{d} t}$, which rightfully does not drop the explicit dependency.
So to avoid having to derive the algorithm again, but with explicit time-dependency in mind, the most direct strategy would be to eliminate all explicit time-dependencies.
This would result in a completely variational effective Hamiltonian $\Heffvcn{N}{\vec{\eta}}$, where all that is time-dependent is encoded in the current state of $\vec{\eta}$.

Following the linear strategy of \autoref{eq:hamiltonian-vcn}, the resulting structure is presented in \autoref{eq:h-eff-vcn}.

\begin{equation}
    \label{eq:h-eff-vcn}
    \begin{split}
        \Heffvcn{N}{\vec{\eta}} &= \sum\limits_{n=1}^{\text{\#}(\text{var.})} \eta^\text{var.}_{n} \cdot \Phi^\text{var.}_{n}(N)
        + \sum\limits_{l=0}^{L} \eta^\text{b.e.}_{l} \cdot \Phi^\text{b.e.}_{l}(N)
         = \lsum[l] \eta_l \cdot \Phi_l(N)\\
         \vec{\eta} &= \left(
            \eta^\text{var.}_{1},\,\eta^\text{var.}_{2},\, \dots,\, \eta^\text{var.}_{\text{\#}(\text{var.})},\,
            \eta^\text{b.e.}_{0},\,\eta^\text{b.e.}_{1},\, \dots,\, \eta^\text{b.e.}_{L}
           \right)^\text{T}
    \end{split}
\end{equation}

The $\eta^\text{var.}_{n}$ and $\Phi^\text{var.}_{n}(N)$ are the same as in \autoref{sec:theory-variational-classical-networks-application} and still depend on the chosen parametrization of $\HNOft$.
When compared with \autoref{eq:base-energy}, \autoref{eq:base-energy-variational-parameters} presents itself as an ideal choice for $\eta^\text{b.e.}_{l}$ and $\Phi^\text{b.e.}_{l}(N)$, as it upholds $\sum\limits_{l=0}^{L} \eta^\text{b.e.}_{l} \cdot \Phi^\text{b.e.}_{l}(N) = -i \cdot E_0(N) \cdot t$ for a lattice with $L$ sites.

\begin{equation}
    \label{eq:base-energy-variational-parameters}
    \begin{split}
        E_0(N) &\stackrel{\ref{eq:base-energy}}{=} U \cdot \lsum[m] n_{m,\,\up}n_{m,\,\down} + \lsum[l,\,\sigma] \epsl n_{l,\,\sigma}\\
        \eta^\text{b.e.}_{l} &\stackrel{\phantom{\ref{eq:base-energy}}}{=} \begin{cases}
            l=0: \quad -i\cdot U \cdot t             \\
            l>0: \quad -i\cdot \epsl[l] \cdot t
        \end{cases}\\
        \Phi^\text{b.e.}_{l}(N) &\stackrel{\phantom{\ref{eq:base-energy}}}{=} \begin{cases}
            l=0: \quad \lsum[m] n_{m,\,\up}n_{m,\,\down}             \\
            l>0: \quad \lsum[\sigma] n_{l,\,\sigma}
        \end{cases}
    \end{split}
\end{equation}

Other partitions of the parameters are also possible, e.g. having multiple parameters for the encoding of the double-occupation term.
With this modification, the equations in \autoref{sec:theory-variational-classical-networks-basics} can be used as derived and the explicit time-dependency is eliminated.