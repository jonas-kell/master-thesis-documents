Depending on the chosen parametrization, different \HvcnOft{\vec{\eta}} might be obtained. 
This changes the value of the variational derivatives in \autoref{eq:variational-derivatives-definition}.
The reference \cite{VCNsolutionForRBM} states how to apply the method to \emph{restricted Boltzmann machines} (RBM) \cite{neuralNetworkQuantumStates}.
In this thesis, a simple linear dependency on the parameters is chosen:

\begin{equation}
    \label{eq:hamiltonian-vcn}
    \HNOft = \lsum[i] C_i(t) \cdot \Phi_i(N) \quad\leftrightarrow\quad \HvcnOft{\vec{\eta}} = \lsum[i] \eta_i(t) \cdot \Phi_i(N)
\end{equation}

$C_\ast(t)$ are here analytically calculated functions, while the $\eta_\ast(t)$ are complex parameters to replace them.
The choice of the $\Phi_i(N)$ dictates the range of interaction that could technically be covered.
For the structure in \autoref{eq:hamiltonian-vcn}, the comparison with the cumulant-expansion yields expressions for $\Phi_l(N)$ and $C_l(t)$ in a controlled manner.
By identifying the terms in \autoref{eq:hn-integrated-first-order}, a possible choice for the first terms could be chosen like in \autoref{eq:psi-and-c-choices} (absolute lattice-site enumeration like in \autoref{fig:geometry-of-square-system}). 

% TODO large formula, make sure it stays at the correct location

\begin{equation}
    \label{eq:psi-and-c-choices}
    \begin{split}
        C_1(t) = \VhamiltonianAnalyticalPartIntegrated{A}{0}{1}{t}
        \quad &\quad
        \Phi_1(N) = J \lsum[l] \lsum[\biggerNeighborX{l}{m}] 
        \lsum[K] \frac{\psiN[K]}{\psiN}
            \bracketHelper{N}{
                \VhamiltonianOperatorPart{A}{l}{m} 
            }{K}\\
        C_2(t) = \VhamiltonianAnalyticalPartIntegrated{B}{0}{1}{t}
        \quad &\quad
        \Phi_2(N) = J \lsum[l] \lsum[\biggerNeighborX{l}{m}] 
        \lsum[K] \frac{\psiN[K]}{\psiN}
            \bracketHelper{N}{
                \VhamiltonianOperatorPart{B}{l}{m} 
            }{K}\\
        C_3(t) = \VhamiltonianAnalyticalPartIntegrated{C}{0}{1}{t}
        \quad &\quad
        \Phi_3(N) = J \lsum[l] \lsum[\biggerNeighborX{l}{m}] 
        \lsum[K] \frac{\psiN[K]}{\psiN}
            \bracketHelper{N}{
                \VhamiltonianOperatorPart{C}{l}{m} 
            }{K}\\
        C_4(t) = \VhamiltonianAnalyticalPartIntegrated{A}{1}{0}{t}
        \quad &\quad
        \Phi_4(N) = J \lsum[l] \lsum[\smallerNeighborX{l}{m}] 
        \lsum[K] \frac{\psiN[K]}{\psiN}
            \bracketHelper{N}{
                \VhamiltonianOperatorPart{A}{l}{m} 
            }{K}\\
        C_5(t) = \VhamiltonianAnalyticalPartIntegrated{B}{1}{0}{t}
        \quad &\quad
        \Phi_5(N) = J \lsum[l] \lsum[\smallerNeighborX{l}{m}] 
        \lsum[K] \frac{\psiN[K]}{\psiN}
            \bracketHelper{N}{
                \VhamiltonianOperatorPart{B}{l}{m} 
            }{K}\\
        C_6(t) = \VhamiltonianAnalyticalPartIntegrated{C}{1}{0}{t}
        \quad &\quad
        \Phi_6(N) = J \lsum[l] \lsum[\smallerNeighborX{l}{m}] 
        \lsum[K] \frac{\psiN[K]}{\psiN}
            \bracketHelper{N}{
                \VhamiltonianOperatorPart{C}{l}{m} 
            }{K}\\
        C_7(t) = \VhamiltonianAnalyticalPartIntegrated{A}{0}{M}{t}
        \quad &\quad
        \Phi_7(N) = J \lsum[l] \lsum[\biggerNeighborY{l}{m}] 
        \lsum[K] \frac{\psiN[K]}{\psiN}
            \bracketHelper{N}{
                \VhamiltonianOperatorPart{A}{l}{m} 
            }{K}\\
        \cdots\quad \quad \quad\quad \quad &\quad\quad \quad\quad\quad \quad \cdots
    \end{split}
\end{equation}

The dissection separates the structure into 12 different factors. The terms $C_1(t)$ to $C_6(t)$ describe the neighbor interaction in x-direction, the terms $C_7(t)$ to $C_{12}(t)$ are identical with the interactions in y-direction (not printed, as they are identical with the x swapped for y and the 1 swapped for an $M$).
The neighbor convention was already introduced earlier, to reiterate \biggerNeighborX{l}{m} means \glqq all sites $m$ that have an x-coordinate that is strictly larger than the one of site $l$ and are nearest-neighbors of $l$\grqq{}.
For the chain-geometry only the terms 1 through 6 are relevant, as there are no neighbors in y-direction.

Combining all this, one arrives at \autoref{eq:variational-derivatives-our-case} for the first terms of the variational derivatives.
\begin{equation}
    \label{eq:variational-derivatives-our-case}
    O_{k}(N) \stackrel{\ref{eq:variational-derivatives-definition},\,\ref{eq:hamiltonian-vcn}}{=}
    \Phi_i(N)
\end{equation}
Because of this and the fact that the calculation of $\HeffOft[\vphantom{N}\smash{\tilde{N}},\,\vec{\eta}]-\HeffOft[N,\,\vec{\eta}]$ mostly boils down to calculating differences of $\Phi_\ast(\vphantom{N}\smash{\tilde{N}})-\Phi_\ast(N)$, these $\Phi_\ast(N)$ and differences thereof should be able to be evaluated most efficiently.