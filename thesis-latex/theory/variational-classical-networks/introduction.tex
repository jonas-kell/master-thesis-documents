When performing time-evolution with a perturbative approach, one ought to watch out for resonances.
In the case of degenerate energy solutions, so-called \emph{secular terms} \cite{secularTermsPerturbation} might be introduced if non-degenerate perturbation-theory is used to find the solution to a degenerate problem. 
Often inaccuracies manifest as linear terms in a function where only oscillating terms would be expected.
The linear term constantly grows over time, guaranteeing the approximation is only valid for short times and then diverges further and further from the exact solution.

For the examined hamiltonian, the first occurrence of such terms can be spotted in the second order cumulant expansion of the hamiltonian.
\autoref{eq:time-ordered-integral-part1} and \autoref{eq:time-ordered-integral-part2} specifically lists extra solutions for the cases where the two energies are the same.
This resonance causes a linear term to appear in the expansion, while all other pre-factors so far were complex exponentials - meaning they are oscillating.

The idea to combat this, is to replace the perturbative parameters that were calculated analytically with variational parameters instead.
This has the motive to generate a physically motivated structure, but the fine details of the parameter weights are adjusted by optimization.
E.g. for the \emph{neural-network quantum state} \cite{neuralNetworkQuantumStates} approach a neural network structure is used to represent a wave-function.
Reason for this being the hope, that the exponentially complicated wave-function can be expressed as a \glqq higher truth\grqq{} that requires only few parameters to describe a seemingly complicated wave-function.
While it would be basically impossible to always find the few-parameter-solution, \emph{approximating} it close enough with optimization akin to machine-learning should be a suitable replacement.

The following section aims to achieve better approximations than the ones found by cumulant expansion, by letting the parameters be optimized during the step to reflect the situation more closely.