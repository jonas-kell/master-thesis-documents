Achieving the goal of calculating observables for a specific time-evolved Hamiltonian and system geometry is mathematically straight forward.
However, the computational expense for solving the time-evolution for arbitrary quantum-mechanical problems - in relation to the system size - is too large for practical purposes.

Here, the goal is to bring down the necessary number of sampled states from \bigo{2^n} to something around the \bigo{n}-range (depending on the required precision) with the use of perturbative simplifications and Monte-Carlo sampling. 

The time-evolution will be executed by employing an expansion in the Interaction Picture to a fixed degree, which results in a computational complexity of \bigo{n^{2\cdot \text{degree}}} per evaluated state in the worst case.
Depending on the geometry (number of nearest neighbors $c$), this can be reduced to \bigo{\mathrm{f}(c) \cdot n}, with $\mathrm{f}(c)$ the number of neighbors reachable from any fixed starting site in at most as many nearest-neighbor-hops as the highest degree of approximation.

By placing a specific restriction on the initial state, all instances where an effective Hamiltonian needs to be evaluated, in fact only require the evaluation of the difference of the effective Hamiltonian on two states that differ only in a fixed number of occupations.
Through this, the act of choosing the next state in the Monte-Carlo-Markov chain can be brought down from \bigo{\mathrm{f}(c) \cdot n} to \bigo{\mathrm{f}(c)} and similarly also the cost of evaluating the most expensive observable drops to \bigo{\mathrm{f}(c)} (except the model-quality gauging energy and its fluctuation, that will have a cost \bigo{\mathrm{f}(c) \cdot n} for a single evaluation).
