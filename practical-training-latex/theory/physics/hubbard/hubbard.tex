The system in question is described by a \emph{Hubbard Model Hamiltonian}, that includes an electrical field. It is shown in \autoref{eq:main-hamiltonian}.


\begin{equation}
    \label{eq:main-hamiltonian}
    \pictureHamiltonian = \HzeroHamiltonian + \Vhamiltonian
\end{equation}

\begin{equation}
    \label{eq:main-hamiltonian-h0}
    \HzeroHamiltonian = U \cdot \lsum \nop{l}{\up}\nop{l}{\down} + \lsum[l,\,\sigma] \underbrace{\left(\vec{E} \cdot \vec{r_l}\right)}_{\text{\normalsize \epsl}} \nop{l}{\sigma}
\end{equation}

\begin{equation}
    \label{eq:main-hamiltonian-pertubation}
    \Vhamiltonian =  - J \cdot \neighborsumWSpin{l}{m}{\sigma} \left(\withspinop{l}{\sigma}{\dagger} \withspinop{m}{\sigma}{} + \withspinop{m}{\sigma}{\dagger} \withspinop{l}{\sigma}{} \right)
\end{equation}

$\vec{E}$ describes the vector of said electrical field and $\vec{r_l}$ the position of the site with index $l$.
The number operators $\nop{l}{\sigma} = \withspinop{l}{\sigma}{\dagger}\withspinop{l}{\sigma}{}$ measure the occupation on site $l$ with the respective spin $\sigma$. 

$U$, $J$ and \epsl[] are constants that describe the interaction strength. 
They all have the unit of energy. 
The scalar Product that is defined as \epsl[] can be evaluated, based on the system geometry. 
In this case, the system described by a regular square pattern that can be seen in \autoref{fig:geometry-of-square-system}. 
For such a system, the energy difference that is acquired from hopping from site $l$ to $m$ ($\Delta E_{l \rightarrow m}$) (provided, $l$ and $m$ are nearest neighbors) is described by \autoref{eq:energy-difference-hopping}. By assuming a default value for \epsl[0] one gets the relation from \autoref{eq:epsl} for \epsl{} (using the telescoping sum over nearest-neighbors-hopping along the path from $\vec{r_0}$ to $\vec{r_l}$). 

\begin{figure}[htbp]
    \centering
            
\def\getFromArray#1{%
    \ifcase#1 0\or
    1\or
    2\or
    \dots\or
    (M-1)\or
    M\or
    M+1\or
    M+2\or
    \dots\or
    M+(M-1)\or
    2\cdot M\or
    2\cdot M+1\or
    2\cdot M+2\or
    \dots\or
    2\cdot M+(M-1)\or
    \dots\or
    \dots\or
    \dots\or
    \dots\or
    \dots\or
    (M-1)\cdot M\or
    M^2-M+1\or
    M^2-M+2\or
    \dots\or
    M^2 -1\or
    \else\fi
}


\begin{tikzpicture}[scale=0.38]
    \def\m{5} % Change this value to adjust the grid size (m x m)
    \def\side{6} % Change this value to adjust the square side length
    \def\labelsize{\side/6} % Adjust label size
    
    % Draw dots and labels with custom text using a loop
    \foreach \x in {0,...,\numexpr\m-1\relax} {
        \foreach \y in {0,...,\numexpr\m-1\relax} {
            \pgfmathtruncatemacro{\index}{(\m-\y-1) * \m + \x}
            \node[scale=\labelsize] at (\x*\side + \side/2,\y*\side + \side/2) {$\getFromArray{\index}$};
            \fill[color=dblue] (\x*\side + \side/2,\y*\side + \side/2 + \side/4) circle (0.4);
        }
    }

    % Store coordinates for reference
    \coordinate (top) at (current bounding box.north west);

    % Draw horizontal black line
    \draw[line width=2pt,overlay,lightgray,dashed] (top) ++ (0,-0.55*\side) -- ++(\m*\side,0);
        
    % Draw green arrow pointing downwards
    \draw[green, ->, overlay, line width=2pt,>=stealth] (top) ++ (1.3*\side,-0.55*\side) -- ++(3*\side,-3*\side) node[pos=0.05,right] {$\varphi$};

    % Draw angle circle
    \draw[overlay, green, line width=1.5pt] (top) ++ (1.3*\side,-0.55*\side) +(0:0.5*\side) arc (0:-45:0.5*\side);
\end{tikzpicture}
    \vspace{0.8cm}
    \caption{Graphical representation of how the examined square system is laid out. The sites are labeled with the index $l$ they later can be identified by. Each of the sites has 4 nearest neighbors, except the ones on the borders, that have 3 or only 2 nearest neighbors (as the system is not periodic). A system of side-length $M$ with size $N$ is depicted, which means that it has $M^2 = N$ sites.
    In green, the Electrical field vector $\vec{E}$ is depicted, it is parametrized by the field strength $E$ and the angle $\varphi$.}
    \label{fig:geometry-of-square-system}
\end{figure}

\begin{equation}
    \label{eq:energy-difference-hopping}
    \begin{split}
        \Delta E_{l \rightarrow m} &=  \epsl[m] - \epsl[l] = \vec{E} \cdot \left(\vec{r_m}-\vec{r_l}\right)\\
        &= E \cdot \left[\cos(\varphi)\cdot \left(m \% M-l \% M\right) + \sin(\varphi)\cdot \left(\floor[\bigg]{\frac{m}{M}}-\floor[\bigg]{\frac{l}{M}}\right)\right]
    \end{split}
\end{equation}
\vspace{0.5cm} % equations are not well enough separated otherwise % TODO check in final render
\begin{equation}
    \label{eq:epsl}
    \begin{split}
        \epsl[0] = 0 \Rightarrow
        \epsl[l] &= \epsl[l] - \epsl[0] = \Delta E_{0 \rightarrow x} + \dots + \Delta E_{y \rightarrow l}\\
        &= E \cdot \left[\cos(\varphi)\cdot l \% M + \sin(\varphi)\cdot \floor[\bigg]{\frac{l}{M}}\right]
    \end{split}
\end{equation}

The Hubbard Model \cite{hubbardModelOriginalDerivation} is a spin-depending quantum-mechanical model. 
Because of this, the Hamiltonian includes terms for both the spin directions $\up$ and $\down$ (most of the time denoted by a summation over $\sigma$).

In principle, the operators $\withspinop[]{l}{\sigma}{(\dagger)}$ can be any type of quantum-mechanical ladder operator, like e.g. \emph{bosonic} or \emph{fermionic} ones.
In this application, the desicion was made to use \emph{hard-core bosons} \cite[]{hardCoreBosonsBasics}.
Typically for bosons, the Hubbard Model is tranformed into the \emph{Bose-Hubbard Model} \cite{boseHubbardModelOriginalDerivation}.
However hard-core bosons have the added property, that the can only have the occupation numbers of $0$ and $1$ per-spin-and-site in the investigated energy range.
This means they behave basically like fermions, only that their wave-function symmetrizes instead of anti-symmetrizes \cite{schwablBook}.
Because of the restriction one term of the Bose-Hubbard Model drops out and the studied model in equations \ref{eq:main-hamiltonian}, \ref{eq:main-hamiltonian-h0} and \ref{eq:main-hamiltonian-pertubation} is produced.
Also the typical Bose-Hubbard Model is not spin-dependent, while this property is kept here.
Hard-core bosons on their own have relevant properties that makes them worth studying. 
Experimental research on hard-core bosons can give insight into e.g. \emph{ultra-cold gasses} and \emph{Bose-Einstein condensation} \cite{hardCoreBosonsBasics}.
For this reason, the calculations are performed on hard-core bosons in this application and the operators obey the \emph{commutation-relations} described in \autoref{eq:commutation-relations}, with $\delta$ the \emph{Kronecker-Delta} \cite{schwablBookII}. 
But in principle, the calculations may be performed on fermions analogously. 

\begin{equation}
    \label{eq:commutation-relations}
    \begin{split}
        \left[\withspinop[]{l}{\sigma}{},\,\withspinop[]{l'}{\sigma'}{}\right] &= 
        \left[\withspinop[]{l}{\sigma}{\dagger},\,\withspinop[]{l'}{\sigma'}{\dagger}\right] = 0\\
        \left[\withspinop[]{l}{\sigma}{},\,\withspinop[]{l'}{\sigma'}{\dagger}\right] &= \delta(l,l')\cdot\delta(\sigma,\sigma') 
    \end{split}
\end{equation}

For ease of notation (most importantly in the code), the two spin degrees of freedom may be described by the alternate naming scheme provided in \autoref{eq:alt-naming-scheme}.

\begin{equation}
    \label{eq:alt-naming-scheme}
    \withspinop{l}{\up}{(\dagger)} \leftrightarrow \cop{l}{(\dagger)}
    \qquad\qquad
    \withspinop{l}{\down}{(\dagger)} \leftrightarrow \dop{l}{(\dagger)}
\end{equation}

For simplification purposes, the two spin directions are assumed to have the same coupling constants ($U$, $J$ and \epsl).
This means, all later results must be symmetrical in terms of the spin directions $\up$ and $\down$.


Translated into the alternative notation, \autoref{eq:main-hamiltonian-h0} reads like \autoref{eq:main-hamiltonian-h0-alt-notation} and \autoref{eq:main-hamiltonian-pertubation} reads like \autoref{eq:main-hamiltonian-pertubation-alt-notation}.

\begin{equation}
    \label{eq:main-hamiltonian-h0-alt-notation}
    \HzeroHamiltonian = U \cdot \lsum \cop[]{l}{\dagger}\cop[]{l}{} \dop[]{l}{\dagger}\dop[]{l}{} + \lsum\epsl \cop[]{l}{\dagger}\cop[]{l}{} \lsum\epsl \dop[]{l}{\dagger}\dop[]{l}{}
\end{equation}

\begin{equation}
    \label{eq:main-hamiltonian-pertubation-alt-notation}
    \Vhamiltonian =  - J \cdot \neighborsum{l}{m} \left(\cop[]{l}{\dagger}\cop[]{m}{} +\cop[]{m}{\dagger}\cop[]{l}{} + \dop[]{l}{\dagger}\dop[]{m}{} +\dop[]{m}{\dagger}\dop[]{l}{} \right)
\end{equation}

Several of the previous equations feature the \emph{nearest neighbor} summing notation with pointy brackets $\langle\rangle$.
This is supposed to express, that in the case of $\neighborsum[]{l}{m}$ for each possible $l$, $m$ will take on the indices of all neighbor sites of $l$.
However, if a pair $l=x,\,m=y$ is part of the sum, then the inverse $l=y,\,m=x$ will not be.
To be more in-line with the code implementation, \autoref{eq:fullneigborsum} gives an alternative notation using $[]$.
In such sums, each pair $l=x,\,m=y$ and $l=y,\,m=x$ will be included in the sum for each nearest-neighbor-relation $x,\,y$.

\begin{equation}
    \label{eq:fullneigborsum}
    \begin{split}
        &\neighborsum[]{l}{m} \left( F(l,m) \cop[]{l}{\dagger}\cop[]{m}{} + F(m,l) \cop[]{m}{\dagger}\cop[]{l}{} \right) \stackrel{\stackrel{\text{\scriptsize rename-swap}}{\text{$l$ and $m$}}}{=} \\
        &\neighborsum[]{l}{m} F(l,m) \cop[]{l}{\dagger}\cop[]{m}{} + \neighborsum[]{m}{l} F(l,m) \cop[]{l}{\dagger}\cop[]{m}{} = \\
        &\fullneighborsum[]{l}{m} F(l,m) \cop[]{l}{\dagger}\cop[]{m}{} \stackrel{\text{! caution !}}{\neq}
        2 \cdot \neighborsum[]{l}{m} F(l,m) \cop[]{l}{\dagger}\cop[]{m}{}
    \end{split}
\end{equation}

With that convention, you can rewrite \autoref{eq:main-hamiltonian-pertubation-alt-notation} to \autoref{eq:main-hamiltonian-pertubation-full-sum}.

\begin{equation}
    \label{eq:main-hamiltonian-pertubation-full-sum}
    \begin{split}
        \Vhamiltonian =  - J \cdot \fullneighborsum{l}{m} \left(\cop[]{l}{\dagger}\cop[]{m}{} + \dop[]{l}{\dagger}\dop[]{m}{} \right)
    \end{split}
\end{equation}
