The system in question is described by a \emph{Hubbard Model Hamiltonian}, that includes an electrical field. It is shown in \autoref{eq:main-hamiltonian}.


\begin{equation}
    \label{eq:main-hamiltonian}
    \pictureHamiltonian = \HzeroHamiltonian + \Vhamiltonian
\end{equation}

\begin{equation}
    \label{eq:main-hamiltonian-h0}
    \HzeroHamiltonian = U \cdot \lsum \nop{l}{\up}\nop{l}{\down} + \lsum[l,\,\sigma] \underbrace{\left(\vec{E} \cdot \vec{r_l}\right)}_{\text{\normalsize \epsl}} \nop{l}{\sigma}
\end{equation}

\begin{equation}
    \label{eq:main-hamiltonian-pertubation}
    \Vhamiltonian =  - J \cdot \neighborsumWSpin{l}{m}{\sigma} \left(\withspinop{l}{\sigma}{\dagger} \withspinop{m}{\sigma}{} + \withspinop{m}{\sigma}{\dagger} \withspinop{l}{\sigma}{} \right)
\end{equation}

$\vec{E}$ describes the vector of said electrical field and $\vec{r_l}$ the position of the site with index $l$.
The number operators $\nop{l}{\sigma} = \withspinop{l}{\sigma}{\dagger}\withspinop{l}{\sigma}{}$ measure the occupation on site $l$ with the respective spin $\sigma$. 

$U$, $J$ and \epsl[] are constants that describe the interaction strength. 
They all have the unit of energy. 
The scalar Product that is defined as \epsl[] can be evaluated, based on the system geometry. 
In this case, the system described by a regular square pattern that can be seen in \autoref{fig:geometry-of-square-system}. 
For such a system, the energy difference that is acquired from hopping from site $l$ to $m$ ($\Delta E_{l \rightarrow m}$) (provided, $l$ and $m$ are nearest neighbors) is described by \autoref{eq:energy-difference-hopping}. By assuming a default value for \epsl[0] one gets the relation from \autoref{eq:epsl} for \epsl{} (using the telescoping sum over nearest-neighbors-hopping along the path from $\vec{r_0}$ to $\vec{r_l}$). 

\begin{figure}[htbp]
    \centering
            While the optimizations described in the previous section greatly increase computational efficiency, they need to be applied for each $\HNOft$-order and each $\Phi_\ast(N)$ separately.
The problem is that for two dimensions the number of interactions to consider grows quickly, while visualizing the symmetries of what terms are equivalent gets progressively harder.
When looking at a chain, each higher order expands the interaction range by a one-site-step. 
So taking one more order into account, two sites are added to the possibilities for each interaction - this linearity is still somewhat manageable.
In two dimensions, the number of affected sites in relation to the interaction range grows quadratically, as visualized in \autoref{fig:growing-interaction-range}.

\begin{figure}[htbp]
    \centering
    \begin{tikzpicture}[scale=0.38]
    \definecolor{relevantsitecol}{HTML}{00AA00} % green

    \definecolor{concolhor}{HTML}{AA0000} % red
    \definecolor{concolver}{HTML}{FF8E00} % orange

    \def\m{5} % Change this value to adjust the grid size (m x m)
    \def\side{3} % Change this value to adjust the square side length
    \def\labelsize{\side/6} % Adjust label size
    
    \def\sp{1} % Change this value to adjust horizontal skip between the grids
    \def\order{2}

    \foreach \orderindex in {0,...,\numexpr\order\relax} {

        % Draw connections
        \foreach \x in {0,...,\numexpr\m-1\relax} {
            \foreach \y in {0,...,\numexpr\m-1\relax} {
                \pgfmathsetmacro{\orderindexincr}{\orderindex + 1}

                % Get shift amount
                \pgfmathsetmacro{\shft}{(\m + \sp - 1)*\orderindex*\side}
                
                % Get current node position
                \pgfmathsetmacro{\cx}{\x*\side + \side/2 + \shft}
                \pgfmathsetmacro{\cy}{\y*\side + \side/2 + \side/4}

                % Compute CURRENT Manhattan distance from center
                \pgfmathtruncatemacro{\dista}{abs(\x - (\m-1)/2) + abs(\y - (\m-1)/2)}


                % Connect to right neighbor
                \ifnum\x<\numexpr\m-1\relax

                    \pgfmathsetmacro{\usex}{\x+1}
                    \pgfmathsetmacro{\usey}{\y}

                    \pgfmathtruncatemacro{\distb}{abs(\usex - (\m-1)/2) + abs(\usey - (\m-1)/2)}

                    \pgfmathsetmacro{\nx}{\usex*\side + \side/2 + \shft}
                    \pgfmathsetmacro{\ny}{\usey*\side + \side/2 + \side/4}

                    % Color if both are smaller Manhattan distance
                    \ifnum\dista<\orderindexincr
                        \ifnum\distb<\orderindexincr
                            \ifnum\orderindex=1
                                \edef\useconcol{concolhor}
                            \else
                                \edef\useconcol{gray}
                            \fi
                        \else
                            \edef\useconcol{black}
                        \fi
                    \else
                        \edef\useconcol{black}
                    \fi

                    \draw[line width=2\pgflinewidth, color=\useconcol] (\cx,\cy) -- (\nx,\ny);
                \fi
                

                % Connect to top neighbor
                \ifnum\y<\numexpr\m-1\relax

                    \pgfmathsetmacro{\usex}{\x}
                    \pgfmathsetmacro{\usey}{\y+1}

                    \pgfmathtruncatemacro{\distb}{abs(\usex - (\m-1)/2) + abs(\usey - (\m-1)/2)}

                    \pgfmathsetmacro{\nx}{\usex*\side + \side/2 + \shft}
                    \pgfmathsetmacro{\ny}{\usey*\side + \side/2 + \side/4}

                    % Color if both are smaller Manhattan distance
                    \ifnum\dista<\orderindexincr
                        \ifnum\distb<\orderindexincr
                            \ifnum\orderindex=1
                                \edef\useconcol{concolver}
                            \else
                                \edef\useconcol{gray}
                            \fi
                        \else
                            \edef\useconcol{black}
                        \fi
                    \else
                        \edef\useconcol{black}
                    \fi

                    \draw[line width=2\pgflinewidth, color=\useconcol] (\cx,\cy) -- (\nx,\ny);
                \fi
            }
        }

        % Draw dots and labels with custom text using a loop
        \foreach \x in {0,...,\numexpr\m-1\relax} {
            \foreach \y in {0,...,\numexpr\m-1\relax} {
                % Get shift amount
                \pgfmathsetmacro{\shft}{(\m + \sp - 1)*\orderindex*\side}
                \pgfmathtruncatemacro{\index}{(\m-\y-1) * \m + \x}

                % Get current node position
                \pgfmathsetmacro{\cx}{\x*\side + \side/2 + \shft}
                \pgfmathsetmacro{\cy}{\y*\side + \side/2 + \side/4}

                % Compute Manhattan distance from center
                \pgfmathtruncatemacro{\dist}{abs(\x - (\m-1)/2) + abs(\y - (\m-1)/2)}
                
                % Change color if distance equals \orderindex
                \pgfmathsetmacro{\orderindexincr}{\orderindex + 1}
                \ifnum\dist<\orderindexincr
                    \fill[color=relevantsitecol] (\cx,\cy) circle (0.4);
                \else
                    \fill[color=dblue] (\cx,\cy) circle (0.4);
                \fi
            }
        }
    }

\end{tikzpicture}
    \caption{
        A depiction of the relevant sites and interactions that might be affected when a modification-event (e.g. a single flip on the center site) occurs.
        From left to right, the graphic shows the case for \emph{base-energy only}, \emph{first order perturbation theory} and \emph{second order perturbation theory}.
        In case of a modification to the center-site the occupation of all green sites possibly influences the outcome of the calculation (in the corresponding order). For the first and second order the terms can be identified with their corresponding colored edges.
    }
    \label{fig:growing-interaction-range}
\end{figure}

As show in \fullref{sec:theory-optimizations-analytical}, the described methods require calculating the differences between the effective Hamiltonians on two states that only differ in localized modifications.
Each modification flips the occupation of at least one site. 
To guarantee the differences of the effective Hamiltonians are evaluated correctly, only the terms that contain a modified site-occupation need to be evaluated. 
The terms that only contain non-modified occupation-numbers are guaranteed to cancel.

For the zeroth-order (base-energy) this is swiftly calculated and can be written down in simple terms like in e.g. equations \ref{eq:new-n-flipping} or \ref{eq:simplified-base-energy-hopping}.
In the case of the first order, it is simple enough to identify the terms.
Looking back at \autoref{eq:hn-integrated-first-order}, it becomes clear that all terms are described by \emph{edges} of two adjacent sites.
These relevant edges are colored red and orange in \autoref{fig:growing-interaction-range}. 
The symmetry is clear enough to spot that all horizontal and vertical edges respectively have the same analytical value.
This is because the $\VhamiltonianAnalyticalPartIntegrated{$\ast$}{l}{m}{t}$ are translationally invariant and are complex conjugates of each other when inverting the indices - which is the reason for arriving at $2\cdot 3 \cdot 2 = 12$ terms in \autoref{eq:psi-and-c-choices}, of which 6 are identified in \autoref{table:first-order-identification} (times two for horizontal/vertical versions).

\begin{table}[htbp]
    \centering
    \begin{tabular}{cc|cccccccccccccccc} 
        \toprule
             $l$ & \up      & 0 & 0 & 0 & 0   & 0 & 0 & 0 & 0   & 1 & 1 & 1 & 1   & 1 & 1 & 1 & 1    \\
             $l$ & \down    & 0 & 0 & 0 & 0   & 1 & 1 & 1 & 1   & 0 & 0 & 0 & 0   & 1 & 1 & 1 & 1    \\
             $m$ & \up      & 0 & 0 & 1 & 1   & 0 & 0 & 1 & 1   & 0 & 0 & 1 & 1   & 0 & 0 & 1 & 1    \\
             $m$ & \down    & 0 & 1 & 0 & 1   & 0 & 1 & 0 & 1   & 0 & 1 & 0 & 1   & 0 & 1 & 0 & 1    \\
        \midrule   
    \multicolumn{2}{c|}{$\VhamiltonianAnalyticalPartIntegrated{$\ast$}{l}{m}{t}$}
                            &   &   &   &    
                                              & A &   & C & 
                                                                & A & C &   &   
                                                                                  & B & A & A &      \\
    \multicolumn{2}{c|}{$\VhamiltonianAnalyticalPartIntegrated{$\ast$}{m}{l}{t}$}
                            &   & A & A & B  
                                              &   &   & C & A
                                                                &   & C &   & A 
                                                                                  &   &   &   &      \\
        \bottomrule
    \end{tabular}
    \vspace{0.5cm}
    \caption{
        A list of all occupation-configurations for the first order terms.
        The indices of the two involved sites are $l$ and $m$. Each hold a site for spin up and down particles \up and \down.
        Not all 16 configurations have a representative term that results from the perturbation theory.
        The ones that do reference the letters A, B and C from the $\VhamiltonianAnalyticalPartIntegrated{$\ast$}{l}{m}{t}$ in \autoref{eq:hn-integrated-first-order}.
        A second line takes the other combination $l \leftrightarrow m$ into account.
    }
    \label{table:first-order-identification}
\end{table}

Looking at all 16 occupation-number-combinations, the complexity is already extreme. 
Some combinations contribute to no terms, some contribute multiple times and some terms have more states contributing to them, some less.
Identifying and analytically optimizing the first order therefore is \emph{just} possible, albeit complicated.
For the variational parametrization, it would be possible and maybe even more efficient to just have 16 parameters, mapping one to one onto the occupation-number-combination of the respective bond. 
In second order, however, this all breaks down.
Taking a second look at \autoref{fig:growing-interaction-range}, each term in \autoref{eq:hn-integrated-second-order-final} now corresponds to \emph{two pairs of connections}, of which are 34 different ones with the edges adjacent (assuming no edge-cases, because at the border all is different again) and even more disconnected ones that might or might not be cases from the first order.
% 6 L -> * 4 for turning
% 5 I -> * 2 for turning
For identifying these, one must take rotational and mirror symmetries into account and the check for the border is vastly more complicated.
Because each of these double-bounds is formed by 3 sites with each two spin-degrees, there are in total $2^{3\cdot 2} = 64$ base-cases that need to be mapped.
The numbers of base-cases 16 and 64 here are only comparably low, as they take symmetries into account. 
For brute-forcing all combinations without the manual identification of symmetries, one would get $2^{5\cdot 2} = 1.024$ cases for the first order and $2^{13\cdot 2} = 67.108.864$ - it would most definitely not be a viable strategy to convert each of these into their own parameter.
Even if it was possible, this would diminish the physical bias we hope to get from defining the $\Phi_\ast(N)$ according to the cumulant expansion.

Even more problematic is the case of double-flips, because the interaction-spheres of these two modifications might be independent, overlapping slightly or even so much that both modifications are inside one bond.
While not impossible for the second order, it is highly complicated to analytically compute the most efficient formula for the difference of effective Hamiltonians.
And the process is not easily generalizable to higher orders because of order-specific symmetries.


\subsubsection*{Pre-Computed Interaction Spheres}

In this case, this was solved by defining pre-computed sets of indices that must be taken into account.

Given a formula that needs to sum over all indices once (like \autoref{eq:hn-integrated-first-order}), or once over all indices for each index (like \autoref{eq:hn-integrated-second-order-final} or \autoref{eq:v-squared-hard-computation}), it is clear that they require a runtime-complexity of \bigo{\text{\#}(\text{sites})} or even \bigo{(\text{\#}(\text{sites}))^2} to evaluate.
Yet all of them have so far also appeared as differences of two sums - one for the un-modified state and one for a state that had undergone a localized modification - of which most elements cancel.

One can assume it is possible to compute the neighbors of a site in constant time - still it takes computational time.
This and the problem from before can be solved by taking time at the start of the program to generate a cache of all tuples of indices, which would appear in the sum that is being optimized, but only restricted to a pre-set range around the index of the modification - a \emph{sphere of influence}.
Because even if the generation takes \bigo{(\text{\#}(\text{sites}))^2} to generate this cache for a two-site-modification - after generation the lookup becomes practically \bigo{1} (and is faster than computing the neighbor indices, even if all of that has asymptotically constant complexity).

This means effectively always a constant number of indices needs to be taken into account for all calculations (which then can be evaluated in constant time).
While it may be possible to analytically derive a more optimal solution with more canceling terms, this method easily scales.
Only the radius of the pre-computed cache needs to be chosen to be large enough to be compatible with the used order of the expansion.
These geometry-dependent caches are generated in \filepath{\cite{selfCode}}{/computation-scripts/systemgeometry.py} and the caches are used at various places.

The difference that this makes depends on the system size. 
As the number of influenced indices is constant, the strategy gets relatively more effective the more sites the system has (so that a comparably large portion of them is left out).
For a chain, optimizations will be noticed comparably earlier than for the square lattice - this is because in two dimensions the sphere of influence grows as an area and not linearly at the two edges.
But in the limit this strategy allows to keep the targeted complexity class, while scaling to the order of the expansion.
    \vspace{0.8cm}
    \caption{Graphical representation of how the examined square system is laid out. The sites are labeled with the index $l$ they later can be identified by. Each of the sites has 4 nearest neighbors, except the ones on the borders, that have 3 or only 2 nearest neighbors (as the system is not periodic). A system of side-length $M$ with size $N$ is depicted, which means that it has $M^2 = N$ sites.
    In green, the Electrical field vector $\vec{E}$ is depicted, it is parametrized by the field strength $E$ and the angle $\varphi$.}
    \label{fig:geometry-of-square-system}
\end{figure}

\begin{equation}
    \label{eq:energy-difference-hopping}
    \begin{split}
        \Delta E_{l \rightarrow m} &=  \epsl[m] - \epsl[l] = \vec{E} \cdot \left(\vec{r_m}-\vec{r_l}\right)\\
        &= E \cdot \left[\cos(\varphi)\cdot \left(m \% M-l \% M\right) + \sin(\varphi)\cdot \left(\floor[\bigg]{\frac{m}{M}}-\floor[\bigg]{\frac{l}{M}}\right)\right]
    \end{split}
\end{equation}
\vspace{0.5cm} % equations are not well enough separated otherwise % TODO check in final render
\begin{equation}
    \label{eq:epsl}
    \begin{split}
        \epsl[0] = 0 \Rightarrow
        \epsl[l] &= \epsl[l] - \epsl[0] = \Delta E_{0 \rightarrow x} + \dots + \Delta E_{y \rightarrow l}\\
        &= E \cdot \left[\cos(\varphi)\cdot l \% M + \sin(\varphi)\cdot \floor[\bigg]{\frac{l}{M}}\right]
    \end{split}
\end{equation}

The Hubbard Model \cite{hubbardModelOriginalDerivation} is a spin-depending quantum-mechanical model. 
Because of this, the Hamiltonian includes terms for both the spin directions $\up$ and $\down$ (most of the time denoted by a summation over $\sigma$).

For ease of notation (most importantly in the code), the two spin degrees of freedom may be described by the alternate naming scheme provided in \autoref{eq:alt-naming-scheme}.

\begin{equation}
    \label{eq:alt-naming-scheme}
    \withspinop{l}{\up}{(\dagger)} \leftrightarrow \cop{l}{(\dagger)}
    \qquad\qquad
    \withspinop{l}{\down}{(\dagger)} \leftrightarrow \dop{l}{(\dagger)}
\end{equation}

For simplification purposes, the two spin directions are assumed to have the same coupeling constants ($U$, $J$ and \epsl).

% TODO tex the translated hamiltonian
