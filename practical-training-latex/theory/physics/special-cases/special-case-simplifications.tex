Generally, handling the computation of the difference of the effective Hamiltonian for two states $\HeffOft[\tilde{N}]-\HeffOft[N]$ is the most expensive computation per Monte-Carlo-step.
It is required in \autoref{eq:form-heff-difference} for the calculation of the spin-polarized kinetics.
As stated in \fullref{sec:theory-monte-carlo} it is also needed to compute the transition probability between two sampled states in \autoref{eq:transition-probability-final}.
The calculation of $\HeffOft[N]$ on its own requires an effort of \bigo{\text{\#}(\text{sites}) \cdot \text{\#}(\text{nearest neighbors})}.
However, for the states \ketN[N] and \ketN[{\vphantom{N}\smash{\tilde{N}}}] that are connected with only small changes, most of the elements in the sum cancel and the complexity may be here even reduced to \bigo{$\text{\#}(\text{nearest neighbors})$}.
This is quite attractive for, because it de-couples the per-step computational costs from the number of lattice sites - a vital step for simulating large systems efficiency - however this requires extra analytical calculations.

\paragraph*{Initial State} \makebox{}\\

The initial state of the system before the start of the time-evolution is encoded in $\psiN$, defined by \autoref{eq:base-expansion-state}. Different configurations can be considered here.

One natural choice would be the preparation of a single defined base-state which would result in all $\psiN$ being $0$ but one that would have the value of $1$. 
This case needs to be considered with a specialized analytical calculation, because there are many instances in this calculation, where a division by $\psiN$ is performed. 

However the current calculation is compatible with the concept of one base-state having a probability $y$ that is $x$ times (e.g. $x=100$-times) as large as the probability $z = y/x$ of all the other states.
This effectively also concentrates the initial state to one base-state.

With the additional restriction of normalization from \autoref{eq:normalization} (factor 2 in the exponent for the spin-degree) the individual values for \psiN{} can be calculated (see \autoref{eq:psi-n-concentrated}).

\begin{equation}
    \label{eq:normalization}
    \lsum[N] \psiN{}^2 = 1 \qquad \text{\#}(\text{states}) = 2^{\text{\#}(\text{sites}) \cdot 2}
\end{equation}

\begin{equation}
    \label{eq:psi-n-concentrated}
    \begin{split}
        \stackrel{\ref{eq:normalization}}{\Rightarrow} y^2 + \left[\text{\#}(\text{states}) - 1 \right] z^2 = 1\\
        y^2 + \left[\text{\#}(\text{states}) - 1 \right] \frac{y^2}{x^2} = 1\\
        y = \sqrt{\left(1+\left[\text{\#}(\text{states}) - 1 \right]\frac{1}{x^2}\right)^{-1}}
    \end{split}
\end{equation}

This configuration was tested and verified for small systems with summation over the full Hilbert-space. 
However when Monte-Carlo sampled, the configuration produced different results than for exact sampling.
This can probably be explained by the fact that the energy landscape of base-states for this configuration is not smooth enough. 
Causing the Metropolis-Algorithm not to perform as intended for the number of samples.
As stated, treating non-uniform distribution requires a different analytical approach. Therefore no further experimentation on this was done.\\

A second natural choice would be the perfectly uniform distribution of probability for all the base-states. 
This results in a smooth energy-landscape, that can be properly sampled. 

The calculation for this is straight-forward and listed in \autoref{eq:psi-n-homogenous}

\begin{equation}
    \label{eq:psi-n-homogenous}
    \psiN{} = \frac{1}{\sqrt{\text{\#}(\text{states})}} \stackrel{\ref{eq:normalization}}{=} \frac{1}{\sqrt{2^{\text{\#}(\text{sites}) \cdot 2}}} = \frac{1}{2^{\text{\#}(\text{sites})}}
\end{equation}

One last advantage of the uniform distribution of base-state-probabilities is, that all $\psiN$ are equal, which makes it possible to optimize sums, as $\psiN{}/\psiN[\tilde{N}]{} = 1$.
The following two calculations require this assumption.

\paragraph*{Hopping-only modification} \makebox{}\\


\paragraph*{Flipping-only modification} \makebox{}\\


All simplifications in this section are implemented as options that are selectable in the control-script     \filepath{\cite{selfCode}}{/computation-scripts/script.py}.
 
% TODO