Doing math on paper is mostly two things. 
Trying and thinking what to do and executing steps by writing them down.

The problem of \glqq what to do\grqq{} can become basically of unlimited difficulty. 
There are programs that attempt to provide complex computations like integral evaluation or something similar. 
Most of them are purpose-built calculation machines that require at the minimum some expertise on how to operate them.
This program is not supposed to take any of this load, nor could it.
The program Math-Manipulator is designed to help with the \glqq writing down\grqq{} and \glqq executing steps\grqq{} part of the calculation.

\begin{itemize}
    \item 
    How to make the denominator of the fraction real-valued?
    \item
    How to expand a product of sums to a sums of product?
    \item
    How to remove al zero- and canceling terms from a sum?
    \item
    How to rename this one variable and copy all the rest of the equation?
\end{itemize}

All these problems are computationally easy, just tedious and error-prone if done by hand for many terms. 
This is the problem the program is designed to solve. 
A helper that mimics how math would be done by hand, but with the extra \glqq Expand these 50 terms, cancel and order for me\grqq{} built in.

This requires many small algorithms and functions that are purpose built for every operator.
Operations can be simple - like returning the numerical value of a constant \ref{appendix:numericals}.
They become more difficult, when recursion over children may be required - computing a percent-sing operator \ref{appendix:numericals}.
And can even become gigantic algorithms like for the general distribution operation \ref{appendix:distribute}.

This report is not a place to present them in any shape or form, but just give an outlook, where some of the very interesting implementations can be found if one is interested. 