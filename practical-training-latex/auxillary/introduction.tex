In preparation for the master thesis, as well as to get accustomed to the process of working in a scientific context, the module \emph{Practical Training} (\glqq Fachpraktikum\grqq{}) is one of the key modules that need to be completed at the end of the masters degree program in physics at Augsburg University.
In this case, work has been performed at the Chair for theoretical Physics III under the guidance of  Prof.~Dr.~Markus Heyl.
To fulfill the module requirements, this report is supposed to give an overview of my work during the past six months, understanding and preparing the basics of the target subject.
Following this report, the research on the subject will be continued, in order to be able to generate scientific content that possibly is of interest to the established community in the scope of my master thesis.

Several, different projects were attempted in the scope of this Practical Training. 
Central element were the analytical calculations performed on the Hubbard-Model, that will be presented in \autoref{sec:theory-hubbard-hamiltonian}.
Subsequent sections will deal with the theory of time evolution and measurement of the operators in the interaction picture necessary for the final implementation in \autoref{sec:implementation-analytical-calculations}.\\

Several of the required analytical calculations seemed to be to repetitive, error-prone or time-consuming to be done by hand. 
During my work and studies at TP III, my background in Computer Science made me question multiple times, whether the established sentiment of doing thins \glqq by hand\grqq{} was really an acceptable use of the staffs time.
Therefore in the past six months, I tried employing computers to aid with completing analytical calculations of such kind in a more timely and less frustrating way.
In this spirit was one calculation performed with help of the package Sympy \cite{sympyPackage}.
Also a bigger chunk of my time was spent, writing a custom application to aid in performing certain \glqq written-on-paper-like\grqq{} calculations with computational assistance. This application is called \emph{Math-Manipulator}.

The methods are touched superficial in \autoref{sec:implementation-math-manipulator} in this limited report. 
However they became central part of my \emph{Project Work Presentation}, that is available alongside this very LaTeX document \cite{selfDocument} and was presented at Augsburg University in 2024.