\section{Variational Classical Networks}
    \begin{frame}[t]
        \frametitle{Why and How should Parameters be variational?}

        \vspace{-0.6cm}
        \begin{columns}[t]
            \column{0.4\textwidth}
                \begin{itemize}
                    \item Theoretically: always better approximation through higher order terms \pause
                    \begin{itemize}
                        \item Hard to calculate analytically and expensive to evaluate \pause
                    \end{itemize}
                    \item Hope: Better behavior if parameters are \emph{learned}/\emph{optimized}, starting from the analytical result \pause
                \end{itemize}
    
            \column{0.4\textwidth}
                \vspace{-0.3cm}
                \makebox[\textwidth][c]{
                    \includegraphics[width=0.95\textwidth,page=1]{main-content/variational-classical-networks/variational-classical-networks-theory.pdf}
                }
                \pause
                \makebox[\textwidth][c]{
                    \includegraphics[width=0.7\textwidth,page=2]{main-content/variational-classical-networks/variational-classical-networks-theory.pdf}
                }%
                \pause
                \vspace{1cm}
                \makebox[\textwidth][c]{
                    \includegraphics[width=1.1\textwidth,page=3]{main-content/variational-classical-networks/variational-classical-networks-theory.pdf}
                }
                \pause
                \makebox[\textwidth][c]{
                    \includegraphics[width=0.7\textwidth,page=4]{main-content/variational-classical-networks/variational-classical-networks-theory.pdf}
                }
    
        \end{columns}
        
        % notes 
        \onslide % on all slides of frame
        \note[item] {
            Argument:
            \begin{itemize}
                \item Could go on performing the controlled expansion for all perturbation-strengths and get nice result
                \item But higher orders complicated and expensive
                \item Try to get more from the lower orders, by using Variational parameters (optimized by the problem, we hope they can incorperate influence from higher order terms without them being present)
            \end{itemize}
        }
        \note[item] {
            Time dependent variational principle (TDVP)
            \begin{itemize}
                \item Find a parametrized \emph{wavefunction}
                \item Update the parameters so that a tiny nudge into the direction of the \emph{time DERIVATIVE} of the parameter vector has the same effect as a time evolution with the full hamiltonian
                \item The first order numerical integration (could do higher orders, but for the concept here)
            \end{itemize}
        }
        \note[item] {
            O: Variational derivative\,\,\,            E: local energy (we had already earlier)
        }
    \end{frame}


    \begin{frame}[t]
        \frametitle{Why and How should Parameters be variational?}

        \vspace{-0.6cm}
        \begin{columns}[t]
            \column{0.4\textwidth}
                \begin{itemize}
                    \item Theoretically: always better approximation through higher order terms
                    \begin{itemize}
                        \item Hard to calculate analytically and expensive to evaluate
                    \end{itemize}
                    \item Hope: Better behavior if parameters are \emph{learned}/\emph{optimized}, starting from the analytical result
                \end{itemize}
    
            \column{0.4\textwidth}
                \vspace{-0.3cm}
                \makebox[\textwidth][c]{
                    \includegraphics[width=0.95\textwidth,page=1]{main-content/variational-classical-networks/variational-classical-networks-theory.pdf}
                }
                \makebox[\textwidth][c]{
                    \includegraphics[width=0.7\textwidth,page=2]{main-content/variational-classical-networks/variational-classical-networks-theory.pdf}
                }%
                \pause
                \vspace{0.5cm}
                \makebox[\textwidth][c]{
                    \includegraphics[width=1.1\textwidth,page=5]{main-content/variational-classical-networks/variational-classical-networks-theory.pdf}
                }%
                \pause
                \vspace{0.15cm}
                \makebox[\textwidth][c]{
                    \includegraphics[width=0.6\textwidth,page=6]{main-content/variational-classical-networks/variational-classical-networks-theory.pdf}
                }
    
        \end{columns}
        
        % notes 
        \onslide % on all slides of frame
        \note[item] {
            Pseudo inversion
        }
        \note[item] {
            S: covariance matrix
        }
        \note[item] {
            F: TDVP force
        }
        \note[item] {
            \emph{time DERIVATIVE}
        }
    \end{frame}

    \begin{frame}[t]
        \frametitle{First Try: Cumulant Expansion Prefactors}

        \begin{itemize}
            \item Idea: generate a variational Hamiltonian by replacing the analytical prefactors
        \end{itemize}
        \vspace{-0.2cm}
        \makebox[\textwidth][c]{
            \includegraphics[width=0.7\textwidth,page=1]{main-content/variational-classical-networks/variational-classical-networks-application.pdf}
        }%
        \vspace{0.4cm}
        \pause
        \makebox[\textwidth][c]{
            \hspace{-1.5cm}
            \makebox[0.5\textwidth][c]{
                \includegraphics[width=0.5\textwidth,page=2]{main-content/variational-classical-networks/variational-classical-networks-application.pdf}%
            }
            \makebox[0.5\textwidth][c]{
                \raisebox{0.55cm}{
                    \includegraphics[width=0.5\textwidth,page=3]{main-content/variational-classical-networks/variational-classical-networks-application.pdf}
                }
            }
        }
        
        % notes 
        \onslide % on all slides of frame
        \note[item] {
            Idea: replace prefactors that depend on the time and keep the shape of teh terms that depend on the occupation of the state
        }
        \note[item] {
            Verification: as the TDVP is also derived from an \emph{action principle} that must mean as per noethers theorem that the energy MUST be conserved by a correct TDVP step
        }
    \end{frame}

    \begin{frame}[t]
        \frametitle{Correction: Watch Explicit Time-Dependency}

        \begin{itemize}
            \item First strategy is not suitable
            \begin{itemize}
                \item Energy- and variance is not conserved
            \end{itemize}
            \pause
            \item Problem: explicit time-dependency of base energy
        \end{itemize}

        \makebox[\textwidth][c]{
            \includegraphics[width=0.55\textwidth,page=1]{main-content/variational-classical-networks/variational-classical-networks-application-explicit.pdf}
        }

        \pause
        \begin{itemize}
            \item Solution: Replace base energy factors with variational parameters
        \end{itemize}

        \makebox[\textwidth][c]{
            \includegraphics[width=0.3\textwidth,page=3]{main-content/variational-classical-networks/variational-classical-networks-application-explicit.pdf}
            \raisebox{1cm}{
                \includegraphics[width=0.62\textwidth,page=2]{main-content/variational-classical-networks/variational-classical-networks-application-explicit.pdf}
            }
        }
        
        % notes 
        \onslide % on all slides of frame
        \note[item] {
            Fails because not taken the explicit time-dependency into account as it was \emph{differentiated away} wrongfully 
        }
        \note[item] {
            As a solution: introduce variational base-energy factors - which is a successful strategy
        }
    \end{frame}
