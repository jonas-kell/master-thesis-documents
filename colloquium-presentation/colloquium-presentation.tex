\documentclass[aspectratio=169]{beamer} % includes \pause in render
% \documentclass[aspectratio=169, handout]{beamer} % do not include \pause in render

\usetheme[neutralbackground]{uniamntf}

\title[Master Colloquium Presentation]{\vspace{-2em}Classical networks for the Hubbard model with a tilted potential}
\subtitle{ - Master Thesis Colloquium - }

\author{Jonas Kell}
\institute[TP III]{Chair for theoretical Physics III}

\date[12.03.2025]{$12^{\text{th}}$ of March 2025}
% Date for the notes
\day=12\relax
\month=3\relax
\year=2025\relax

\acknowledgement{
    Jonas Kell\\
    University of Augsburg\\
    jonas.kell@student.uni-augsburg.de\\
    www.uni-augsburg.de
}

\newcommand{\blankfootnote}[1]{%
\let\thefootnote\relax\footnotetext{#1}%
}
\newcommand{\tab}{%
\,\,\,\,
}

% bibtex/biber
\usepackage[backend=biber, style=phys, biblabel=brackets]{biblatex} % citations with "modern" backend and an physics-accepted citation style
\addbibresource{literature.bib}

\newenvironment{wideitemize}{\itemize\addtolength{\itemsep}{0.3em}}{\enditemize}

%! notes control

%\setbeameroption{hide notes} % Only slides
%\setbeameroption{show only notes} % Only notes
\setbeameroption{show notes on second screen=right} % Both, use for pympress mode

% presentation tool https://github.com/Cimbali/pympress
% run: pympress colloquium-presentation.pdf 

%! notes control

% !! THIS ONLY COMPILES AFTER A SUCCESSFUL COMPILATION OF THE MAIN DOCUMENT

\usepackage{hyperref}

\begin{document}
    \begin{frame}[t,plain] 
        \maketitle
    \end{frame}

    \note[enumerate]{
        \item Thank you for coming
        \item Title: Classical networks for the Hubbard model with a tilted potential
        \item Quickly what this will be about
    }

    \begin{frame}
        \frametitle{Outline}
        \tableofcontents[pausesections] % pause toc before each section
        % \tableofcontents % all toc at once

        % toc notes do need to live inside the frame, to appear on all animated slides
        \note[item] {
            \begin{itemize}
                \item Introduction to the physical theory
                \item In depth: Time evolution - why and how
                \item Quickly: Sampling Strategy and simplifications
                \item To give the title justice: Variational classical networks - what is is all about
                \item End (if there is time or if questions lead into that direction): Implementation and experiments
                \item Outlook
            \end{itemize}
        }
    \end{frame}

    \section{TODO}
    \begin{frame}[t]
        \frametitle{TODO}
        
        \vspace{-0.5em}
        \begin{itemize}
            \item TODO
        \end{itemize}

        % notes 
        \onslide % on all slides of frame
        \note[item] {
            TODO
        }
    \end{frame}

    \section{Time Evolution}

    \begin{frame}[t]
        \frametitle{Goal of the Calculation}

        \begin{itemize}
            \item Evaluate the time-evolution of the system for various observables \pause
            \begin{itemize}
                \item Uses calculation in the \emph{Interaction Picture} \pause
                \item Introduce operator-less \emph{effective Hamiltonian}
            \end{itemize}
        \end{itemize}

        \vspace{1cm}

        \makebox[\textwidth][c]{
            \includegraphics[width=0.165\textwidth,page=1]{main-content/time-evolution/effective-hamiltonian-definition.pdf}
            \includegraphics[width=0.45\textwidth,page=2]{main-content/time-evolution/effective-hamiltonian-definition.pdf}
        }

        % notes 
        \onslide % on all slides of frame
        \note[item] {
            TODO
        }
    \end{frame}

    \begin{frame}[t]
        \frametitle{The effective Hamiltonian}

        \makebox[\textwidth][c]{
            \includegraphics[width=0.55\textwidth,page=3]{main-content/time-evolution/effective-hamiltonian-definition.pdf}
        }

        \pause
        \begin{itemize}
            \item Constructed from the contributions of the base energy and the perturbation \pause
            \item Base energy contribution:
        \end{itemize}

        \makebox[\textwidth][c]{
            \includegraphics[width=0.65\textwidth,page=4]{main-content/time-evolution/effective-hamiltonian-definition.pdf}
        }

        % notes 
        \onslide % on all slides of frame
        \note[item] {
            TODO
        }
    \end{frame}

    \begin{frame}[t]
        \frametitle{The effective Hamiltonian}

        \begin{itemize}
            \item To evaluate the time-evolution of the perturbation
        \end{itemize}

        \makebox[\textwidth][c]{
            \includegraphics[width=0.50\textwidth,page=1]{main-content/time-evolution/v-in-interaction.pdf}
        }

        \pause
        \begin{itemize}
            \item Solve the equation of motion for the ladder operators
        \end{itemize}
        
        \vspace{-0.1cm}
        \makebox[\textwidth][c]{
            \includegraphics[width=0.50\textwidth,page=2]{main-content/time-evolution/v-in-interaction.pdf}
        }


        % notes 
        \onslide % on all slides of frame
        \note[item] {
            Solving the equations of motions in the interaction picture requires the number operator being \emph{idempotent}
        }
    \end{frame}

    \begin{frame}[t]
        \frametitle{The effective Hamiltonian}

        \begin{itemize}
            \item Insert and reorder the operators
        \end{itemize}

        \makebox[\textwidth][c]{
            \includegraphics[width=0.80\textwidth,page=1]{main-content/time-evolution/v-in-interaction-solution.pdf} % TODO
        }%
        \vspace{-0.2cm}
        \pause
        \makebox[\textwidth][c]{
            \includegraphics[width=0.80\textwidth,page=2]{main-content/time-evolution/v-in-interaction-solution.pdf}
        }

        % notes 
        \onslide % on all slides of frame
        \note[item] {
            TODO
        }
    \end{frame}

    \begin{frame}[t]
        \frametitle{The effective Hamiltonian}

        \begin{itemize}
            \item Evaluate the contribution to the effective Hamiltonian \pause
            \begin{itemize}
                \item Requires previously calculated value of the V-operator in the Interaction Picture \pause
                \item Controllable \emph{cumulant expansion}
            \end{itemize}
        \end{itemize}

        \pause[2]
        \vspace{0.2cm}
        \makebox[\textwidth][c]{
            \includegraphics[width=0.80\textwidth,page=1]{main-content/time-evolution/effective-hamiltonian-n.pdf}
        }

        % notes 
        \onslide % on all slides of frame
        \note[item] {
            TODO
        }
    \end{frame}

    \begin{frame}[t]
        \frametitle{Handling of Observables}

        \begin{itemize}
            \item This allows for general evaluation of expectation values \pause
            \begin{itemize}
                \item Requires a transition probability
                \item Requires a local observable
            \end{itemize}
        \end{itemize}

        \pause[1]
        \vspace{0.2cm}
        \makebox[\textwidth][c]{
            \includegraphics[width=0.28\textwidth,page=1]{main-content/time-evolution/time-evolution-of-observable.pdf}
        }
        \pause[2]
        \makebox[\textwidth][c]{
            \includegraphics[width=0.80\textwidth,page=2]{main-content/time-evolution/time-evolution-of-observable.pdf}
        }

        % notes 
        \onslide % on all slides of frame
        \note[item] {
            Does only require difference of effective Hamiltonian for evaluation (notice for later)
        }
    \end{frame}

    \begin{frame}[t]
        \frametitle{Simple Observalbles}

        \begin{itemize}
            \item Local observable for double occupation measurement \pause
            \begin{itemize}
                \item Observables generally are very sparse matrices
                \item Reduces to pure occupation measurement
            \end{itemize}
        \end{itemize}

        \makebox[\textwidth][c]{
            \includegraphics[width=0.7\textwidth,page=1]{main-content/time-evolution/local-observables-simple-ops.pdf}
        }

        \pause
        \begin{itemize}
            \item Local observable for particle current measurement \pause
            \begin{itemize}
                \item Requires evaluation of the difference of two effective Hamiltonians
            \end{itemize}
        \end{itemize}

        \makebox[\textwidth][c]{            
            \includegraphics[width=0.8\textwidth,page=2]{main-content/time-evolution/local-observables-simple-ops.pdf}
        }

        % notes 
        \onslide % on all slides of frame
        \note[item] {
            Complex observable, make sure this cancels
        }
    \end{frame}

    \begin{frame}[t]
        \frametitle{Access to Density-Matrics}

        \begin{itemize}
            \item Non-classical (\emph{quantum}) measurements depend on the (reduced) density matrix \pause
                \begin{itemize}
                    \item Access to \emph{purity}, \emph{concurrence} and other entanglement measures/monotones\pause
                \end{itemize}
            \item Direct calculation not possible
        \end{itemize}

        \vspace{-0.2cm}
        \makebox[\textwidth][c]{
            \includegraphics[width=0.65\textwidth,page=1]{main-content/time-evolution/reduced-density-matrix.pdf}
        }

        \vspace{-0.2cm}
        \pause
        \begin{itemize}
            \item Use Pauli matrices to expand the complex 4x4 matrix
        \end{itemize}

        \makebox[\textwidth][c]{
            \includegraphics[width=0.65\textwidth,page=2]{main-content/time-evolution/reduced-density-matrix.pdf}
        }

        % notes 
        \onslide % on all slides of frame
        \note[item] {
            TODO
        }
    \end{frame}

    \begin{frame}[t]
        \frametitle{Access to Density-Matrics}

        \makebox[\textwidth][c]{
            \includegraphics[width=0.75\textwidth,page=3]{main-content/time-evolution/reduced-density-matrix.pdf}
        }

        % notes 
        \onslide % on all slides of frame
        \note[item] {
            TODO
        }
    \end{frame}
    \section{Sampling \& Simplifications}
    \begin{frame}[t]
        \frametitle{Probability Calculations}
        
        \vspace{-0.6cm}

        \begin{columns}[t]
            \column{0.4\textwidth}
                \begin{itemize}
                    \item Full probability still requires normalization \pause
                    \item Switch to the Metropolis \emph{Hastings} algorithm
                    \begin{itemize}
                        \item Non-exponential number of samples \pause
                        \item Depends only on a transition probability
                        \item Does not need normalization because it cancels
                    \end{itemize}
                \end{itemize}
    
            \onslide
            \column{0.4\textwidth}
                \vspace{0.0cm}
                \pause[3]
                \makebox[\textwidth][c]{
                    \includegraphics[width=1.3\textwidth,page=2]{main-content/simplifications/mc-sampling.pdf}
                }
    
        \end{columns}

        \pause[2]
        \vspace{0.15cm}
        \makebox[\textwidth][c]{
            \includegraphics[width=0.80\textwidth,page=1]{main-content/simplifications/mc-sampling.pdf}
        }

        % notes 
        \onslide % on all slides of frame
        \note[item] {
            Normalization requires knowing exponential amount of states and we need to sample an exponential amount to get all
        }
    \end{frame}

    \begin{frame}[t]
        \frametitle{Analytical Simplifications}
        
        \begin{itemize}
            \item Choose a helpful initial state
        \end{itemize}
    
        \pause
        \vspace{-0.35cm}
        \makebox[\textwidth][c]{
            \includegraphics[width=0.60\textwidth,page=1]{main-content/simplifications/analytical-simplification-example.pdf}
        }

        \vspace{-0.1cm}
        \begin{itemize}
            \item Now \emph{almost all} terms cancel in \emph{differences} of the effective Hamiltonian \pause
            \begin{itemize}
                \item Exemplary for single flip on base energy term
            \end{itemize} 
        \end{itemize}

        \vspace{-0.15cm}
        \makebox[\textwidth][c]{
            \includegraphics[width=0.62\textwidth,page=2]{main-content/simplifications/analytical-simplification-example.pdf}
        }

        % notes 
        \onslide % on all slides of frame
        \note[item] {
            TODO
        }
    \end{frame}

    \section{Variational Classical Networks}
    \begin{frame}[t]
        \frametitle{Why and How should Parameters be variational?}

        \vspace{-0.6cm}
        \begin{columns}[t]
            \column{0.4\textwidth}
                \begin{itemize}
                    \item Theoretically: always better approximation through higher order terms \pause
                    \begin{itemize}
                        \item Hard to calculate analytically and expensive to evaluate \pause
                    \end{itemize}
                    \item Hope: Better behavior if parameters are \emph{learned}/\emph{optimized}, starting from the analytical result \pause
                \end{itemize}
    
            \column{0.4\textwidth}
                \vspace{-0.3cm}
                \makebox[\textwidth][c]{
                    \includegraphics[width=0.95\textwidth,page=1]{main-content/variational-classical-networks/variational-classical-networks-theory.pdf}
                }
                \pause
                \makebox[\textwidth][c]{
                    \includegraphics[width=0.7\textwidth,page=2]{main-content/variational-classical-networks/variational-classical-networks-theory.pdf}
                }%
                \pause
                \vspace{1cm}
                \makebox[\textwidth][c]{
                    \includegraphics[width=1.1\textwidth,page=3]{main-content/variational-classical-networks/variational-classical-networks-theory.pdf}
                }
                \pause
                \makebox[\textwidth][c]{
                    \includegraphics[width=0.7\textwidth,page=4]{main-content/variational-classical-networks/variational-classical-networks-theory.pdf}
                }
    
        \end{columns}
        
        % notes 
        \onslide % on all slides of frame
        \note[item] {
            Time dependent variational principle (TDVP)
        }
        \note[item] {
            Variational classical network (VCN)
        }
    \end{frame}


    \begin{frame}[t]
        \frametitle{Why and How should Parameters be variational?}

        \vspace{-0.6cm}
        \begin{columns}[t]
            \column{0.4\textwidth}
                \begin{itemize}
                    \item Theoretically: always better approximation through higher order terms
                    \begin{itemize}
                        \item Hard to calculate analytically and expensive to evaluate
                    \end{itemize}
                    \item Hope: Better behavior if parameters are \emph{learned}/\emph{optimized}, starting from the analytical result
                \end{itemize}
    
            \column{0.4\textwidth}
                \vspace{-0.3cm}
                \makebox[\textwidth][c]{
                    \includegraphics[width=0.95\textwidth,page=1]{main-content/variational-classical-networks/variational-classical-networks-theory.pdf}
                }
                \makebox[\textwidth][c]{
                    \includegraphics[width=0.7\textwidth,page=2]{main-content/variational-classical-networks/variational-classical-networks-theory.pdf}
                }%
                \pause
                \vspace{0.5cm}
                \makebox[\textwidth][c]{
                    \includegraphics[width=1.1\textwidth,page=5]{main-content/variational-classical-networks/variational-classical-networks-theory.pdf}
                }%
                \pause
                \vspace{0.15cm}
                \makebox[\textwidth][c]{
                    \includegraphics[width=0.6\textwidth,page=6]{main-content/variational-classical-networks/variational-classical-networks-theory.pdf}
                }
    
        \end{columns}
        
        % notes 
        \onslide % on all slides of frame
        \note[item] {
            Pseudo inversion
        }
    \end{frame}

    \begin{frame}[t]
        \frametitle{First Try: Cumulant Expansion Prefactors}

        \begin{itemize}
            \item Idea: generate a variational Hamiltonian by replacing the analytical prefactors
        \end{itemize}
        \vspace{-0.2cm}
        \makebox[\textwidth][c]{
            \includegraphics[width=0.7\textwidth,page=1]{main-content/variational-classical-networks/variational-classical-networks-application.pdf}
        }%
        \vspace{0.4cm}
        \pause
        \makebox[\textwidth][c]{
            \hspace{-1.5cm}
            \makebox[0.5\textwidth][c]{
                \includegraphics[width=0.5\textwidth,page=2]{main-content/variational-classical-networks/variational-classical-networks-application.pdf}%
            }
            \makebox[0.5\textwidth][c]{
                \raisebox{0.55cm}{
                    \includegraphics[width=0.5\textwidth,page=3]{main-content/variational-classical-networks/variational-classical-networks-application.pdf}
                }
            }
        }
        
        % notes 
        \onslide % on all slides of frame
        \note[item] {
            TODO
        }
    \end{frame}

    \begin{frame}[t]
        \frametitle{Correction: Watch Explicit Time-Dependency}

        \begin{itemize}
            \item First strategy is not suitable
            \begin{itemize}
                \item Energy- and variance is not conserved
            \end{itemize}
            \pause
            \item Problem: explicit time-dependency of base energy
        \end{itemize}

        \makebox[\textwidth][c]{
            \includegraphics[width=0.55\textwidth,page=1]{main-content/variational-classical-networks/variational-classical-networks-application-explicit.pdf}
        }

        \pause
        \begin{itemize}
            \item Solution: Replace base energy factors with variational parameters
        \end{itemize}

        \makebox[\textwidth][c]{
            \includegraphics[width=0.3\textwidth,page=3]{main-content/variational-classical-networks/variational-classical-networks-application-explicit.pdf}
            \raisebox{1cm}{
                \includegraphics[width=0.62\textwidth,page=2]{main-content/variational-classical-networks/variational-classical-networks-application-explicit.pdf}
            }
        }
        
        % notes 
        \onslide % on all slides of frame
        \note[item] {
            TODO
        }
    \end{frame}

    \section{Implementation \& Experiments}
    \begin{frame}[t]
        \frametitle{Implementation}
        
        \begin{itemize}
            \item Too much code to properly show 
            \begin{itemize}
                \item Look at repository for overview
            \end{itemize}
            \pause
            \item Testing: validation code to check the correct implementation
            \begin{itemize}
                \item Independent implementations for same functionality
                \item Sanity checks for state of data and program
            \end{itemize}
            \pause
            \item Optimization: runtime comparison for different code styles
            \begin{itemize}
                \item For generated code fast execution over readability
            \end{itemize}
        \end{itemize}

        % notes 
        \onslide % on all slides of frame
        \note[item] {
            While nothing is looked at in detail maybe interesting for some
        }
        \note[item] {
            Testing: validation and assertion based programming
        }
        \note[item] {
            Optimization: code style runtime experiments
        }
    \end{frame}

    \begin{frame}[t]
        \frametitle{Numerical Experiments}
        
        \begin{itemize}
            \item Sadly no presentation possible in the small time allocated for the presentation
            \item The plots are in the presentation in the supplementary material
        \end{itemize}

        \makebox[\textwidth][c]{\includegraphics[width=0.6\textwidth]{./../thesis-latex/plotgeneration/concurrence-from-spin/concurrence-comparison.pdf}}

        % notes 
        \onslide % on all slides of frame
        \note[item] {
            See supplementary material for more plots and stuff
        }
    \end{frame}

    \section{Conclusion \& Outlook}
    \begin{frame}[t]
        \frametitle{Conclusion \& Outlook}
        
        \begin{exampleblock}{Conclusion:}
            \begin{itemize}
                \item Successfully transferred and validated the theory
                \item Working implementation
                \begin{itemize}
                    \item Extensively tested
                    \item Configurable and extensible
                \end{itemize}
                \item Started validating expectations with measurements on the compute cluster
            \end{itemize}
        \end{exampleblock}
        
        \pause

        \begin{block}{Outlook:}
            \begin{itemize}
                \item Re-write in lower level language to gain performance over Python
                \item Start more real experiments now that the theory is validated to work
            \end{itemize}
        \end{block}

        % notes 
        \onslide % on all slides of frame
        \note[item] {
            Conclusion:
            \begin{itemize}
                \item Managed to Solve all problems 
                \item Provide heavily \emph{tested} and \emph{extensible} library of code AND theoretical summary
                \item Started to validate expectations with computations on the cluster
            \end{itemize}
        }
        \note[item] {
            Future plans:
            \begin{itemize}
                \item re-write in performant language after testing is successful
                \item Go through the parameters to start investigating the behavior of the systems
            \end{itemize}
        }
    \end{frame}


    \section*{Summary \& Conclusion}
    {
        \setbeamertemplate{frametitle}[uniamntfack] % use the acknowledgment-style for this slide
        \begin{frame}[plain]{Acknowledgment}
            \vspace{1cm}

            Thank you for your kind attention\\

            \vspace{1cm}
            \makebox[0.4\textwidth][c]{
                \includegraphics[width=0.3\textwidth]{extra-slides/qr-github.png}
            }
        \end{frame}
    }

    \note[enumerate]{
        \item Thank you for your kind attention
        \item All tools and other resources are referenced in the presentation
        \item You can also find everything on my Github
    }

    \begingroup % group to not count pages from here
        \begin{frame}[allowframebreaks,noframenumbering]
            \frametitle{References}
            \nocite{*}
            \printbibliography[title={Bibliography}]
        \end{frame}
        
        \section*{Extra slides}
            
\begin{frame}[noframenumbering]{Theme alternative: Faculty of App. Computer Science}
    \hspace{1cm}\includegraphics[height=.7\paperheight]{./../latex-beamer-template/rendered-preview-pictures/FAICompilation.png}
\end{frame}

\note[enumerate]{
    \item Template does also do "Faculty of Applied Computer Science" presentations out of the box
    \item Alternate styles (e.g. title slide) available
}

\begin{frame}[noframenumbering]{Backup-Solution Problem 1}
    \hspace{2cm}\includegraphics[height=.72\paperheight]{./math-manipulator-calculations/fkp_page_15_backup_solution.png}
\end{frame}

\begin{frame}[noframenumbering]{Backup-Solution Problem 2}
    \hspace{0.6cm}\includegraphics[height=.72\paperheight]{./math-manipulator-calculations/fkp_page_33_backup_solution.png}
\end{frame}

\begin{frame}[noframenumbering, t]{Backup: Simplification V-Parts}
    \vspace{-0.2cm}
    \begin{minipage}[t]{0.6\textwidth}
        \vspace{0pt}
        \hspace{-2em}
        \includegraphics[width=.5\paperwidth]{./math-manipulator-calculations/simplification_cparts_backup_solution.png}
    \end{minipage}%
    \begin{minipage}[t]{0.6\textwidth}
        \vspace{0pt}
        \hspace{-3em}
        \includegraphics[width=.41\paperwidth]{./math-manipulator-calculations/simplification_cparts.pdf}
    \end{minipage}
\end{frame}

\note[enumerate]{
    \item Better optimization was found, rewriting in number operators DOES save on terms (however only for B and C, A still same number of terms and computational requirement, even if it looks nicer)
    \item Doesn't help at all with computing more efficiently, so python script still exceptinally usefull
    \item Wouldn't have come so far, if simple "brute-force" solution wasn't done in the first place
}

    \endgroup

\end{document}
