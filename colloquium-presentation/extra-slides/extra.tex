
\begin{frame}[noframenumbering]{Extra Plots: Overview}
    \begin{itemize}
        \item \hyperlink{plot:exact-comparison}{Exact comparison density matrix}
        \item \hyperlink{plot:energy-variance}{First orders: energy and variance comparison}
        \item \hyperlink{plot:j-sweep}{J parameter sweep}
        \item \hyperlink{plot:mc-convergence}{Monte Carlo sampling convergence}
        \item \hyperlink{plot:runtime-comparison}{Runtime validation}
        \item \hyperlink{plot:vcn-failing}{VCN: parametrization failiure}
        \item \hyperlink{plot:vcn-success-energy-conserved}{VCN: energy conservation can be controlled}
        \item \hyperlink{plot:vcn-small}{VCN: minimal square reference system}
        \item \hyperlink{plot:vcn-big}{VCN: larger square reference system in direct comparison}
        \item \hyperlink{plot:vcn-end-to-end-test}{VCN: system size dependency}
    \end{itemize}
\end{frame}

\begin{frame}[noframenumbering]{Exact Comparison: Concurrence}
    \label{plot:exact-comparison}
    \vspace{-0.4cm}
    \makebox[\textwidth][c]{\includegraphics[height=.8\paperheight]{./../thesis-latex/plotgeneration/concurrence-from-spin/concurrence-comparison.pdf}}
    \note[item]{
        Concurrence comparison of small system with exact calculations and comparison thereof
    }
    \note[item]{
        See extreme boost of second order 
    }
    \note[item]{
        NOTICE: how the External calculation is done COMPLETELY on density matrices and diagonalization, so prooves we can sample the reduced density matrix out
    }
\end{frame}
\begin{frame}[noframenumbering]{Exact Comparison: Purity}
    \vspace{-0.4cm}
    \makebox[\textwidth][c]{\includegraphics[height=.8\paperheight]{./../thesis-latex/plotgeneration/concurrence-from-spin/purity-comparison.pdf}}
    \note[item]{
        Purity measurements for the graph of the measurement from one page before
    }
\end{frame}
\begin{frame}[noframenumbering]{Exact Comparison: Pauli Measurements}
    \vspace{-0.6cm}
    \makebox[\textwidth][c]{\includegraphics[height=.83\paperheight]{./../thesis-latex/plotgeneration/concurrence-from-spin/pauli-measurements.pdf}}
    \note[item]{
        Portrait mode format, because extracted from the thesis. 
    }
    \note[item]{
        You can see what a problem the z-contributions are
    }
\end{frame}

\begin{frame}[noframenumbering]{First Orders: Energy Comparison}
    \label{plot:energy-variance}
    \vspace{-0.4cm}
    \makebox[\textwidth][c]{\includegraphics[height=.8\paperheight]{./../thesis-latex/plotgeneration/energy-variance/energy.pdf}}
    \note[item]{
        Second picture zoomed in
    }
\end{frame}
\begin{frame}[noframenumbering]{First Orders: Variance Comparison}
    \vspace{-0.4cm}
    \makebox[\textwidth][c]{\includegraphics[height=.8\paperheight]{./../thesis-latex/plotgeneration/energy-variance/variance.pdf}}
    \note[item]{
        Second picture zoomed in
    }
\end{frame}

\begin{frame}[noframenumbering]{J-Sweep: Current Border}
    \label{plot:j-sweep}
    \vspace{-0.4cm}
    \makebox[\textwidth][c]{\includegraphics[height=.8\paperheight]{./../thesis-latex/plotgeneration/j-sweep/current-border.pdf}}
    \note[item]{
        Logarithmic plot
    }
    \note[item]{
        Measurements show the trend of the convergence of the relative error for small perturbations
    }
    \note[item]{
        Depends on the observable though, still
    }
\end{frame}
\begin{frame}[noframenumbering]{J-Sweep: Current Center}
    \vspace{-0.4cm}
    \makebox[\textwidth][c]{\includegraphics[height=.8\paperheight]{./../thesis-latex/plotgeneration/j-sweep/current-center.pdf}}
\end{frame}
\begin{frame}[noframenumbering]{J-Sweep: Single Occupation Center}
    \vspace{-0.4cm}
    \makebox[\textwidth][c]{\includegraphics[height=.8\paperheight]{./../thesis-latex/plotgeneration/j-sweep/single-occ-center.pdf}}
\end{frame}
\begin{frame}[noframenumbering]{J-Sweep: Double Occupation Border}
    \vspace{-0.4cm}
    \makebox[\textwidth][c]{\includegraphics[height=.8\paperheight]{./../thesis-latex/plotgeneration/j-sweep/double-occ-border.pdf}}
\end{frame}

\begin{frame}[noframenumbering]{Monte Carlo Convergence: Occupation}
    \label{plot:mc-convergence}
    \vspace{-0.4cm}
    \makebox[\textwidth][c]{\includegraphics[height=.8\paperheight]{./../thesis-latex/plotgeneration/monte-carlo-variance-test/occupation.pdf}}
    \note[item]{
        Well as we can see, the appriximation converges as the variance gos to zero
    }
\end{frame}
\begin{frame}[noframenumbering]{Monte Carlo Convergence: Occupation}
    \vspace{-0.4cm}
    \makebox[\textwidth][c]{\includegraphics[height=.8\paperheight]{./../thesis-latex/plotgeneration/monte-carlo-variance-test/spin-current.pdf}}
\end{frame}

\begin{frame}[noframenumbering]{Runtime Validation}
    \label{plot:runtime-comparison}
    \vspace{-0.4cm}
    \makebox[\textwidth][c]{\includegraphics[height=.8\paperheight]{./../thesis-latex/plotgeneration/runtime-complexity/runtime.pdf}}
\end{frame}

\begin{frame}[noframenumbering]{VCN Failiure: Non-Explicit Parametrization}
    \label{plot:vcn-failing}
    \vspace{-0.4cm}
    \makebox[\textwidth][c]{\includegraphics[height=.8\paperheight]{./../thesis-latex/plotgeneration/vcn-eff-stepsize/energy-variance.pdf}}
    \note[item]{
        Complex part now properly vanishes (was long a problem)
    }
    \note[item]{
        See how the vcn goes like the base energy curve only in the beginning (suspiciously)
    }
\end{frame}
\begin{frame}[noframenumbering]{VCN Failiure: Non-Explicit Parametrization}
    \vspace{-0.4cm}
    \makebox[\textwidth][c]{\includegraphics[height=.8\paperheight]{./../thesis-latex/plotgeneration/vcn-eff-stepsize/vcn-parameters.pdf}}
    \note[item]{
        See complex part jetting off to infinity because it doesn't knwo how to compensate
    }
\end{frame}

\begin{frame}[noframenumbering]{VCN Validation: Energy Conservation Controllable}
    \label{plot:vcn-success-energy-conserved}
    \vspace{-0.4cm}
    \makebox[\textwidth][c]{\includegraphics[height=.8\paperheight]{./../thesis-latex/plotgeneration/vcn-energy-conservation/energy-conservation.pdf}}
    \note[item]{
        Still no perfect conservation for infinite steps, but this is probably due to first order integration only
    }
\end{frame}

\begin{frame}[noframenumbering]{VCN: Small Square System}
    \label{plot:vcn-small}
    \vspace{-0.4cm}
    \makebox[\textwidth][c]{\includegraphics[height=.8\paperheight]{./../thesis-latex/plotgeneration/vcn-square-small/energy-variance.pdf}}
    \note[item]{
        Notice how VCN now is MUCH better
    }
\end{frame}
\begin{frame}[noframenumbering]{VCN: Small Square System}
    \vspace{-0.4cm}
    \makebox[\textwidth][c]{\includegraphics[height=.8\paperheight]{./../thesis-latex/plotgeneration/vcn-square-small/energy-variance-step-dependent.pdf}}
    \note[item]{
        Step size dependent version of previous slide
    }
\end{frame}

\begin{frame}[noframenumbering]{VCN: Bigger Square System}
    \label{plot:vcn-big}
    \vspace{-0.4cm}
    \makebox[\textwidth][c]{\includegraphics[height=.8\paperheight]{./../thesis-latex/plotgeneration/vcn-square-comparison/energy-variance.pdf}}
    \note[item]{
        Now not a guarantee for immediately better results, but very promising
    }
\end{frame}
\begin{frame}[noframenumbering]{VCN: Bigger Square System}
    \vspace{-0.4cm}
    \makebox[\textwidth][c]{\includegraphics[height=.8\paperheight]{./../thesis-latex/plotgeneration/vcn-square-comparison/energy-variance-step-dependent.pdf}}
    \note[item]{
        Here one can see very quickly more steps doesn't only make it better
    }
\end{frame}

\begin{frame}[noframenumbering]{VCN End to End Test: System Size Dependency}
    \label{plot:vcn-end-to-end-test}
    \vspace{-0.4cm}
    \makebox[\textwidth][c]{\includegraphics[height=.8\paperheight]{./../thesis-latex/plotgeneration/system-size-dependency/system-size-energy.pdf}}
    \note[item]{
        This is the energy for validation
    }
\end{frame}
\begin{frame}[noframenumbering]{VCN End to End Test: System Size Dependency}
    \vspace{-0.4cm}
    \makebox[\textwidth][c]{\includegraphics[height=.8\paperheight]{./../thesis-latex/plotgeneration/system-size-dependency/system-size-purity.pdf}}
    \note[item]{
        This is the purity
    }
    \note[item]{
        Does not seem to have a huge variation
    }
\end{frame}
\begin{frame}[noframenumbering]{VCN End to End Test: System Size Dependency}
    \vspace{-0.4cm}
    \makebox[\textwidth][c]{\includegraphics[height=.8\paperheight]{./../thesis-latex/plotgeneration/system-size-dependency/system-size-concurrence.pdf}}
    \note[item]{
        This is the concurrence
    }
    \note[item]{
        Poorly resolved, this should be looked into possibly in the future
    }
\end{frame}
\begin{frame}[noframenumbering]{VCN End to End Test: System Size Dependency}
    \vspace{-0.4cm}
    \makebox[\textwidth][c]{\includegraphics[height=.8\paperheight]{./../thesis-latex/plotgeneration/system-size-dependency/system-size-current.pdf}}
    \note[item]{
        This is the current
    }
    \note[item]{
        Most obvious effect, but still extremely stable over a order of magnitude of sites
    }
    \note[item]{
        See assymmetric behavior, because electrical field is set to not be symmetric
    }
\end{frame}
